%%
%% $Revision: $"
%%

\documentclass[a4paper,12pt]{article}
\usepackage[utf8]{inputenc}
\usepackage[spanish]{babel}
\usepackage{geometry}
\usepackage{hyperref}
\usepackage{url}
\hypersetup{
    colorlinks=true,
    linkcolor=blue,
    filecolor=magenta,      
    urlcolor=cyan,
  }

\title{Servicios y Aplicaciones Telemáticas \\
Grado en Ingeniería en Tecnologías de Telecomunicación \\
Programa del curso 2021/2022}
\author{Jesús M. González Barahona, Gregorio Robles Martínez, \\
  Francisco Servant \\
GSyC, Universidad Rey Juan Carlos}

%\date{}

% Para que el margen derecho no se desborde
\sloppy

\begin{document}
\maketitle

\tableofcontents

\newpage

%%--------------------------------------------------------------------------
%%--------------------------------------------------------------------------
%%--------------------------------------------------------------------------
\section{Datos generales}

\begin{tabular}{ll}
\textbf{Título:} & Servicios y Aplicaciones Telemáticas \\
\textbf{Titulación:} & Grado en Ingeniería en Tecnologías de Telecomunicación \\
\textbf{Cuatrimestre:} & Tercer curso, segundo cuatrimestre \\
\textbf{Créditos:} & 6 (3 teóricos, 3 prácticos) \\
\textbf{Horas lectivas:} & 4 horas semanales \\
\textbf{Horario:} & martes, 11:00--13:00 \\
                  & jueves, 11:00--13:00 \\
\textbf{Profesores:}
& Francisco Servant \\
& \hspace{1cm}francisco.servant @ urjc.es \\
& \hspace{1cm}Despacho 101, Departamental III\\
& Gregorio Robles Martínez\\
& \hspace{1cm}grex @ gsyc.es \\
& \hspace{1cm}Despacho 110, Departamental III\\
\textbf{Sedes telemáticas:} & \url{https://aulavirtual.urjc.es} \\
                            & \url{https://cursosweb.github.io} \\
                            & \url{https://gitlab.etsit.urjc.es/cursosweb} \\
\textbf{Aulas:} & Laboratorio 207, Edif. Laboratorios III
\end{tabular}

\newpage

%%-------------------------------------------------------------------------
%%-------------------------------------------------------------------------
%%-------------------------------------------------------------------------
\section{Objetivos}

En esta asignatura se pretende que el alumno obtenga conocimientos detallados sobre los servicios y aplicaciones comunes en las redes de ordenadores, y en particular en Internet. Se pretende especialmente que conozcan las tecnologías básicas que los hacen posibles.

\section{Metodología}

La asignatura tiene un enfoque eminentemente práctico. Por ello se realizará en la medido de lo posible en el laboratorio, y las prácticas realizadas (incluyendo especialmente el proyecto final) tendrán gran importancia en la evaluación de la asignatura. Los conocimientos teóricos necesarios se intercalarán con los prácticos, en gran media mediante metodologías apoyadas en la resolución de problemas. En las clases teóricas se utilizan, en algunos casos, transparencias que sirven de guión. En todos los casos se recomendarán referencias (usualmente documentos disponibles en Internet) para profundizar conocimientos, y complementarias de los detalles necesarios para la resolución de los problemas prácticos. En el desarrollo diario, las sesiones docentes incluirán habitualmente tanto aspectos teóricos como prácticos.

Se usa un sistema de apoyo telemático a la docencia (aula virtual de la URJC) para realizar actividades complementarias a las presenciales, y para organizar parte de la documentación ofrecida a los alumnos. La mayoría de los contenidos utilizados en la asignatura están disponibles o enlazados desde el sitio web CursosWeb. Asimismo, se utiliza el servicio GitLab de la ETSIT como repositorio, tanto de los materiales de la asignatura, como para entregar las prácticas por parte de los alumnos.

\newpage

%%--------------------------------------------------------------------------
%%--------------------------------------------------------------------------
%%--------------------------------------------------------------------------
\section{Evaluación}

% \subsection{Cambios excepcionales para el curso 2019-2020}

% Debido a las circunstancias excepcionales de este curso académico, hemos tenido que realizar algunos cambios a los criterios de evaluación. Estos cambios suponen sólo un cambio en cuanto a cómo se realizarán alguna de las actividades evaluables, y en cuanto a sus condiciones y plazos de entrega. No suponen en general un cambio en los criterios de evaluación, salvo que se tratará de tener una visión más global de la evaluación de la asignatura, sabiendo que los alumnos se han podido tener que enfrentar a situaciones excepcionales, y que se hará un esfuerzo por asegurar que el alumno ha conseguido los objetivos personales de conocimiento y habilidad que se esperan en esta asignatura.

% Los cambios son los siguientes:

% \begin{itemize}
% \item Se tendrán en cuenta las microprácticas diarias y las miniprácticas preparatorias aunque se hayan entregado con posterioridad a la fecha indicada. No se tendrán en cuenta exactamente igual que si se hubieran entregado en plazo, sino (aproximadamente) con un 75\% de valoración. Aún así se podrán conseguir notas cercanas a los dos puntos (que es máximo para el total de estas prácticas) en casos en que estén realizadas especialmente bien, o se hayan entregado prácticamente todas y bien realizadas.
  
% \item En lugar de la prueba de teoría se propondrán unas preguntas para evaluar el contenido más teórico, en relación a el proyecto final de cada alumno. Estas preguntas cubrirán conocimientos y habilidades similares a los evaluados normalmente con la prueba de teoría. Las preguntas se entregarán, vía indicaciones en el campus virtual, el miércoles 10 de junio, y tendrán una fecha límite de entrega de viernes 12 de junio (detalles en las condiciones de entrega de la practica final).

% \item Se realizará una entrevista personal, vía videoconferencia, o si eso no fuera posible por otros medios, con el mayor número posible de alumnos. Estas entrevistas tendrán lugar en las fechas que se indica en el enunciado de el proyecto final. El objetivo de esta entrevista es evaluar que el alumno puede explicar, en el contexto de los conocimientos y habilidades necesarios para aprobar esta asignatura, lo que ha realizado tanto en el proyecto final como en su respuesta a las preguntas que se usan este curso para evaluar el contenido más teórico. el resultado de esta entrevista puede tener un gran impacto en la calificación final, por la vía de aclarar o invalidar las respuestas, o el proyecto final entregada. Es fundamental que el alumno pueda explicar y razonar cualquier aspecto de ambas, y hacerlo en el contexto de los conocimientos y habilidades de la asignatura.
% \end{itemize}

% Estos cambios se aplicarán tanto para la convocatoria ordinaria como para la extraordinaria. Para la práctica extraordinaria se publicarán nuevas fechas de entrega y entrevistas, y el enunciado de el proyecto final (que será similar, pero no igual, a la propuesta para la convocatoria ordinaria).

\subsection{Criterios de evaluación}

Parámetros generales:

\begin{itemize}
\item Teoría (obligatorio): 0 a 4.
\item Microprácticas diarias: 0 a 1
\item Miniprácticas preparatorias: 0 a 1
\item Proyecto final (obligatorio): 0 a 2.
\item Opciones y mejoras del proyecto final: 0 a 3
\item Nota final: Suma de notas, moderada por la interpretación del profesor
\item Mínimo para aprobar:
      \begin{itemize}
      \item aprobado en teoría (2) y proyecto final (1)
      \item 5 puntos de nota final
      \end{itemize}
\end{itemize}

Evaluación teoría: prueba escrita

Evaluación microprácticas diarias (evaluación continua):

\begin{itemize}
\item entre 0 y 1
\item preguntas y ejercicios en foro y entragados en GitLab
\item es muy recomendable hacerlas
\end{itemize}

Evaluación proyecto final:

\begin{itemize}
\item posibilidad de examen presencial para proyecto final
\item tiene que funcionar en el laboratorio
\item enunciado mínimo obligatorio supone 1, se llega a 2 sólo con calidad y cuidado en los detalles
\item realización individual de la práctica
\end{itemize}

Opciones y mejoras proyecto final:

\begin{itemize}
\item permiten subir la nota mucho
\end{itemize}

Evaluación extraordinaria:

\begin{itemize}
\item prueba escrita (si no se aprobó la ordinaria)
\item nuevo proyecto final (si no se aprobó la ordinaria)
\item entrega de ejercicios de evaluación continua (con penalización)
\end{itemize}

\newpage


%%--------------------------------------------------------------------------
%%--------------------------------------------------------------------------
%%--------------------------------------------------------------------------
\section{Calendario}

\newcommand{\martesA}{25 de enero}
\newcommand{\martesB}{1 de febrero}
\newcommand{\martesC}{8 de febrero}
\newcommand{\martesD}{15 de febrero}
\newcommand{\martesE}{22 de febrero}
\newcommand{\martesF}{1 de marzo}
\newcommand{\martesG}{8 de marzo}
\newcommand{\martesH}{15 de marzo}
\newcommand{\martesI}{22 de marzo}
\newcommand{\martesJ}{29 de marzo}
\newcommand{\martesK}{5 de abril}
\newcommand{\martesL}{19 de abril}
\newcommand{\martesM}{26 de abril}
\newcommand{\martesN}{3 de mayo}
\newcommand{\martesO}{10 de mayo}

\newcommand{\juevesA}{27 de enero}
\newcommand{\juevesB}{3 de febrero}
\newcommand{\juevesC}{10 de febrero}
\newcommand{\juevesD}{17 de febrero}
\newcommand{\juevesE}{24 de febrero}
\newcommand{\juevesF}{3 de marzo}
\newcommand{\juevesG}{10 de marzo}
\newcommand{\juevesH}{17 de marzo}
\newcommand{\juevesI}{24 de marzo}
\newcommand{\juevesJ}{31 de marzo}
\newcommand{\juevesK}{7 de abril}
\newcommand{\juevesL}{21 de abril}
\newcommand{\juevesM}{28 de abril}
\newcommand{\juevesN}{5 de mayo}

\begin{tabular}{|l|l|}
   \hline
  Martes & Jueves \\ \hline \hline
  \nameref{cal:martesA} & \nameref{cal:juevesA} \\ \hline
                        % & \nameref{cal:juevesAa} \\ \hline
  \nameref{cal:martesB} & \nameref{cal:juevesB} \\
                        & \nameref{cal:juevesBb} \\  \hline
  \nameref{cal:martesC} & \nameref{cal:juevesC} \\ \hline
  \nameref{cal:martesD} & \nameref{cal:juevesD} \\
                        & \nameref{cal:juevesDa} \\ \hline
  \nameref{cal:martesE} & \nameref{cal:juevesE} \\  \hline
  \nameref{cal:martesF} & \nameref{cal:juevesF} \\ \hline
  \nameref{cal:martesG} & \nameref{cal:juevesG} \\ \hline
  \nameref{cal:martesH} & \nameref{cal:juevesH} \\ \hline
  \nameref{cal:martesI} & \nameref{cal:juevesI} \\ \hline
  \nameref{cal:martesJ} & \nameref{cal:juevesJ} \\ \hline
  \nameref{cal:martesK} & \nameref{cal:juevesK} \\ \hline
  \nameref{cal:martesL} & \nameref{cal:juevesL} \\ \hline
  \nameref{cal:martesM} & \nameref{cal:juevesM} \\ \hline
  \nameref{cal:martesN} & \nameref{cal:juevesN} \\ \hline
  \nameref{cal:martesO} & \\ \hline
\end{tabular}

\newpage

%%--------------------------------------------------------------------------
%%--------------------------------------------------------------------------
%%--------------------------------------------------------------------------
\section{Programa de teoría}

Programa de la asignatura (el detalle evoluciona según avanza el curso).

%%--------------------------------------------------------------------------
%%--------------------------------------------------------------------------
\subsection{00 - Presentación}

%%-----------------------------------------------------------------------
\subsubsection{\martesA: Presentación  (2 horas)}
\label{cal:martesA}

\begin{itemize}
\item \textbf{Presentación:} Presentación de la temática de la asignatura
\item \textbf{Presentación:} Qué son las aplicaciones web del lado del servidor (``back-end'') y del lado del cliente (``front-end''), y cómo se relacionan.
\item \textbf{Presentación:} Detalles de la asignatura: teoría y prácticas, estructura de las clases, evaluación, etc.
\item \textbf{Presentación:} Materiales de la asignatura: sitios web y documentos fundamentales que tenéis a vuestra disposición.
\item \textbf{Material:} Transparencias, tema ``Presentación''.
\item \textbf{Ejercicio propuesto (voluntario, entrega en el foro):} ``Web 2.0'' (ejercicio~\ref{subsec:web-20}) \\
  Entrega recomendada: antes del \juevesA.
\end{itemize}


%%----------------------------------------------------------------------
\subsection{01 - Conceptos básicos de aplicaciones web}

Sesión, mantenimiento de estado, persistencia.

%%------------------------------------------------------------------------
\subsubsection{\martesB: Conceptos básicos I (2 horas)}
\label{cal:martesB}

Páginas dinámicas (diferentes según cómo y cuándo se invocan). Cómo realizar sesiones en HTTP. Profundización en el concepto de sesión, y técnicas para conseguirla, incluyendo cookies y otros mecanismos.

\begin{itemize}
\item \textbf{Ejercicio (presentación en clase):} ``Espía a tu navegador (Firefox Developer Tools)'' (ejercicio~\ref{subsec:firefox-devel}) \\
  Centrado en la pestaña de ``Red''.
\item \textbf{Ejercicio propuesto (entrega en el foro):} ``Explora tus cookies'' (ejercicio~\ref{subsec:explora-cookies}) \\
  Entrega recomendada: antes del \juevesB.
\item \textbf{Ejercicio (presentación):} ``Última búsqueda'' (ejercicio~\ref{subsec:ultima-busqueda})
\end{itemize}


%%----------------------------------------------------------------
\subsubsection{\martesC: Conceptos básicos II (2 horas)}
\label{cal:martesC}

Datos persistentes entre operaciones HTTP diferentes. Concepto de estado persistente frente a caídas del servidor.

\begin{itemize}
\item Resolución de ejericios pendientes.
\item \textbf{Ejercicio (presentación en clase):} ``Espía a tu navegador (Firefox Developer Tools)'' (ejercicio~\ref{subsec:firefox-devel}) \\
  Centrado en la pestaña de ``Inspección''.
\item \textbf{Presentación:} Cookies
\item \textbf{Material:} Transparencias, tema ``Cookies''
\item \textbf{Ejercicio (discusión en clase, entrega en el foro):} ``Última búsqueda'' (ejercicio~\ref{subsec:ultima-busqueda}) \\
  Entrega recomendada: antes del \juevesC. \\
  Se introducen las dos solucines típicas: cookie con la pregunta, y cookie con un identificador más tabla en el servidor.
\end{itemize}



%%----------------------------------------------------------------
\subsubsection{\martesD: Conceptos básicos III (2 horas)}
\label{cal:martesD}

\begin{itemize}
\item \textbf{Ejercicio (discusión en clase):} ``Última búsqueda'' (ejercicio~\ref{subsec:ultima-busqueda}) \\
  Discusión sobre el uso de cookies (u otros mecanismos) para conseguir la funcionalidad requerida. Se discute también el almacenamiento en el lado del servidor y en el lado del cliente. Relación entre peticiones HTTP. Cookies como herramienta para ambas situaciones. Analogía con la agencia de viajes física. Cómo tienen que ser las cookies de identificación para que no sean fácilmente ``adivinables''. Introducción al uso de cookies para autenticación (sólo queda planteado)
\item \textbf{Ejercicio propuesto (entrega en el foro):} ``Explora tus cookies (2)'' (ejercicio~\ref{subsec:explora-cookies-2}) \\
  Entrega recomendada: antes del \juevesD.
\end{itemize}

%%----------------------------------------------------------------
\subsubsection{\martesE: Conceptos básicos IV (2 horas)}
\label{cal:martesE}

\begin{itemize}
\item \textbf{Ejercicio (discusión en clase):} ``Explora tus cookies (2)'' (ejercicio~\ref{subsec:explora-cookies-2}).
\item \textbf{Discusión:} Usos de las cookies. \\
  Uso de las cookies para identificación de visitantes (como en el ejercicio~\ref{subsec:ultima-busqueda}), para autenticación (interacción de autenticación y cookie de sesión posterior), para almacenamiento (como en el ejercicio~\ref{subsec:ultima-busqueda}, con última búsqueda en la cookie). Implicaciones de trasladar una cookie de identificación o de sesión de un ordenador a otro. Implicaciones de almacenar datos en el lado del navegador.
\item \textbf{Ejercicio propuesto:} ``Cookies en tu navegador'' (ejercicio~\ref{subsec:cookies-navegador}) \\
\item \textbf{Ejercicio propuesto:} ``Cookies en tu navegador avanzado'' (ejercicio~\ref{subsec:cookies-navegador-2}) \\
\item \textbf{Ejercicio propuesto:} ``Contador simple'' (ejercicio~\ref{subsec:contador-simple})
\item \textbf{Discusión de ejercicio:} ``cURL básico'' (ejercicio~\ref{subsec:curl-basico})
\item \textbf{Discusión de ejercicio:} ``Depurador básico'' (ejercicio~\ref{subsec:depurador-basico})
\item \textbf{Ejercicio propuesto (entrega en el foro:} ``Transplante de cookies'' (ejercicio~\ref{subsec:transplante-cookies}) \\
  Entrega recomendada: antes del \martesF.
\end{itemize}

%%----------------------------------------------------------------
\subsubsection{\martesF: Conceptos básicos V (2 horas)}
\label{cal:martesF}

\begin{itemize}
\item \textbf{Discusión:} Medición de audiencias y visitas únicas por un sitio web.
\item \textbf{Ejercicio propouesto (entrega en el foro):} ``Traza de historiales de navegación por terceras partes'' (ejercicio~\ref{subsec:navegacion-terceras-partes}) \\
  Entrega recomendada: antes del \martesG.
\end{itemize}

%%----------------------------------------------------------------
\subsubsection{\martesG: Conceptos básicos VI (2 horas)}
\label{cal:martesG}

\begin{itemize}
\item Relación de traza de historias de navegación con identidades personales.
\item \textbf{Ejercicio (demo):} ``Trackers en páginas web'' (ejercicio~\ref{subsec:trackers-paginas-web})
\item Presentación de la minipráctica 1 (enunciado en~\ref{subsec:practica-vol-1-2016})
\end{itemize}


%%----------------------------------------------------------------
%%----------------------------------------------------------------
\subsection{02 - Servicios web que interoperan}

Invocaciones a aplicaciones web desde aplicaciones web. Servicios web como un conjunto de aplicaciones que interoperan.

%%----------------------------------------------------------------
\subsubsection{\martesH: Interoperación web I (2 horas)}
\label{cal:martesH}

\begin{itemize}
\item \textbf{Ejercicio (discusión en clase):} ``Contador simple con varios 
navegadores intercalados'' (ejercicio~\ref{subsec:contador-simple-varios-intercalados})
\item \textbf{Discusión:} Introducción al problema de los rearranques.
\item \textbf{Ejercicio (entrega en el foro):} ``Contador simple con rearranques'' (ejercicio~\ref{subsec:contador-simple-rearranques}).
\end{itemize}

%%----------------------------------------------------------------
\subsubsection{\martesI: Interoperación web II (2 horas)}
\label{cal:martesI}

\begin{itemize}
\item Introducción al diseño de APIs HTTP
\item \textbf{Ejercicio (discusión en clase)}: ``Lista de la compra'' (ejercicio~\ref{subsec:lista-compra}). \\
  Trabajo en grupos y discusión de los detalles del ejercicio.
\item \textbf{Presentación:} Arquitectura REST
\item \textbf{Material:} Transparencias, tema ``REST''
%\item \textbf{Discusión de ejercicio:} ``Cache de contenidos'' (ejercicio~\ref{subsec:cache-contenidos}). \\
%  Trabajo en grupos y discusión de los detalles del ejercicio.
%  Repo GitLab: \url{https://gitlab.etsit.urjc.es/CursosWeb/X-Serv-App-Cache}
\item \textbf{Ejercicio (presentación):} ``Listado de lo que tengo en la nevera'' (ejercicio~\ref{subsec:contenido-nevera}).
\end{itemize}

%%----------------------------------------------------------------
\subsubsection{\martesJ: Interoperación web III (2 horas)}
\label{cal:martesJ}

\begin{itemize}
\item \textbf{Discusión:} Introducción a las operaciones idempotentes.
\item \textbf{Ejercicio (discusión en clase):} ``Listado de lo que tengo en la nevera'' (ejercicio~\ref{subsec:contenido-nevera}).  
\item \textbf{Ejercicio (entrega en el foro):} ``Listado de lo que tengo en la nevera'' (ejercicio~\ref{subsec:contenido-nevera}) \\
  Entrega recomendada: \juevesI
\end{itemize}

%% \item \textbf{Ejercicio (discusión en clase, entrega en el foro):} ``Calculadora simple versión REST'' (ejercicio~\ref{subsec:calc-simple-rest}). \\
%%   Entrega recomendada: antes del 16 de marzo. \\
%%   Repo GitLab: \url{https://github.com/CursosWeb/X-Serv-App-Calculadora-REST}

%%\item \textbf{Ejercicio (discusión en clase, entrega en GitLab):} ``Aplicación redirectora'' (ejercicio~\ref{subsec:aplweb-redirectora}). \\
%%  Entrega recomendada: antes del 9 de marzo. \\
%%  Repo GitLab: \url{https://github.com/CursosWeb/X-Serv-App-Redirectora}

%% \item \textbf{Ejercicio (entrega en el foro):} ``Cache de contenidos anotado'' (ejercicio~\ref{subsec:cache-contenidos-anotado}) \\
%%   Entrega recomendada: antes del 23 de marzo. \\
%%   Repo GitLab: \url{https://github.com/CursosWeb/X-Serv-App-Cache-Anotada}

%% \item \textbf{Ejercicio (discusión en clase, entrega en el foro):} ``Sistema de transferencias bancarias'' (ejercicio~\ref{subsec:transferencias-bancarias}). \\
%% Entrega recomendada: antes del 12 de marzo.
%% \item \textbf{Ejercicio (entrega en el foro):} ``Sistema REST para calcular Pi'' (ejercicio~\ref{subsec:rest-pi}) \\
%% Entrega recomendada: antes del 12 de marzo.

%% \item \textbf{Ejercicio:} ``Arquitectura escalable'' (ejercicio~\ref{subsec:arq-escalable}).
%% \item \textbf{Ejercicio:} ``Gestor de contenidos multilingue preferencias del navegador'' (ejercicio~\ref{subsec:contentappmulti-navegador}).
%% \item \textbf{Ejercicio:} ``Gestor de contenidos multilingue con elección en la aplicación'' (ejercicio~\ref{subsec:contentappmulti-apli}).
%\end{itemize}

%%----------------------------------------------------------------
%%----------------------------------------------------------------
\subsection{03 - Modelo-vista-controlador}

Explicación del patrón de diseño ``modelo-vista-controlador''.

%%----------------------------------------------------------------
\subsubsection{\martesK: MVC (2 horas)}
\label{cal:martesK}

\begin{itemize}
\item \textbf{Presentación:} ``Tres implementaciones de una aplicación web simple: Counter''. \\
  Comparación de tres formas de implementar una aplicación web muy sencilla, identificando los componentes y estructuras que se repiten, con el objetivo de ver cómo cuando pasamos a Django seguimos construyendo el mismo tipo de apicaciones, aunque el marco de programación nos proporcione ya muchos elementos que no tenemos que construir.
\item \textbf{Código:} Programas \verb|counter-server-1.py| (directorio \verb|Python-Web/counter|, \verb|counterapp.py| (directorio \verb|Python-Web/http-server-classes/counterapp.py|) y projecto Django \verb|django-counter| (directorio \verb|Python-Django|).
%\item \textbf{Videos:} ``Implementación de aplicaciones web: Counter Server'', ``Implementación de aplicaciones web: Counter WebApp'', ``Implementación de aplicaciones web: Counter Django''.
\end{itemize}

%%----------------------------------------------------------------
\subsubsection{\martesL: MVC II (2 horas)}
\label{cal:martesL}

\begin{itemize}
\item \textbf{Presentación:} ``Modelo-vista-controlador''.
\item \textbf{Material:} Transparencias ``Modelo-vista-controlador''.
\item \textbf{Presentación:} ``Componentes de aplicaciones Django y MVC''.
  Repaso de los componentes principales de una aplicación Django y su relación con el patrón modelo-vista-controlador.
\item \textbf{Video:} ``Arquitectura Modelo-Vista-Controlador''
\end{itemize}


%%----------------------------------------------------------------
%%----------------------------------------------------------------
\subsection{04 - Introducción a XML y JSON}

Uso de XML en aplicaciones web.

%%----------------------------------------------------------------
\subsubsection{\martesM: XML, JSON I (2 horas)}
\label{cal:martesM}

\begin{itemize}
\item \textbf{Presentación:} ``XML: Conceptos fundamentales''. \\
  Introducción a XML, sintaxis básica, equivalencia con el árbol XML correspondiente a un documento.
\item \textbf{Video:} ``XML: Conceptos fundamentales''.

\item \textbf{Presentación:} ``XML: Lenguajes de definición de vocabularios XML''. \\
  Formas de especificar vocabularios XML, ejemplo de DTD simple.
\item \textbf{Video:} ``XML: Lenguajes de definición de vocabularios XML''.
\item \textbf{Presentación:} ``XML: Módulos habituales para trabajar con XML desde lenguajes de programación''. \\
  Reconocedores SAX y DOM, y su uso en aplicaciones web.
\item \textbf{Video:} ``XML: Módulos habituales para trabajar con XML desde lenguajes de programación''.
\item \textbf{Presentación:} ``XML: Usos en aplicaciones web''. \\
  Usos habituales de XML en aplicaciones web, incluyendo el DOM de los navegadores y los canales RSS.
\item \textbf{Video:} ``XML: Usos en aplicaciones web''.
\item \textbf{Material:} Transparencias ``Introducción a XML''.
\item \textbf{Ejercicio (discusión en clase):} ``Chistes XML'' (ejercicio~\ref{subsec:xml-chistes}).
\item \textbf{Ejercicio (discusión en clase):} ``Videos en canal de YouTube'' (ejercicio~\ref{subsec:xml-youtube}).
\item \textbf{Ejercicio (entrega en GitLab):} ``Videos en canal de YouTube (con descarga)'' (ejercicio~\ref{subsec:xml-youtube-descarga}) \\
  Repositorio: \url{https://gitlab.etsit.urjc.es/cursosweb/practicas/server/youtube-descarga}
\end{itemize}

%%----------------------------------------------------------------
\subsubsection{\martesN: XML, JSON II (2 horas)}
\label{cal:martesN}

\begin{itemize}
\item \textbf{Presentación:} ``JSON (JavaScript Object Nottion)''. \\
  JSON como formato de intercambio de datos en aplicaciones web.
\item \textbf{Video:}``JSON (JavaScript Object Nottion)''.
\item \textbf{Presentación:} ``XML y JSON: Ejemplos reales''. \\
  Ejemplo de canal XML de YouTube y documento JSON de GitLab.
\item \textbf{Video:}``XML y JSON: Ejemplos reales''.
\item \textbf{Ejercicio (discusión en clase):} ``Forks de un repositorio GitLab''
  (ejercicio~\ref{subsec:json-gitlab-forks})
\item \textbf{Ejercicio (discusión en clase):} ``Gestor de contenidos con videos de YouTube (simple)'' (ejercicio~\ref{subsec:django-cms-youtube})
\item \textbf{Ejercicio (entrega en GitLab):} ``Gestor de contenidos con videos de YouTube (2)'' (ejercicio~\ref{subsec:django-cms-youtube-2}) \\
  Repositorio: \url{https://gitlab.etsit.urjc.es/cursosweb/practicas/server/django-youtube} \\
  Entrega: \juevesL
\item \textbf{Ejercicio voluntario:} ``Municipios JSON via HTTP'' (ejercicio~\ref{subsec:json-municipios-http}).
%\item \textbf{Ejercicio (entrega en GitLab):} ``Gestor de contenidos con video de Youtube (tests) (ejercicio~\ref{subsec:django-cms-youtube-tests}) \\
%  Repositorio: \url{https://gitlab.etsit.urjc.es/cursosweb/practicas/server/django-youtube-tests}
\end{itemize}

%%----------------------------------------------------------------
%\subsubsection{Sesión del ... (2 horas)}

%\item \textbf{Ejercicio (complementario):} ``Gestor de contenidos con titulares de BarraPunto versión SQL'' (ejercicio~\ref{subsec:contentapp-barrapunto-sql}). 
%%----------------------------------------------------------------
%\subsubsection{Sesión del ... (2 horas)}
%
%\begin{itemize}
%\item \textbf{Presentación:} ``Introducción a XML''. \\
%Uso básico de HTML DOM desde JavaScript. 
%\item \textbf{Material:} Transparencias ``Introducción a XML''.
%\item \textbf{Demo:} Manejo de HTML DOM con JavaScript.
%\item \textbf{Ejercicio (entrega en el foro):} ``Modificación del contenido de una página HTML'' (ejercicio~\ref{subsec:xml-modificacion-html}). \\
% Entrega recomendada: antes del 2 de mayo.
%
%\item \textbf{Material:} Fichero dom.html
%\end{itemize}
%
%
%%----------------------------------------------------------------
%%----------------------------------------------------------------
\subsection{05 - Hojas de estilo CSS}

Hojas de estilo CSS, separación entre contenido y presentación.

%%----------------------------------------------------------------
\subsubsection{\martesO: CSS (2 horas)}
\label{cal:martesO}

Hojas de estilo CSS, y su uso para manejar la apariencia de las páginas HTML.

\begin{itemize}
\item \textbf{Presentación:} ``Hojas de estilo CSS''. Introducción a CSS. Principales elementos.
 \item \textbf{Material:} Transparencias, tema ``CSS''.
\item \textbf{Demo:} Inspección de datos de aspecto y hojas CSS con el depurador de Firefox.
\item \textbf{Ejercicio (discusión en clase):} ``Django cms\_css simple'' (ejercicio~\ref{subsec:django-cms-css}).
\item \textbf{Ejercicio (entrega en GitLab):} ``Django cms\_css elaborado'' (ejercicio~\ref{subsec:django-cms-css-2}). \\
  Entrega recomendada: antes del 1 de mayo.
\end{itemize}


%% ----------------------------------------------------------------
%% ----------------------------------------------------------------
%% \subsection{06 - AJAX}

%% Introducción a Ajax, mashups y otros tipos de aplicaciones web con código en el lado del cliente.

%% ----------------------------------------------------------------
%% \subsubsection{Sesión del 3 de mayo (2 horas)}


%% \begin{itemize}
%% \item \textbf{Presentación:} Aplicaciones web con código en el lado del cliente. DHTML, SPA, AJAX
%% \item \textbf{Material:} Transparencias de la asignatura, tema ``AJAX''.
%% \item \textbf{Ejercicio (discusión en clase):} ``SPA Sentences generator'' (ejercicio~\ref{subsec:spa-sentences-generator})
%% \item \textbf{Ejercicio (discusión en clase):} ``AJAX Sentences generator'' (ejercicio~\ref{subsec:ajax-sentences-generator})
%% \end{itemize}

%%% %----------------------------------------------------------------
%\subsubsection{Sesión del ... (2 horas)}
%
%Ejercicios con gadgets y mashups.
%
%\begin{itemize}
% \item \textbf{Presentación:} Aplicaciones web con código en el lado del cliente. Web 2.0, mashups, etc.
% \item \textbf{Material:} Transparencias de la asignatura, tema ``Ajax''.
% \item \textbf{Ejercicio opcional:} Gadget de Google (ejercicio~\ref{subsec:gadget-google}).
%% \item \textbf{Ejercicio opcional:} EyeOS (ejercicio~\ref{subsec:eyeos}).
%\end{itemize}
%
%%% \subsubsection{Sesión del 21 de noviembre}
%
%%% Ejercicios con gadgets y mashups.
%
%%% \begin{itemize}
%%% \item \textbf{Ejercicio opcional:} Gadget de Google en Django cms (ejercicio~\ref{subsec:gadget-google-cms}).
%%% \item \textbf{Ejercicio:} EzWeb (ejercicio~\ref{subsec:ezweb}).
%%% \end{itemize}
%

\newpage

%%----------------------------------------------------------------------------
%%----------------------------------------------------------------------------
%%----------------------------------------------------------------------------
\section{Programa de prácticas}

Programa de las prácticas de la asignatura (tentativo).

%%----------------------------------------------------------------------------
%%----------------------------------------------------------------------------
\subsection{P1 - Introducción a Python}

Introducción al lenguaje de programación Python, que se utilizará para la realización de las prácticas de la asignatura.

%%----------------------------------------------------------------------------
\subsubsection{\juevesA: Python I (2 horas)}
\label{cal:juevesA}

\begin{itemize}
% \item \textbf{Presentación:} ``Introducción a Python'' (introducción, entorno de ejecución, características básicas del lenguaje, ejecución en el intérprete, strings, listas, estructuras condicionales (if else), bucles for).
\item \textbf{Presentación:} ``Introducción a Python'' (introducción, entorno de ejecución, características básicas del lenguaje, tipos, anotación de tipos, métodos).
\item \textbf{Material:} Transparencias ``Introducción a Python''
% \item \textbf{Material:} Ejercicios ``Uso interactivo del intérprete de Python'' (ejercicio~\ref{subsec:practicas-interprete}) y ``Haz un programa en Python'' (ejercicio~\ref{subsec:eje-python-primer-programa}).
% \item \textbf{Ejercicio propuesto (entrega en el foro):} ``Ficheros y listas'' (ejercicio~\ref{subsec:ficheros-listas}).
   % Entrega recomendada: antes del 23 de febrero.
\end{itemize}

%%----------------------------------------------------------------------------
\subsubsection{\juevesB: Lab remoto (0.5 horas)}
\label{cal:juevesB}

\begin{itemize}
\item Mecanismos de conexión remota al laboratorio: VNC desde el navegador, ssh, etc.
\end{itemize}

%%----------------------------------------------------------------------------
\subsubsection{\juevesB: Python II (1.5 horas)}
\label{cal:juevesBb}

\begin{itemize}
\item \textbf{Presentación:} ``Introducción a Python'' (estructuras condicionales (if else), listas, bucles (for))
\item \textbf{Material:} Transparencias ``Introducción a Python''
% \item \textbf{Material:} Ejercicio ``Ficheros y listas'' (ejercicio~\ref{subsec:ficheros-listas}).
% \item \textbf{Ejercicio:} ``Ficheros, diccionarios y excepciones''~\ref{subsec:ficheros-dic-excep}.
% \item \textbf{Ejercicio propuesto (entrega en GitLab):} ``Calculadora'' (ejercicio~\ref{subsec:calculadora}).
   % Entrega recomendada: antes del 2 de marzo.
\end{itemize}

%%----------------------------------------------------------------------------
\subsubsection{\juevesC: Python III (2 horas)}
\label{cal:juevesC}

\begin{itemize}
\item \textbf{Presentación:} ``Introducción a Python'' (ficheros, strings, depurador, diccionarios)
\item \textbf{Material:} Transparencias ``Introducción a Python''

\item \textbf{Ejercicio propuesto (entrega en el foro):} ``Ficheros y listas'' (ejercicio~\ref{subsec:ficheros-listas}).
   Entrega recomendada: antes del \juevesD.

% \item \textbf{Ejercicio propuesto (discusión en clase:} ``Calculadora'' (ejercicio~\ref{subsec:calculadora}).

\end{itemize}



%%----------------------------------------------------------------------------
\subsubsection{\juevesD: Python IV (1 horas)}
\label{cal:juevesD}

\begin{itemize}
\item \textbf{Presentación:} ``Introducción a Python'' (revisión ejercicio ficheros y listas, excepciones, guardar datos intermedios en memoria)
\item \textbf{Material:} Transparencias ``Introducción a Python''

\end{itemize}


%% --------------------------------------------------------------------------
\subsubsection{\juevesD: GitLab (1 horas)}
\label{cal:juevesDa}

\begin{itemize}
\item \textbf{Presentación:} Introducción a la entrega de prácticas en GitLab (seccion~\ref{sec:eje-entrega-practicas-incr}).

\item \textbf{Ejercicio propuesto (discusión en clase:} ``Calculadora'' (ejercicio~\ref{subsec:calculadora}).
   Entrega recomendada: antes del \juevesE.
\end{itemize}


%%----------------------------------------------------------------------------
%%----------------------------------------------------------------------------
\subsection{P2 - Aplicaciones web simples}


Construcción de aplicaciones web mínimas sobre la biblioteca Sockets de Python.


%%----------------------------------------------------------------------------
\subsubsection{\juevesE: Aplicaciones web (2 horas)}
\label{cal:juevesE}

\begin{itemize}
 \item \textbf{Ejercicio:} ``Aplicación web hola mundo'' (ejercicio~\ref{subsec:aplweb-hola-mundo}) \\
   Se muestra la solución del ejercicio, y se comenta en clase. Se pide a los alumnos que lo ejecuten, lo modifiquen y se fijen en las cabeceras HTTP enviadas por el cliente y que el servidor muestra en pantalla (pero no hay entrega específica).
 \item \textbf{Ejercicio:} ``Variaciones de la aplicación web hola mundo'' (ejercicio~\ref{subsec:aplweb-hola-mundo-var}).
\item \textbf{Explicación de ejercicio:} ``Aplicación web generadora de URLs aleatorias'' (ejercicio~\ref{subsec:aplweb-urls-aleatorias})
\item \textbf{Ejercicio propuesto (entrega en GitLab):} ``Aplicación redirectora'' (ejercicio~\ref{subsec:aplweb-redirectora})


   Entrega recomendada: antes del \juevesF.
\end{itemize}


%%%----------------------------------------------------------------------------
\subsection{P3 - Servidores simples de contenidos}

Construcción de algunos servidores de contenidos que permitan comprender la estructura básica de una aplicación web, y de cómo implementarlos aprovechando algunas características de Python.

%%----------------------------------------------------------------------------
\subsubsection{\juevesF: Servidores con clase (2 horas)}
\label{cal:juevesF}

\begin{itemize}
%\item \textbf{Ejercicio:} ``Sumador simple'' (ejercicio~\ref{subsec:sumador-simple}) (en una fase) 
\item \textbf{Comentario de ejercicio:} ``Sumador simple'' (ejercicio~\ref{subsec:sumador-simple}) 
\item \textbf{Trabajo y explicación del ejercicio:} ``Clase servidor de aplicaciones'' (ejercicio~\ref{subsec:clase-serv-aplis}) \\
  Explicación de la estructura general que tienen las aplicaciones web, y fundamentos de cómo esta estructura se puede encapsular en una clase.
\item \textbf{Comentario de ejercicio:} ``Clase servidor de aplicaciones, generador de URLs aleatorias'' (ejercicio~\ref{subsec:aplweb-clase-urls-aleatorias}). 
 \item \textbf{Comentario de ejercicio:}  ``Clase contentApp'' (ejercicio~\ref{subsec:contentapp}) 
 \item \textbf{Ejercicio propuesto (entrega en GitLab):} ``Clase contentPostApp'' (ejercicio~\ref{subsec:contentpostapp}).
  Entrega recomendada: antes del \juevesG.
\end{itemize}


%%----------------------------------------------------------------------------
%\subsubsection{\martesF: Servidores simples I (2 horas)}
%\label{cal:martesF}

%\begin{itemize}

%\item \textbf{Ejercicio:} ``Clase servidor de aplicaciones, sumador'' (ejercicio~\ref{subsec:clase-sumador-simple}). 
%\item \textbf{Ejercicio:}  ``Clase servidor varias aplicaciones'' (ejercicio~\ref{subsec:clase-serv-aplis-multi}) \\
%  Explicación de la estructura principal de una clase que gestiona varias aplicaciones (o varios recursos, cada uno manejado por una aplicación)
% \item \textbf{Ejercicio:} ``Clase servidor, cuatro aplis'' (ejercicio~\ref{subsec:clase-serv-aplis-varias}).
% \item \textbf{Ejercicio:}  ``Clase contentApp'' (ejercicio~\ref{subsec:contentapp}) \\
%   Explicación de la estructura principal de una aplicación que sirve contenidos previamente almacenados.
% \item \textbf{Ejercicio:} ``Instalación y prueba de Poster'' (ejercicio~\ref{subsec:inst-poster}).
% \item \textbf{Ejercicio:} ``Clase contentPostApp'' (ejercicio~\ref{subsec:contentpostapp}).
% \item \textbf{Ejercicio:} ``Clase contentPutApp'' (ejercicio~\ref{subsec:contentputapp}).
%Entrega en GitLab. Fecha de entrega: antes del 3 de marzo.
%\end{itemize}


%%%----------------------------------------------------------------------------
%\subsubsection{\martesG: Servidores simples II (2 horas)}
%\label{cal:martesG}

%\begin{itemize}
%  \item \textbf{Discusión en clase:} ``Clase contentPutApp'' (ejercicio~\ref{subsec:contentputapp}).
%  \item Presentación de la primera práctica de entrega voluntaria (\ref{subsec:practica-vol-1-2016}). Entrega en GitLab. Fecha de entrega: antes del 10 de marzo.
%\end{itemize}


%%%----------------------------------------------------------------------------
%%%----------------------------------------------------------------------------
\subsection{P4 - Introducción a Django}

%%%----------------------------------------------------------------------------
\subsubsection{\juevesG: Django I (2 horas)}
\label{cal:juevesG}

Presentación de Django como sistema de construcción de aplicaciones web.

\begin{itemize}
 \item \textbf{Presentación:} Introducción a Django (primera parte)
 \item \textbf{Ejercicio:} ``Instalación de Django'' (ejercicio~\ref{subsec:django-install}).
 \item \textbf{Ejercicio:} ``Django Intro'' (ejercicio~\ref{subsec:django-intro}).
 \item \textbf{Material:} Transparencias ``Introducción a Django''
 \item \textbf{Ejercicio (discusión en clase:} ``Django Primera Aplicación'' (ejercicio~\ref{subsec:django-primera}).
 \item \textbf{Material:} Guión \url{https://gsyc.urjc.es/grex/cursosweb/guion.html}
 \item \textbf{Ejercicio (discusión en clase):} ``Django Primera Aplicación'' (ejercicio~\ref{subsec:django-primera}).

  \item \textbf{Ejercicio (entrega en GitLab:} ``Django calc'' (ejercicio~\ref{subsec:django-calc}). \\
    Entrega recomendada: antes del 6 de abril.
\end{itemize}


%%%----------------------------------------------------------------------------
\subsubsection{\juevesH: Django II (2 horas)}
\label{cal:juevesH}

Primeros ejercicios con base de datos.

Usuarios, administración y autenticación con Django.

\begin{itemize}
 \item \textbf{Presentación:} Introducción a Django (segunda parte)
  \item \textbf{Material:} Guión \url{https://gsyc.urjc.es/grex/cursosweb/guion2.html}
 \item \textbf{Ejercicio  (discusión en clase):} ``Django cms'' (ejercicio~\ref{subsec:django-cms}). \\
 \item \textbf{Ejercicio (entrega en GitLab):} ``Django cms\_put'' (ejercicio~\ref{subsec:django-cms-put}). \\
  Entrega recomendada: hasta el 13 de abril.
\end{itemize}

%%%----------------------------------------------------------------------------
\subsubsection{\juevesI: Django III (2 horas)}
\label{cal:juevesI}

\begin{itemize}
 \item \textbf{Presentación:} Introducción a Django (tercera parte)
  \item \textbf{Material:} Guión \url{https://gsyc.urjc.es/grex/cursosweb/guion3.html}
 \item \textbf{Ejercicio (discusión en clase):} ``Django cms\_templates'' (ejercicio~\ref{subsec:django-templates}).
 \item \textbf{Ejercicio (discusión en clase):} ``Django cms\_post'' (ejercicio~\ref{subsec:django-post}) \\
  \item Presentación de la \textbf{Práctica 2} (ejercicio~\ref{subsec:practica-vol-2-2015}) \\
  Entrega recomendada: antes del 20 de abril.
\end{itemize}

%%%----------------------------------------------------------------------------
\subsubsection{\juevesJ: Django IV (2 horas)}
\label{cal:juevesJ}

\begin{itemize}
 \item \textbf{Presentación:} Tests con Django y GitLab
  \item \textbf{Material:} Repositorio \url{https://gitlab.etsit.urjc.es/cursosweb/practicas/server/testing-example}
\end{itemize}


%%%----------------------------------------------------------------------------
\subsubsection{\juevesK: Django V (2 horas)}
\label{cal:juevesK}

\begin{itemize}
 \item \textbf{Presentación:} Introducción a Django (quinta parte)
 \item \textbf{Material:} Guión \url{https://gsyc.urjc.es/grex/cursosweb/guion4.html}
 \item \textbf{Material:} Transparencias ``Introducción a Django''
 \item \textbf{Ejercicio (discusión en clase):} ``Django cms\_users'' (ejercicio~\ref{subsec:django-users}).
 \item \textbf{Ejercicio (discusión en clase):} ``Django cms\_users\_put'' (ejercicio~\ref{subsec:django-users-put}). \\
  Entrega recomendada: antes del 27 de abril.
\end{itemize}

%%%----------------------------------------------------------------------------
\subsubsection{\juevesL: Django VI (2 horas)}
\label{cal:juevesL}

\begin{itemize}
%  \item \textbf{Presentación:} Posibilidades de feedparser.py
  %  \item \textbf{Presentación:} Posibilidades de BeautifulSoup.py

 \item \textbf{Material:} Guión \url{https://gsyc.urjc.es/grex/cursosweb/guion5.html}
 \item \textbf{Material:} Transparencias ``Introducción a Django''
 \item \textbf{Ejercicio (discusión en clase):} ``Django cms\_post'' (ejercicio~\ref{subsec:django-post}).
  Entrega recomendada: antes del 4 de mayo.
%    \item \textbf{Material:} Documentación de Django: ``Working with forms'' \\
%          \url{http://docs.djangoproject.com/en/1.4/topics/forms}
%  \item \textbf{Ejercicio voluntario:} ``Django feed\_expander'' (ejercicio~\ref{subsec:django-feed-expander}).
%  \item \textbf{Presentación:} proyecto final (enunciado~\ref{practica-final-2019-05})
\end{itemize}

%%%----------------------------------------------------------------------------
\subsubsection{\juevesM: Django VII (2 horas)}
\label{cal:juevesM}

\begin{itemize}
  \item \textbf{Presentación:} Introducción a Django (formularios)
  \item \textbf{Ejercicio:} ``Django cms\_forms'' (ejercicio~\ref{subsec:django-forms}) \\
\end{itemize}



%%----------------------------------------------------------------------------
 \subsubsection{\juevesN: Detalles finales I  (2 horas)}
 \label{cal:juevesN}

 \begin{itemize}
%   \item \textbf{Presentación:} Introducción a CSS
%   \item \textbf{Ejercicio:} ``Django cms\_css simple'' (ejercicio~\ref{subsec:django-cms-css}) \\
   \item \textbf{Ejercicio:} ``Django cms\_bootstrap cuadrícula'' (ejercicio~\ref{subsec:django-cms-bootstrap-1})
   \item \textbf{Presentación:} Práctica final (apartado~\ref{practica-final-2021-05})
\end{itemize}

%%----------------------------------------------------------------
% \subsubsection{\juevesN: Detalles finales II (2 horas)}
% \label{cal:juevesN}

% \begin{itemize}
%    \item \textbf{Ejercicio:} ``Django cms\_bootstrap componentes'' (ejercicio~\ref{subsec:django-cms-bootstrap-2})
% \item \textbf{Ejercicio:} ``Django cms\_bootstrap componentes personalizados'' (ejercicio~\ref{subsec:django-cms-bootstrap-3})
% \item \textbf{Ejercicio (discusión en clase):} ``Gestor de contenidos con videos de YouTube (despliegue)'' (ejercicio~\ref{subsec:django-cms-youtube-despliegue}).
%  \item \textbf{Ejercicio:} ``Extractor de información de un documento HTML'' (ejercicio~\ref{subsec:html-extractor})
%  \item \textbf{Presentación:} CSS Garden \\
%    \url{https://csszengarden.com}
%  \item \textbf{Presentación:} Awesome Django \\
%    \url{https://github.com/wsvincent/awesome-django#readme}
%  \item \textbf{Presentación:} Otras formas de despliegue de la práctica final.
%  \item \textbf{Preguntas y comentarios:} Práctica final.
%  \end{itemize}



%%%----------------------------------------------------------------------------
% \subsubsection{\martesN: Django (2 horas)}
% \label{cal:martesN}

% \begin{itemize}
%   \item \textbf{Presentación:} Introducción a Django (repaso general)
%   \item \textbf{Ejercicio:} ``Django Conciertos'' (ejercicio~\ref{subsec:django-conciertos}) \\
%   Entrega recomendada: antes del 5 de mayo.

% %  \item \textbf{Presentación:} Introducción a Django (ejercicio final)
% \end{itemize}


%%----------------------------------------------------------------------------
%% \subsubsection{Sesión del XX de abril (2 horas)}

%% \begin{itemize}
%%   \item Tutoría para el proyecto final. Esta clase se aprovechará para presentar o asentar, si fuera necesario, cuestiones relacionadas con el proyecto final.
%% \end{itemize}

%%----------------------------------------------------------------------------
%% \subsubsection{Sesión del XXX de abril (2 horas)}

%% \begin{itemize}
%%   \item Tutoría para el proyecto final. Esta clase se aprovechará para presentar o asentar, si fuera necesario, cuestiones relacionadas con el proyecto final.
%% \end{itemize}

%%% %----------------------------------------------------------------------------
%\subsubsection{Sesión del ...}
%
%Prácticas con almacenamiento en base de datos
%
%\begin{itemize}
% \item \textbf{Ejercicio (entrega en foro):} ``Django feed\_expander\_db'' (ejercicio~\ref{subsec:django-feed-expander-db}).
%\end{itemize}


%
%%% %%----------------------------------------------------------------------------
%%% \subsubsection{Sesión del ... (unos minutos)}

%% \begin{itemize}
%% \item \textbf{Presentación de ejercicio:} Práctica 2 de entrega voluntaria (ejercicio~\ref{subsec:practica-vol-2-2012}) \\
%% Entrega recomendada: antes del 19 de noviembre de 2012.
%% \end{itemize}

%% %%----------------------------------------------------------------------------
%% \subsubsection{Sesión del ... (20 min.)}

%% \begin{itemize}
%% \item \textbf{Presentación:} Enunciado de la práctica 2 de entrega voluntaria (\ref{subsec:practica-vol-2-2011}),
%% \end{itemize}


%%----------------------------------------------------------------------------
%% \subsubsection{Sesión del ... (2 horas)}

%% Recopilación final

%% \begin{itemize}
%% \item \textbf{Explicación de ejercicio:} ``Django feed\_expander'' (ejercicio~\ref{subsec:django-feed-expander}). Incluye implementación de referencia.
%% \item \textbf{Explicación de ejercicio:} ``Django feed\_expander\_db'' (ejercicio~\ref{subsec:django-feed-expander-db}).
%% \item \textbf{Presentación:} Algunos aspectos relevantes de las proyecto final
%% \end{itemize}

%%----------------------------------------------------------------------------
%%----------------------------------------------------------------------------
%% \subsection{P4 - Servidores simples de contenidos}

%% Construcción de algunos servidores de contenidos que permitan comprender la estructura básica de una aplicación web, y de cómo implementarlos aprovechando algunas características de Python.

%% %%----------------------------------------------------------------------------
%% \subsubsection{Sesión del 26 de septiembre (1 hora)}

%% \begin{itemize}
%% \item \textbf{Ejercicio propuesto (entrega en foro):}  ``Clase contentApp'' (ejercicio~\ref{subsec:contentapp}) \\
%%   Explicación de la estructura principal de una aplicación que sirve contenidos previamente almacenados.
%% \end{itemize}

%% %%----------------------------------------------------------------------------
%% \subsubsection{Sesión del 13 de octubre}

%% \begin{itemize}
%% \item \textbf{Ejercicio:} ``Instalación y prueba de Poster'' (ejercicio~\ref{subsec:inst-poster}).
%% \item \textbf{Ejercicio propuesto (entrega en foro):} ``Clase contentPutApp'' (ejercicio~\ref{subsec:contentputapp}).
%% \item \textbf{Ejercicio propuesto (entrega en foro):} ``Clase contentPostApp'' (ejercicio~\ref{subsec:contentpostapp}).
%% \item \textbf{Presentación:} Enunciado de la práctica 1 de entrega voluntaria (\ref{subsec:practica-vol-1-2011}),
%% %% \item \textbf{Ejercicio propuesto (entrega en foro):} ``Clase contentPersistentApp'' (ejercicio~\ref{subsec:contentpersistentapp}).

%% \end{itemize}

%% \subsubsection{Sesión del 13 de octubre}

%% \begin{itemize}
%% \item \textbf{Ejercicio propuesto (entrega en foro):} Clase ``contentStorageApp'' (ejercicio~\ref{subsec:contentstorageapp}).
%% \item \textbf{Ejercicio propuesto (entrega en foro):} ``Gestor de contenidos con usuarios'' (ejercicio~\ref{subsec:contentappusers}). Sólo se plantea su entrega, se explicará en la próxima sesión.
%% \end{itemize}


%% \subsubsection{Sesión del 20 de octubre (1.5 horas)}

%% \begin{itemize}
%% \item \textbf{Explicación de ejercicio:}  ``Gestor de contenidos con usuarios'' (ejercicio~\ref{subsec:contentappusers}).
%% \item \textbf{Ejercicio propuesto (entrega en foro):} ``Gestor de contenidos con usuarios, con control estricto de actualización'' (ejercicio~\ref{subsec:contentappusersstrict}).

%% \item \textbf{Presentación:} Bibliotecas Python potencialmente útlies.
  
%%   Para las próximas sesiones, hasta que introduzcamos Django, quien quiera puede considerar usar los siguientes módulos, de la biblioteca estándar de Python
  
%%   \begin{itemize}
%%   \item BaseHTTPServer. Proporciona las clases básicas para construcción de servidores web. SimpleHTTPServer y CGIHTTPServer heredan de ellas. Probablmente no te interese tocar la clase BaseHTTPServer.HTTPServer, pero sí te venga bien hacer que BaseHTTPServer.BaseHTTPRequestHandler sea la raíz de tu jerarquía de clases, redefiniendo probablmente el método handle, y usando send\_error, send\_response, send\_header, etc.
    
%%   \item SimpleHTTPServer. Proporciona un servidor HTTP que sirve ficheros. Probablmente incluye funcionalidad que no necesitas, pero tiene métodos do\_GET, do\_PUT que te podrían interesar.

%%   \item CGIHTTPServer. Proporciona un servidor HTTP que entiende el protocolo CGI-BIN. Probablmente incluye demasiada funcionalidad que no necesitas.

%%   \item Cookie. Gestión de cookies. Un poquito complicado de usar, pero puede serte muy útil.

%%   \item mimetools. Te servirá para manejar las cabeceras de HTTP. Un poco complejo, pero también te puede ser muy útil.
%%   \end{itemize}
%% \end{itemize}

%% %%----------------------------------------------------------------------------
%% %%----------------------------------------------------------------------------
%% \subsection{P4 - Aplicaciones web con base de datos}

%% Construcción de aplicaciones web con almacenamiento estable en base de datos.


%% \subsubsection{Sesión del 20 de octubre (0.5 horas)}

%% \begin{itemize}
%% \item \textbf{Ejercicio:} ``Introducción a SQLite3 con Python'' (ejercicio~\ref{subsec:sqlite3-python}). 

%% \item \textbf{Ejercicio (entrega en el foro):} ``Gestor de contenidos con base de datos'' (ejercicio~\ref{subsec:gestor-contenidos-bbdd}). 

%% \end{itemize}

%% \subsubsection{Sesión del 27 de octubre (1.5 horas)}

%% \begin{itemize}
%% \item \textbf{Presentación:} ``SQL básico''
%% \item \textbf{Material:} Transparecias ``SQL básico''

%% \textbf{Ejercicio (entrega en el foro):} ``Gestor de contenidos con usuarios, control estricto de actualización y base de datos'' (ejercicio~\ref{subsec:gestor-contenidos-usuarios-bbdd}).

%% \end{itemize}

%% \subsubsection{Sesión del 1 de diciembre}

%% Repaso de prácticas.

%% \begin{itemize}
%% \item \textbf{Ejercicio:} Trabajo con el proyecto final.
%% \end{itemize}


%% \subsubsection{Sesión del XXX de diciembre}

%% Algunos aspectos más de Django. Modelos: relación entre tablas. Internacionalización. Generación de canales RSS y Atom.

%% \begin{itemize}
%% \item \textbf{Presentación:} Introducción a Django (sexta parte).
%% \item \textbf{Material:} Transparencias ``Introducción a Django''.
%% \item \textbf{Ejercicio:} Trabajo con el proyecto final.
%% \end{itemize}

%% \subsubsection{Sesión del XXX de diciembre}

%% Repaso de prácticas.

%% \begin{itemize}
%% \item \textbf{Presentación:} Algunos aspectos e ideas sobre el proyecto final.
%% \item \textbf{Ejercicio:} Trabajo con el proyecto final.
%% \end{itemize}


\newpage

%%----------------------------------------------------------------------------
%%----------------------------------------------------------------------------
\section{Proyecto final: LoVisto (2021, mayo)}
\label{practica-final-2021-05}

[ \textbf{Nota importante:} Por ahora esto es sólo es un borrador. Aún estamos definiendo cómo será el enunciado definitivo. ]
%[ \textbf{Nota importante:} Este enunciado es aún tentativo, y puede sufrir cambios ]

La práctica final de la asignatura consiste en la creación de una aplicación web, llamada ``LoVisto'', que permitirá gestionar aportaciones de los usuarios, que serán enlaces (URLs) a vídeos, noticias y otra información que los usuarios vayan viendo por la red y les resulte interesante. Cuando las aportaciones correspondan con ciertos patrones de recurso, de ciertos sitios (patrones reconocidos), se mostrará cierta información ofrecida por esos sitios, a modo de una ``previsualización''. Los usuarios podrán añadir aportaciones con enlaces a páginas que hayan visto, comentarlos, puntuarlos, compartirlos, etc. A continuación se describe el funcionamiento y la arquitectura general de la aplicación, la funcionalidad mínima que debe proporcionar, y otra funcionalidad optativa que podrá tener.

%%----------------------------------------------------------------------------
\subsection{Arquitectura y funcionamiento general}

Arquitectura general:

\begin{itemize}

\item La práctica se construirá como un proyecto Django/Python3, que incluirá una o varias aplicaciones (\emph{apps}) Django que implementen la funcionalidad requerida.

\item Para el almacenamiento de datos persistente se usará SQLite3, con tablas definidas en modelos de Django.

\item Para implementar usuarios, cuando sea necesario, se usará como base el sistema de autenticación de usuarios que proporciona Django\footnote{User Authentication in Django:\\ \url{https://docs.djangoproject.com/en/3.0/topics/auth/}}.

\item Todas las bases de datos que contenga la aplicación tendrán que ser accesibles vía la interfaz que proporciona el ``Admin Site'' (además de lo que pueda hacer falta para que funcione al aplicación).

\item Se utilizarán plantillas Django (a ser posible, una jerarquía de plantillas, para que la práctica tenga un aspecto similar) para definir las páginas que se servirán a los navegadores de los usuarios. Estas plantillas incluirán en todas las páginas al menos:

  \begin{itemize}
  \item Un \emph{banner} (imagen) del sitio, preferentemente en la parte superior.
  \item Un formulario para entrar (hacer login en el sitio), o para salir (si ya se ha entrado).
    \begin{itemize}
    \item En caso de que no se haya entrado en una cuenta, este formulario permitirá al visitante introducir su identificador de usuario y su contraseña.
    \item En caso de que ya se haya entrado, este formulario mostrará el identificador del usuario y permitirá salir de la cuenta (logout). Este formulario aparecerá preferentemente en la parte superior derecha.
    \end{itemize}
  \item Un menú de opciones, como barra, preferentemente debajo de los dos elementos anteriores (banner y caja de entrada o salida). Los contenidos de este menú se indican más adelante en el enunciado.
  \item Un pie de página con una nota de atribución, indicando ``Esta aplicación enlaza a XXX, YYY, ZZZ y otros muchos sitios'', siendo XXX, YYY y ZZZ los tres últimos sitios que los usuarios han referenciado en sus aportaciones, y siendo cada uno de ellos un enlace al sitio en cuestión.
  \end{itemize}

  Cada una de estas partes estará construida dentro de un elemento \texttt{div}, marcada con un atributo \texttt{id} en HTML, para poder ser referenciadas fácilmente en hojas de estilo CSS. Cuando sea conveniente, se podrán utilizar en lugar de \texttt{div} elementos de HTML5 (\texttt{header}, \texttt{footer}, \texttt{nav}, etc).

\item Se utilizarán hojas de estilo CSS para determinar la apariencia de la práctica. Estas hojas definirán al menos el color y el tamaño de la letra, y el color de fondo de cada una de las partes (elementos) marcadas con un \emph{id}, tal como se indica en el apartado anterior. Además, elementos que deban tener el mismo aspecto deberían estar en una misma clase, para poder gestionarlo de forma común.

\item Se utilizará Bootstrap para la maquetación (\emph{layout}) de las páginas, de forma que funcionen adecuadamente tanto en navegadores de escritorio como en móviles.
  
\item Para obtener información correspondiente a un recurso reconocido, se utilizará la API del sitio al que corresponda, o quizás en algunos casos, ser hará un análisis de las páginas HTML del sitio. En general, la forma de funcionamiento será la siguiente:

  \begin{itemize}
  \item El usuario que quiera realizar una aportación, rellenará un formulario con la URL de la aportación que realiza, y un título para ella.
  \item La aplicación recibirá esta URL, y la analizará, obteniendo a partir de ella el nombre del sitio, y el nombre del recurso.
  \item Si el recurso es uno de los reconocidos, se obtendrá del sitio correspondiente (vía API, o de otras formas) la información extendida ese recurso. Ver detalles en ``Recursos reconocidos'' (apartado~\ref{sec:practica-final-2021-05:reconocidos}), más adelante.
  \item Si no es uno de los recursos reconocidos, se extraerá información directamente de la página HTML del recurso. Ver detalles en ``Recursos no reconocidos'' (apartado~\ref{sec:practica-final-2021-05:noreconocidos}).
  \item En ambos casos, esa información se almacenará en la base de datos (normalmente como string HTML), y se mostrará junto a la URL cuando se especifique que se muestra la ``información extendida'' para esa URL.
  \end{itemize}

\end{itemize}

Funcionamiento general:

\begin{itemize}
\item En general, para utilizar el sitio, no hará falta autenticarse con una cuenta. Si no se está autenticado, se podrá ver toda la información, salvo la página de usuario, y no se podrán realizar aportaciones, votar ni poner comentarios.

\item Cuando un visitante quiera, podrá autenticarse en una cuenta ya existente. En este caso, la funcionalidad quedará ligada a su cuenta. En ese momento podrá ya realizar aportaciones, votar, o poner comentarios. También podrá acceder a la página de usuario, que mostrará la inforamción de su usuario.
  
\end{itemize}


%%----------------------------------------------------------------------------
\subsection{Recursos reconocidos}
\label{sec:practica-final-2021-05:reconocidos}

La aplicación reconocerá ciertos patrones recurso, que se describen a continuación. Si la URL que se incluye con una aportación es uno de estos recursos reconocidos, la aplicación tendrá que recoger información adicional, y almacenarla en la base de datos.

La práctica tendrá que funcionar con al menos tres patrones de recursos reconocidos, entre los siguientes, uno de los cuales deberá ser la predicción AEMET para un municipio.

%%----------------------------------------------------------------------------
\subsubsection{Predicción AEMET para un municipio}

  \begin{itemize}
  \item Ejemplo: \\
    \url{http://www.aemet.es/es/eltiempo/prediccion/municipios/getafe-id28065}
  \item Patrones reconocidos:
    \begin{itemize}
    \item Sitio: \texttt{www.aemet.es}, \texttt{aemet.es}
    \item Patrón de recurso: \\
      \texttt{/es/eltiempo/prediccion/municipios/\{municipio\}-id\{num\}}
    \end{itemize}
  \item Información extendida:
    \begin{itemize}
    \item Ejemplo: \\
      \url{https://www.aemet.es/xml/municipios/localidad_28065.xml}
    \item Patrón de url: \\
      \texttt{https://www.aemet.es/xml/municipios/localidad\_\{num\}.xml}
    \item Formato: XML
    \item Datos mínimos:
      \begin{itemize}
      \item Municipio, provincia
      \item Para todos los días que tengan predicción, temperatura máxima y mínima, sensación térmica máxima y mínima, humedad relativa máxima y mínima.
      \item Nota de copyright
      \end{itemize}
    \item Ejemplo:

{\footnotesize
\begin{verbatim}
<div class="aemet">
  <p>Datos AEMET para Getafe (Madrid)</p>
  <ul>
  <li>2021-05-08. Temperatura: 11/22, sensación: 16/25, humedad: 60/75.</li>
  <li>2021-05-09. Temperatura: 10/18, sensación: 19/21, humedad: 50/66.</li>
  ...
  <li>2021-05-14. Temperatura: 9/12, sensación: 10/15, humedad: 90/95.</li>
  </ul>
  <p>Copyright AEMET.
    <a href="http://www.aemet.es/es/.../getafe-id28065">Página
      original en AEMET</p>
</div>
\end{verbatim}
}
  \end{itemize}
    \end{itemize}


%%----------------------------------------------------------------------------
\subsubsection{Página Wikipedia en español}

  \begin{itemize}
  \item Ejemplo: \\
    \url{https://es.wikipedia.org/wiki/Astronauta}
  \item Patrones reconocidos:
    \begin{itemize}
    \item Sitio: \texttt{es.wikipedia.org}
    \item Patrón de recurso: \\
      \texttt{/wiki/\{articulo\}}
    \end{itemize}
  \item Información extendida:
    \begin{itemize}
    \item Ejemplo texto: \\
      \url{https://es.wikipedia.org/w/api.php?action=query&format=xml&titles=Astronauta&prop=extracts&exintro&explaintext}
    \item Ejemplo imagen: \\
      \url{https://es.wikipedia.org/w/api.php?action=query&titles=Astronauta&prop=pageimages&format=json&pithumbsize=100}
    \item Patrón de URL (texto):\\
      \texttt{https://es.wikipedia.org/w/api.php?}\\
      \texttt{action=query\&format=xml\&titles=\{articulo\}}\\
      \texttt{\&prop=extracts\&exintro\&explaintext}
    \item Patrón de URL (imagen):\\
    \texttt{https://es.wikipedia.org/w/api.php?}\\
    \texttt{action=query\&titles=\{articulo\}}\\
    \texttt{\&prop=pageimages\&format=json\&pithumbsize=100}
    \item Formato: XML (text) y JSON (image)
    \item Datos mínimos:
      \begin{itemize}
      \item Primeros 400 caracteres del texto.
      \item Imagen.
      \item Nota de copyright
      \end{itemize}
    \item Ejemplo:

{\footnotesize
\begin{verbatim}
<div class="wikipedia">
  <p>Artículo Wikipedia: Astronauta</p>
  <img src="url de la imagen principal">
  <p>[Texto del principio de la página]</p>
  <p>Copyright Wikipedia.
    <a href="https://es.wikipedia.org/wiki/Astronauta">Artículo
      original</a>.</p>
</div>
\end{verbatim}
}
    \end{itemize}
  \end{itemize}


%%----------------------------------------------------------------------------
\subsubsection{Vídeo de YouTube}

  \begin{itemize}
  \item Ejemplo: \\
    \url{https://www.youtube.com/watch?v=IfoSqaxJsAM}
  \item Patrones reconocidos:
    \begin{itemize}
    \item Sitio: \texttt{www.youtube.com}, \texttt{youtube.com}
    \item Patrón de recurso: \\
      \texttt{/watch?v=\{video\}}
    \end{itemize}
  \item Información extendida:
    \begin{itemize}
    \item Ejemplo: \\
      \url{https://www.youtube.com/oembed?format=json&url=https://www.youtube.com/watch?v=IfoSqaxJsAM}
    \item Patrón de URL:\\
      \texttt{https://www.youtube.com/oembed}\\
      \texttt{?format=json\&url=https://www.youtube.com/watch?v=\{video\}}
    \item Formato: JSON
    \item Datos mínimos:
      \begin{itemize}
      \item Título.
      \item Autor.
      \item Vídeo embebido en iframe.
      \end{itemize}
    \item Ejemplo:

{\footnotesize
\begin{verbatim}
<div class="youtube">
  <p>Video YouTube: Presentación de la asignatura</p>
  <iframe width="560" height="315"
    src="https://www.youtube.com/embed/IfoSqaxJsAM"
    title="YouTube video player" frameborder="0"
    allow="accelerometer; encrypted-media; gyroscope; picture-in-picture"
    allowfullscreen>
  </iframe>
  <p>Autor: CursosWeb.
    <a href="https://www.youtube.com/watch?v=IfoSqaxJsAM">Video
      en YouTube</a></p>
</div>
\end{verbatim}
}
    \end{itemize}
  \end{itemize}


%%----------------------------------------------------------------------------
\subsubsection{Nota en Reddit}

\begin{itemize}
  \item Ejemplo: \\
    \url{https://www.reddit.com/r/django/comments/n842st/is_there_a_public_repo_that_shows_productionlevel/}
  \item Patrones reconocidos:
    \begin{itemize}
    \item Sitio: \texttt{www.reddit.com}, \texttt{reddit.com}
    \item Patrón de recurso: \\
      \texttt{/r/\{subreddit\}/comments/\{id\}/\{titulo\}/}
    \end{itemize}
  \item Información extendida:
    \begin{itemize}
    \item Ejemplo: \\
      \url{https://www.reddit.com/r/django/comments/n842st/.json}
    \item Patrón de URL:\\
      \texttt{https://www.reddit.com/r/django/comments/\{id\}/.json}\\
    \item Formato: JSON
    \item Datos mínimos (obtenidos de [0][data][children][0][data]):
      \begin{itemize}
      \item Subreddit
      \item Título (title)
      \item Texto (selftext)
      \item Aprobación (upvote\_ratio)
      \item URL (url)
      \end{itemize}
    \item Comentarios. Si la URL es del sitio \texttt{https://i.redd.it/}, normalmente se refiere a una imagen, y por lo tanto habrá que presentarla como tal en la información extendida. Si no aparecen los campos anteriores, puede considerarse que la información extendida sólo tiene los campos que se hayan encontrado. Si se mejora el código para que se reconozcan otros casos, se considerará una mejora a la práctica: documentar esos cambios en el documento de descripción de la práctica.
    \item Ejemplos:

{\footnotesize
\begin{verbatim}
<div class="reddit">
  <p>Nota Reddit: Is there a public repo that shows...</p>
  <p>[texto]</p>
  <p><a href="https://www.reddit.com/r/django/comments/n842st">Publicado
    en [subreddit]</a>. Aprobación: [aprobacion].</p>
</div>
\end{verbatim}
}

{\footnotesize
\begin{verbatim}
<div class="reddit">
  <p>Nota Reddit: Distribution of the surname Ryan...</p>
  <img src="[imagen_url]">
  <p><a href="https://www.reddit.com/r/dataisbeautiful/comments/n8azu6/">Publicado
    en [subreddit]</a>. Aprobación: [aprobacion].</p>
</div>
\end{verbatim}
}

    \end{itemize}
  \end{itemize}

%%----------------------------------------------------------------------------
\subsubsection{Otros recursos reconocidos}


Además de las anteriores, puedes proponer otros tipos de recursos reconocidos para tu aplicación. La información extendida de estos recursos reconocidos ha de ser accesibles públicamente (el acceso mediante un token de aplicación se considera público), y proporcionar datos en formato XML o JSON. Si hay algún tipo de recurso reconocido querrías utilizar, coméntalo con los profesores para que te indiquen si es válido. En caso de ser aceptado como válido, estos tipos de recursos reconocidos serán puntuados positivamente, teniendo en cuenta la iniciativa del alumno que los propuso. En este caso, documéntalos en el documento de descripción de tu práctica, de forma similar a como se han documentado los anteriores (ejemplo de URL, parones reconocidos, información extendida, etc.). 

Si quieres buscar servicios que ofrezcan APIs que podrían ser recursos reconocidos, puedes buscarlos en Internet. Una lista por la que puedes comenzar es la que mantiene ProgrammableWeb\footnote{Programmable Web API Directory: \\\url{https://www.programmableweb.com/apis/directory}}.

%%----------------------------------------------------------------------------
\subsection{Recursos no reconocidos}
\label{sec:practica-final-2021-05:noreconocidos}

En el caso de que el recurso que indique el usuario no sea uno de los recursos reconocidos, habrá que extraer información extendida del propio documento HTML, si esto es posible. Para ello se utilizarán dos estrategias (primero una, y si no funciona, la otra):

\begin{itemize}
\item Si hay disponibles propiedades Open Graph\footnote{https://ogp.me/}, se extraerán las propiedades \texttt{og:title} y \texttt{image} (o la que exista de ellas), y se usarán para componer una información extendida que incluya el titulo y la imagen.

  Las propiedades Open Graph normalmente se encuentran como metadatos en la cabecera (\texttt{head}) del documento HTML. Por ejemplo:

\begin{verbatim}
<html>
  <head>
  <title>Este es el titulo</title>
  <meta property="og:title" content="Este es el titulo" />
  <meta property="og:image" content="https://..../imagen.jpg" />
  ...
</head>
...
\end{verbatim}

  Ejemplo de información extendida generada de esta forma:

\begin{verbatim}
<div class="og">
  <p>Este es el titulo</p>
  <img src="https://..../imagen.jpg">
</div>
\end{verbatim}

\item Si hay disponible un elemento \texttt{title}, se usará su contenido para componer la información extendida. Por ejemplo:

\begin{verbatim}
<div class="html">
  <p>Este es el titulo</p>
</div>
\end{verbatim}
  
\end{itemize}

En ambos casos, se usará el módulo \texttt{html.parser}\footnote{\texttt{html.parser}: \url{https://docs.python.org/3/library/html.parser.html}}\footnote{Ejemplo de uso de \texttt{html.parser}: \url{https://stackoverflow.com/a/36650753/2075265}} de Python3, o algún otro módulo que ayude en la identificación de elementos HTML, como BeautifulSoup4\footnote{\url{https://pypi.org/project/beautifulsoup4/}}.

Si no funciona ninguna de esas estrategias, o la URL no corresponde con una página HTML (pero se puede descargar), se incluirá una información extendida que indique no hay información extendida:

\begin{verbatim}
<div class="html">
  <p>Información extendida no disponible</p>
</div>
\end{verbatim}


%%----------------------------------------------------------------------------
\subsection{Funcionalidad mínima}

La aplicación servirá las siguientes páginas:

\begin{itemize}
  \item Página principal de la aplicación:
  
    \begin{enumerate}
    \item Listado con los 10 últimas aportaciones del el sitio. Para cada una se mostrará su información en formato completo (ver más abajo).
    \end{enumerate}

    Si el visitante está autenticado como usuario, se mostrará también:

    \begin{itemize}
    \item Para cada aportación que aparezca en la página se mostrarán también dos botones para votar (positivo, negativo), resaltando de alguna forma que el valor que se haya votado, si se hubiera votado ya ese item.
    \item Formulario para realizar una aportación. Tendrá campos para un título, un texto de descripción, y un enlace (que será el enlace a la aportación en el sitio donde se ha visto). Al enviar el formulario, se creará una nueva aportación a nombre de usuario que la ha enviado
    \item Listado con las últimos 5 aportaciones (formato resumido) realizadas por el usuario.
    \item Enlace a la página del usuario.
    \end{itemize}

  \item Página de la aportación (para cada aportación):

    \begin{itemize}
    \item Datos de la aportación (formato completo).
    \item Comentarios que haya recibido la aportación. Para cada comentario se mostrará el texto del comentario, el identificador de quien lo puso, y la fecha en que se puso.
    \end{itemize}

    Si el visitante está además autenticado como usuario, se mostrará también:

    \begin{itemize}
    \item Dos botones para votar (positivo, negativo), resaltando de alguna forma que el valor que se haya votado, si se hubiera votado ya esa aportación.
    \item Formulario para poner un comentario. Tras poner el comentario, se volverá a ver la misma página de la aportación.
    \end{itemize}

  \item Página del usuario (siempre el mismo nombre de recurso, sólo para usuarios autenticados):

    \begin{itemize}
    \item Listado de todas las aportaciones del usuario. Cada aportación, en formato resumido.
    \item Listado de todos los comentarios del usuario. Cada comentario, junto con la aportación a la que hizo ese comentario, en formato resumido.
    \item Listado de todas las aportaciones votadas por el usuario. Cada voto, junto con la aportación a la que votó, en formato resumido, y una indicación de si el voto fue positivo o negativo.
    \end{itemize}
    
  \item Página de todas las aportaciones:

    \begin{itemize}
    \item Listado de todas las aportaciones realizadas al sitio, en formato resumido, ordenadas por fecha de aportación (primero las más recientes).
    \end{itemize}

  La página de todas las aportaciones se ofrecerá también como un documento XML y como un documento JSON, que incluiría la misma información (el mismo listado de todas las aportaciones, cada una con su título y su enlace a la página de la aportación). Este documento se ofrecerá cuando se pida la página principal, concatenando al final \verb|?format=xml| o \verb|?format=json|.
 
  \item Página de información: Página con información en HTML indicando la autoría de la práctica, explicando su funcionamiento y una brevísima documentación.

\end{itemize}

Información para una aportación (formato resumido):

\begin{itemize}
\item La información para una aportación estará compuesta por:
  \begin{itemize}
  \item Título
  \item Enlace a la página de la aportación    
  \end{itemize}

\item Por ejemplo, esta información se podrá presentar como:

{\footnotesize
\begin{verbatim}
<p class="resumida"><a href="[enlace_pagina]">[titulo]</a></p>
\end{verbatim}
}

\end{itemize}

Información para una aportación (formato completo):

\begin{itemize}
\item La información completa para una aportación estará compuesta por:
  \begin{itemize}
  \item Título
  \item Enlace a la página de la aportación
  \item Descripción
  \item Usuario
  \item Votos positivos
  \item Votos negativos
  \item Fecha de la aportación
  \item Número de comentarios
  \item Información extendida
  \end{itemize}

\item Por ejemplo, esta información se podrá presentar como:

{\footnotesize
\begin{verbatim}
<div class="aportacion">
  <h2>[titulo]</h2>
  <p class="descripcion">[descripcion]</p>
  <p class="datos">Contribución de [usuario], enviada en [fecha],
    [num_comentarios] comentarios,
    <a href="[enlace_pagina]">más info</p>
  <p class="votos">Positivos: [positivos]. Negativos: [negativos]</p>
  [info_extendida]
</div>
\end{verbatim}
}

\end{itemize}

La aplicación se encargará de controlar que no haya más de un voto (positivo o negativo) por usuario para cada aportación. Por lo tanto, si un usuario ya ha votado una aportación, y vuelve a votarlo, se ignorará su voto (si es igual que el que está almacenado) o se anotará el nuevo (si es distinto). Por ejemplo, si había votado una aportación con positivo, y ahora vuelve a votarla con positivo, se ignorará el segundo voto. Si vuelve a votarla, pero ahora con negativo, se cambiará el voto a negativo.

En todos los casos en que se vote, tras votar se volverá a ver la misma página en que se estaba, ahora con el voto contabilizado.

En todas las páginas habrá un menú desde el que se podrá acceder a la página principal (con el texto ``Inicio''), a la de información (con el texto ``Información''), a la de todas las aportaciones en formato ``normal'' (HTML) (con el rexto ``Todo''), y descargar el listado de todas las aportaciones en formato XML (``Descarga como fichero XML'') y JSON (``Descarga como fichero JSON'') (que apuntará a la página de todas las aportaciones en el formato correspondiente). Estas opciones de menú estarán en cada página, salvo si la opción apunta a la página en la que se está, en cuyo caso no saldrá la opción correspondiente.

Desde este menú también se podrá cambiar, si se está autenticado como usuario, a un ``modo oscuro'' (se se está en el modo normal), o volver al modo normal (si se está en el modo oscuro). Este cambio se realizara cambiando el CSS que servirá para todas las páginas del sitio, mientras el visitante no vuelva a cambiar de modo.

Además, la práctica incluirá tests, que se ejecutarán con \verb|python3 manage.py test|, y que incluirán al menos un test de API HTTP para cada recurso que sirva la aplicación, y para cada método (GET, POST) que admita cada recurso. Además, al menos la mitad de los test incluirán comprobar algo distinto del código HTTP retornado por la petición.

%%----------------------------------------------------------------------------
\subsection{Despliegue}
\label{sec:practica-2021-05:despliegue}

La práctica deberá estar desplegada en algún sitio de Internet, de forma que pueda accederse a ella. Deberá mantenerse desplegada y activa al menos desde el día de entrega de la práctica, hasta el día del cierre de actas.

Para el despliegue, se puede utilizar Python Anywhere\footnote{Python Anywhere: \url{https://pythonanywhere.com}}, que proporciona un plan gratuito que incluye suficientes recursos como para poder desplegar la práctica.

Si el alumno así lo desea, puede considerarse desplegar en un ordenador dedicado (por ejemplo, una Raspberry Pi accesible directamente desde Internet, alojada en su hogar), o en servicios como Google Computing Engine\footnote{GCP Engine Free: \url{https://cloud.google.com/free/}}. En general, dado que este tipo de despliegues no podrá contar con una ayuda detallada por los profesores, estará algo más valorado.

En el caso de que la práctica se despliegue en Python Anywhere, hay que tener en cuenta que sus máquinas virtuales tienen cortado el acceso a todos los sitios de Internet salvo los que están en una ``lista blanca''\footnote{Lista blanca de Python Anywhere: \url{https://www.pythonanywhere.com/whitelist/}} (\emph{whitelist}). Esto afectará a vuestro despliegue de dos formas:

\begin{itemize}
\item Al clonar vuestro repositorio git dentro de la máquina virtual, para tener el código de vuestra aplicación. No debería dar problemas, porque el sitio GitLab de la ETSIT, donde está vuestro código fuente, está ya en la lista blanca.
\item Cuando vuestra aplicación se conecte para obtener la información extendida de un recurso, si el sitio al que la aplicación se tiene que conectar para conseguir el documento JSON o XML no está en la lista blanca, vuestra aplicación no se podrá conectar. YouTube y algunos otros sitios para los que se pueden implementar recursos reconocidos (como Flickr) están ya en la lista blanca. Pero otros sitios que podéis estar usando, no.
\end{itemize}

Para evitar los problemas con los sitios que no estén en la lista blanca, os pedimos que si hacéis el despliegue en Python Anywhere:

\begin{itemize}
\item Tiene que funcionar al menos con recursos reconocidos de YouTube, recogiendo los documentos XML correspondientes, como indica el enunciado.
\item Para los demás recursos reconocidos que hayáis implementado, tenéis dos opciones:
  \begin{itemize}
  \item Si están en la lista blanca de Python Anywhere, funcionarán sin problemas sin hacer nada especial, si os funcionaban ya en las pruebas locales, así que también deberían funcionar en el despliegue.
  \item Si no están en la lista blanca de Python Anywhere, aseguraos de que la base de datos que subís al despliegue de vuestra práctica incluya datos de recursos reconocidos de la plataforma en cuestión. Por ejemplo, si tenéis implementados recursos reconocidos de Last.fm, basta con que tengáis en la base de datos algunas aportaciones de un par de recursos de este sitio, que muestren que en la versión local os funcionó.
  \end{itemize}
\end{itemize}

Tened en cuenta que si usáis otras plataformas para el despliegue, puede que os encontréis problemas similares. Y tened en cuenta también que en cualquier caso, nosotros probaremos la práctica en otros despliegues, así que todos los recursos reconocidos que hayáis implementado deben funcionar correctamente si no hay restricciones de conexión.

Tenéis más detalles sobre cómo se hace un despliegue de una aplicación Django en Python Anywayre en el video ``Django: Despliegue en Python Anywhere''\footnote{\url{https://www.youtube.com/watch?v=hlZPC5L2Itc}}, que explica cómo desplegar allí la aplicación \texttt{django-youtube-4}\footnote{\url{https://github.com/CursosWeb/Code/tree/master/Python-Django/django-youtube-4}} (ver también el fichero \texttt{README.md} de esa aplicación para más detalles).

%%----------------------------------------------------------------------------
\subsection{Funcionalidad optativa}

De forma optativa, se podrá incluir cualquier funcionalidad relevante en el contexto de la asignatura. Se valorarán especialmente las funcionalidades que impliquen el uso de técnicas nuevas, o de aspectos de Django no utilizados en los ejercicios previos, y que tengan sentido en el contexto de esta práctica y de la asignatura.

En el formulario de entrega se pide que se justifique por qué se considera funcionalidad optativa lo que habéis implementado. Sólo a modo de sugerencia, se incluyen algunas posibles funcionalidades optativas:

\begin{itemize}
  \item Inclusión de un \emph{favicon} del sitio
  
  \item Visualización de cualquier página en formato JSON y/o XML, de forma similar a como se ha indicado para la página principal.

  \item Generación de un documento XML y/o JSON para los comentarios puestos en el sitio.

  \item Incorporación de datos de otros tipos de recurso además de los obligatorios. Se valorará especialmente la búsqueda de otros tipos de recurso no descritos en este enunciado, la implementación de tipos de recurso que requieran token de autenticación (en este caso, atención a no subir el token de autenticación a GitLab).
 
  \item Atención al idioma indicado por el navegador. El idioma de la interfaz de usuario de la aplicación tendrá en cuenta lo que especifique el navegador.

  \item Inclusión de imágenes (no solo texto) en los comentarios. Esto puede hacerse de dos formas: quien suba un comentario, además de rellenar una caja de texto con el comentario, puede indicar también la URL de una imagen, que se mostrará junto al comentario, o bien subiendo una imagen a la aplicación, que se mostrará junto al comentario (se valorará más la segunda opción, y se pueden implementar las dos).
    
  \item Mejora de los tests de la práctica, incluyendo test de condiciones de error, test de escenarios con más de una invocación de recurso, tests de API Python, etc.
\end{itemize}

%%----------------------------------------------------------------------------
\subsection{Entrega de la práctica}

\begin{itemize}
  \item \textbf{Fecha límite de entrega (convocatoria ordinaria):} domingo, 6 de junio de 2021 a las 23:55 (hora española peninsular)
       %{\bf Convocatoria de junio:} miércoles, 24 de junio de 2015 a las 23:59 (hora peninsular española).

  \item \textbf{Notificación de alumnos que tendrán que realizar entrevista:} martes, 8 de junio, en el aula virtual.
%{\bf Convocatoria de junio:} viernes, 26 de junio, en la plataforma Moodle.

  \item \textbf{Realización de entrevistas:} miércoles, 9 de junio, en la aplicación Teams. Si es necesario, se realizarán también el día 10.
    
  \item \textbf{Fecha de publicación de notas:} jueves, 10 de junio, en el aula virtual.
%{\bf Convocatoria de junio:} viernes, 26 de junio, en la plataforma Moodle.

  \item \textbf{Fecha de revisión:} viernes, 11 de junio, a las 9:00, en la aplicación Teams.
%{\bf Convocatoria de junio:} martes, 30 de junio a las 13:30. Se requerirá a algunos alumnos que asistan a la revisión {\bf en persona}; se informará de ello en el mensaje de publicación de notas.
\end{itemize}

La entrega de la práctica consiste en:

\begin{enumerate}

  \item {\bf Rellenar un formulario} enlazado en el sitio de la asignatura en el aula virtual.
  
  \item {\bf Subir tu práctica a un repositorio en el GitLab de la Escuela}. El repositorio contendrá todos los ficheros necesarios para que funcione la aplicación (ver detalle más abajo). Es muy importante que el alumno haya realizado una derivación (fork) del repositorio que se indica a continuación, porque si no, la práctica no podrá ser identificada: 

\url{https://gitlab.etsit.urjc.es/cursosweb/practicas/server/final-lovisto/}

Recordad que es importante ir haciendo commits de vez en cuando y que sólo al hacer push estos commits son públicos. Antes de entregar la práctica, haced un push. Y cuando la entreguéis y sepáis el nombre del repositorio, podéis cambiar el nombre del repositorio desde el interfaz web de GitLab. 

Se recomienda mantener el repositorio como privado, hasta el momento en que se entregue la práctica.

 \item {\bf Entregar un vídeo de demostración de la parte obligatoria, y otro vídeo de demostración de la parte opcional}, si se han realizado opciones avanzadas. Los vídeos serán de una {\bf duración máxima de 3 minutos} (cada uno), y consistirán en una captura de pantalla de un navegador web utilizando la aplicación, y mostrando lo mejor posible la funcionalidad correspondiente (básica u opcional). Siempre que sea posible, el alumno comentará en el audio del vídeo lo que vaya ocurriendo en la captura. Los vídeos se colocarán en algún servicio de subida de vídeos en Internet (por ejemplo, Vimeo, Twitch, o YouTube). Los vídeos de más de tres minutos tendrán penalización.

Hay muchas herramientas que permiten realizar la captura de pantalla. Por ejemplo, en GNU/Linux puede usarse Gtk-RecordMyDesktop o Istanbul (ambas disponibles en Ubuntu). OBS Studio\footnote{OBS Studio: \url{https://obsproject.com/}} está disponible para varias plataformas (Linux, Windows, MacOS). Es importante que la captura sea realizada de forma que se distinga razonablemente lo que se grabe en el vídeo.

En caso de que convenga editar el vídeo resultante (por ejemplo, para eliminar tiempos de espera) puede usarse un editor de vídeo, pero siempre deberá ser indicado que se ha hecho tal cosa con un comentario en el audio, o un texto en el vídeo. Hay muchas herramientas que permiten realizar esta edición. Por ejemplo, en GNU/Linux puede usarse OpenShot o PiTiVi.

\end{enumerate}

Sobre la entrega del repositorio:
\begin{itemize}
  \item Se han de entregar los siguientes ficheros:

\begin{itemize}
  \item El repositorio en la instancia GitLab de la ETSIT deberá contener un proyecto Django completo y listo para funcionar en el entorno del laboratorio, incluyendo la base de datos. Deberá poder ejecutarse directamente con \verb|python3 manage.py runserver| desde un entorno virtual en el que esté instalado Django~3.0.3. También ejecutará los tests con \verb|python3 manage.py test|, desde el mismo entorno virtual.

  \item La base de datos habrá de tener datos suficientes como para poder probarlo. Estos datos incluirán al menos dos usuarios, con al menos diez aportaciones en total, cinco comentarios puestos en total, y al menos seis aportaciones votadas por cada usuario.

  \item Un fichero \verb|requirements.txt|, con un nombre de paquete Python por línea, para indicar Cualquier biblioteca Python que pueda hacer falta para que la aplicación funcione, si es que fuera el caso. Este fichero no ha de incluir Django, dado que ya se supone que hace falta. Si es posible, se recomienda escribir este fichero en el formato que entiende \verb|pip install -r requirements.txt|

  \item Cualquier fichero auxiliar que pueda hacer falta para que funcione la práctica, si es que fuera el caso.
\end{itemize}

  \item Se incluirán en el fichero README.md los siguientes datos (la mayoría de estos datos se piden también en el formulario que se ha de rellenar para entregar la práctica: se recomienda hacer un copia y pega de estos datos en el formulario):

\begin{itemize}
  \item Nombre y titulación.
  \item Nombre de su cuenta en el laboratorio del alumno.
  \item URL del vídeo demostración de la funcionalidad básica
  \item URL del vídeo demostración de la funcionalidad optativa, si se ha realizado funcionalidad optativa
  \item URL de la aplicación desplegada
  \item Cuenta (login) y contraseña de los usuarios que están dados de alta en la aplicación.
  \item Resumen de las peculiaridades que se quieran mencionar sobre lo implementado en la parte obligatoria.
  \item Lista de funcionalidades opcionales que se hayan implementado, y breve descripción de cada una.
\end{itemize}

Estos datos se escribirán siguiendo estrictamente el siguiente formato:

\begin{verbatim}
# Entrega practica

## Datos

* Nombre:
* Titulación:
* Despliegue (url):
* Video básico (url):
* Video parte opcional (url):
* Despliegue (url):
*

## Cuenta Admin Site

* usuario/contraseña

## Cuentas usuarios

* usuario/contraseña
* usuario/contraseña
* ...

## Resumen parte obligatoria

## Lista partes opcionales

* Nombre parte:
* Nombre parte:
* ...
\end{verbatim}

Asegúrate de que las URLs incluidas en este fichero están adecuadamente escritas en Markdown, de forma que la versión HTML que genera GitLab los incluya como enlaces ``pinchables''.
\end{itemize}


%%----------------------------------------------------------------------------
\subsection{Notas y comentarios}

La práctica deberá funcionar en el entorno GNU/Linux (Ubuntu) del laboratorio de la asignatura con la versión de Django que se ha usado en prácticas.

La práctica deberá funcionar desde el navegador Firefox disponible en el laboratorio de la asignatura.

Los documentos XML que genere la práctica deberán ser correctos desde el punto de vista de la sintaxis XML, y por lo tanto reconocibles por un reconocedor XML, como por ejemplo el del módulo xml.sax de Python. Los documentos JSON generados deberán ser correctos desde el punto de vista de la sintaxis JSON, y por lo tanto reconocibles por un reconocedor JSON, como por ejemplo el del módulo json de Python

%%----------------------------------------------------------------------------
\subsection{Preguntas frecuentes}
\label{sec:practica-2021-05:preguntas}

A continuación, algunas preguntas relacionadas con el enunciado de esta práctica, junto con sus respuestas:

\begin{itemize}

\item Cuando despliego mi práctica en Python Anywhere, algunos recursos reconocidos no me funcionan, pero otros (YouTube entre ellos), sí. Todo me funciona bien en mi versión local. ¿Qué está pasando?

  Las máquinas virtuales de Python Anywhere están limitadas en cuanto a los sitios a los que se pueden conectar: sólo se pueden conectar a aquellos que están en una cierta ``lista blanca''. Por eso, si el sitio al que tu programa se tiene que conectar para obtener recursos no está en la lista blanca, no va a poder descargarse el documento XML o JSON de esos recursos. Para evitar problemas, en el caso de despliegue en Python Anywhere pedimos que funcionen bien los recursos reconocidos que están en la lista blanca, y para los demás, que tengan recursos en la base de datos de despliegue. Más detalles en el apartado sobre despliegue de este enunciado (\ref{sec:practica-2021-05:despliegue}).
  
\item En la pagina de información que se menciona en el enunciado, ¿qué hay que incluir en el apartado de documentación?

Casi que lo que queráis, lo importante es tener la página. Puede ser por ejemplo un resumen de un párrafo de lo que hace la aplicación.

\item ¿Qué es la ``API key'' en la API de algunos sitios, como Last.fm y GitLab?
\label{sec:practica-2021-05:preguntas-apikey}

  Algunas API de servicio, entre ellas la de Last.fm y GitLab, requieren el uso de una ``API key'' (clave de API) para poder usarla. Normalmente, estas claves las usa el servicio para evitar abusos, o para limitar lo que se puede hacer con su API. El caso es que si no se incluye la clave de API en cada GET que se hace al servicio, no se reciben los datos (el documento XML o JSON).

  Es habitual que estas claves se obtengan creándose una cuenta en el servicio en cuestión, y luego obteniendo la clave en una página al efecto, estando autenticados con el servicio.

  Por ejemplo, en el caso de Last.fm, hay que ir a la página de petición de claves de API\footnote{Lastfm Create API Account: \url{https://www.last.fm/api/account/create}}, donde (una vez autenticados con una cuenta de Last.fm), rellenaremos los dato que nos pide: ``contact email'', ``application name'' (cualquier nombre de aplicación, por ejemplo LoVisto), ``application description'' (cualquier descripción por ejemplo ``App to manage Last.fm artists''). En este caso, puedes ignorar los campos ``callback url'' y ``application homepage''. Cuando se hayan enviado estos datos, te devolverá entre otros datos tu ``API key''. Esa es la que tendrá que usar en tus llamadas a Last.fm.

  Como las claves de API son personales, mantenlas en secreto. En particular, no las subas a repositorios públicos, pues cualquiera podrá verlas (y usarlas). Si quieres que el repositorio de tu práctica sea público, incluye la clave en un fichero que tengas sólo en tu disco, y no subas al repositorio git. Por ejemplo, puedes poner la clave en un fichero \verb|apikeys.py| del estilo de este:

\begin{verbatim}
LASTFM_APIKEY = "012345678"
\end{verbatim}

Luego, en el módulo Python que la uses (por ejemplo \verb|views.py|), pondrás algo como:

\begin{verbatim}
from .apikeys import LASTFM_APIKEY
\end{verbatim}

Y ya puedes usar la clave en tu código. Este fichero \verb|apikeys.py| no lo subirás al repositorio git público.

Si usas claves de API en tu práctica, indícalo claramente en el fichero de entrega de la práctica, y o bien sube una clave de API válida (si el repositorio de entrega es privado) para que la podamos probar, o bien indica en qué fichero hay que ponerla, y cómo se consigue una clave API válida para el servicio que estés usando, de forma que la podamos conseguir y ejecutar tu práctica.

  
\item ¿Qué quiere decir ``vídeo empotrado'', en el caso de YouTube? ¿Cómo se hace?

  ``Vídeo empotrado'' es en ese caso la traducción de ``embedded video'', y quiere decir que el video aparezca directamente en la página, normalmente en un elemento \verb|iframe|. En el caso de YouTube, puedes obtener el código HTML para empotrar cualquier video si, cuando lo estás viendo en el navegador, pulsas el botón de compartir (``Share''), y eliges la opción ``Embed''. Esto muestra un código HTML como el siguiente:

\begin{verbatim}
<iframe width="560" height="315"
  src="https://www.youtube.com/embed/HZOodD84MR8"
  frameborder="0"
  allow="accelerometer; autoplay; encrypted-media; gyroscope; picture-in-picture"
  allowfullscreen></iframe>
\end{verbatim}

Basta con que en él cambies el código del video (en este caso ``HZOodD84MR8''  por el del video que quieres empotrar, y ya está. Puedes ver también esta información, incluyendo el HTML para el iframe de un video, en el fichero de información sobre un video (ver el enunciado cuando se especifican los recursos reconocidos de YouTube).
  
\item ¿Es necesario utilizar los mecanismos provistos por Django para el control de sesiones y autenticación?

  En principio, esa es la solución recomendada. El principal problema suele ser asegurarse de que cualquier mecanismo alternativo funciona al menos tan bien como el de Django, lo que no es en general trivial. De todas formas, salvo muy buenos motivos, la aplicación es una aplicación Django, y por lo tanto cuantas más facilidades de Django se usen (bien usadas), mejor.

\item Los archivos CSS para ``modo oscuro'' y ``modo normal'', ¿dónde y cómo debemos guardarlos?

  La forma recomendada de hacerlo es mediante plantillas:

  \begin{itemize}
  \item En el directorio de plantillas incluirías una para la hoja CSS del sitio. Esa plantilla tendría como variables de plantilla los valores que cambien de estilo normal a estilo oscuro (color de tipo de letra, color de fondo, etc.).
  \item Además, para cada usuario, tendrás una tabla en la base de datos donde se indicará el modo (normal o oscuro).
  \item Tendrás una vista en \texttt{views.py} que se encargará de generar la hoja CSS a partir de la plantilla. Esa vista es la que comprobará si la petición que está atendiendo corresponde a un usuario (en cuyo caso tendrá que obtener los valores para ese usuario de la tabla anterior), o no (en cuyo caso usará valores por defecto, siempre para el modo ``normal''). Con los valores que obtenga, generará la hoja CSS a partir de la plantilla anterior.
  \item Por último, en \texttt{urls.py} tendrás una línea para indicar que si te piden el recurso que sirve la hoja de estilo, llamas a la vista anterior.
  \end{itemize}

\item ¿Dónde puedo realizar el despliegue de la aplicación?

  El despliegue puede realizarse en cualquier ordenador que esté conectado permanentemente a Internet durante el periodo de corrección, en una dirección accesible desde cualquier navegador conectado a su vez a Internet. Esto puede ser por ejemplo un ordenador personal en un domicilio con acceso permanente a Internet, adecuadamente configurado (puede ser una Raspberry Pi o similar, si se busca una solución simple y de bajo coste). También puede ser un servicio en Internet, por ejemplo uno gratuito como los que ofrecen Google (instrucciones\footnote{GCP Quickstart Using a Linux VM:\\ \url{https://cloud.google.com/compute/docs/quickstart-linux}}, precios\footnote{Google Compute Engine Pricing:\\ \url{https://cloud.google.com/compute/pricing}}), o PythonAnywhere (instrucciones\footnote{Capítulo ``Deploy!'' de Django Girls Tutorial:\\ \url{https://tutorial.djangogirls.org/en/deploy/}}, precios\footnote{PythonAnywhere Plans and Pricing:\\ \url{https://www.pythonanywhere.com/pricing/}}). Los profesores podremos ayudar de forma más detallada con PythonAnywhere.

\end{itemize}

%% ejercicios.tex
%%

%% Ejercicios (comunes para SAT y SARO), 2013-2014.

%%---------------------------------------------------------------------------
%%---------------------------------------------------------------------------
%%---------------------------------------------------------------------------
\section{Entrega de prácticas incrementales}
\label{sec:eje-entrega-practicas-incr}

Para la entrega de prácticas incrementales se utilizarán repositorios git públicos alojados en GitLab. Para cada práctica entregable los profesores abrirán un repositorio público en el proyecto CursosWeb~\footnote{\url{https://gitlab.etsit.urjc.es/CursosWeb}}, con un nombre que comenzará por ``X--Serv--'', seguirá con el nombre del tema en el que se inscribe la práctica (por ejemplo, ``Python'' para el tema de introducción a Python) y el identificador del ejercicio (por ejemplo, ``Calculadora''). Este repositorio incluirá un fichero README.md, con el enunciado de la práctica, y cualquier otro material que los profesores estimen conveniente.

Cada alumno dispondrá de una cuenta en GitLab, que usará a efectos de entrega de prácticas. Esta cuenta deberá ser apuntada en una lista, en el sitio de la asignatura en el campus virtual, cuando los profesores se lo soliciten. Si el alumno desea que no sea fácil trazar su identidad a partir de esta cuenta, puede elegir abrir una cuenta no ligada a sus datos personales: a efectos de valoración, los profesores utilizará la lista anterior. Si el alumno lo desea, puede usar la misma cuenta en GitLab para otros fines, además de para la entrega de prácticas.

Para trabajar en una práctica, los alumnos comenzarán por realizar una copia (fork) de cada uno de estos repositorios. Esto se realiza en GitLab, visitando (tras haberse autenticado con su cuenta de usuario de GitLab para entrega de prácticas) el repositorio con la práctica, y pulsando sobre la opción de realizar un fork. Una vez esto se haya hecho, el alumno tendrá un fork del repositorio en su cuenta, con los mismos contenidos que el repositorio original de la práctica. Visitando este nuevo repositorio, el alumno podrá conocer la url para clonarlo, con lo que podrá realizar su clon (copia) local, usando la orden \verb|git clone|.

A partir de este momento, el alumno creará los ficheros que necesite en su copia local, los irá marcando como cambios con \verb|git commit| (usando previamente \verb|git add|, si es preciso, para añadirlos a los ficheros considerados por git), y cuando lo estime conveniente, los subirá a su repositorio en GitLab usando \verb|git push|.

Por lo tanto, el flujo normal de trabajo de un alumno con una nueva práctica será:

\begin{verbatim}
[En GitLab: visita el repositorio de la práctica en CursosWeb,
y le hace un fork, creando su propia copia del repositorio]

git clone url_copia_propia

[Se cera el directorio copia_propia, copia local del repositorio propio]

cd copia_propia
git add ... [ficheros de la práctica]
git commit .
git push
\end{verbatim}

Conviene visitar el repositorio propio en GitLab, para comprobar que efectivamente los cambios realizados en la copia local se han propagado adecuadamente a él, tras haber invocado \verb|git push|.

%%-----------------------------------------------------------------------------
%%-----------------------------------------------------------------------------
%%-----------------------------------------------------------------------------
\section{Ejercicios 01: Conceptos básicos de aplicaciones web}

%%-----------------------------------------------------------------------------
%%-----------------------------------------------------------------------------
\subsection{Web 2.0}
\label{subsec:web-20}

\textbf{Enunciado:}

Seguramente has oído hablar muchas veces de la ``web 2.0''. ¿Qué es lo que significa esta expresión? Si puedes, cita referencias en la Red al respecto.

%%-----------------------------------------------------------------------------
%%-----------------------------------------------------------------------------
\subsection{Última búsqueda}
\label{subsec:ultima-busqueda}

\textbf{Enunciado:}

¿Cómo mostrar la última búsqueda en un  buscador?

Se quiere que un cierto buscador web muestre a sus usuarios la última búsqueda que hicieron en él. Para ello, se utilizarán cookies. Son relevantes tres interacciones HTTP: la primera, en la que el navegador pide la página HTML con el formulario de búsquedas, la segunda, en la que el navegador envía la cadena de búsqueda que el usuario ha escrito en el navegador, y la tercera, que se realizará en cualquier momento posterior, en la que el navegador vuelve a pedir la página con el formulario de búsquedas, que ahora se recibe anotada con la cadena de la última búsqueda. Se pide indicar dónde van las cookies, cómo son éstas, y cómo solucionan el problema.

\textbf{Solución:}

Se puede hacer utilizando identificador de sesión en las cookies. Pero también es posible hacerlo sin que el servidor (el buscador) tenga que almacenar todos los identificadores de sesión  junto con la última búsqueda realizada, lo que tiene varias ventajas.

Para identificador de sesión, basta con un número aleatorio grande que se almacena en la cookie. La cookie la envía el buscador al navegador en la respuesta al HTTP GET que se realiza para obtener la página del buscador. Luego, esa cookie va en cada POST que hace el navegador (para realizar una nueva búsqueda). Si no se quiere que el buscador almacena la última pregunta para cada sesión, se puede enviar la propia búsqueda en la cookie.   

\textbf{Discusiones relacionadas:}

\begin{itemize}
\item Ventajas y desventajas de utilizar identificadores de sesión, o de almacenar las preguntas en cookies en el navegador.
\item ¿Serviría el mismo esquema para un servicio de banca electrónica? (en lugar de ``recordar'' la última pregunta, se quiere recordar qué usuario se autenticó.
\item Cómo implementarlo usando el identificador de usuario y la contraseña en la cookie. Implicaciones para la seguridad. El problema de la salida de la sesión.
\end{itemize}

%%--------------------------------------------------------------------
%%--------------------------------------------------------------------
\subsection{Espía a tu navegador (Firefox Developer Tools)}
\label{subsec:firefox-devel}

\textbf{Enunciado:}

El navegador hace una gran cantidad de tareas interesantes para esta asignatura. Es muy útil poder ver cómo lo hace, y aprender de los detalles que veamos. De hecho, también, en ciertos casos, se puede modificar su comportamiento. Para todo esto, se pueden usar herramientas específicas. En nuestro caso, vamos a usar las ``Firefox Developer Tools'', que vienen ya preinstaladas en Firefox.

El ejercicio consiste en:

\begin{itemize}
\item Ojear las distintas herramientas de Firefox Developer Tools.
\item Utilizarlas para ver la interacción HTTP al descargar una página web real.
\item Utilizarlas para ver el árbol DOM de una página HTML real.
\end{itemize}

Más adelante, lo utilizaremos para otras cosas, así que si quieres jugar un rato con lo que permiten hacer estas herramientas, mucho mejor.

\textbf{Referencias}

Sitio web de Firefox Developer Tools: \\
\url{https://developer.mozilla.org/en/docs/Tools}



%%--------------------------------------------------------------------
%%--------------------------------------------------------------------
\subsection{Espía a tu navegador (Firebug)}
\label{subsec:firebug}

\textbf{Enunciado:}

El navegador hace una gran cantidad de tareas interesantes para esta asignatura. Es muy útil poder ver cómo lo hace, y aprender de los detalles que veamos. De hecho, también, en ciertos casos, se puede modificar su comportamiento. Para todo esto, se pueden usar herramientas específicas. En nuestro caso, vamos a usar el módulo ``Firebug'' de Firefox (también disponible para otros navegadores).

El ejercicio consiste en:

\begin{itemize}
\item Instalar el módulo Firebug en tu navegador
\item Utilizarlo para ver la interacción HTTP al descargar una página web real.
\item Utilizarlo para ver el árbol DOM de una página HTML real.
\end{itemize}

Más adelante, lo utilizaremos para otras cosas, así que si quieres jugar un rato con lo que permite hacer Firebug, mucho mejor.

\textbf{Referencias}

Sitio web de Firebug: \url{https://getfirebug.com/}


%%-----------------------------------------------------------------------------
%%-----------------------------------------------------------------------------
\subsection{Explora tus cookies}
\label{subsec:explora-cookies}

\textbf{Enunciado:}

En este ejercicio vamos a ver las cookies que intercambia nuestro navegador con un servidor simple. El servidor que vamos a usar es \verb|cookies-server-6.py| (en la carpeta \verb|cookies|). Ejecuta el servidor, y luego, utilizando las herramientas de desarrollador de Firefox (ver ejercicio~\ref{subsec:firefox-devel}, observa las cookies que se intercambian entre este servidor y el navegador. En concreto, carga en el navegador la página principal del servidor, escribe algo en el formulario que te aparecerá, y contesta a este ejercicio escribiendo las cookies que observes en la interacción que se produce cuando le das al botón ``Submit'' para enviar al servidor el texto que has escrito.

%%-----------------------------------------------------------------------------
%%-----------------------------------------------------------------------------
\subsection{Explora tus cookies (2)}
\label{subsec:explora-cookies-2}

\textbf{Enunciado:}

Vamos a explorar las diferencias entre dos programas que tratan de ``recordarte''
lo último que escribiste en un formulario. Ambos son servidores HTTP, y están en el directorio \verb|Python-Web/cookies| (en el repositorio de código de la asignatura), y son \verb|cookies-server-8.py| y \verb|cookies-server-9.py|. El ejercicio consiste en ejecutar cada uno de ellos de esta forma:

\begin{itemize}
\item Lanza el programa servidor que vas a probar.
\item En el navegador, carga la página correspondiente al recurso principal de ese servidor.
\item Borra las cookies que pueda haber para ese servidor.
\item Recarga la página que tienes en el navegador
\item En el formulario que tienes en la página, escribe ``Primero'', y envíalo al servidor. Llamaremos al resultado de este envío (la página que muestre el navegador al recibir la respuesta del servidor ``Página~1''.
\item En el formulario que tienes ahora en la Página~1, escribe ``Segundo'', y vuelve a enviarlo al servidor. Llamaremos al resultado de este envío (la página que muestre el navegador al recibir la respuesta del servidor ``Página~2''.
\item En el formulario que tienes ahora en la Página~2, escribe ``Tercero'', y vuelve a enviarlo al servidor. Llamaremos al resultado de este envío (la página que muestre el navegador al recibir la respuesta del servidor ``Página~3''.
\end{itemize}

Compara qué ves en Página~1, Página~2 y Página~3 en los dos casos (cuando lanzas cada uno de los dos servidores, y sigues el proceso con ellos). Trata de explicar lo que ocurre viendo en el navegador las cookies que envía el servidor en cada uno de los casos. A continuación, trata de explicarlo mirando el código de los dos servidores. Puedes utilizar la herramienta \verb|diff| para ver las diferencias en el código de ambos, si eso te ayuda.

Escribe como respuesta:

\begin{itemize}
\item Las diferencias que observes entre Página~1, Página~2 y Página~3 (descritas, acompañadas de capturas de pantalla si lo ves útil).
\item La explicación que hayas podido encontrar a las diferencias mirando las cookies en el navegador.
\item La explicación que hayas podido encontrar a las diferencias mirando el código de los dos servidores.  
\end{itemize}

%%-----------------------------------------------------------------------------
%%-----------------------------------------------------------------------------
\subsection{Servidores que recuerdan}
\label{subsec:servidores-recuerdan}

\textbf{Enunciado:}

En el directorio \verb|Python-Web/cookies| (en el repositorio de código de la asignatura) puedes encontrar los programas \verb|content-server-1.py| y \verb|content-server-2.py|. Ambos utilizan cookies de datos para ``recordar'' el último texto que se introdujo en el formulario que proporciona el servidor. El primero, utiliza GET para enviar al servidor el contenido del formulario, y el segundo utiliza POST.

Este ejercicio consiste en entender el código de ambos programas, y escribir otros dos, \verb|content-server-3.py| y \verb|content-server-4.py|, que hagan lo mismo, pero utilizando cookies de sesión (que identifican el navegador, y utilizan el identificador para buscar el último contenido del formulario en un diccionario que mantienen).

%%-----------------------------------------------------------------------------
%%-----------------------------------------------------------------------------
\subsection{Servicio horario}
\label{subsec:ej-servicio-horario}

\textbf{Enunciado:}

Queremos construir una aplicación web que cuando se consulta, devuelva la hora actual. Además, queremos que cuando se consulta por segunda vez, devuelva la hora actual y la hora en que se consultó por última vez. Explicar cómo se pueden usar cookies para conseguirlo.

\textbf{Enunciado avanzado:}

Igual que el anterior, pero se quiere que se muestre no sólo la hora en que se consultó por última vez, sino las horas de todas las consultas previas (además de la hora actual).

%%-----------------------------------------------------------------------------
%%-----------------------------------------------------------------------------
\subsection{Última búsqueda: números aleatorios o consecutivos}
\label{subsec:ultima-busqueda-aleconsec}

\textbf{Enunciado:}

En el ejercicio ``Última búsqueda'' (ejercicio~\ref{subsec:ultima-busqueda}) una de las soluciones pasa por usar cookies con identificadores de sesión. En principio, se han propuesto dos posibilidades para esos identificadores:

\begin{itemize}
\item Números enteros aleatorios sobre un espacio de números grande (por ejemplo entre 0 y $2^{128}-1$)
\item Números enteros consecutivos, comenzando por ejemplo por 0.
\end{itemize}

Comenta cuál de las dos soluciones te parece mejor, y si crees que alguna de ellas no sirve para resolver el problema. En ambos casos, indica las razones que te llevan a esa conclusión


%%------------------------------------------------------------------------------
%%------------------------------------------------------------------------------
\subsection{Cookies en tu navegador}
\label{subsec:cookies-navegador}

\textbf{Enunciado:}

Busca dónde tiene tu navegador accesible la lista de cookies que mantiene, y mírala. ¿Cuántas cookies tienes? ¿Qué sitio te ha puesto más cookies? ¿Cuál es la cookie más antigua que tienes? Explica también qué navegador usas (nombre y versión), desde cuándo más o menos, y cómo has podido ver las cookies en él.


%%------------------------------------------------------------------------------------
%%------------------------------------------------------------------------------------
\subsection{Cookies en tu navegador avanzado}
\label{subsec:cookies-navegador-2}

\textbf{Enunciado:}

Con el módulo adecuado, pueden editarse las cookies del navegador, lo que permite manejarlas con gran flexibilidad. Utiliza uno de estos módulos (por ejemplo, Cookie Quick Manager para Firefox) para manipular las cookies que tenga tu navegador. Utilízalo para ``traspasar'' una sesión de un navegador a otro. Por ejemplo, puedes buscar las cookies que te autentican con un servicio (el campus virtual, una red social en la que tengas cuenta, etc.), guardarlas en un fichero, transferirlas a otro ordenador con otro navegador, e instalarlas en él, para comprobar cómo puedes continuar con la sesión desde él.

\textbf{Referencias}

Cookie Quick Manager: \url{https://addons.mozilla.org/en-US/firefox/addon/cookie-quick-manager/}


%%--------------------------------------------------------------------------
%%--------------------------------------------------------------------------
\subsection{Sumador simple con varios navegadores}
\label{subsec:sumador-simple-varios}

\textbf{Enunciado:}

Igual que el ejercicio ``Sumador simple'' (\ref{subsec:sumador-simple}), pero ahora puede haber varios navegadores invocando la aplicación web. Se supone que los navegadores no se interfieren (esto es, uno completa una suma antes de que otro la empiece).

\textbf{Comentario:}

No hacen falta modificaciones al código del ejercicio ``Sumador simple'' (\ref{subsec:sumador-simple}).


%%------------------------------------------------------------------------------------
%%------------------------------------------------------------------------------------
\subsection{Sumador simple con varios navegadores intercalados}
\label{subsec:sumador-simple-varios-intercalados}

\textbf{Enunciado:}

Igual que``Sumador simple con varios navegadores'' (\ref{subsec:sumador-simple-varios}), pero ahora un navegador puede comenzar una suma en cualquier momento, incluyendo momentos en los que otro navegador no la haya terminado.

\textbf{Comentarios:}

En una primera versión, se implementa con una cookie simple que incluye el primer operando, de forma que el servidor de aplicaciones no tiene que almacenar los operandos ni las cookies.

En una segunda versión, se utiliza una cookie de sesión más clásica, con un entero aleatorio, y se almacena el estado en un diccionario indexado por ese entero.

%%------------------------------------------------------------------------------------
%%------------------------------------------------------------------------------------
\subsection{Sumador simple con rearranques}
\label{subsec:sumador-simple-rearranques}

\textbf{Enunciado:}

Igual que el ejercicio ``Sumador simple con varios navegadores intercalados'' (\ref{subsec:sumador-simple-varios-intercalados}), pero ahora desde que el navegador inicia la suma hasta que la completa, puede haberse caído la aplicación web.

\textbf{Comentario:}

La aplicación web tendrá que  almacenar su estado en almacenamiento estable. Hay que detectar cuál es ese estado, y almacenarlo en  un fichero, en  una base de datos, etc.


%%--------------------------------------------------------------------------
%%--------------------------------------------------------------------------
\subsection{Contador simple}
\label{subsec:contador-simple}

\textbf{Enunciado:}

Construir una aplicación web que funcione como contador inverso. Ofrecerá un recurso que, cuando sea invocado mediante un método GET, devolverá un número entero. La primera vez que se invoque, el número devuelto será un 5. Cuando se le invoque sucesivamente, el número obtenido se irá decrementando en uno (4, 3, 2...). Cuando se haya obtenido un 0, el siguiente número será de nuevo el 5 (esto es, el contador funciona como un contador inverso cíclico).

\textbf{Comentario:}

Es importante hacer énfasis en la solución en la estructura de la aplicación, tratando de estructurar el código de forma que las diferentes acciones que hará la aplicación web queden claras.

%%--------------------------------------------------------------------------
%%--------------------------------------------------------------------------
\subsection{cURL básico}
\label{subsec:curl-basico}

\textbf{Enunciado:}

Prueba el ejercicio ``Contador simple'' (\ref{subsec:contador-simple}) con el programa \verb|curl|. Utilizalo para ver el documento que se recibe de tu servidor, para ver las cabeceras que te envía, para ver tanto cabeceras (de ida y vuelta) como cabeceras...

\textbf{Materiales:}

\begin{itemize}
\item \href{https://linuxacademy.com/guide/13852-understanding-curl-and-http-headers/}{Understanding CURL and HTTP Headers} (tutorial)
\item \href{https://curl.haxx.se/}{cURL} (sitio web)
\item \href{https://curl.haxx.se/book.html}{Everything curl} (libro)
\end{itemize}

\textbf{Solución:}

\begin{verbatim}
curl -XGET http://localhost:1234/
curl -XGET -I http://localhost:1234/
curl -XGET -Iv http://localhost:1234/
\end{verbatim}


%%--------------------------------------------------------------------------
%%--------------------------------------------------------------------------
\subsection{Depurador básico}
\label{subsec:depurador-basico}

\textbf{Enunciado:}

Prueba el ejercicio ``Contador simple'' (\ref{subsec:contador-simple}) con el depurador de Python, por ejemplo, ejecutándolo desde PyCharm.


%%--------------------------------------------------------------------------
%%--------------------------------------------------------------------------
\subsection{Contador simple con varios navegadores}
\label{subsec:contador-simple-varios}

\textbf{Enunciado:}

Igual que el ejercicio ``Contador simple'' (\ref{subsec:contador-simple}), pero ahora puede haber varios navegadores invocando la aplicación web. Se supone que los navegadores no se interfieren (esto es, uno completa todas sus operaciones con el contador antes de que otro la empiece).

\textbf{Comentario:}

No hacen falta modificaciones al código del ejercicio ``Contador simple'' (\ref{subsec:contador-simple}), si se asume que cuando se invoque el contador, éste puede empezar por cualquier número de su ciclo. Si por el contrario se quiere que comience como ``la primera vez'' (por 5) es preciso detectar que se está sirviendo a un nuevo navegador, y habrá que prever algún mecanismo al respecto.

Puede consultarse la implementación de referencia disponible en
\href{https://github.com/CursosWeb/Code/blob/master/Python-Web/counter/counter-server-1.py}{counter-server-1.py}

%%-------------------------------------------------------------------------
%%-------------------------------------------------------------------------
\subsection{Contador simple con varios navegadores intercalados}
\label{subsec:contador-simple-varios-intercalados}

\textbf{Enunciado:}

Igual que ``Contador simple con varios navegadores'' (\ref{subsec:contador-simple-varios}), pero ahora un navegador puede comenzar a trabajar con el contador en cualquier momento, incluyendo momentos en los que otro navegador no haya terminado aún.

\textbf{Comentarios:}

En una primera versión, se implementa con una cookie simple que incluye el número que ha servido el contador de forma que la aplicación no tiene que almacenar el valor del contador para cada navegador.

En una segunda versión, se utiliza una cookie de sesión más clásica, con un entero aleatorio, y se almacena el estado del contador correspondiente en un diccionario indexado por ese entero.

En una tercera versión, se podría añadir una operación para crear un contador único para cada navegador, con un nombre de recurso propio. Cada navegador conocería su recurso, y sólo utilizaría ese. Se puede evitar que un navegador utilice un recurso que no le corresponde haciendo que su nombre no sea fácilmente descubrible.

Puede consultarse la implementación de referencia disponible en
\href{https://github.com/CursosWeb/Code/blob/master/Python-Web/counter/counter-server-2.py}{counter-server-2.py}

%%------------------------------------------------------------------------------------
%%------------------------------------------------------------------------------------
\subsection{Contador simple con rearranques}
\label{subsec:contador-simple-rearranques}

\textbf{Enunciado:}

Igual que el ejercicio ``Contador simple con varios navegadores intercalados'' (\ref{subsec:contador-simple-varios-intercalados}), pero ahora desde que el navegador inicia el trabajo con el contador hasta que la completa, puede haberse caído la aplicación web.

\textbf{Comentario:}

La aplicación web tendrá que  almacenar su estado en almacenamiento estable. Hay que detectar cuál es ese estado, y almacenarlo en  un fichero, en  una base de datos, etc.

Puede consultarse la implementación de referencia disponible en
\href{https://github.com/CursosWeb/Code/blob/master/Python-Web/counter/counter-server-3.py}{counter-server-3.py} y
\href{https://github.com/CursosWeb/Code/blob/master/Python-Web/counter/counter-server-4.py}{counter-server-4.py}.


%%-----------------------------------------------------------------------------
%%-----------------------------------------------------------------------------
\subsection{Traza de historiales de navegación por terceras partes}
\label{subsec:navegacion-terceras-partes}

Cuando un navegador realiza un GET sobre una página web HTML lanza a continuación, de forma automática, otras operaciones GET sobre los elementos cargables automáticamente que contenga esa página, como por ejemplo, las imágenes empotradas. Cada vez que se realiza uno de estos GET, se pueden recibir una o más cookies de los servidores que las sirven (y que en general pueden ser diferentes del que sirve la página HTML).

De esta forma, sirviendo imágenes para diferentes páginas HTML en diferentes sitios, una tercera parte puede trazar historiales de navegación, ligándolos a identificadores únicos. ¿Cómo?

Además, si la tercera parte en cuestión tiene acceso a información de un sitio web que permita identificar identidades, esos historiales pueden también ser ligados a identidades. ¿Cómo?

\textbf{Comentarios:}

La liga con identificadores únicos se puede lograr de varias formas, Por ejemplo se puede incluir en cada página HTML a trazar una imagen con nombre único, todas servidas por la tercera parte. La primera vez que sirve una imagen a un navegador dado, le envía también una cookie con identificador único. Todas las peticiones de imagen que se reciban serán escritas en un historial, junto con el identificador único de la cookie.

Para poder ligar este historial a una identidad, basta con que, en un servidor que ha identificado una identidad, sirva una imagen de la tercera parte con un nombre que permita posteriormente ligarlo a la identidad.

%%-----------------------------------------------------------------------------
%%-----------------------------------------------------------------------------
\subsection{Trackers en páginas web}
\label{subsec:trackers-paginas-web}

Instala el \emph{plug-in} Lightbeam para tu navegador. Este \emph{plug-in} permite detectar todos los sitios web que se acceden al descargar una página, incluyendo los forzados por ``trackers'' (objetos incluidos en una página web para trazar a quienes descargan esa página). Utilízalo para encontrar páginas web con muchos trackers. Una vez lo hayas hecho, indica las dos páginas (de sitios distintos) en las que hayas encontrado más trackers.

\textbf{Referencias}

Sitio web de Lightbeam: \url{https://www.mozilla.org/lightbeam/}

%%-----------------------------------------------------------------------------
%%-----------------------------------------------------------------------------
\subsection{Trackers en páginas web (Ghostery)}
\label{subsec:trackers-paginas-web-2}

Instala el \emph{plug-in} Ghostery para tu navegador. Este \emph{plug-in} permite detectar ``trackers'', objetos incluidos en una página web para trazar a quienes descargan esa página. Utilízalo para encontrar páginas web con muchos trackers. Una vez lo hayas hecho, indica las dos páginas (de sitios distintos) en las que hayas encontrado más trackers.

\textbf{Referencias}

Sitio web de Ghostery: \url{http://www.ghostery.com/}


%%-----------------------------------------------------------------------------
%%-----------------------------------------------------------------------------
%%-----------------------------------------------------------------------------
\section{Ejercicios 02: Servicios web que interoperan}



%%-----------------------------------------------------------------------------
\subsection{Arquitectura escalable}
\label{subsec:arq-escalable}

\textbf{Enunciado:}

Diseñar una arquitectura para una aplicación distribuida que cumpla las siguientes condiciones:

\begin{itemize}
\item Puede ser usada por millones de usuarios simultáneamente.
\item Hay miles de equipos de desarrollo trabajando sobre ella. Entre los equipos hay poca comunicación pero no deben tener conflictos entre sí.
\item Cada uno de los equipos podrían extender lo que habían hecho los otros sin que estos lo sepan y sin que la evolución de cada sistema rompiera la integración.
\end{itemize}

\textbf{Comentarios:}

Desde luego, hay otros sistemas, pero el web, entendido en sentido amplio, es uno que cumple bien estos requisitos.


%%-----------------------------------------------------------------------------
\subsection{Arquitectura distribuida}
\label{subsec:arq-distribuida}

\textbf{Enunciado:}

Diseñar una arquitectura para una aplicación distribuida que cumpla las siguientes condiciones:

\begin{itemize}
\item Pueda gestionar elementos en mi casa desde remoto, en particular, mi comida. Por tanto, tendrá que gestionar los alimentos que se encuentran en la nevera, la despensa, el bol de frutas, etc.
\item Pueda interactuar tanto con máquinas como con humanos
\item Sea lo más sencilla posible
\item Sea escalable
\end{itemize}

En particular, usando REST, define algunos recursos, y las operaciones que se podrían hacer sobre ellos. Explica también qué necesitaría para poder interoperar con los recursos correspondientes, por ejemplo, a la casa de tus amigos.

\textbf{Comentarios:}

Desde luego, hay muchas maneras de hacerlo y eso favorecerá el debate.

Una primera idea es modelar los elementos como objetos (la nevera, los alimentos, etc.) y hacerlo llegar de alguna manera al otro lado de la red, donde está mi portátil (esta es una solución que siguen muchos web services, o incluso CORBA). Hay que entender que esto hará que en el lado del portátil tengamos que conocer cómo funcionan los elementos (sus atributos y sus métodos). Es como tener que aprender el manual de instrucciones (los verbos) para cada cacharro que tengamos en la cocina.

Al ver esta solución, nos damos cuenta de que contamos en el otro lado con sustantivos. Éstos tienen una localización única, que especificamos mediante una URL. Asimismo, existe la URN, que permite especificar unívocamente un elemento según su nombre, pero se ha de tener en cuenta de que puede haber un URN para muchos elementos (es como el ISBN, que hay uno para toda la edición, o sea para muchos libros). Dado la URL localizamos un URN de manera unívoca.

Mientas, en el otro lado (en el cliente) tendremos un número mínimo de acciones (los verbos). Vemos el primero: GET. Éste no obtiene el sustantivo, sino una representación del mismo. Los sustantivos son recursos. Y estos recursos pueden venir expresados de varias maneras. Así, por ejemplo, si pedimos manzanas desde un portátil la representación podría ser una imagen muy detallada; para el móvil, la imagen será más pequeña; y si el que lo pide es una máquina, podría ser un XML. Vemos los demás métodos: PUT, POST y DELETE.

Vistos los métodos discutimos si cambian el estado (vemos que sólo GET no lo hace) y si el resultado de realizar varios consecutivos es igual a hacerlo una vez (lo que llamamos idempotencia, vemos que sólo POST no lo es).

Introducimos el concepto de elemento y colección de elementos (cuando pedimos una colección, nos da un listado de los elementos que contiene; este listado contiene enlaces a los mismos) y qué pasa cuando aplicamos un método a cada uno. Hacemos especial hincapié en la diferencia entre PUT y POST.

Introducimos el concepto de REST y sus reglas. Hay varias las hemos visto ya: URLs, enlaces, representaciones y métodos. Nos falta por ver que las comunicaciones son sin estado. Los recursos pueden tenerlo, pero no la comunicación. Discutimos qué significa esto con respecto a lo que hemos visto en la asignatura hasta ahora, en particular con respecto a las sesiones (y las cookies).

Finalmente discutimos porque la web no es así, si en realidad los diseñadores de HTTP habían diseñado el protocolo para que todo fuera REST. Comentamos que con los navegadores sólo podemos hacer GETs y POSTs (y contamos que podemos utilizar los demás métodos mediante plug-ins como Poster (ver ejercicio~\ref{subsec:inst-poster}). Mostramos que, más allá de los navegadores, ya estamos en disposición de crear programas para interactuar con servidores REST, de manera que podemos comunicar máquinas entre sí siguiendo estas reglas. Discutimos las ventajas de este enfoque en esos casos.


%%-----------------------------------------------------------------------------
\subsection{Lista de la compra}
\label{subsec:lista-compra}

\textbf{Enunciado:}

Vamos a diseñar una API HTTP para un servicio que permite guardar una lista de la compra. Supongamos que esta lista está compuesta únicamente por los items que quiero comprar en un momento dado (zanahorias, yogures, etc.) y un número natural para cada item que tengo, que expresa la cantidad que quiero comprar. Por ejemplo, en un momento dado, la lista podría ser:

\begin{itemize}
\item Zanahorias: 5
\item Yogures: 4
\item Leche: 2
\end{itemize}

Puede haber items en la lista de la compra con valor 0, si se encuentra que es útil por algún motivo.

Las operaciones que permitirá la API del servicio son: consultar la lista, añadir un item (con su correspondiente cantidad) a la lista, modificar la cantidad de un item en la lista, y borrar un item de la lista.

Se pide diseñar una API HTTP para esta aplicación (servicio) web, identificando cuáles son los recursos relevantes, las operaciones HTTP válidas sobre ellas, y describiendo la semántica de cada una de ellas.

Podemos imaginar que el servicio web va a ser usado, por ejemplo, desde una aplicación en el móvil, que podrá hacer GET, PUT, POST y DELETE (usando HTTP). Cada vez que quiero apuntar o borrar algo de la lista de la compra, o quiero consultar la lista, utilizo la app del móvil para acceder, mediante HTTP, al servicio de lista de la compra.

%%-----------------------------------------------------------------------------
\subsection{Listado de lo que tengo en la nevera}
\label{subsec:contenido-nevera}

\textbf{Enunciado:}

Este es un ejercicio muy similar a ``Lista de la compra'' (\ref{subsec:contenido-nevera}). Vamos a diseñar una API REST para una aplicación que concreta el ejercicio ``Arquitectura distribuida'' (\ref{subsec:arq-distribuida}) en un caso bien simple: una aplicación web que mantenga la lista de lo que tengo en la nevera. El tipo de lista será el mismo que se indica para en el ejercicio de la lista de la compra, pero ahora trataremos de que la API cumpla los principios REST.

\textbf{Comentarios:}

Hay muchas soluciones posibles para este problema, que sobre todo pretende que se reflexione sobre las características que hacen de una interfaz HTTP una interfaz REST. Pero en cualquier caso, como mínimo hay que definir cuáles serán los recursos (y sus nombres), las operaciones sobre cada uno de ellos, y un breve comentario sobre su funcionamiento.

Una posible solución sería: \\

\begin{tabular}{l|l|p{10cm}}
  Recurso & Método & Descripción \\ \hline \hline
  /       & GET    & Lista de items en la nevera (enlaces a recursos) \\
          & POST   & Crea un nuevo item, \verb|item=xx&cantidad=yy| \\
  /[item] & GET    & Obten el valor (número) del alimento ``item'' \\
          & PUT    & Actualiza el valor del alimento ``item'' \\
          & DELETE & Borra el item (elimina el recurso) \\
\end{tabular}

Esta interfaz HTTP podría usarse desde la aplicación como se ve en los siguiente ejemplos.

La primera vez que se introducen zanahorias (supongamos que se introducen 5):

\begin{itemize}
\item Petición (para ver si hay zanahorias):
\begin{verbatim}
GET / HTTP/1.1
\end{verbatim}

\item Respuesta:
\begin{verbatim}
HTTP/1.1 200 OK

<a href="/leche"></a>
<a href="/chorizo"></a>
\end{verbatim}

\item Petición (para crear el recurso para las zanahorias):
\begin{verbatim}
POST / HTTP/1.1

item=zanahorias&cantidad=5
\end{verbatim}

\item Respuesta:
\begin{verbatim}
HTTP/1.1 200 OK

<a href="/chorizo"></a>
\end{verbatim}

\end{itemize}

Si sacamos 3 zanahorias:

\begin{itemize}
\item Petición (para ver cuántas zanahorias hay):
\begin{verbatim}
GET /zanahorias HTTP/1.1
\end{verbatim}

\item Respuesta:
\begin{verbatim}
HTTP/1.1 200 OK

5
\end{verbatim}

\item Petición (para actualizar al nuevo número de zanahorias):
\begin{verbatim}
PUT /zanahorias HTTP/1.1

2
\end{verbatim}

\item Respuesta:
\begin{verbatim}
HTTP/1.1 200 OK

2
\end{verbatim}

\end{itemize}

Si sacamos otras 2 zanahorias:

\begin{itemize}
\item Petición (para ver cuántas zanahorias hay):
\begin{verbatim}
GET /zanahorias HTTP/1.1
\end{verbatim}

\item Respuesta:
\begin{verbatim}
HTTP/1.1 200 OK

2
\end{verbatim}

\item Petición (para eliminar el recurso, porque quedaría a cero):
\begin{verbatim}
DELETE /zanahorias HTTP/1.1
\end{verbatim}

\item Respuesta:
\begin{verbatim}
HTTP/1.1 200 OK
\end{verbatim}

\end{itemize}



%%-----------------------------------------------------------------------------
\subsection{Sumador simple versión REST}
\label{subsec:sumador-simple-rest}

\textbf{Enunciado:}

Desarrollar una versión RESTful de ``Sumador simple'' (ejercicio~\ref{subsec:sumador-simple}). ¿Plantea problemas si se usa simultáneamente desde varios navegadores? ¿Plantea problemas si se cae el servidor entre dos invocaciones por parte del mismo navegador?

\textbf{Comentarios:}

Hay varias formas de hacer el diseño, pero por ejemplo, cada sumando podría ser un recurso, y el resultado obtenerse en un tercero (o bien como respuesta al actualizar el segundo sumando). Cada suma podría también realizarse en un espacio de nombres de recurso distinto (con su propio primer sumando, segundo sumando y resultado).


%%-----------------------------------------------------------------------------
\subsection{Calculadora simple versión REST}
\label{subsec:calc-simple-rest}

\textbf{Enunciado:}

Realizar una calculadora de las cuatro operaciones aritméticas básicas (suma, resta, multiplicación y división), siguiendo los principios REST, a la manera del sumador simple versión REST (ejercicio~\ref{subsec:sumador-simple-rest}).

\textbf{Comentarios:}

Este ejercicio, más que para proponer una solución concreta, está diseñado para debatir sobre las posibles soluciones que se le podrían dar. Por ejemplo, tenemos primero la versiones donde se supone un único usuario:

\begin{itemize}
\item Versión con un recurso por tipo de operación (``/suma'', ``resta'', etc.). Se actualiza con PUT, que envía los operandos (ej: 4,5), se consulta con GET, que devuelve el resultado (ej: 4+5=9).
\item Versión con un único recurso, ``/operacion''. Se actualiza con PUT, que envía en el cuerpo la operación (ej: 4+5), se consulta con GET, que devuelve el resultado (ej: 4+5=7).
\item Versión actualizando por separado los elementos de la operación, con un único recurso ``/operacion''. PUT podrá llevar en el cuerpo ``Primero: 4'' o ``Segundo: 5'', o ``Op: +''. Cada uno de ellos actualiza el elemento correspondiente de la operación. GET de ese recurso, devuelve el resultado de la operación con los elementos que tiene en este momento.
\item Versión actualizando por separado los elementos de la operación, con un único recurso ``/operación''. PUT podrá llevar en el cuerpo un número si es la primera o segunda vez que se invoca, un símbolo de operación si es la tercera. GET dará el resultado si se han especificado todos los elementos de la operación, error si no. Es ``menos REST'', en el sentido que guarda más estado en el lado del servidor. Pero cumple los requisitos generales de REST si consideramos que le cliente es responsable de mantener su estado y saber en qué fase de la operación está en cada momento.
\item Versión donde cada elemento se envía con un PUT a un recurso (``/operacion/primeroperando'', ``/operacion/segundooperando'', ``/operacion/signo''), y el resultado se obtiene con ``GET /operacion/resultado''. No es REST, porque el estado del recurso ``/operacion/resultado'' depende del estado de los otros recursos
.
\end{itemize}

También podemos extender el diseño a versiones con varios usuarios:

\begin{itemize}
\item Podría tenerse un identificador para cada operación. ``POST /operaciones'' podría devolver el enlace a una nueva operación creada, como ``/operaciones/2af434ad3''. Cada una de estas sumas se comportaría como las ``sumas con un usuario'' que se han comentado antes. ``DELETE  /operaciones/2af434ad3'' destruiría una operación.
\end{itemize}

\textbf{Material:}

\begin{itemize}
\item \texttt{simplecalc.py}: Programa con una posible solución a este ejercicio. Proporciona cuatro recursos ``calculadora'', uno para cada operación matemática (suma, resta, multiplicación, división). Cada calculadora mantiene un estado (operación matemática) que se actualiza con PUT y se consulta con GET.
\item Vídeo que muestra el funcionamiento de \texttt{simplecalc.py} \\
  \url{http://vimeo.com/31427714}
\item Vídeo que describe el programa \texttt{simplecalc.py} \\
  \url{http://vimeo.com/31430208}
\item \texttt{multicalc.py}: Programa con otra posible solución a este ejercicio. Proporciona un recurso para crear calculadoras (mediante POST). Al crear una calculadora se especifica de qué tipo (operación) es. Cada calculadora mantiene un estado (operación matemática) que se actualiza con PUT y se consulta con GET. Se apoya en las clases definidas en \texttt{simplecalc.py} para implementar las calculadoras.
\item \texttt{webappmulti.py}: Clase que proporciona la estructura básica para los dos programas anteriores (clase raíz de servicio web, de \emph{aplis}, etc.)
\end{itemize}


%%-----------------------------------------------------------------------------
%%-----------------------------------------------------------------------------
\subsection{Cache de contenidos}
\label{subsec:cache-contenidos}

\textbf{Enunciado:}

Vamos a construir una aplicación web que no sólo recibe peticiones de un cliente, sino que también hace peticiones a otros servicios web. El ejercicio consiste en construir una aplicación que, dada una URL (sin ``http://'') como nombre de recurso, devuelve el contenido de la página correspondiente a esa URL. Esto es, si se le pide http://localhost:1234/gsyc.es/ devuelve el contenido de la página http://gsyc.es/. Además, lo guarda en un diccionario, de forma que si se le vuelve a pedir, lo devuelve directamente de ese diccionario.

%Puede usarse como base ``Django cms'' (ejercicio~\ref{subsec:django-cms}), y si se quiere, el módulo estándar de Python ``urllib''.
Puede usarse como base ContentApp, y si se quiere, el módulo estándar de Python urllib.

\textbf{Comentarios:}

Pueden discutirse muchos detalles de esta aplicación. Por ejemplo, cómo gestionar las cabeceras, y en particular las cookies. También, cómo saber si la página ha cambiado en el sitio original antes de decidir volver a bajarla (cabeceras relacionadas con ``cacheable'', peticiones ``HEAD'' para ver fechas, etc.)

Para la implementación de la aplicación sólo se pide lo más básico: no hay tratamiento de cabeceras, y no se vuelve a bajar el original una vez está en la cache.

%%-----------------------------------------------------------------------------
%%-----------------------------------------------------------------------------
\subsection{Cache de contenidos versión Django}
\label{subsec:cache-contenidos-django}

\textbf{Enunciado:}

Construir una aplicación que implemente una cache de contenidos, como la descrita en el ejercicio~\ref{subsec:cache-contenidos}, pero sobre Django.
Puede usarse como base ``Django cms'' (ejercicio~\ref{subsec:django-cms}), y si se quiere, el módulo estándar de Python ``urllib''.

%%-----------------------------------------------------------------------------
%%-----------------------------------------------------------------------------
\subsection{Cache de contenidos anotado}
\label{subsec:cache-contenidos-anotado}

\textbf{Enunciado:}

Construir una aplicación como ``Cache de contenidos'' (ejercicio~\ref{subsec:cache-contenidos}), pero que anote cada página, en la primera línea, con un enlace a la página original, y que incluya también un enlace para ``recargar'' la página (volverla a refrescar a partir del original), otro enlace para ver el HTTP (de ida y de vuelta, si fuera posible) que se intercambió para conseguir la página original, y otro enlace para ver el HTTP de la consulta del navegador y de la respuesta del servidor al pedir esta página (de nuevo si fuera posible).

\textbf{Comentarios:}

Téngase en cuenta que por lo tanto cada página que sirva la aplicación, además de los contenidos  HTML correspondientes (obtenidos de la cache o directamente de Internet) tendrá cuatro enlaces en la primera línea:

\begin{itemize}
\item Enlace a la página original.
\item Enlace a un recurso de la aplicación que permita recargar.
\item Enlace a un recurso de la aplicación que permita ver el HTTP que se intercambió con el servidor que tenía la página.
\item Enlace a un recurso de la aplicación que permita ver el HTTP que se intercambió cuando se cargó en cache esa página.
\end{itemize}

Estos enlaces conviene introducirlos en el cuerpo de la página HTML que se va a servir. Así, por ejemplo, si la página que se bajó de Internet es como sigue:

\begin{verbatim}
<html>
  <head> ... </head>
  <body>
    Text of the page
  </body>
</html>
\end{verbatim}

Debería servirse anotada como sigue:

\begin{verbatim}
<html>
  <head> ... </head>
  <body>
    <a href="original_url">Original webpage</a>
    <a href="/recurso1">Reload</a>
    <a href="/recurso2">Server-side HTTP</a>
    <a href="/recurso3">Client-side HTTP</a></br>
    Text of the page
  </body>
</html>
\end{verbatim}

Para poder hacer esto, es necesario localizar el elemento $<body>$ en la página HTML que se está anotando. Hay que tener en cuenta que este elemento puede venir tal cual o con atributos, por ejemplo:

\begin{verbatim}
<body class="all" id="main">
\end{verbatim}

Por eso no basta con identificar dónde está la cadena ``$<body>$'' en la página, sino que habrá que identificar primero dónde está ``$<body$'' y, a partir de ahí, el cierre del elemento, ``$>$''. Será justo después de ese punto donde deberán colocarse las anotaciones. Para encontrar este punto puede usarse el método \texttt{find} de las variables de tipo \emph{string}, o expresiones regulares.

Para que los enlaces que se enlazan desde estas anotaciones funcionen, la aplicación tendrá que atender a tres nuevos recursos para cada página:

\begin{itemize}
\item /recurso1: Recarga de la página en la cache.
\item /recurso2: Devuelve el HTTP con el servidor (que tendrá que estar previamente almacenado en, por ejemplo, un diccionario).
\item /recurso3: Devuelve el HTTP con el navegador (que tendrá que estar previamente almacenado en, por ejemplo, un diccionario).
\end{itemize}

Naturalmente, cada página necesitará estos tres recursos, por lo tanto lo mejor será diseñar tres espacios de nombres donde estén los recursos correspondientes para cada una de las páginas. Por ejemplo, todos los recursos de recarga podrían comenzar por ``/reload/'', de forma que ``/reload/gsyc.es'' sería el recurso para recargar la página ``http://gsyc.es''.

Para poder almacenar el HTTP con el servidor, es importante darse cuenta de que el que se envía al servidor lo produce la propia aplicación. Si se usa \texttt{urllib}, no es posible acceder directamente a lo que se está enviando, pero se puede inferir a partir de lo que se indica a \texttt{urllib}. Por lo tanto, cualquier petición HTTP ``razonable'' para los parámetros dados será suficiente, aunque no sea exactamente lo que envíe \texttt{urllib}.

El HTTP que se recibe del servidor habrá que obtenerlo usando \texttt{urllib}, en la medida de lo posible.

Para poder almacenar el HTTP con el cliente, es importante darse cuenta de que el que se envía al navegador lo produce la propia aplicación, por lo que basta con almacenarlo antes de enviarlo. El que se recibe del navegador habrá que obtenerlo de la petición recibida.

%%-----------------------------------------------------------------------------
\subsection{Gestor de contenidos miltilingüe versión REST}
\label{subsec:contentappmulti}

\textbf{Enunciado:}

Diseño y construcción de ``Gestor de contenidos miltilingüe versión REST''. Retomamos la aplicación ContentApp, pero ahora vamos a proporcionarle una interfaz multilingüe simple. Para empezar, trabajaremos con español (``es'') e inglés (``en''). Siguiendo la filosofía REST, cada recurso lo vamos a tener ahora disponible en dos URLs distintas, según en qué idioma esté. Los recursos en español empezarán por ``/es/'', y los recursos en inglés por ``/en/''. Además, si a un recurso no se le especifica de esta forma en qué idioma está, se servirá en el idioma por defecto (si está disponible), o en el otro idioma (si no está en el idioma por defecto, pero sí en el otro). Como siempre, los recursos que no estén disponibles en ningún idioma producirán un error ``Resource not available''.

\textbf{Comentarios:}

Para construir esta aplicación puedes usar dos diccionarios de contenidos (uno para cada idioma), o quizás mejor un diccionario de diccionarios, donde para cada recurso tengas como dato un diccionario con los idiomas en que está disponible, que tienen a su vez como dato la página HTML a servir.


%%-----------------------------------------------------------------------------
\subsection{Sistema de transferencias bancarias}
\label{subsec:transferencias-bancarias}

\textbf{Enunciado:}

Diseñar un sistema RESTful sobre HTTP fiable para realizar una transferencia bancaria vía HTTP.

\begin{itemize}
\item Debe poder confirmarse que la transferencia ha sido realizada.
\item Debe poder prevenirse que la transferencia se haga más de una vez.
\item Datos de la transferencia: cuenta origen, cuenta destino, cantidad.
\item También debe poder consultarse el saldo de la cuenta (datos: cuenta)
\item En un segundo escenario, puede suponerse todo lo anterior, pero considerando que hay una contraseña que protege el acceso a operaciones sobre una cuenta data (una contraseña por cuenta), tanto transferencias como consultas de saldo
.
\end{itemize}

Indica el esquema de recursos (URLs) que ofrecerá la aplicación, y los verbos (comandos) HTTP que aceptará para cada uno, y con qué semántica.

%%-----------------------------------------------------------------------------
\subsection{Gestor de contenidos multilingüe preferencias del navegador}
\label{subsec:contentappmulti-navegador}

\textbf{Enunciado:}

Diseñar y construir la aplicación web ``Gestor de contenidos multilingüe preferencias del navegador''. En la aplicación ``Gestor de contenidos multilingüe versión REST'' (ejercicio~\ref{subsec:contentappmulti}) se especificaban como parte del nombre e recurso el idioma en que se quiere recibir un recurso. Pero el navegador tiene habitualmente una forma de especificar en qué idioma quieres recibir las páginas cuando están disponibles en varios. Para ver cómo funciona esto, prueba a cambiar tus preferencias idiomáticas en Firefox, y consulta la página \url{http://debian.org}.

Implementa una aplicación web que sea como la anterior, pero que además, haga caso de las preferencias del navegador con que la invoca, al menos para el caso de los idiomas ``es'' y ``en''.

\textbf{Material complementario:}
\begin{itemize}
\item Descripción de ``Accept-Language'' en la especificación de HTTP (RFC 2616) \\
  \url{http://www.w3.org/Protocols/rfc2616/rfc2616-sec14.html#sec14.4}
\end{itemize}

\textbf{Comentario:}

¿Qué recibe el servidor para poder hacer la selección de idioma? Utiliza una de tus aplicaciones para ver lo que le llega al servidor. Verás que lo que utiliza el navegador para indicar las preferencias idiomáticas del usuario es la cabecera ``Accept-Language''

%%-----------------------------------------------------------------------------
\subsection{Gestor de contenidos multilingüe con elección en la aplicación}
\label{subsec:contentappmulti-apli}

\textbf{Enunciado:}

Diseñar y construir la aplicación web ``Gestor de contenidos multilingüe con elección en la aplicación''. Ahora vamos a construir un servidor de contenidos multilingüe que además de los dos mecanismos anteriores (interfaz REST y preferencias del navegador, ejercicios~\ref{subsec:contentappmulti} y~\ref{subsec:contentappmulti-navegador}) permita que el usuario elija el idioma específicamente en la propia aplicación.

Para ello, el gestor de contenidos atenderá a peticiones GET sobre recursos de la forma ``/language/es'' (para indicar que se quieren recibir las páginas en español), ``/language/en'' (para indicar que se quieren recibir las páginas en inglés) o ``language/browser'' (para indicar que se quieren recibir las páginas en el idioma que indique las preferencias del navegador).

El mecanismo de especificación de idioma mediante nombre de recurso (``/en'' o /es'') tendrá precedencia sobre el mecanismo de especificación en la aplicación, y éste sobre el de preferencias del navegador.

Cada página incluirá, además del contenido en el idioma especificado, una lista (con enlaces) de los idiomas en que está disponible esa página, y una lista (con enlaces) de los idiomas que se pueden elegir en la aplicación. Por ejemplo, si estamos consultando una página en español que está disponible también en inglés, veremos un enlace ``This page in English'' que apuntará a la URL REST de esa página en inglés. Además, habrá enlaces a ``Ver páginas preferentemente en español'' (que apuntará al recurso /language/es), ``See pages preferently in English'' (que apuntará al recurso /language/en) y ``Ver páginas según preferencias del navegador'' (que apuntará a /language/browser).

\textbf{Comentario:}

Para implementar la elección especificándolo en la propia aplicación se podrán usar cookies, aunque no haya sistema de cuentas en la aplicación, como es el caso.

%%-----------------------------------------------------------------------------
\subsection{Sistema REST para calcular Pi}
\label{subsec:rest-pi}

\textbf{Enunciado:}

Diseñar un sistema RESTful sobre HTTP que permita calcular el número pi como una operación asíncrona.

\begin{itemize}
\item El usuario solicita el comienzo del cálculo indicando el número de decimales deseado
\item El usuario debe poder consultar a partir de ese momento el estado del cálculo
\end{itemize}

Indica el esquema de recursos (URLs) que ofrecerá la aplicación, y los verbos (comandos) HTTP que aceptará para cada uno, y con qué semántica.

Háganse dos versiones: en la primera, se supone que hay un sólo usuario (navegador) del sistema. En la segunda, puede haber varios, pero no simultáneamente: si un usuario solicita el comienzo del cálculo mientras hay otro cálculo en curso, le devuelve un mensaje de error.

\textbf{Comentario:}

Quien esté interesado puede realizar una implementación de una aplicación web para este diseño. Puede usar, por ejemplo, el método Monte Carlo, aplicando incrementalmente números cada vez más altos de números aleatorios.

\textbf{Materiales:}

Explicación del cálculo de Pi mediante el método Monte Carlo, incluyendo ejemplo en Python: \\
\url{http://www.eveandersson.com/pi/monte-carlo-circle}


%%-----------------------------------------------------------------------------
%%-----------------------------------------------------------------------------
%%-----------------------------------------------------------------------------
\section{Ejercicios 03: Introducción a XML}


%%-----------------------------------------------------------------------------
%%-----------------------------------------------------------------------------
\subsection{Chistes XML}
\label{subsec:xml-chistes}

\textbf{Enunciado:}

Estudia y modifica el programa xml-parser-jokes.py (que funciona con el fichero jokes.xml), hasta que entiendas los rudimentos del manejo de reconocedores SAX con Python.

\textbf{Material:}

\begin{itemize}
  \item jokes.xml. Fichero XML con descripciones de chistes.
  \item xml-parser-jokes.py. Programa que lee el fichero anterior, y usando un parser SAX lo reconoce y muestra en pantalla el contenido de los chistes.
\end{itemize}

%%-----------------------------------------------------------------------------
%%-----------------------------------------------------------------------------
\subsection{Modificación del contenido de una página HTML}
\label{subsec:xml-modificacion-html}

\textbf{Enunciado:}

Estudia y modifica el documento HTML dom.html, de forma que:

\begin{itemize}
\item Al pulsar con el ratón sobre un texto, se recargue la página (invocando para ello una función JavaScript). Este texto ha de estar disponible para poder pulsar sobre él una vez la página haya cambiado de contenido.
\item Al pulsar con el ratón sobre un botón, se modificará alguna parte del contenido mostrando la hora y fecha del momento.
\end{itemize}

\textbf{Material:}

\begin{itemize}
  \item dom.html. Documento HTML, que incluye algo de código JavaScript, y que hay que modificar.
\end{itemize}

%%-----------------------------------------------------------------------------
%%-----------------------------------------------------------------------------
\subsection{Titulares de BarraPunto}
\label{subsec:xml-barrapunto}

\textbf{Enunciado:}

Descargar el fichero RSS de BarraPunto\footnote{\url{http://barrapunto.com}}, y construir un programa que produzca como salida sus titulares en una página HTML. Si se carga esa página en un navegador,  picando sobre un titular, el navegador deberá cargar la página de BarraPunto con la noticia correspondiente. Como base puede usarse lo aprendido estudiando los programas xml-parser-jokes.py y xml-parser-barrapunto.py.

\textbf{Material:}

\begin{itemize}
\item \url{http://barrapunto.com/index.rss}: URL del fichero RSS de BarraPunto.
\item xml-parser-barrapunto.py: Programa que muestre en pantalla los titulares y las URLs que se describen en el fichero barrapunto.rss.
\item barrapunto.rss: Fichero con el contenido del canal RSS de BarraPunto en un momento dado.
\end{itemize}

Repositorio de entrega en GitLab: \\
\url{https://gitlab.etsit.urjc.es/cursosweb/practicas/server/xml-barrapunto}


%%-----------------------------------------------------------------------------
%%-----------------------------------------------------------------------------
\subsection{Videos en canal de YouTube}
\label{subsec:xml-youtube}

\textbf{Enunciado:}

Descargar el fichero RSS con los videos del canal CursosWeb de Youtube\footnote{\url{https://www.youtube.com/channel/UC300utwSVAYOoRLEqmsprfg}}, y construir un programa que produzca como salida sus títulos en una página HTML. Si se carga esa página en un navegador,  picando sobre un titular, el navegador deberá cargar la página de Youtube con el video correspondiente. Como base puede usarse lo aprendido estudiando el programa \verb|xml-parser-jokes.py|.

\textbf{Ejemplo de ejecución:}

Al ser ejecutado el programa, producirá la página HTML descrita anteriormente. Por lo tanto, podemos redirigir la salida estándar el programa a un fichero, que podrá ser visualizado mediante un navegador:

\begin{verbatim}
programa > pagina.html
\end{verbatim}

\textbf{Material:}

\begin{itemize}
\item \url{https://www.youtube.com/feeds/videos.xml?channel_id=UC300utwSVAYOoRLEqmsprfg}: URL del documento RSS del canal CursosWeb de Youtube.
\item xml-parser-barrapunto.py: Programa que muestre en pantalla los titulares y las URLs que se describen en el fichero barrapunto.rss.
\item barrapunto.rss: Fichero con el contenido del canal RSS de BarraPunto en un momento dado.
\end{itemize}

%%-----------------------------------------------------------------------------
%%-----------------------------------------------------------------------------
\subsection{Videos en canal de YouTube (con descarga)}
\label{subsec:xml-youtube-descarga}

\textbf{Enunciado:}

Realizar un programa con la funcionalidad descrita en ``Videos en canal de Youtube'' (ejercicio~\ref{subsec:xml-youtube}), pero realizanda la descarga del canal desde YouTube. El programa admitirá un único argumento, que será un identificador de canal de YouTube, y producirá como salida estándar la página HTML que producía el ejercicio mencionado anteriormente.

\textbf{Ejemplo de ejecución:}

Al ser ejecutado el programa, producirá la página HTML descrita anteriormente. Por lo tanto, podemos redirigir la salida estándar el programa a un fichero, que podrá ser visualizado mediante un navegador:

\begin{verbatim}
programa UC300utwSVAYOoRLEqmsprfg > pagina.html
\end{verbatim}

\textbf{Material:}

\begin{itemize}
\item Módulo urllib de Python: \\
  \url{https://docs.python.org/3/library/urllib.html}
\item Tutorial sobre urllib de Python: \\
  \url{https://pythonspot.com/urllib-tutorial-python-3/}
\end{itemize}

%%-----------------------------------------------------------------------------
%%-----------------------------------------------------------------------------
\subsection{Gestor de contenidos con titulares de BarraPunto}
\label{subsec:contentapp-barrapunto}

\textbf{Enunciado:}

Partiendo de contentApp (``Gestor de contenidos'', ejercicio~\ref{subsec:contentapp}), realiza contentAppBarraPunto. Esta versión devolverá, para cada recurso para el cuál tenga un contenido asociado en el diccionario de contenidos, una página que incluirá el contenido en cuestión, y los titulares de BarraPunto (para cada uno, título y URL).

Para ello, podéis hacer por un lado una aplicación que sirva para bajar el canal RSS de la portada de BarraPunto, y lo almacene en un objeto persistente (usando, por ejemplo, Shelve). Por otro lado, contentBarraPuntoApp leerá, antes de devolver una página, ese objeto, y utilizará sus datos para componer esa página a devolver. 

%%-----------------------------------------------------------------------------
%%-----------------------------------------------------------------------------
\subsection{Gestor de contenidos con titulares de BarraPunto versión SQL}
\label{subsec:contentapp-barrapunto-sql}

\textbf{Enunciado:}

Realiza contentDBAppBarraPuntoSQL, con la misma funcionalidad que contentAppBarraPunto (ejercicio~\ref{subsec:contentapp-barrapunto}), pero usando una base de datos SQLite en lugar de un diccionario persistente gestionado con Shelve.

%%-----------------------------------------------------------------------------
%%-----------------------------------------------------------------------------
\subsection{Gestor de contenidos con titulares de BarraPunto versión Django}
\label{subsec:django-cms-barrapunto}

\textbf{Enunciado:}

%% Realiza una aplicación Django con la misma funcionalidad que contentDBAppBarraPuntSQL (ejercicio~\ref{subsec:contentapp-barrapunto-sql}), pero usando el entorno de desarrollo Django.

Realiza una aplicación Django con la misma funcionalidad que ``Django cms'' (ejercicio~\ref{subsec:django-cms}), pero que devuelva para cada recurso para el cuál tenga un contenido asociado en su tabla de la base de datos una página que incluirá el contenido en cuestión, y los titulares de BarraPunto (para cada uno, título y URL).

%% Para ello, podéis hacer por un lado una aplicación que sirva para bajar el canal RSS de la portada de BarraPunto, y lo almacene en un objeto persistente (usando, por ejemplo, Shelve). Por otro lado, contentBarraPuntoApp leerá, antes de devolver una página, ese objeto, y utilizará sus datos para componer esa página a devolver. 

Para reutilizar código, puedes partir de ``Django cms'' (ejercicio~\ref{subsec:django-cms}) o  ``Django cms\_put'' (ejercicio~\ref{subsec:django-cms-put}).

En particular, puedes implementar la consulta a BarraPunto de una de las siguientes formas:

\begin{itemize}
\item Cada vez que se pida un recurso, se mostrará el contenido asociado a él, anotado con los titulares de BarraPunto, que se descargarán (vía canal RSS) en ese mismo momento.
\item Habrá un recurso especial, ``/update'', que se usará para actualizar una tabla con los contenidos de BarraPunto. Cuando se invoque este recurso, se bajarán los titulares (vía canal RSS) de BarraPunto, y se almacenarán en una tabla en la base de datos que mantiene Django. Cada vez que se pida cualquier otro recurso, se mostrará el contenido asociado a él, anotado con los titulares de BarraPunto, que se extraerán de esa tabla, sin volver a pedirlos a BarraPunto.
\end{itemize}

Basta con mostrar por ejemplo los últimos tres o cinco titulares de BarraPunto (cada uno como un enlace a la URL correspondiente).

Repositorio de entrega en  GitLub: \\
\url{https://gitlab.etsit.urjc.es/cursosweb/practicas/server/django-cms-barrapunto}

%%-----------------------------------------------------------------------------
%%-----------------------------------------------------------------------------
\subsection{Gestor de contenidos con videos de YouTube (simple)}
\label{subsec:django-cms-youtube}

\textbf{Enunciado:}

Realiza una aplicación Django que utilice parte de la funcionalidad de ``Django cms'' (ejercicio~\ref{subsec:django-cms}) para construir un archivador de videos de un canal Youtube. Para ello, en el recurso ``/'', la aplicación ofrecerá, como documento HTML, dos listados:

\begin{itemize}
\item Listado de videos seleccionados
\item Listado de videos seleccionables
\end{itemize}

Cualquier otro recurso devolverá una página de error.

El listado de videos seleccionables incluirá un listado de todos los videos del canal, junto con un botón ``Seleccionar'' a su lado. Si se pulsa el botón de ``Seleccionar'' para un video, se añadirá éste al listado de videos seleccionados.

El listado de videos seleccionados, que estará vacío inicialmente, incluirá los videos que hayan sido seleccionados, según se indica anteriormente. Junto a cada video, habrá un botón ``Eliminar'', que quitará el video del listado de videos seleccionados.

Un video dado aparecerá sólo en uno de los dos listados.

Para cada video, aparecerá su título, y un enlace al video en cuestión, tanto en el listado de seleccionables como de seleccionados.

La aplicación funcionará cargando el listado del canal cuando arranque, a partir del listado XML de ese canal. Puede usarse el canal ``CursosWeb'' como canal con el que funcionará la aplicación:

\begin{itemize}
\item HTML: \\
  \url{https://www.youtube.com/channel/UC300utwSVAYOoRLEqmsprfg}
\item XML: \\
  \url{https://www.youtube.com/feeds/videos.xml?channel_id=UC300utwSVAYOoRLEqmsprfg}
\end{itemize}

\textbf{Comentarios:}

La descarga del documento XML con los contenidos del canal debería hacerse una vez. Esto debería hacerse en el momento en que la aplicación arranca. Una solución elegante para realizarlo sería utilizar el enganche (hook) \verb|AppConfig.ready|.

Para usarlo, habría que escribir código siguiendo el siguiente esquema. En la app que esté implementando la solución al ejercicio (llamémosla \verb|myapp|) escribiríamos un fichero \verb|myapp/apps.py| con código de este estilo:

\begin{verbatim}
from django.apps import AppConfig
class MyAppConfig(AppConfig):
    name = 'myapp'
    def ready(self):
        url = 'https://www.youtube.com/feeds/videos.xml?channel_id=' \
            + sys.argv[1]
        xmlStream = urllib.request.urlopen(url)
\end{verbatim}

Y luego, para que este código se ejecture, en el fichero \verb|myapp/__init__.py|:

\begin{verbatim}
default_app_config = 'myapp.apps.MyAppConfig'
\end{verbatim}

Documentación sobre este enganche: \\
\url{https://docs.djangoproject.com/en/stable/ref/applications/#django.apps.AppConfig.ready}

%%-----------------------------------------------------------------------------
%%-----------------------------------------------------------------------------
\subsection{Gestor de contenidos con videos de YouTube (2)}
\label{subsec:django-cms-youtube-2}

\textbf{Enunciado:}

Vamos a extender la aplicación desarrollada en el ejercicio ``Gestor de contenidos con videos de Youtube'' (ejercicio~\ref{subsec:django-cms-youtube-2}), con la funcionalidad que se detalla a continuación.

Además del recurso ``/'', esta aplicación servirá recursos de la forma ``/[id]'', siendo ``[id]'' el identificador de un video seleccionado en la página principal (y sólo si no ha sido eliminado). Como identificador utilizaremos el que aparece en el elemento ``yt:videoId'' del documento XML que describe un canal. En cada uno de estos recursos se servirá una página HMTL con el siguiente contenido:

\begin{itemize}
\item Enlace a la página principal de la aplicación
\item Título del video (que será un enlace a la url del video)
\item Imagen del video (obtenida del elemento ``media:thumbnail'' del documento XML que describe el canal (que será un enlace a la url del video)
\item Nombre del canal (que será un enlace a la url del canal)
\item Fecha de publicación del video
\item Descripción del video
\end{itemize}

El recurso ``/'', en el listado de videos seleccionados, en lugar de incluir un enlace al video en Youtube ofrecerá un enlace a la página del video en la aplicación, tal y como se ha indicado anteriormente.

%%-----------------------------------------------------------------------------
%%-----------------------------------------------------------------------------
\subsection{Municipios JSON}
\label{subsec:json-municipios}

\textbf{Enunciado:}

Crea un programa Python que lea el fichero \verb|municipios.json|,
y muestra en pantalla el nombre de cada municipio y su id (campo ``url'').
El fichero \verb|municipios.json| contiene una lista de diccionarios,
uno por municipio, con varios campos.

\textbf{Materiales:}

Estos materiales pueden encontrarse en el
directorio \verb|Python-JSON| del repositorio de código de la
asignatura. 

\begin{itemize}
\item Fichero: \verb|municipios.json| \\
  Originalmente, este fichero fue recogido de: \\
  \url{https://opendata.aemet.es/opendata/api/maestro/municipios/?api_key=XXX}
\item Solución de referencia: \verb|json-municipios.py|
\end{itemize}

%%-----------------------------------------------------------------------------
%%-----------------------------------------------------------------------------
\subsection{Municipios JSON via HTTP}
\label{subsec:json-municipios-http}

\textbf{Enunciado:}

Igual que ``Municipios JSON'' (ejercicio~\ref{subsec:json-municipios}),
pero recogiendo el documento JSON de la red, vía HTTP.

\textbf{Materiales:}

Estos materiales pueden encontrarse en el
directorio \verb|Python-JSON| del repositorio de código de la
asignatura. 

\begin{itemize}
\item Documento JSON con los municipios: \\
  \url{https://raw.githubusercontent.com/CursosWeb/Code/master/Python-JSON/municipios.json}
\item Solución de referencia: \verb|json-municipios-http.py|
\end{itemize}


%%-----------------------------------------------------------------------------
%%-----------------------------------------------------------------------------
%%-----------------------------------------------------------------------------
\section{Ejercicios 04: Hojas de estilo CSS}

%%-------------------------------------------------------------------------
%%-------------------------------------------------------------------------
\subsection{Django cms\_css simple}
\label{subsec:django-cms-css}

\textbf{Enunciado:}

Crea una hoja de estilo en la URL ``/css/main.css'' para manejar la apariencia de la página ``/about'' en ``Django cms\_put'' (ejercicio~\ref{subsec:django-cms-put}). La hoja tendrá el siguiente contenido:

\begin{verbatim}
body {
  margin: 10px 20% 50px 70px;
  font-family: sans-serif;
  color: black;
  background: white;
}
\end{verbatim}

La página ``/about'' tendrá el contenido que estimes conveniente. Ambos contenidos (el de ``/about'' y el de ``/css/main.css'') se subirán al gestor de contenidos mediante un PUT, igual que cualquier otro contenido.

Explica en el fichero `README.md` del repositorio de entrega cómo has solucionado la práctica.

Repositorio de entrega en GitLab: \\
\url{https://gitlab.etsit.urjc.es/cursosweb/practicas/server/django-cms-css}

%%-------------------------------------------------------------------------
%%-------------------------------------------------------------------------
\subsection{Django cms\_css elaborado}
\label{subsec:django-cms-css-2}

\textbf{Enunciado:}

Modifica tu solución para ``Django cms\_put'' (ejercicio~\ref{subsec:django-cms-put}) de forma que:

\begin{itemize}
\item Si el recurso está bajo ``/css/'', se almacene tal cual al recibirlo (mediante PUT) y se sirva tal cual (cuando se recibe un GET).
\item Si el recurso tiene cualquier otro nombre, se almacene de tal forma cuando se reciba (mediante PUT) que el contenido almacenado sea el cuerpo (lo que va en el elemento $<BODY>$) de las páginas que se sirvan (cuando se reciba el GET correspondiente). Para servir las páginas utiliza una plantilla (\emph{template}) que incluya el uso de la hoja de estilo ``/css/main.css'' para manejar la apariencia de todas las páginas.
\end{itemize}

Repositorio de entrega en GitLab: \\
\url{https://gitlab.etsit.urjc.es/cursosweb/practicas/server/django-cms-css-2}

%%-----------------------------------------------------------------------------
%%-----------------------------------------------------------------------------
%%-----------------------------------------------------------------------------
\section{Ejercicios 05: AJAX}

Ejercicios con AJAX y tecnologías relacionadas.

%%-------------------------------------------------------------------------
%%-------------------------------------------------------------------------
\subsection{SPA Sentences generator}
\label{subsec:spa-sentences-generator}

\textbf{Enunciado:}

Prueba el fichero sentences\_generator.html, que incluye una aplicación SPA simple que genera frases de forma aleatoria, a partir de componentes de tres listas de fragmentos de frases. En particular, observa dónde se obtiene una referencia al nodo del árbol DOM donde se quiere colocar la frase, y cómo se manipula éste árbol para colocarla ahí, una vez está generada.

Una vez lo hayas entendido, modifícalo para que en lugar de usar tres fragmentos para cada frase, use cuatro, cogiendo cada uno, aleatoriamente, de una lista de fragmentos.

\textbf{Material:}
\begin{itemize}
\item sentences\_generator.html: Aplicación SPA que muestra frases componiendo fragmentos.
\end{itemize}

%%-------------------------------------------------------------------------
%%-------------------------------------------------------------------------
\subsection{Ajax Sentences generator}
\label{subsec:ajax-sentences-generator}

\textbf{Enunciado:}

Construye una aplicación con funcionalidad similar a ``SPA Sentences generator'' (ejercicio~\ref{subsec:spa-sentences-generator}), pero realizada mediante una aplicación AJAX que pide los datos a un servidor implementado en Django.

El servidor atenderá GET sobre los recursos /first, /second y /third, dando para cada uno de ellos la parte correspondiente (primera, segunda o tercera) de una frase, devolviendo un fragmento de texto aleatorio de una lista con fragmentos que tenga para cada uno de ellos (esto es, habrá una lista para los ``primeros'' fragmentos, otra para los segundos, y otra para los terceros).

La aplicación AJAX solicitará los tres fragmentos que necesita, y los compondrá mostrando la frase resultante, de forma similar a como lo hace la aplicación ``SPA Sentences generator''.

\textbf{Material:}

\begin{itemize}
\item words\_provider.tar.gz: Proyecto Django que sirve como servidor que proporciona fragmentos de frases para la aplicación AJAX anterior. Incluye apps/sentences\_generator.html, aplicación AJAX que muestra frases componiendo fragmentos que obtiene de un sitio web, utilizando llamadas HTTP síncronas, y apps/async\_sentences\_generator.html (similar, pero con llamadas asíncronas).
\end{itemize}

%%-------------------------------------------------------------------------
%%-------------------------------------------------------------------------
\subsection{Gadget de Google}
\label{subsec:gadget-google}

\textbf{Enunciado:}

Inclusión de un gadget de Google, adecuadamente configurado, en una página HTML estática.

\textbf{Referencias:}

\url{http://www.google.com/ig/directory?synd=open}

%%-------------------------------------------------------------------------
%%-------------------------------------------------------------------------
\subsection{Gadget de Google en Django cms}
\label{subsec:gadget-google-cms}

\textbf{Enunciado:}

Crear una versión del gestor de contenidos Django (Django cms, ejercicio~\ref{subsec:django-cms}) con un gadget de Google en cada página (el mismo en todas ellas).

%%-------------------------------------------------------------------------
%%-------------------------------------------------------------------------
\subsection{EzWeb}
\label{subsec:ezweb}

\textbf{Enunciado:}

Abrir una cuenta en el sitio de EzWeb, y crear allí un nuevo espacio de trabajo donde se conecten algunos gadgets.

\textbf{Referencia:}

\url{http://ezweb.tid.es}

%%-------------------------------------------------------------------------
%%-------------------------------------------------------------------------
\subsection{EyeOS}
\label{subsec:eyeos}

\textbf{Enunciado:}

Abrir una cuenta en el sitio de EyeOS, y visitar el entorno que proporciona.

\textbf{Referencia:}

\url{http://www.eyeos.org/}

%%-----------------------------------------------------------------------------
%%-----------------------------------------------------------------------------
%%-----------------------------------------------------------------------------
\section{Ejercicios P1: Introducción a Python}

Estos ejercicios pretenden ayudar a conocer el lenguaje de programación Python. Los ejercicios suponen que previamente el alumno se ha documentado sobre el lenguaje, usando las referencias ofrecidas en clase, u otras equivalentes que pueda preferir.

Aunque es fácil encontrar soluciones a los ejercicios propuestos, se recomienda al alumno que realice por si mismo todos ellos.

El primer ejercicio has de hacerlo directamente en el intérprete de Python (invocándolo sin un  programa fuente como argumento). Para los demás, puedes usar un editor (Emacs, gedit, o el que quieras) )o un IDE (Eclipse con el módulo PyDev, o el que quieras).

%%-------------------------------------------------------------------------
%%-------------------------------------------------------------------------
\subsection{Uso interactivo del intérprete de Python}
\label{subsec:practicas-interprete}

\textbf{Enunciado:}

Invoca el intérprete de Python desde la shell. Crea las siguientes
variables:

\begin{itemize}
\item un entero
\item una cadena de caracteres con tu nombre
\item una lista con cinco nombres de persona
\item un diccionario de cuatro
entradas que utilice como llave el nombre de uno de tus amigos y como valor su
número de móvil
\end{itemize}

Comprueba con la sentencia \verb|print nombre_variable| que todo lo
que has hecho es correcto.

Fíjate en particular que la lista mantiene el
orden que has introducido, mientras el diccionario no lo hace. Prueba a mostrar 
los distintos elementos de la lista y del diccionario con \verb|print|.

%%------------------------------------------------------------------------
%%------------------------------------------------------------------------
\subsection{Haz un programa en Python}
\label{subsec:eje-python-primer-programa}

\textbf{Enunciado:}

Haz un programa en Python que haga cualquier cosa, y escriba algo en la salida estándar (en el terminal, cuando lo ejecutes normalmente).

%En tu respuesta a este ejercicio, explica brevemente qué hace, y súbelo como anexo a esa respuesta (opción "Agregar" cuando estés editando la respuesta).

%%------------------------------------------------------------------------
%%------------------------------------------------------------------------
\subsection{Tablas de multiplicar}
\label{subsec:eje-python-tablas}

\textbf{Enunciado:}

Utilizando bucles for, y funciones range(), escribe un programa que muestre en su salida estándar (pantalla) las tablas de multiplicar del 1 al 10, de la siguiente forma:

\begin{verbatim}
Tabla del 1
-----------
1 por 1 es 1
1 por 2 es 2
1 por 3 es 3
...
1 por 10 es 10
Tabla del 2
-----------
2 por 1 es 2
2 por 2 es 4
...
Tabla del 10
------------
...
10 por 10 es 100
\end{verbatim}

%%------------------------------------------------------------------------------------
%%------------------------------------------------------------------------------------
\subsection{Ficheros y listas}
\label{subsec:ficheros-listas}

\textbf{Enunciado:}

Crea un script en Python que abra el fichero \verb|/etc/passwd|, tome todas sus líneas en una lista de Python e imprima, para cada identificador de usuario, la shell que utiliza.

Imprime también el número de usuarios que hay en esta máquina. Utiliza para
ello un método asociado a la lista, no un contador de la iteración.

Puedes partir del siguiente repositorio: \verb|https://github.com/CursosWeb/X-Serv-Python-Ficheros-y-Listas|

%%------------------------------------------------------------------------------------
%%------------------------------------------------------------------------------------
\subsection{Ficheros, diccionarios y excepciones}
\label{subsec:ficheros-dic-excep}

\textbf{Enunciado:}

Modifica el script
anterior, de manera que en vez de imprimir para cada identificador de usuario el tipo
de shell que utiliza, lo introduzca en un diccionario. Una vez introducidos todos, imprime por pantalla los valores para el usuario 'root' y para el
usuario 'imaginario'. El segundo produce un error, porque no existe. ¿Sabrías evitarlo mediante el uso de
excepciones?

Puedes partir del siguiente repositorio: \verb|https://github.com/CursosWeb/X-Serv-Python-FichDicExp|

%%------------------------------------------------------------------------------------
%%------------------------------------------------------------------------------------
\subsection{Calculadora}
\label{subsec:calculadora}

\textbf{Enunciado:}

Crea un programa que sirva de calculadora y que incluya las funciones
básicas (sumar, restar, multiplicar y dividir). El programa ha de poder ejecutarse desde la línea de comandos de la siguiente manera: \texttt{python calculadora.py
función operando1 operando2}. No olvides capturar las excepciones.

Parte del repositorio en GitLab \verb|https://gitlab.etsit.urjc.es/cursosweb/x-serv-13.6-calculadora|

\newpage

%%------------------------------------------------------------------------------------
%%------------------------------------------------------------------------------------
%%------------------------------------------------------------------------------------
\section{Ejercicios P2: Aplicaciones web simples}

Estos ejercicios presentan al alumno unas pocas aplicaciones web que, aunque de funcionalidad mínima, van introduciendo algunos conceptos fundamentales.

%%-----------------------------------------------------------------------
%%-----------------------------------------------------------------------
\subsection{Aplicación web hola mundo}
\label{subsec:aplweb-hola-mundo}

\textbf{Enunciado:}

Construir una aplicación web, en Python, que muestre en el navegador ``Hola mundo'' cuando sea invocada. La aplicación usará únicamente la biblioteca socket. Construir la aplicación de la forma más simple posible, mientras proporcione correctamente la funcionalidad indicada.

\textbf{Motivación:}

Este ejercicio sirve para construir el primer ejemplo de aplicación web. Con ella se muestra ya la estructura típica genérica de una aplicación web: inicialización y bucle de atención a peticiones (a su vez dividido en recepción y análisis de petición, proceso y lógica y de aplicación, y respuesta). Todo está muy simplificado: no se hace análisis de la petición, porque se considera que todo vale, no se realiza proceso de la petición, porque siempre se hace lo mismo, y la respuesta es en realidad mínima.

Aunque no se usará mucho en la asignatura la biblioteca socket (pues trabajaremos a niveles de abstracción superiores), esta práctica sirve para ayudar a entender los detalles que normalmente oculta un marco de desarrollo de aplicaciones web.

La práctica también sirve para introducir el esquema típico de prueba (carga de la página principal de la aplicación con un navegador, colocación en un puerto TCP de usuario, etc.).

\textbf{Material:}

Se ofrecen dos soluciones en \verb|https://gitlab.etsit.urjc.es/grex/x-serv-14.1-webserver|. La más simple es \verb|servidor-http-simple.py|. La otra, \verb|servidor-http-simple-2.py|, permite conexiones desde fuera de la máquina huésped, y es capaz de reusar el puerto de forma que se puede rearrancar en cuanto muere.


%%---------------------------------------------------------------------
%%---------------------------------------------------------------------
\subsection{Variaciones de la aplicación web hola mundo}
\label{subsec:aplweb-hola-mundo-var}

\textbf{Enunciado:}

Basándose en la aplicación ``Hola mundo'' construida para el ejercicio~\ref{subsec:aplweb-hola-mundo}, crear tres aplicaciones diferentes, con la siguiente funcionalidad cada una:

\begin{itemize}
\item Aplicación web que devuelva siempre la misma página HTML, que tendrá que tener al menos una imagen (usando un elemento IMG).

\item Aplicación web que devuelva un código de error 404 y muestre un mensaje en el navegador.

\item Aplicación web que produzca una redirección a la página \url{http://gsyc.es/}
\end{itemize}

\textbf{Material:}

%Soluciones de referencia:

%\begin{itemize}
%\item \verb|02-aplicaciones-web-simples/servidor-http-simple-img.py|
%\item \verb|02-aplicaciones-web-simples/servidor-http-simple-404.py|
%\item \verb|02-aplicaciones-web-simples/servidor-http-simple-301.py|
%\end{itemize}

%%------------------------------------------------------------------------------------
%%------------------------------------------------------------------------------------
\subsection{Aplicación web generadora de URLs aleatorias}
\label{subsec:aplweb-urls-aleatorias}

\textbf{Enunciado:}

Construcción de una aplicación web que devuelva URLs aleatorias. Cada vez que os conectéis al servidor, debe aparecer en el navegador ``Hola. Dame otra'', donde ``Dame otra'' es un enlace a una URL aleatoria bajo \verb|localhost:1234| (esto es, por ejemplo, \url{http://localhost:1234/324324234}). Esa URL ha de ser distinta cada vez que un navegador se conecte a la aplicación.

Parte para ello (i.e., haz un \emph{fork}) del siguiente repositorio: \url{https://gitlab.etsit.urjc.es/grex/x-serv-14.3-urlsaleatorias}

\textbf{Motivación:}

Explorar una aplicación web como extensión muy simple de ``Aplicación web hola mundo''.

%%-----------------------------------------------------------------------------
\subsection{Aplicación redirectora}
\label{subsec:aplweb-redirectora}

\textbf{Enunciado:}

Construir un programa en Python que sirva cualquier invocación que se le realice con una redirección (códigos de resultado HTTP en el rango 3xx) a otro recurso (aleatorio) de si mismo. 

\textbf{Comentarios:}

Este programa se realiza muy fácilmente a partir de la solución de ``Aplicación web generadora de URLs aleatorias'' (ejercicio~\ref{subsec:aplweb-urls-aleatorias}).

Para poder observar con más facilidad en el navegador lo que está ocurriendo, se puede hacer que la aplicación devuelva, en el cuerpo de la respuesta HTTP, un texto HTML indicando que se va a realizar una redirección, y a qué url va a realizarse. Para que este mensaje sea visible durante un tiempo razonable, se puede hacer que la aplicación, al recibir una petición, se quede ``parada'' durante unos segundos antes de contestar con la redirección.

\textbf{Motivación:}

Entender cómo funciona la redirección, y cómo reacciona un navegador ante ella.

%%------------------------------------------------------------------------------------
%%------------------------------------------------------------------------------------
\subsection{Sumador simple}
\label{subsec:sumador-simple}

\textbf{Enunciado:}

{\bf a) En una fase:} Construye una aplicación web que suma en una fases. Invocamos una URL del tipo \url{http://localhost:1234/sumar/1/2}, aportando la operación, el primer y el segundo operando. La aplicación nos devuelve el resultado de la suma.

{\bf b) En dos fases: } Construye una aplicación web que suma en dos fases. En la primera, invocamos una URL del tipo \url{http://localhost:1234/5}, aportando el primer sumando (el número que aparece como nombre de recurso). En la segunda, invocamos una URL similar, proporcionando el segundo sumando. La aplicación nos devuelve el resultado de la suma. En esta primera versión, suponemos que la aplicación es usada desde un solo navegador, y que las URLs siempre le llegan ``bien formadas''.

Repositorio de inicio: \url{https://gitlab.etsit.urjc.es/grex/x-serv-14.5-sumador-simple}

\textbf{Nota:}

Muchos navegadores, cuando se invoca con ellos una URL, lanzan un GET para ella, y a continuación uno o varios GET para el recurso \texttt{favicon.ico} en el mismo sitio. Por ello, hace falta tener en cuenta este caso para que funcione la aplicación web con ellos.

%%------------------------------------------------------------------------------------
%%------------------------------------------------------------------------------------
\subsection{Clase servidor de aplicaciones}
\label{subsec:clase-serv-aplis}

\textbf{Enunciado:}

Reescribe el programa ``Aplicación web hola mundo'' usando clases, y reutilizándolas, haz otro que devuelva ``Adiós mundo cruel'' en lugar de ``Hola mundo''. Para ello, define una clase \texttt{webApp} que sirva como clase raíz, que al especializar permitirá tener aplicaciones web que hagan distintas cosas (en nuestro caso, \texttt{holaApp} y \texttt{adiosApp}).

Esa clase \texttt{webApp} tendrá al menos:

\begin{itemize}
\item Un método \texttt{Analyze} (o \texttt{Parse}), que devolverá un objeto con lo que ha analizado de la petición recibida del navegador (en el caso más simple, el objeto tendrá un nombre de recurso)
\item Un método \texttt{Compute} (o \texttt{Process}), que recibirá como argumento el objeto con lo analizado por el método anterior, y devolverá una lista con el código resultante (por ejemplo, ``200 OK'') y la página HTML a devolver
\item Código para inicializar una instancia que incluya el bucle general de atención a clientes, y la gestión de sockets necesaria para que funcione.
\end{itemize}

Una vez la clase \texttt{webApp} esté definida, en otro módulo define la clase \texttt{holaApp}, hija de la anterior, que especializará los métodos Parse y Process como haga falta para implementar el ``Hola mundo''.

El código \verb|__main__| de ese módulo instanciará un objeto de clase \texttt{holaApp}, con lo que tendremos una aplicación ``Hola mundo'' funcionando.

Luego, haz lo mismo para \texttt{adiosApp}.

Conviene que en el módulo donde se defina la clase \texttt{webApp} se incluya también código para, en caso de ser llamado como programa principal, se cree un objeto de ese tipo, y se ejecute una aplicación web simple.

\textbf{Motivación:}

Explorar el sistema de clases de Python, y a la vez construir la estructura básica de una aplicación web con un esquema muy similar al que proporciona el módulo Python \texttt{SocketServer}.

%%------------------------------------------------------------------------------------
%%------------------------------------------------------------------------------------
\subsection{Clase servidor de aplicaciones, generador de URLs aleatorias}
\label{subsec:aplweb-clase-urls-aleatorias}

\textbf{Enunciado:}

Realiza el servidor especificado en el ejercicio ``Aplicación web generadora de URLs aleatorias'' (ejercicio~\ref{subsec:aplweb-urls-aleatorias}) utilizando el esquema de clases definido en el ejercicio ``Clase servidor de aplicaciones'' (ejercicio~\ref{subsec:clase-serv-aplis}).

Repositorio de inicio: \url{https://gitlab.etsit.urjc.es/grex/x-serv-14.7-servurlaleat}

%%------------------------------------------------------------------------------------
%%------------------------------------------------------------------------------------
\subsection{Clase servidor de aplicaciones, sumador}
\label{subsec:clase-sumador-simple}

\textbf{Enunciado:}

Realizar el servidor especificado en el ejercicio ``Sumador simple'' (ejercicio~\ref{subsec:sumador-simple}) utilizando el esquema de clases definido en el ejercicio ``Clase servidor de aplicaciones'' (ejercicio~\ref{subsec:clase-serv-aplis}).

Repositorio de inicio: \url{https://gitlab.etsit.urjc.es/grex/X-Serv-14.8-Servidor-Aplicaciones-Sumador}

%%----------------------------------------------------------------------------
%%----------------------------------------------------------------------------
\subsection{Clase servidor de varias aplicaciones}
\label{subsec:clase-serv-aplis-multi}

\textbf{Enunciado:}

Realizar una nueva clase, similar a la que se construyó en el ejercicio ``Clase servidor de aplicaciones'' (ejercicio~\ref{subsec:clase-serv-aplis}), pero preparada para servir varias aplicaciones (\emph{aplis}). Cada \emph{apli} se activará cuando se invoquen recursos que comiencen por un cierto prefijo.

Cada una de estas \emph{aplis} será a su vez una instancia de una clase con origen en una básica con los dos métodos ``parse'' y ``process'', con la misma funcionalidad que tenían en ``Clase servidor de aplicaciones''. Por lo tanto, para tener una cierta apli, se extenderá la jerarquía de clases para \emph{aplis} con una nueva clase, que redefinirá ``parse'' y ``process'' según la semántica de la apli.

Para especificar qué \emph{apli} se activará cuando llegue una invocación a un nombre de recurso, se creará un diccionario donde para cada prefijo se indicará la instancia de \emph{apli} a invocar. Este diccionario se pasará como parámetro al instanciar la clase que sirve varias aplicaciones.

Repositorio de inicio: \url{https://gitlab.etsit.urjc.es/grex/X-Serv-14.9-ServVariasApps}

%%----------------------------------------------------------------------------
%%----------------------------------------------------------------------------
\subsection{Clase servidor, cuatro aplis}
\label{subsec:clase-serv-aplis-varias}

\textbf{Enunciado:}

Utilizando la clase creada para ``Clase servidor de varias aplicaciones'' (ejercicio~\ref{subsec:clase-serv-aplis-multi}), crea una una aplicación web con varias aplis:

\begin{itemize}
\item Si se invocan recursos que comiencen por ``/hola'', se devuelve una página HTML en la que se vea el texto ``Hola''.
\item Si se invocan recursos que comiencen por ``/adios'', se devuelve una página HTML en la que se vea el texto ``Adiós''.
\item Si se invocan recursos que comiencen por ``/suma/'', se proporciona la funcionalidad de ``Sumador simple'' (ejercicio~\ref{subsec:sumador-simple}), esperando que los sumandos se incluyan justo a continuación de ``/suma/''.
\item Si se invocan recursos que comiencen por ``/aleat/'', se proporciona la funcionalidad de ``Aplicación web generadora de URLs aleatorias'' (ejercicio~\ref{subsec:aplweb-urls-aleatorias}).
\end{itemize}

Repositorio de inicio: \url{https://gitlab.etsit.urjc.es/grex/X-Serv-14.10-CuatroAplis}

%%----------------------------------------------------------------------------
%%----------------------------------------------------------------------------
\subsection{Herramientas de Web Developer}
\label{subsec:inst-web-developer}

\textbf{Enunciado:}

Introducción a las herramientas de \emph{Web Developer}, que ayudan en el desarrollo
y depuración de aplicaciones web en Firefox.

%%----------------------------------------------------------------------------
%%----------------------------------------------------------------------------
%%----------------------------------------------------------------------------
\section{Ejercicios P3: Introducción a Django}

%%----------------------------------------------------------------------------
%%----------------------------------------------------------------------------
\subsection{Instalación de Django}
\label{subsec:django-install}

\textbf{Enunciado:}

Instala la versión de Django que utilizaremos en prácticas.

\textbf{Comentarios:}

Utilizaremos la versión Django 2.1.7.

\textbf{Material:}

\begin{itemize}
\item Transparencias ``Introducción a Django''
\item Descarga de Django: \url{http://www.djangoproject.com/download/} \\
  (``Option 1: Get the latest official version'')
\end{itemize}


%%----------------------------------------------------------------------------
%%----------------------------------------------------------------------------
\subsection{Introducción a Django}
\label{subsec:django-intro}

\textbf{Enunciado:}

Realización de un proyecto Django de prueba (myproject), siguiendo el ejemplo de las transparencias ``Introducción a Django''. Creación de las tablas de su base de datos, con Django, y consulta de la base de datos creada con sqlitebrowser.

\textbf{Material:}

Se puede encontrar un ejemplo de solución del ejercicio ``Django intro'' en el directorio \verb|Python-Django/django-intro| del repositorio de codigo.

Además, se recomienda consultar:

\begin{itemize}
\item Django Getting Started: \\
  \url{https://docs.djangoproject.com/en/2.1/intro/}
\end{itemize}

%%----------------------------------------------------------------------------
%%----------------------------------------------------------------------------
\subsection{Django primera aplicación}
\label{subsec:django-primera}

\textbf{Enunciado:}

Realización de una aplicación Django que haga cualquier cosa, aún sin usar datos en almacenamiento estable. Por ejemplo, puede simplemente responder a ciertos recursos con páginas HTML definidas en el propio programa (en el correspondiente fichero \texttt{views.py}).

%%----------------------------------------------------------------------------
%%----------------------------------------------------------------------------
\subsection{Django calc}
\label{subsec:django-calc}

\textbf{Enunciado:}

Realiza una calculadora con Django. Esta calculadora responderá a URLs de la forma ``/suma/num1/num2'', ``/multi/num1/num2'', ``/resta/num1/num2'', ``/div/num1/num2'', realizando las operaciones correspondientes, y devolviendo error ``Not Found'' para las demás.

Parte del repositorio en GitLab: \verb|https://gitlab.etsit.urjc.es/cursosweb/x-serv-15.4-django-calc|. El proyecto Django se llamará \texttt{project} y la aplicación \texttt{calc}. 
Recuerda que sólo tendrás que modificar los siguientes ficheros: \texttt{urls.py} (modificando el del proyecto y creando el de la app) y \texttt{views.py}. 

%\textbf{Material:}
%
%calc.tar.gz: Ejemplo de solución del ejercicio ``Django calc''

%%----------------------------------------------------------------------------
%%----------------------------------------------------------------------------
\subsection{Django cms}
\label{subsec:django-cms}

\textbf{Enunciado:}

Realizar una sistema de gestión de contenidos muy simple con Django. Corresponderá con la funcionalidad de ``contentApp'' (ejercicio~\ref{subsec:contentapp}), almacenando los contenidos en una base de datos. La aplicación Django se ha de llamar \texttt{cms}.

El ejercicio ha de entregarse en el siguiente repositorio en GitLab: 
\url{https://gitlab.etsit.urjc.es/grex/x-serv-15.5-django-cms}. El repositorio contiene
un archivo check.py para comprobar que se han entregado todos los fichero necesarios (básicamente todos los ficheros con código Python del proyecto (\texttt{manage.py} y los contenidos en el directorio \texttt{myproject} y de la aplicación en \texttt{cms}, así como la base de datos en un fichero \texttt{db.sqlite3}), además de comprobar que el código en
\texttt{views.py} sigue con las reglas de estilo de Python (PEP8).

%\textbf{Material:}
%
%cms.tar.gz: Ejemplo de solución del ejercicio ``Django cms''


%%----------------------------------------------------------------------------
%%----------------------------------------------------------------------------
\subsection{Django cms\_put}
\label{subsec:django-cms-put}

\textbf{Enunciado:}

Realizar una sistema de gestión de contenidos muy simple con Django. Corresponderá con la funcionalidad de ``contentPutApp'' (ejercicio~\ref{subsec:contentputapp}), almacenando los contenidos en una base de datos. En otras palabras, será como ``Django cms'' (ejercicio~\ref{subsec:django-cms}), añadiendo la funcionalidad de que el usuario pueda poner contenidos mediante PUT, tal y como se explicó en el ejercicio de ``contentPutApp''. La aplicación Django se ha de llamar \texttt{cms\_put}.


\textbf{Comentario:}

Para realizar este ejercicio, consultar el manual de Django, donde explica cómo se comporta el objeto HTTPRequest, que es siempre primer argumento en los métodos que estamos definiendo en \texttt{views.py}. En particular, nos interesarán sus atributos ``method'' (que sirve para saber si nos está llegando un GET o un PUT) y ``body'', que nos da acceso a los datos (cuerpo) de la petición en bytes. A pesar de su nombre, este último atributo tiene esos datos tanto si la petición es un POST como si es un PUT. 

El ejercicio ha de entregarse en el siguiente repositorio en GitLab: 
\url{https://gitlab.etsit.urjc.es/cursosweb/x-serv-15.6-django-cms-put}. Has de subir el proyecto y la aplicación entera al repositorio.

%\textbf{Material:}
%
%cms\_put.tar.gz: Ejemplo de solución del ejercicio ``Django cms\_put''.

%%----------------------------------------------------------------------------
%%----------------------------------------------------------------------------
\subsection{Django cms\_users}
\label{subsec:django-users}

\textbf{Enunciado:}

Realizar un proyecto Django con la misma funcionalidad que ``Django cms\_put'', pero incluyendo un módulo de administración (lo que proporciona el ``Admin site'' de Django) y recursos para login y logout de usuarios. Además, cada página de contenidos (o cada mensaje indicando que una página no está disponible) deberá quedar anotada con la cadena ``Not logged in. Login'' (siendo ``Login'' un enlace al recurso de login) si no se está autenticado como usuario, o con la cadena ``Logged in as name. Logout'' (siendo ``name'' el nombre de usuario, y ``Logout'' un enlace al recurso de logout) si se está autenticado como usuario.

\textbf{Comentarios:}

\begin{itemize}
  \item Cada página tendrá, por tanto, la misma funcionalidad que cms\_put, pero además una línea en la parte superior que dependerá de si el usuario que la visita está registrado o no.

  \item Se puede ver cómo realizar la funcionalidad de login y de logout en las páginas de Django, en particular, en \url{https://docs.djangoproject.com/en/dev/topics/auth/default/#auth-web-requests}.

  \item No hace falta tener una página de registro. Si queremos registrar un usuario, lo haríamos a través del interfaz de ``admin''.
\end{itemize}

%Las aplicación Django ``Admin site'', entre otras, utiliza el middleware CSRF (protección frente a ``Cross Site Request Forgery''), que hay que tener en cuenta, especialmente en los formularios POST. En particular, es importante asegurarse de que se han referenciado los módulos de CSRF en settings.py:

%\begin{verbatim}
%MIDDLEWARE_CLASSES = (
%    ...
%    'django.middleware.csrf.CsrfViewMiddleware',
%    'django.middleware.csrf.CsrfResponseMiddleware',
%    ...
%\end{verbatim}

%Más información sobre este tema: \\
%\url{http://docs.djangoproject.com/en/dev/ref/contrib/csrf/}


%%----------------------------------------------------------------------------
%%----------------------------------------------------------------------------
\subsection{Django cms\_users\_put}
\label{subsec:django-users-put}

\textbf{Enunciado:}

Realizar un proyecto Django con la misma funcionalidad que ``Django cms\_users'' (ejercicio~\ref{subsec:django-users}), tratando de que el proceso de login y logout sea lo más razonable posible e incluyendo la funcionalidad de que sólo los usuarios que estén autenticados pueden cambiar el contenido de cualquier página, mientras que los que no lo están sólo pueden ver las páginas (funcionalidad similar a la de ``Gestor de contenidos con usuarios'').

Repositorio en GitLab para entregar el ejercicio: \\ 
\url{https://gitlab.etsit.urjc.es/grex/X-Serv-15.8-CmsUsersPut}.


%%----------------------------------------------------------------------------
%%----------------------------------------------------------------------------
\subsection{Django cms\_templates}
\label{subsec:django-templates}

\textbf{Enunciado:}

Realizar un proyecto Django con la misma funcionalidad que ``Django cms\_users\_put'' (ejercicio~\ref{subsec:django-users-put}), pero atendiendo a una nueva familia de recursos: ``/annotated/''. Cualquier recurso que comience con ``/annotated/'' se servirá usando una plantilla, y por lo demás, con la misma funcionalidad que teníamos en ``Django cms\_users\_put'' al recibir un GET para el nombre de recurso.

%Repositorio en GitHub para entregar el ejercicio: \\ 
%\url{https://github.com/CursosWeb/X-Serv-15.9-Django-CMS-Templates}.

%%----------------------------------------------------------------------------
%%----------------------------------------------------------------------------
\subsection{Django cms\_post}
\label{subsec:django-post}

\textbf{Enunciado:}
 Realizar un proyecto Django con la misma funcionalidad que ``Django cms\_templates'' (ejercicio~\ref{subsec:django-templates}), pero atendiendo a una nueva familiar de recursos: ``/edit/''. Cuando se acceda con un GET a un recurso que comience por ``/edit'', la aplicación web devolverá un formulario que permita editarlo (si se detecta un usuario autenticado, y si el nombre de recurso existe como página en la base de datos de la aplicación). Ese formulario tendrá un único campo que se precargará con el contenido de esa página. Si se accede con POST a un recurso que comience por ``/edit/'', se utilizará el valor que venga en él para actualizar la página correspondiente, si el usuario está autenticado y la página existe. Además, volverá a devolver el formulario igual que con el GET, para que el usuario pueda continuar editando si así lo desea.
 
Repositorio en GitLab: \\
\url{https://gitlab.etsit.urjc.es/cursosweb/practicas/server/django-cms-post}.

%%----------------------------------------------------------------------------
%%----------------------------------------------------------------------------
\subsection{Django cms\_forms}
\label{subsec:django-forms}

\textbf{Enunciado:}

Realizar el ejercicio ``Django cms\_post'' (ejercicio~\ref{subsec:django-post}) utilizando la clase Forms de Django. 

Además, se ha de intentar que un cambio en el modelo (p.ej. añadir un campo nuevo) sólo afecte al modelo y a la clase Form derivada del mismo, y pueda realizarse sin modificar ni las vistas ni las plantillas.

Repositorio en GitLab para entregar el ejercicio: \\ 
\url{https://gitlab.etsit.urjc.es/cursosweb/practicas/server/django-cms-forms}.


%%----------------------------------------------------------------------------
%%----------------------------------------------------------------------------
\subsection{Django feed\_expander}
\label{subsec:django-feed-expander}

Utilizando Django y, en la medida que te parezca conveniente, \texttt{feedparser.py} (\url{https://github.com/kurtmckee/feedparser}) y \texttt{BeautifulSoup.py} (\url{http://www.crummy.com/software/BeautifulSoup/}), realiza un servicio que expanda el contenido del canal de un usuario de Twitter. El servicio atenderá peticiones a recursos de la forma \verb|/feed/user| (siendo  \texttt{user} el identificador de un usuario de Twitter), devolviendo una página HTML con:

\begin{itemize}
\item Los cinco últimos \emph{tweets} del usuario.
\item Para cada uno de ellos, la lista de URLs que incluye (considerando como tales, por ejemplo, las subcadenas de caracteres delimitadas por espacios y que comiencen por ``http://''.
\item Para cada una de estas URLs:
  \begin{itemize}
  \item El texto del primer elemento $<p>$ de la página correspondiente, si existe.
  \item Las imágenes (identificadas como elementos $<img>$ que contenga la página correspondiente, si existen.
  \end{itemize}
\end{itemize}

Los canales de usuarios de Twitter están disponibles en formato RSS mediante el servicio Twitrss en URLs como \url{https://twitrss.me/twitter_user_to_rss/?user=user}, para el usuario \texttt{user}.

Pueden usarse también las bibliotecas \texttt{urllib} para la descarga de páginas mediante HTTP, y \texttt{urlparse} para manipular URLs (ambos son módulos estándar de Python).

\textbf{Referencias:}

\begin{itemize}
\item Documentación sobre feedparser.py: \\
  \url{https://pythonhosted.org/feedparser/}
\item Presentación sobre feedparser.py: \\
  \url{http://www.slideshare.net/LindseySmith1/feedparser}
\item Documentación sobre BeautifulSoup.py: \\
  \url{http://www.crummy.com/software/BeautifulSoup/documentation.html}
\end{itemize}


Repositorio en GitLab para entregar el ejercicio: \\ 
\url{https://gitlab.etsit.urjc.es/grex/X-Serv-15.12-Django-feedexpander}.


%%----------------------------------------------------------------------------
%%----------------------------------------------------------------------------
\subsection{Django feed\_expander\_db}
\label{subsec:django-feed-expander-db}

Realiza un servicio que proporcione la misma funcionalidad que ``Django feed\_expander'' (ejercicio~\ref{subsec:django-feed-expander}), pero almacenando los datos en tablas en una base de datos. Más en detalle:

\begin{itemize}
\item El recurso \verb|/feed/user| seguirá haciendo lo mismo, para el usuario ``user'' de Twitter. Pero además de mostrar la página web resultante, almacenará entablas en la base de datos:

  \begin{itemize}
  \item Los cinco últimos \emph{tweets} del usuario, y el usuario al que se refieren
  \item La lista de URLs de cada \emph{tweet}
  \item El texto del primer elemento $<p>$ de la página referenciada por cada URL.
  \item Las imágenes de dicha página.
  \end{itemize}

En todos estos casos, la información se añadirá a la que haya ya previamente en las tablas correspondientes.

\item El recurso \verb|/db/user| mostrará la misma página que se muestra para \verb|/feed/user|, pero incluyendo toda la información disponible en la base de datos para ese usuario (esto es, no limitado a los cinco últimos \emph{tweets}, si hubiera más den la base de datos). Para mostrar la página mencionada, no se accederá a ningún recurso externo: sólo a la información en la base de datos.
\end{itemize}

\textbf{Comentarios:}

Se pueden realizar varios diseños de tablas en la base de datos para este ejercicio. Entre ellos, se sugieren los basados en el siguiente esquema:

\begin{itemize}
\item Tabla de \emph{tweets}, con dos campos: usuarios y \emph{tweets}, ambos cadenas de texto (además, Django mantendrá un campo id para cada \emph{tweet}).
\item Tabla de URLs, con dos campos: id de \emph{tweet} y URL, el primero un id, el segundo cadena de texto (además, Django mantendrá un campo id para cada URL).
\item Tabla de textos, con dos campos: id de URL y texto (contenido de $<p>$, cadena de texto).
\item Tabla de imágenes, con dos campos: id de URL e imagen (cadena de texto con la URL de la imagen).
\end{itemize}

Desde luego, este esquema se puede simplificar y complicar, pero quizás sea un buen punto medio para empezar a trabajar.

%%----------------------------------------------------------------------------
%%----------------------------------------------------------------------------
%%----------------------------------------------------------------------------
\section{Ejercicios P4: Servidores simples de contenidos}

Construcción de algunos servidores de contenidos que permitan comprender la estructura básica de una aplicación web, y de cómo implementarlos aprovechando algunas características de Python.

%%----------------------------------------------------------------------------
%%----------------------------------------------------------------------------
\subsection{Clase contentApp}
\label{subsec:contentapp}

\textbf{Enunciado:}

Esta clase, basada en el esquema de clases definido en el ejercicio ``Clase servidor de aplicaciones'' (ejercicio~\ref{subsec:clase-serv-aplis}), sirve el contenido almacenado en un diccionario Python. La clave del diccionario es el nombre de recurso a servir, y el valor es el cuerpo de la página HTML correspondiente a ese recurso.

La solución de este ejercicio se encuentra disponible en el siguiente repositorio de 
GitLab: \url{https://gitlab.etsit.urjc.es/grex/x-serv-16.3-contentputapp}.

% GitHub:\url{https://github.com/CursosWeb/X-Serv-16.3-contentPutApp}.

%%----------------------------------------------------------------------------
%%----------------------------------------------------------------------------
%%----------------------------------------------------------------------------
\subsection{Instalación y prueba de Poster}
\label{subsec:inst-poster}

\textbf{Enunciado:}

Instalación y prueba de Poster, \emph{add-on} de Firefox

\textbf{Referencias:}

Poster Firefox add-on: \\
\url{https://addons.mozilla.org/es/firefox/addon/poster/}

También se puede utilizar RestClient, que tiene funcionalidad parecida: \\
\url{https://addons.mozilla.org/en-US/firefox/addon/restclient/}

%%----------------------------------------------------------------------------
\subsection{Clase contentPutApp}
\label{subsec:contentputapp}

\textbf{Enunciado:}

Construcción de la clase ``contentPutApp'', similar a contentApp (ejercicio~\ref{subsec:contentapp}). En este caso, la clase permite la actualización del contenido mediante peticiones HTTP PUT. Para probarla, se puede usar el add-on de Firefox llamado ``Poster''. La clase será minimalista, basta con que funcione con ``Poster''.

Opcionalmente, puede trabajarse en conseguir que un servidor construido con la clase anterior funcione con Bluefish. Bluefish es un editor de contenidos, que puede cargar una página especificando su URL, y que una vez modificada, puede enviarla, usando PUT, de nuevo a la misma URL. Aunque esto es exactamente lo que espera la clase ``contentPutApp'', hay algunas peculiaridades de funcionamiento de Bluefish que hacen que probablemente la clase haya de ser modificada para que funcione correctamente con esta herramienta.

%%----------------------------------------------------------------------------
\subsection{Clase contentPostApp}
\label{subsec:contentpostapp}

\textbf{Enunciado:}

Construcción de la clase ``contentPostApp'', similar a contentApp (ejercicio~\ref{subsec:contentapp}). En este caso, la clase permite la actualización del contenido mediante peticiones HTTP POST. Cuando se reciba un GET pidiendo cualquier recurso, se buscará en el diccionario de contenidos, y si existe, se servirá. En cualquier caso (exista o no exista el contenido en cuestión) se servirá en la misma página un formulario que permitirá actualizar el contenido del diccionario (o crear una nueva entrada, si no existía) mediante un POST.

\textbf{Referencias:}

Forms in HTML (HTML 4.01 Specification by W3C): \\
\url{http://www.w3.org/TR/html4/interact/forms.html}

%%----------------------------------------------------------------------------
\subsection{Clase contentPersistentApp}
\label{subsec:contentpersistentapp}

\textbf{Enunciado:}

Construcción de la clase contentPersistentApp, similar a contentPutApp (ejercicio~\ref{subsec:contentputapp}), pero incluyendo almacenamiento del diccionario con los contenidos en almacenamiento persistente, de forma que la aplicación mantenga estado al recuperarse después de una caída. Para mantener estado, puede usarse el módulo ``Shelve'' de Python, que permite almacenar y recuperar objetos en ficheros.

Opcionalmente, puede usarse en otra versión el módulo ``dbm'' de Python, que sirve también para gestionar diccionarios persistentes, pero con más limitaciones.

%%----------------------------------------------------------------------------
\subsection{Clase contentStorageApp}
\label{subsec:contentstorageapp}

\textbf{Enunciado:}

Construcción de la clase contentStorageApp similar a contentPersistentApp, pero que use un objeto de clase permanentContentStore para almacenar el estado que ha de sobrevivir a caídas de la aplicación. Esta clase mantendrá variables internas con el estado a salvaguardar persistentemente, y métodos para consultar y actualizar los valores de ese estado.

%%----------------------------------------------------------------------------
\subsection{Gestor de contenidos con usuarios}
\label{subsec:contentappusers}

\textbf{Enunciado:}

Construye la clase contentAppUsers, que amplía el gestor de contenidos que estamos construyendo (clase contentStorageApp, ejercicio~\ref{subsec:contentstorageapp}) con el concepto de usuarios registrados.

Cada usuario registrado tendrá un nombre y una contraseña (que puedes almacenar por ejemplo en un diccionario), y sólo si se ha mostrado al sistema que se es usuario registrado se podrá cambiar contenido del sitio (mediante un PUT). Para mostrar que se es usuario del sistema, se hará un GET a un recurso de la forma ``/login,usuario,contraseña'', donde ``usuario'' y ``contraseña'' son el nombre de un usuario y su contraseña. A partir de ese momento, el sistema reconocerá que los accesos desde el mismo navegador son de ese usuario. 

%%----------------------------------------------------------------------------
\subsection{Gestor de contenidos con usuarios, con control estricto de actualización}
\label{subsec:contentappusersstrict}

\textbf{Enunciado:}

Construye la clase contentAppUsersStrict, que implemente la misma funcionalidad de contentAppUsers (ejercicio~\ref{subsec:contentappusers}), pero que además controle que sólo actualiza un contenido quien lo creó. En otras palabras, cuando la aplicación recibe un PUT, se comprueba que el recurso no existe, y en ese caso, si lo está subiendo un usuario autenticado, se crea. Pero si el recurso existe, sólo lo actualiza si el usuario que está invocando el PUT es el mismo que creó el recurso. Para implementar esta funcionalidad puedes utilizar un diccionario que ``recuerde'' quien creó cada recurso, o añadir, a los datos del diccionario de contenidos (donde sólo había la página HTML para el recurso en cuestión) un nuevo elemento (por ejemplo, usando una lista): el usuario que creó el recurso.


%%----------------------------------------------------------------------------
%%----------------------------------------------------------------------------
%%----------------------------------------------------------------------------
\section{Ejercicios P5: Aplicaciones web con base de datos}

Construcción de aplicaciones web con almacenamiento estable en base de datos.

%%----------------------------------------------------------------------------
%%----------------------------------------------------------------------------
\subsection{Introducción a SQLite3 con Python}
\label{subsec:sqlite3-python}

\textbf{Enunciado:}

Vamos a empezar a usar bases de datos relacionales con nuestras aplicaciones web. En particular, vamos a usar el módulo Python sqlite3, que proporciona enlace con el gestor de bases de datos SQLite3, que utiliza una interfaz SQL. Estudiar \texttt{test-db.py}, para entender cómo se hacen operaciones básicas sobre una base de datos con Python. Modificar ese programa para que añada más registros, y comprobar con sqlitebrowser la base de datos creada.

\textbf{Material:}

\texttt{test-db.py}. Programa que crea una base de datos simple SQLite3, y luego la muestra en pantalla.

%%----------------------------------------------------------------------------
%%----------------------------------------------------------------------------
\subsection{Gestor de contenidos con base de datos}
\label{subsec:gestor-contenidos-bbdd}

\textbf{Enunciado:}

Escribe y prueba la clase contentDBApp, que será una versión de contentApp (ejercicio~\ref{subsec:contentapp}), pero utilizando una base de datos SQLite3 para almacenar sus objetos persistentes.

%%----------------------------------------------------------------------------
%%----------------------------------------------------------------------------
\subsection{Gestor de contenidos con usuarios, con control estricto de actualización y base de datos}
\label{subsec:gestor-contenidos-usuarios-bbdd}

\textbf{Enunciado:}

Escribe y prueba la clase contentDBAppUsersStrict, que será igual que ``Gestor de contenidos con usuarios, con control estricto de actualización'' (contentAppUsersStrict, ejercicio~\ref{subsec:contentappusersstrict}), pero usando base de datos como almacenamiento permanente.

\newpage

%% %%----------------------------------------------------------------------------
%% %%----------------------------------------------------------------------------
%%----------------------------------------------------------------------------
\section{Prácticas de entrega voluntaria}


%%----------------------------------------------------------------------------
%%----------------------------------------------------------------------------
\subsection{Práctica 1 (entrega voluntaria)}
\label{subsec:practica-vol-1-2016}

\textbf{Fecha recomendada de entrega:} Antes del 10 de marzo de 2020.

Esta práctica tendrá como objetivo la creación de una aplicación web simple para acortar URLs. La aplicación tendrá que realizarse según un esquema de clases similar al explicado en clase.

El repositorio de partida es: \url{https://gitlab.etsit.urjc.es/grex/x-serv-18.1-practica1}

El código ha de guardarse en un fichero llamado \emph{practica1.py}.

El funcionamiento de la aplicación será el siguiente:

\begin{itemize}
\item Recurso ``/'', invocado mediante GET. Devolverá una página HTML con un formulario. En ese formulario se podrá escribir una url, que se enviará al servidor mediante POST. Además, esa misma página incluirá un listado de todas las URLs reales y acortadas que maneja la aplicación en este momento.

\item Recurso ``/'', invocado mediante POST. Si el comando POST incluye una \texttt{qs} (query string) que corresponda con una url enviada desde el formulario, se devolverá una página HTML con la URL original y la URL acortada (ambas como enlaces pinchables), y se apuntará la correspondencia (ver más abajo).

Si el POST no trae una \texttt{qs} que se haya podido generar en el formulario, devolverá una página HTML con un mensaje de error.

Si la URL especificada en el formulario comienza por ``http://'' o ``https://'', se considerará que ésa es la URL a acortar. Si no es así, se le añadirá ``http://'' por delante, y se considerará que esa es la url a acortar. Por ejemplo, si en el formulario se escribe ``http://gsyc.es'', la URL a acortar será ``http://gsyc.es''. Si se escribe ``gsyc.es'', la URL a acortar será ``http://gsyc.es''.

Para determinar la URL acortada, utilizará un número entero secuencial, comenzando por 0, para cada nueva petición de acortamiento de una URL que se reciba. Si se recibe una petición para una URL ya acortada, se devolverá la URL acortada que se devolvió en su momento.

Así, por ejemplo, si se quiere acortar

\verb|http://gsyc.urjc.es|

y la aplicación está en el puerto 1234 de la máquina ``localhost'', se invocará (mediante POST) la URL

\verb|http://localhost:1234/|

y en el cuerpo de esa petición HTTP irá la \texttt{qs}

\verb|url=http://gsyc.urjc.es|

si el campo donde el usuario puede escribir en el formulario tiene el nombre ``URL''. Normalmente, esta invocación POST se realizará rellenando el formulario que ofrece la aplicación.

Como respuesta, la aplicación devolverá (en el cuerpo de la respuesta HTTP) la URL acortada, por ejemplo

\verb|http://localhost:1234/3|

Si a continuación se trata de acortar la URL

\verb|http://www.urjc.es|

mediante un procedimiento similar, se recibirá como respuesta la URL acortada

\verb|http://localhost:1234/4|

Si se vuelve a intentar acortar la URL

\verb|http://gsyc.urjc.es|

como ya ha sido acortada previamente, se devolverá la misma URL corta:

\verb|http://localhost:1234/3|

\item Recursos correspondientes a URLs acortadas. Estos serán números con el prefijo ``/''. Cuando la aplicación reciba un GET sobre uno de estos recursos, si el número corresponde a una URL acortada, devolverá un HTTP REDIRECT a la URL real. Si no la tiene, devolverá HTTP ERROR ``Recurso no disponible''.

Por ejemplo, si se recibe 

\verb|http://localhost:1234/3|

la aplicación devolverá un HTTP REDIRECT a la URL

\verb|http://gsyc.urjc.es|

\end{itemize}

%La aplicación funcionará con estado: se supone que cada vez que la aplicación muera y vuelva a ser lanzada, no perderá todo su estado anterior. Para ello, se guardarán las URLs acortadas en un fichero CSV. Al lanzar la aplicación, se leerá el fichero CSV con las URLs acortadas. Y cada vez que se incluya una nueva URL acortada en el sistema, también se guardará esta información en el fichero CSV.

\textbf{Comentario}

Se recomienda utilizar dos diccionarios para almacenar las URLs reales y los números de las URLs acortadas. En uno de ellos, la clave de búsqueda será la URL real, y se utilizará para saber si una URL real ya está acortada, y en su caso saber cuál es el número de la URL corta correspondiente.

En el otro diccionario la clave de búsqueda será el número de la URL acortada, y se utilizará para localizar las URLs reales dadas las cortas. De todas formas, son posibles (e incluso más eficientes) otras estructuras de datos.

Se recomienda realizar la aplicación en varios pasos:

\begin{itemize}
\item Comenzar por reconocer ``GET /'', y devolver el formulario correspondiente.
\item Reconocer ``POST /'', y devolver la página HTML correspondiente (con la URL real y la acortada).
\item Reconocer ``GET /num'' (para cualquier número num), y realizar la redirección correspondiente.
\item Manejar las condiciones de error y realizar el resto de la funcionalidad.
\end{itemize}

%%---------------------------------------------------------------------
%%---------------------------------------------------------------------
\subsection{Práctica 2 (entrega voluntaria)}
\label{subsec:practica-vol-2-2015}

\textbf{Fecha recomendada de entrega:} Hasta el 14 de abril.

Esta práctica tendrá como objetivo la creación de una aplicación web (de nombre \emph{acorta}) simple para acortar URLs utilizando Django (en un nuevo proyecto Django llamado \emph{project}). Su enunciado será igual que el de la práctica 1 de entrega voluntaria (ejercicio~\ref{subsec:practica-vol-1-2016}), salvo en los siguientes aspectos:

\begin{itemize}
  \item Se implementará utilizando Django.
  \item Tendrá que almacenar la información relativa a las URLs que acorta en una base de datos, de forma que aunque la aplicación sea rearrancada, las URLs acortadas sigan funcionando adecuadamente.
  \item Utilizará plantillas, de manera que el código Python y el HTML estarán separados.
\end{itemize}

%Repositorio GitHub de entrega: \\
%\url{https://github.com/CursosWeb/X-Serv-18.2-Practica2}

Repositorio GitLab de partida: \\
\url{https://gitlab.etsit.urjc.es/cursosweb/x-serv-18.2-practica2}

\newpage

%%----------------------------------------------------------------------------
%%----------------------------------------------------------------------------
\section{Práctica final: MiTiempo (2019, mayo)}
\label{practica-final-2019-05}

[ \textbf{Nota importante:} Este enunciado es aún tentativo, y puede sufrir cambios ]

La práctica final de la asignatura consiste en la creación de una aplicación web, llamada ``MiTiempo'', que aglutine información sobre municipios de España, y especialmente información meteorológica sobre ellos. A continuación se describe el funcionamiento y la arquitectura general de la aplicación, la funcionalidad mínima que debe proporcionar, y otra funcionalidad optativa que podrá tener.

La aplicación se encargará de descargar información sobre las condiciones meteorológicas de los municipios, disponibles públicamente en en sitio web de la AEMET, y de ofrecerla a los usuarios para que puedan monitorizar con facilidad las previsiones para aquellos municipios que les parezcan más interesantes, y comentar sobre ellos. De esta manera, un escenario típico es el de un usuario que elija los municipios que le parezcan de interés, y comente lo que le quiera sobre ellos.

%%----------------------------------------------------------------------------
\subsection{Arquitectura y funcionamiento general}

Arquitectura general:

\begin{itemize}

  \item La práctica se construirá como un proyecto Django/Python3, que incluirá una o varias aplicaciones (apps) Django que implementen la funcionalidad requerida.

  \item Para el almacenamiento de datos persistente se usará SQLite3, con tablas definidas en modelos de Django.

  \item Se usará la aplicación Django ``Admin Site'' para crear cuentas a los usuarios en el sistema, y para la gestión general de las bases de datos necesarias. Todas las bases de datos que contenga la aplicación tendrán que ser accesibles vía este ``Admin Site''.

  \item Se utilizarán plantillas Django (a ser posible, una jerarquía de plantillas, para que la práctica tenga un aspecto similar) para definir las páginas que se servirán a los navegadores de los usuarios. Estas plantillas incluirán en todas las páginas al menos:
  \begin{itemize}
  \item Un \emph{banner} (imagen) del sitio, preferentemente en la parte superior izquierda.
  \item Una caja para entrar (hacer login en el sitio), o para salir (si ya se ha entrado).
  \begin{itemize}
    \item En caso de que no se haya entrado en una cuenta, esta caja permitirá al visitante introducir su identificador de usuario y su contraseña. 
    \item En caso de que ya se haya entrado, esta caja mostrará el identificador del usuario y permitirá salir de la cuenta (logout). Esta caja aparecerá preferentemente en la parte superior derecha.
  \end{itemize}
  \item Un menú de opciones, como barra, preferentemente debajo de los dos elementos anteriores (banner y caja de entrada o salida).
  \item Un pie de página con una nota de atribución, indicando ``Esta aplicación utiliza datos proporcionados por la AEMET'', y un enlace al sitio web de AEMET\footnote{Sitio web de AEMET: \url{https://aemet.es}}.
  \end{itemize}

Cada una de estas partes estará construida dentro de un elemento ``div'', marcada con un atributo ``id'' en HTML, para poder ser referenciadas fácilmente en hojas de estilo CSS.

\item Se utilizarán hojas de estilo CSS para determinar la apariencia de la práctica. Estas hojas definirán al menos el color y el tamaño de la letra, y el color de fondo de cada una de las partes (elementos) marcadas con un \emph{id}, tal como se indica en el apartado anterior.

\item Para obtener la información sobre previsión meteorológica de cada municipio se utilizará la información disponible en AEMET:

  \begin{itemize}
  \item Ejemplo de información para un municipio, en formato XML
    (para cada municipio, el número de cinco cifras que finaliza la URL
    se obtiene del documento descrito más abajo): \\
    \url{http://www.aemet.es/xml/municipios/localidad_28058.xml}
  \item Documento JSON con listado de municipios, incluyendo su nombre y
    su identificador para localizar los documentos anteriores (campo ``id\_old''): \\
    \url{https://raw.githubusercontent.com/CursosWeb/Code/master/Python-JSON/municipios.json}
  \end{itemize}
  
\end{itemize}

Funcionamiento general:

\begin{itemize}
  \item Los usuarios serán dados de alta en la práctica mediante el módulo ``Admin Site'' de Django. Una vez estén dados de alta, serán considerados ``usuarios registrados''.

  \item El listado de municipios se cargará de nuevo cada vez que arranque la aplicación, a partir de un fichero que será parte del proyecto Django. El listado se mantendrá en un diccionario en memoria, y no se guardará en almacenamiento persistente en la base de datos.

  \item Los usuarios registrados podrán crear su selección de municipios. Para ello, dispondrán de una página personal. Llamaremos a esta página la ``página del usuario''.

  \item La selección de municipios en su página personal la realizará cada usuario rellenando un formulario que estará en su página de usuario. Este formulario permitirá elegir un nombre de municipio. Si el municipio coincide con uno en el listado de municipios, se considerará válido, y se añadirá a la lista de municipios seleccionados por ese usuario. Si no es así, se le indicará que el nombre del municipio es erróneo.

  \item Cualquier navegador podrá acceder a la interfaz pública del sitio, que ofrecerá la página personal de cada usuario, para todos los usuarios del sitio.

\end{itemize}


%%----------------------------------------------------------------------------
\subsection{Funcionalidad mínima}

La información para cada municipio se obtendrá a partir de la información pública ofrecida por AEMET, en forma de ficheros XML, como se indicaba anteriormente.

La {\bf interfaz pública} contiene los recursos a servir como páginas HTML completas (pensadas para ser vistas en el navegador) para cualquier visitante (sea usuario registrado o no), excepto donde se indica que se servirá una página XML:

\begin{itemize}
  \item /: Página principal de la práctica. Constará de un listado de poblaciones que han sido elegidas por algún usuario, y otro con enlaces a páginas de usuarios:
  
  \begin{enumerate}
    \item Mostrará un listado de los 10 municipios con más comentarios. Si no hubiera 10 municipios con comentarios, se mostrarán sólo los que tengan comentarios. Para cada municipio, incluirá información sobre:
    \begin{itemize}
    \item su nombre (que será un enlace que apuntará a la URL del municipio en el sitio de AEMET)\footnote{Por ejemplo \url{http://www.aemet.es/es/eltiempo/prediccion/municipios/fuenlabrada-id28058} (donde el identificador ``fuenlabrada-id28058'' puede encontrarse en el docuento JSON con el listado de municipios como campo ``url''}, 
    \item su altitud, latitud y longitud,
    \item su previsión de tiempo para mañana: probabilidad de precipitación (0 a 24), temperatura máxima y mínima, y descripción (0 a 24).
    \item y un enlace, ``Más información'', que apuntará a la página del municipio en la aplicación (ver más adelante).
    \end{itemize}
   
  \item También se mostrará un listado, en una columna lateral, con enlaces a las páginas personales disponibles. Para cada página personal mostrará el título que le haya dado su usuario (como un enlace a la página personal en cuestión) y el nombre del usuario. Si a una página personal aún no se le hubiera puesto título, este título será ``Página de usuario'', donde ``usuario'' es el identificador de usuario del usuario en cuestión.
  \item Tambień se mostrará un botón, que al pulsarlo se verán sólo los municipios con probabilidad de precipitacion mayor que cero. Si se vuelve a pulsar, se verán los que tengan ptobabilidad de precipitación igual a cero. Si se vuelve a pulsar una vez más, se volverán a ver todos los municipios.
   \end{enumerate}

  La página principial se ofrecerá también como un documento XML, que incluirá la misma lista de municipios, y un enlace al fichero XML que proporciona para cada uno de ellos la AEMET. Este documento se ofrecerá cuando se pida la URL de municipios, concatenando al final \verb|?format=xml|.

  La página principal en formato HTML includirá un enlace a la página principal en formato XML (``Descarga como fichero XML'').
  
  \item /{usuario}: Página personal de un usuario. Si la URL es ``/usuario'', es que corresponde al usuario ``usuario''. Mostrará los municipios seleccionados por ese usuario. Para cada municipio se mostrará la misma información que en la página principal. Los municipios deben aparecer en el orden en que los ha seleccionado el usuario (primero el que fue seleccionado más recientemente).

  La página de cada usuario se ofrecerá también como un documento XML, que incluirá la lista de municipios seleccionados, y un enlace al fichero XML que proporciona para cada uno de ellos la AEMET. Este documento se ofrecerá cuando se pida la URL del usuario, concatenando al final \verb|?format=xml|.

  La página de cada usuario en formato HTML includirá un enlace a la página de ese mismo usuario en formato XML (``Descarga como fichero XML'').

  \item /municipios: Página con todos los municipos que han sido seleccionados por algún usuario (aunque hayan sido luego ``desseleccionados''. Para cada uno de ellos aparecerá sólo el nombre, como un enlace a su página (ver más abajo), y el número de comentarios que se han puesto sobre él. En la parte superior de la página, habrá un formulario que permita filtrar según la temperatura máxima para mañana: se mostrarán solo los municipios que para mañana tengan previsión de temperatura máxima entre las dos que se indiquen, si se indican.

    La página de municipios se ofrecerá también como un documento XML, que incluirá la misma lista de municipios, y un enlace al fichero XML que proporciona para cada uno de ellos la AEMET. Este documento se ofrecerá cuando se pida la URL de municipios, concatenando al final \verb|?format=xml|.

    La página de municipios en formato HTML includirá un enlace a la página de municipios en formato XML (``Descarga como fichero XML'').

  \item /municipios/{id}: Página de un municipio en la aplicación. Mostrará toda la información razonablemente posible del documento XML obtenido de AEMET (en cuanto a predicción para mañana, en el rango 0 a 24 horas), incluyendo también al menos la que se menciona en otros apartados de este enunciado. También se incluirá un enlace a la página de este municipio en el sitio de AEMET. Además, se mostrarán todos los comentarios que se hayan puesto para este municipio. Esta información se actualizará cuando se consulte esta página de un minicipio, y a partir de este momento se mostrará actualizada en cualquier otra página del sitio. La información no se actualizará en ningún otro momento.
 
  \item /info: Página con información en HTML indicando la autoría de la práctica, explicando su funcionamiento y una brevísima documentación.

\end{itemize}

Todas las páginas de la interfaz pública incluirán un menú desde el que se podrá acceder a todos los municipios (URL /municipios) con el texto ``Todos'' y a la ayuda (URL /info) con el texto ``Info''. Todas las página que no sean la principal tendrán otra opción de menú para la URL /, con el texto ``Inicio''.

La {\bf interfaz privada} contiene los recursos a servir como páginas HTML completas para usuarios registrados (una vez se han autenticado):

\begin{itemize}
  \item Todos los recursos de la interfaz pública.

  \item /municipios/{id}: Además de la información que se muestra de manera pública:

    \begin{enumerate}
    \item Un formulario para poner comentarios sobre este municipio. Los comentarios quedarán a nombre del usuario que los ponga, y sólo se podrán poner por los usuarios registrados, una vez se han autenticado. Por tanto, bastará con que este formulario esté compuesto por una caja de texto, donde se podrá escribir el comentario, y un botón para enviarlo. El sistama anotará automáticamente quién está poniendo el comentario, y mostrará esa información cada vez que muestre el comentario (con el texto ``Comentado por'').
  \end{enumerate}

  \item /{usuario}: Además de la información que se muestra de manera pública:
  
  \begin{enumerate}
    \item Un formulario para cambiar el estilo CSS de todo el sitio para ese usuario. Bastará con que se pueda cambiar el tamaño y el color de la letra y el color de fondo. Si se cambian estos valores, quedará cambiado el documento CSS que utilizarán todas las páginas del sitio para este usuario. Este cambio será visible en cuanto se suba la nueva página CSS.

    \item Un formulario para elegir el título de su página personal.

    \item Un formulario para seleccionar un nuevo municipio. En este formulario se podrá poner el nombre de un municipio, que si existe, quedará seleccionado para este usuario.

    \item Un botón ``Quitar'' que aparecerá asociado a cada municipio seleccionado, que permitirá al usuario ``deseleccionar'' el municipio de su lista.
  \end{enumerate}
\end{itemize}

%Si es preciso, se añadirán más recursos (pero sólo si es realmente preciso) para poder satisfacer los requisitos especificados.

Dados los recursos mencionados anteriormente, no se permitirán los nombres de usuario ``municipios'' ni ``info'' (pero no hay que hacer ninguna comprobación para esto: se asume que no se darán de alta esos usuarios en el Admin Site).


%%----------------------------------------------------------------------------
\subsection{Funcionalidad optativa}

De forma optativa, se podrá incluir cualquier funcionalidad relevante en el contexto de la asignatura. Se valorarán especialmente las funcionalidades que impliquen el uso de técnicas nuevas, o de aspectos de Django no utilizados en los ejercicios previos, y que tengan sentido en el contexto de esta práctica y de la asignatura.

En el formulario de entrega se pide que se justifique por qué se considera funcionalidad optativa lo que habeis implementado. Sólo a modo de sugerencia, se incluyen algunas posibles funcionalidades optativas:

\begin{itemize}
  \item Inclusión de un \emph{favicon} del sitio
  
  \item Visualización de las páginas en formato JSON, de forma similar a como el enunciado indica para XML.

  \item Generación de un canal RSS, XML libre y/o JSON para los comentarios puestos en el sitio.

  \item Incorporación de datos del canal RSS de avisos de AEMET\footnote{Canales RSS de AEMET: \url{http://www.aemet.es/es/rss_info}} a la página principal y/o a otras páginas ofrecidas por la aplicación.
    
  \item Funcionalidad para acceder a datos ofrecidos por AEMT via su API de datos abiertos\footnote{AEMET open data: \url{https://opendata.aemet.es}}
  
  \item Funcionalidad de registro de usuarios: que la aplicación proporcione la funcionalidad de registrarse en el sitio.
  
  \item Uso de Javascript o AJAX para algún aspecto de la práctica (por ejemplo, para seleccionar un municipio para una página de usuario).

  \item Puntuación de municipios. Cada visitante (registrado o no) puede dar un ``+1'' a cualquier municipio que aparezca en el sitio. La suma de ``+'' que ha obtenido un municipio se verá cada vez que se vea el municipio en el sitio.
  
  \item Uso de elementos HTML5 (especificar cuáles al entregar)

  \item Atención al idioma indicado por el navegador. El idioma de la interfaz de usuario del planeta tendrá en cuenta lo que especifique el navegador.

  \item Despliegue de la práctica en algún sitio de Internet, de forma que pueda accederse a ella. Por ejemplo, puede considerarse desplegar en un ordenador dedicado (por ejemplo, Raspberry Pi accesible directamente desde Internet), o en servicios como Google Computing Engine\footnote{GCP Engine Free: \url{https://cloud.google.com/free/}} o Heroku\footnote{Heroku Free: \url{https://www.heroku.com/free}}.
\end{itemize}

%%----------------------------------------------------------------------------
\subsection{Entrega de la práctica}

\begin{itemize}
  \item \textbf{Fecha límite de entrega de la práctica:} viernes, 24 de mayo de 2019 a las 03:00 (hora española peninsular)\footnote{Entiéndase la hora como jueves por la noche, ya entrado en viernes.}
       %{\bf Convocatoria de junio:} miércoles, 24 de junio de 2015 a las 23:59 (hora peninsular española).

  \item \textbf{Fecha de publicación de notas de prácticas:} sábado 25 de mayo, en el aula virtual.
%{\bf Convocatoria de junio:} viernes, 26 de junio, en la plataforma Moodle.

  \item \textbf{Fecha de revisión de prácticas:} martes 28 de mayo, a las 12:00. Se requerirá a algunos alumnos que asistan a la revisión {\bf en persona}; se informará de ello en el mensaje de publicación de notas.
%{\bf Convocatoria de junio:} martes, 30 de junio a las 13:30. Se requerirá a algunos alumnos que asistan a la revisión {\bf en persona}; se informará de ello en el mensaje de publicación de notas.
\end{itemize}

La entrega de la práctica consiste en {\bf rellenar un formulario} (enlazado en el Moodle de la asignatura) y en seguir las instrucciones que se describen a continuación.

\begin{enumerate}
  \item El repositorio contendrá todos los ficheros necesarios para que funcione la aplicación (ver detalle más abajo). Es muy importante que el alumno haya realizado una derivación (fork) del repositorio que se indica a continuación, porque si no, la práctica no podrá ser identificada: 

\url{https://gitlab.etsit.urjc.es/cursosweb/practicas/server/final-mitiempo/}

Los alumnos que no entreguen las práctica de esta forma serán considerados como no presentados en lo que a la entrega de prácticas se refiere. Los que la entreguen podrán ser llamados a realizar también una entrega presencial, que tendrá lugar en la fecha y hora de la revisión. Esta entrega presencial podrá incluir una conversación con el profesor sobre cualquier aspecto de la realización de la práctica.

Recordad que es importante ir haciendo commits de vez en cuando y que sólo al hacer push estos commits son públicos. Antes de entregar la práctica, haced un push. Y cuando la entreguéis y sepáis el nombre del repositorio, podéis cambiar el nombre del repositorio desde el interfaz web de GitLab. 
 
 \item Un vídeo de demostración de la parte obligatoria, y otro vídeo de demostración de la parte opcional, si se han realizado opciones avanzadas. Los vídeos serán de una {\bf duración máxima de 3 minutos} (cada uno), y consistirán en una captura de pantalla de un navegador web utilizando la aplicación, y mostrando lo mejor posible la funcionalidad correspondiente (básica u opcional). Siempre que sea posible, el alumno comentará en el audio del vídeo lo que vaya ocurriendo en la captura. Los vídeos se colocarán en algún servicio de subida de vídeos en Internet (por ejemplo, Vimeo, Twitch, o YouTube). Los vídeos de más de tres minutos tendrán penalización.

Hay muchas herramientas que permiten realizar la captura de pantalla. Por ejemplo, en GNU/Linux puede usarse Gtk-RecordMyDesktop o Istanbul (ambas disponibles en Ubuntu). OBS Studio\footnote{OBS Studio: \url{https://obsproject.com/}} está disponible para varias plataformas (Linux, Windows, MacOS). Es importante que la captura sea realizada de forma que se distinga razonablemente lo que se grabe en el vídeo.

En caso de que convenga editar el vídeo resultante (por ejemplo, para eliminar tiempos de espera) puede usarse un editor de vídeo, pero siempre deberá ser indicado que se ha hecho tal cosa con un comentario en el audio, o un texto en el vídeo. Hay muchas herramientas que permiten realizar esta edición. Por ejemplo, en GNU/Linux puede usarse OpenShot o PiTiVi.

  \item Se han de entregar los siguientes ficheros:

\begin{itemize}
  \item Un fichero README.md que resuma las mejoras, si las hay, y explique cualquier peculiaridad de la entrega (ver siguiente punto).
  \item El repositorio en el GitLab de la ETSIT deberá contener un proyecto Django completo y listo para funcionar en el entorno del laboratorio, incluyendo la base de datos con datos suficientes como para poder probarlo. Estos datos incluirán al menos dos usuarios con sus datos correspondientes, con al menos seis municipios en su página personal, al menos 12 municipios distintos seleccionados, y con al menos cinco comentarios en total.
  \item Cualquier biblioteca Python que pueda hacer falta para que la aplicación funcione, junto con los ficheros auxiliares que utilice, si es que los utiliza.
\end{itemize}

  \item Se incluirán en el fichero README.md los siguientes datos (la mayoría de estos datos se piden también en el formulario que se ha de rellenar para entregar la práctica - se recomienda hacer un corta y pega de estos datos en el formulario):

\begin{itemize}
  \item Nombre y titulación.
  \item Nombre de su cuenta en el laboratorio del alumno.
  \item Resumen de las peculiaridades que se quieran mencionar sobre lo implementado en la parte obligatoria.
  \item Lista de funcionalidades opcionales que se hayan implementado, y breve descripción de cada una.
  \item URL del vídeo demostración de la funcionalidad básica
  \item URL del vídeo demostración de la funcionalidad optativa, si se ha realizado funcionalidad optativa
  \item Cuenta (login) y contraseña de los usuarios que están dados de alta en la aplicación.
  \item URL de la aplicación desplegada (si es que se ha desplegado)
\end{itemize}

Asegúrate de que las URLs incluidas en este fichero están adecuadamente escritas en Markdown, de forma que la versión HTML que genera GitLab los incluya como enlaces ``pinchables''.

\end{enumerate}


%%----------------------------------------------------------------------------
\subsection{Notas y comentarios}

La práctica deberá funcionar en el entorno GNU/Linux (Ubuntu) del laboratorio de la asignatura con la versión de Django que se ha usado en prácticas.

La práctica deberá funcionar desde el navegador Firefox disponible en el laboratorio de la asignatura.

Los canales (feeds) RSS que produce la aplicación web realizada en la práctica deberán funcionar al menos con el navegador Firefox (considerándolos como canales RSS) disponibles en el laboratorio. Los documentos XML deberán ser correctos desde el punto de vista de la sintaxis XML, y por lo tanto reconocibles por un reconocedor XML, como por ejemplo el del módulo xml.sax de Python. Los documentos JSON generados deberán ser correctos desde el punto de vista de la sintaxis JSON, y por lo tanto reconocibles por un reconocedor JSON, como por ejemplo el del módulo json de Python

%%----------------------------------------------------------------------------
\subsection{Preguntas y respuestas}

A continuación, algunas preguntas relacionadas con el enunciado de esta práctica, junto con sus respuestas:

\begin{itemize}
\item ¿Es necesario utilizar los mecanismos provistos por Django para el control de sesiones y autenticación?

  En principio, esa es la solución recomendada. El principal problema suele ser asegurarse de que cuaquier mecanismo alternativo funciona al menos tan bien como el de Django, lo que no es en general trivial. De todas formas, salvo muy buenos motivos, la aplicación es una aplicación Django, y por lo tanto cuantas más facilidades de Django se usen (bien usadas), mejor.
  
\item ¿Puedo guardar en la base de datos los datos referentes a latitud, altitud, etc (datos que no varian nunca) y precipitación, temperatura, descripción, etc y cambiarlos cuando sea necesario (ya que estos si cambian)?

  Pueden almacenarse en tablas en la base de datos los datos correspondientes a poblaciones que han sido seleccionadas por al menos un usuario. En otras palabras, cada vez que un usuario seleccione un municipio, puedes guardar en una tabla en la base de datos los datos sobre ese municipio (includos latitud y longitud). Pero no puedes analizar todos los municipios que hay en el fichero JSON e incorporar su información a la base de datos.

  La información sobre un municipio que puedas almacenar en la base de datos deberá actualizarse cuando se acceda al fichero XML para ese municipio, según indica el enunciado (por ejemplo, porque un usuario selecciona ese municipio, o porque hay un acceso a su página de municipio).

\item Los archivos CSS que pueden modificar los usuarios, ¿dónde y cómo debemos guardarlos?

  La forma recomendada de hacerlo es mediante plantillas:

  \begin{itemize}
  \item En el directorio de plantillas incluirías una para la hoja CSS del sitio. Esa plantilla tendría como variables de plantilla los valores que quieras que los usuarios puedan cambiar (color de tipo de letra, tamaño de tipo de letra, etc.).
  \item Además, para cada usuario, tendrás una tabla donde se almacenarán los valores para ese usuario (normalmente, una fila de la tabla por usuario).
  \item Tendrás una vista en views.py que se encargará de generar la hoja CSS a partir de la plantilla. Esa vista es la que comprobará si la petición que está atendiendo corresponde a un usuario (en cuyo caso tendrá que obtener los valores para ese usuario de la tabla anterior), o no (en cuyo caso usará valores por defecto). Con los valores que obtenga, generará la hoja CSS a partir de la plantilla anterior.
  \item Por último, en urls.py tendrás una línea para indicar que si te piden el recurso que sirve la hoja de estilo, llamas a la vista anterior.
  \end{itemize}

\item ¿Qué partes de la página tiene que modificar el CSS ``customizable'' del usuario? En el enunciado de la práctica dice ``se usarán hojas CSS para cambiar al menos el tamaño y color de la letra, y el color del fondo, para los elementos marcados con un id, tal y como se especifica en el apartado anterior''. En el ``apartado anterior'' lo que se especifica es que el banner, caja de login, menú y pie de página tienen que ir cada uno en un elemento div con una id. ¿Significa esto que el CSS que personaliza el usuario se aplica solo a esos cuatro elementos, o aplica a toda la página? ¿En el caso de ser a cada uno de los cuatro elementos, debería el usuario poder modificar el color y letra de cada uno de ellos por separado, o aplicaría para los cuatro el mismo estilo?

  Creemos que el enunciado no es ambiguo. Debe haber, por un lado, estilos CSS que afecten, como mínimo, al tamaño y color de la letra, y al color de fondo, de los elementos que es obligatorio marcar con un id, según indica el enunciado (efectivamente, el banner, la caja de login, etc.) Pueden llevar todos los mismos valores, o valores diferentes, como quiera quien realice la práctica, pero los estilos tienen que estar aplicados específicamente a esos ids.

  Por otro lado, el usuario puede especificar unos cuantos valores para toda la página (según indica el enunciado: ``Un formulario para cambiar el estilo CSS de todo el sitio para ese usuario. Bastará con que se pueda cambiar el tamaño y el color de la letra y el color de fondo. Si se cambian estos valores, quedará cambiado el documento CSS que utilizarán todas las páginas del sitio para este usuario.`` Esto es, al indicar en el formulario valores para lo que puede personalizar el usuario (como mínimo el color de la letra y el color de fondo) estos valores se cambiarán para todo el sitio. Este color de letra y de fondo pueden aplicarse a todos los elementos que se muestren en el sitio, o sólo a algunos de ellos (por ejemplo, a todos los que no se ven afectados por los id mencionados anteriormente), según quiera el alumno. Lo importante es que el cambio afecte, en los elementos que se vean afectados, a todas las páginas del sitio. Naturalmente, si se decide cambiar por ejemplo la apariencia de todos los elementos del sitio, eso afectará también a los que tengan id Por eso quizás no sea una buena idea cambiar también estos elementos, desde el punto de vista estético, dado que quizás sea mejor que aparezcan con u  color de letra y/o de fondo diferente. Pero eso queda como decisión del alumno.
  
\item Si decido trabajar en la opción de despliegue de la aplicación, ¿dónde puedo realizar este despliegue?

  El despliegue puede realizarse en caulquier ordenador que esté conectado permanentemente a Internet durante el periodo de correción, en una dirección accesible desde cualquier navegador conectado a su vez a Internet. Esto puede ser por ejemplo un ordenador personal en un domicilio con acceso permanente a Intener, adecaudamente configurado (puede ser una Raspberry Pi o similar, si se busca una solución simple y de bajo coste). También puede ser un servicio en Internet, por ejemplo uno gratuito como los que ofrecen Google (instrucciones\footnote{GCP Quickstart Using a Linux VM:\\ \url{https://cloud.google.com/compute/docs/quickstart-linux}}, precios\footnote{Google Compute Engine Pricing:\\ \url{https://cloud.google.com/compute/pricing}}), Heroku (instrucciones\footnote{Heroku Deploying with Git:\\ \url{https://devcenter.heroku.com/categories/deploying-with-git}}, características\footnote{Heroku Free Dyno Hours:\\ \url{https://devcenter.heroku.com/articles/free-dyno-hours}}) o PythonAmywhere (instrucciones\footnote{Capítulo ``Deploy!'' de Django Girls Tutorial:\\ \url{https://tutorial.djangogirls.org/en/deploy/}}, precios\footnote{PythonAnywhere Plans and Pricing:\\ \url{https://www.pythonanywhere.com/pricing/}}).

\item Para las URLs de los documentos XML que ofrece MiTiempo, ¿puedo usar la terminación \verb|format=xml| en lugar de \verb|?format=xml| ?

  Sí. Debido a un error, los primeros enunciados mencionaban la terminación \verb|format=xml| para estos ficheros. Por ello, el alumno puede elegir entre servirlos con ese nombre de recurso, o con el que indica la versión final del enunciado, \verb|?format=xml|. Si aún no se ha realizado la implementación de ninguna de las dos formas, se recomienda hacerlo como indica el enunciado definitivo, porque eso permitirá utilizar la misma vista (view) que se utiliza para el documento HTML correspondiente, simplemente comprobando si la petición incluye una ``query string'' (utilizando los mecanismos pertinentes de Django). Pero como se ha dicho, si el alumno prefiere implementarlo de la otra forma, se considerara de la misma manera.
\end{itemize}


%%----------------------------------------------------------------------------
%%----------------------------------------------------------------------------
\section{Práctica final (2019, junio)}
\label{practica-final-2019-06}

La práctica final para la convocatoria de junio de 2019 será la misma que la descrita para la convocatoria de mayo de 2019, con las siguientes consideraciones:

\begin{itemize}
  \item En vez de ``\emph{La página principal se ofrece rá también como un documento XML, que incluirá la misma lista de municipios, y un enlace al fichero XML que proporciona para cada uno de ellos la AEMET. Este documento se ofrecerá cuando se pida la URL de municipios, concatenando al final} \verb|?format=xml|' ahora será ``\emph{La página principal se ofrecerá también como un documento JSON, que incluirá la misma lista de municipios, y un enlace al fichero XML que proporciona para cada uno de ellos la AEMET. Este documento se ofrecerá cuando se pida la URL de municipios, concatenando al final} \verb|?format=json|''.
  \item En vez de ``\emph{Mostrará un listado de los 10 municipios con más comentarios. Si no hubiera 10 municipios con comentarios, se mostrarán sólo los que tengan comentarios}'' ahora será ``\emph{Mostrará un listado de los 10 municipios más seleccionados por los usuarios. Si no hubiera 10 municipios seleccionados, se mostrarán sólo los que se hayan seleccionado}''.
\end{itemize}

Las fechas de entrega, publicación y revisión de esta convocatoria quedan como siguen:

\begin{itemize}
  \item \textbf{Fecha límite de entrega de la práctica:} lunes, 1 de julio de 2019 a las 05:00 (hora española peninsular)\footnote{Entiéndase la hora como domingo por la noche, ya entrado el lunes.}.

  \item \textbf{Fecha de publicación de notas:} miércoles, 3 de julio de 2019, en la plataforma Moodle.

  \item \textbf{Fecha de revisión:} viernes, 5 de julio de 2019 a las 10:00.

\end{itemize}


\newpage

\newpage


%%----------------------------------------------------------------------------
%%----------------------------------------------------------------------------
\section{Práctica final (2018, mayo)}
\label{practica-final-2018-05}

La práctica final de la asignatura consiste en la creación de una aplicación web que aglutine información sobre museos de la ciudad de Madrid. A continuación se describe el funcionamiento y arquitectura general de la aplicación, la funcionalidad mínima que debe proporcionar, y otra funcionalidad optativa que podrá tener.

La aplicación se encargará de descargar información sobre los mencionados museos, disponibles públicamente en varios formatos en el portal de datos abiertos de Madrid, y de ofrecerlos a los usuarios para que puedan seleccionar los que les parezca más interesantes, y comentar sobre ellos. De esta manera, un escenario típico es el de un usuario que a partir de los museos disponibles, elija los que le parezca más adecuados, y comente sobre los que quiera.

%%----------------------------------------------------------------------------
\subsection{Arquitectura y funcionamiento general}

Arquitectura general:

\begin{itemize}

  \item La práctica se construirá como un proyecto Django/Python3, que incluirá una o varias aplicaciones Django que implementen la funcionalidad requerida.

  \item Para el almacenamiento de datos persistente se usará SQLite3, con tablas definidas en modelos de Django.

  \item Se usará la aplicación Django ``Admin Site'' para crear cuenta a los usuarios en el sistema, y para la gestión general de las bases de datos necesarias. Todas las bases de datos que contenga la aplicación tendrá que ser accesible vía este ``Admin Site''.

  \item Se utilizarán plantillas Django (a ser posible, una jerarquía de plantillas, para que la práctica tenga un aspecto similar) para definir las páginas que se servirán a los navegadores de los usuarios. Estas plantillas incluirán en todas las páginas al menos:
  \begin{itemize}
  \item Un \emph{banner} (imagen) del sitio, en la parte superior izquierda.
  \item Una caja para entrar (hacer login en el sitio), o para salir (si ya se ha entrado).
  \begin{itemize}
    \item En caso de que no se haya entrado en una cuenta, esta caja permitirá al visitante introducir su identificador de usuario y su contraseña. 
    \item En caso de que ya se haya entrado, esta caja mostrará el identificador del usuario y permitirá salir de la cuenta (logout). Esta caja aparecerá en la parte superior derecha.
  \end{itemize}
  \item Un menú de opciones, como barra, debajo de los dos elementos anteriores (banner y caja de entrada o salida).
  \item Un pie de página con una nota de atribución, indicando ``Esta aplicación utiliza datos del portal de datos abiertos de la ciudad de Madrid'', y un enlace a la página con los datos, y a la descripción de los mismos (ver enlaces más abajo).
  \end{itemize}

Cada una de estas partes estará marcada con propiedades ``id'' en HTML, para poder ser referenciadas en hojas de estilo CSS.

\item Se utilizarán hojas de estilo CSS para determinar la apariencia de la práctica. Estas hojas definirán al menos el color y el tamaño de la letra, y el color de fondo de cada una de las partes (elementos) marcadas con id que se indican en el apartado anterior.

\item Se utilizará, para componer la información sobre museos, la disponible en el portal de datos abiertos de la ciudad de Madrid:

  \item Fichero con los datos abiertos de museos proporcionado por el Ayuntamiento de Madrid: \\
    \url{https://datos.madrid.es/portal/site/egob/menuitem.c05c1f754a33a9fbe4b2e4b284f1a5a0/?vgnextoid=118f2fdbecc63410VgnVCM1000000b205a0aRCRD&vgnextchannel=374512b9ace9f310VgnVCM100000171f5a0aRCRD&vgnextfmt=default}

  \item Copia del fichero anterior en el repositorio CursosWeb/Code de GitHub: \\
    \url{https://github.com/CursosWeb/CursosWeb.github.io/tree/master/etc}
\end{itemize}

Funcionamiento general:

\begin{itemize}
  \item Los usuarios serán dados de alta en la práctica mediante el módulo ``Admin Site'' de Django. Una vez estén dados de alta, serán considerados ``usuarios registrados''.

  \item El listado de museos se cargará a partir del fichero XML cuando un usuario indique que se carguen. Hasta que algún usuario indique por primera vez que se carguen los datos, no habrá listado de museos en la base de datos te la aplicación.

  \item Los usuarios registrados podrán crear su selección de museos. Para ello, dispondrán de una página personal. Llamaremos a esta página la ``página del usuario''.

  \item La selección de museos en su página personal la realizará cada usuario a partir de información sobre museos ya disponibles en el sitio.

  \item Cualquier navegador podrá acceder a la interfaz pública del sitio, que ofrecerá la página personal de cada usuario, para todos los usuarios del sitio.

  \item Cualquier usuario podrá indicar que quiere una vista del sitio que incluya sólo los museos (los que en XML tienen el atributo de nombre ``Accesibilidad'' con valor ``1'').
\end{itemize}


%%----------------------------------------------------------------------------
\subsection{Funcionalidad mínima}

Los museos se obtendrán a partir de la información pública ofrecida por el Ayuntamiento de Madrid en el Portal de Datos Abiertos, en forma de ficheros XML, como se indicaba anteriormente.

La {\bf interfaz pública} contiene los recursos a servir como páginas HTML completas (pensadas para ser vistas en el navegador) para cualquier visitante (sea usuario registrado o no):

\begin{itemize}
  \item /: Página principal de la práctica. Constará de un listado de museos y otro con enlaces a páginas personales:
  
  \begin{enumerate}
    \item Mostrará un listado de los cinco museos con más comentarios. Si no hubiera 5 museos con comentarios, se mostrarán sólo los que tengan comentarios. Para cada museo, incluirá información sobre:
    \begin{itemize}
      \item su nombre (que será un enlace que apuntará a la URL del museo en el portal esmadrid), 
      \item su dirección,
      \item y un enlace, ``Más información'', que apuntará a la página del museo en la aplicación (ver más adelante).
    \end{itemize}
   
  \item También se mostrará un listado, en una columna lateral, con enlaces a las páginas personales disponibles. Para cada página personal mostrará el título que le haya dado su usuario (como un enlace a la página personal en cuestión) y el nombre del usuario. Si a una página personal aún no se le hubiera puesto título, este título será ``Página de usuario'', donde ``usuario'' es el identificador de usuario del usuario en cuestión.
    \item Tambień se mostrará un botón, que al pulsarlo se pasará a ver en todos los listados los museos accesibles, y sólo estos. Si se vuelve a pulsar, se volverán a ver todos los museos.
   \end{enumerate}

  \item /{usuario}: Página personal de un usuario. Si la URL es ``/usuario'', es que corresponde al usuario ``usuario''. Mostrará los museos seleccionados por ese usuario (aunque no puede haber más de 5 a la vez; si hay más debería haber un enlace para mostrar las 5 siguientes y así en adelante, siempre de 5 en 5). Para cada museo se mostrará la misma información que en la página principal. Además, para cada museo se deberá mostrar la fecha en la que fue seleccionada por el usuario.

  \item /museos: Página con todos los museos. Para cada uno de ellos aparecerá sólo el nombre, y un enlace a su página. En la parte superior de la página, existirá un formulario que permita filtrar estos museos según el distrito. Para poder filtrar por distrito, se buscará en la base de datos cuáles son los distritos con museos.

  \item /museos/{id}: Página de un museo en la aplicación. Mostrará toda la información razonablemente posible de XML del portal de datos abierto del Ayuntamiento de Madrid, incluyendo al menos la que se menciona en otros apartados de este enunciado, la dirección, la descripción, si es accesible o no, el barrio y el distrito, y los datos de contacto. Además, se mostrarán todos los comentarios que se hayan puesto para este museo.
  
  \item /{usuario}/xml: Canal XML para los museos seleccionados por ese usuario. El documento XML tendrá una entrada para cada museo seleccionado por el usuario, y tendrá una estructura similar (pero no necesariamente igual) a la del fichero XML del portal del Ayuntamiento.

  \item /about: Página con información en HTML indicando la autoría de la práctica, explicando su funcionamiento.

\end{itemize}

Todas las páginas de la interfaz pública incluirán un menú desde el que se podrá acceder a todos los museos (URL /museos) con el texto ``Todos'' y a la ayuda (URL /about) con el texto ``About''. Todas las página que no sean la principal tendrán otra opción de menú para la URL /, con el texto ``Inicio''.

La {\bf interfaz privada} contiene los recursos a servir como páginas HTML completas para usuarios registrados (una vez se han autenticado):

\begin{itemize}
  \item Todos los recursos de la interfaz pública.
  
  \item /museos/{id}: Además de la información que se muestra de manera pública:

    \begin{enumerate}
      \item Un formulario para poner comentarios sobre este museo. Los comentarios serán anónimos, pero sólo se podrán poner por los usuarios registrados, una vez se han autenticado. Por tanto, bastará con que este formulario esté compuesto por una caja de texto, donde se podrá escribir el comentario, y un botón para enviarlo.
  \end{enumerate}

  \item /{usuario}: Además de la información que se muestra de manera pública:
  
  \begin{enumerate}
    \item Un formulario para cambiar el estilo CSS de todo el sitio para ese usuario. Bastará con que se pueda cambiar el tamaño de la letra y el color de fondo. Si se cambian estos valores, quedará cambiado el documento CSS que utilizarán todas las páginas del sitio para este usuario. Este cambio será visible en cuanto se suba la nueva página CSS.

    \item Un formulario para elegir el título de su página personal.
  \end{enumerate}
\end{itemize}

%Si es preciso, se añadirán más recursos (pero sólo si es realmente preciso) para poder satisfacer los requisitos especificados.

Dados los recursos mencionados anteriormente, no se permitirán los nombres de usuario ``museos'' ni ``about'' (pero no hay que hacer ninguna comprobación para esto: se asume que no se darán de alta esos usuarios en el Admin Site).


%%----------------------------------------------------------------------------
\subsection{Funcionalidad optativa}

De forma optativa, se podrá incluir cualquier funcionalidad relevante en el contexto de la asignatura. Se valorarán especialmente las funcionalidades que impliquen el uso de técnicas nuevas, o de aspectos de Django no utilizados en los ejercicios previos, y que tengan sentido en el contexto de esta práctica y de la asignatura.

En el formulario de entrega se pide que se justifique por qué se considera funcionalidad optativa lo que habeis implementado. Sólo a modo de sugerencia, se incluyen algunas posibles funcionalidades optativas:

\begin{itemize}
  \item Inclusión de un \emph{favicon} del sitio
  
  \item Generación de un canal XML para los contenidos que se muestran en la página principal.

  \item Generación de canales, pero con los contenidos en JSON

  \item Generación de un canal RSS para los comentarios puestos en el sitio.
  
  \item Funcionalidad para leer los datos del Ayuntamiento en otros formatos diferentes a XML: CSV, JSON...
  
  \item Funcionalidad de registro de usuarios
  
  \item Uso de Javascript o AJAX para algún aspecto de la práctica (por ejemplo, para seleccionar un museo para una página de usuario).

  \item Puntuación de museos. Cada visitante (registrado o no) puede dar un ``+1'' a cualquier museo del sitio. La suma de ``+'' que ha obtenido un museo se verá cada vez que se vea el museo en el sitio.
  
  \item Uso de elementos HTML5 (especificar cuáles al entregar)

  \item Atención al idioma indicado por el navegador. El idioma de la interfaz de usuario del planeta tendrá en cuenta lo que especifique el navegador.

\end{itemize}


%%----------------------------------------------------------------------------
\subsection{Entrega de la práctica}

\begin{itemize}
  \item \textbf{Fecha límite de entrega de la práctica:} lunes, 21 de mayo de 2018 a las 03:00 (hora española peninsular)\footnote{Entiéndase la hora como domingo por la noche, ya entrado en lunes.}
       %{\bf Convocatoria de junio:} miércoles, 24 de junio de 2015 a las 23:59 (hora peninsular española).

  \item \textbf{Fecha de publicación de notas:} miércoles, 23 de mayo de 2018, en la plataforma Moodle.
%{\bf Convocatoria de junio:} viernes, 26 de junio, en la plataforma Moodle.

  \item \textbf{Fecha de revisión:} jueves, 24 de mayo de 2018 a las 13:00.
%{\bf Convocatoria de junio:} martes, 30 de junio a las 13:30. Se requerirá a algunos alumnos que asistan a la revisión {\bf en persona}; se informará de ello en el mensaje de publicación de notas.
\end{itemize}

La entrega de la práctica consiste en rellenar un formulario (enlazado en el Moodle de la asignatura) y en seguir las instrucciones que se describen a continuación.

\begin{enumerate}
  \item El repositorio contendrá todos los ficheros necesarios para que funcione la aplicación (ver detalle más abajo). Es muy importante que el alumno haya realizado un fork del repositorio que se indica a continuación, porque si no, la práctica no podrá ser identificada: 

\url{https://github.com/CursosWeb/X-Serv-Practica-Museos}

Los alumnos que no entreguen las práctica de esta forma serán considerados como no presentados en lo que a la entrega de prácticas se refiere. Los que la entreguen podrán ser llamados a realizar también una entrega presencial, que tendrá lugar en la fecha y hora de la revisión. Esta entrega presencial podrá incluir una conversación con el profesor sobre cualquier aspecto de la realización de la práctica.

Recordad que es importante ir haciendo commits de vez en cuando y que sólo al hacer push estos commits son públicos. Antes de entregar la práctica, haced un push. Y cuando la entreguéis y sepáis el nombre del repositorio, podéis cambiar el nombre del repositorio desde el interfaz web de GitHub. 
 
 \item Un vídeo de demostración de la parte obligatoria, y otro vídeo de demostración de la parte opcional, si se han realizado opciones avanzadas. Los vídeos serán de una {\bf duración máxima de 3 minutos} (cada uno), y consistirán en una captura de pantalla de un navegador web utilizando la aplicación, y mostrando lo mejor posible la funcionalidad correspondiente (básica u opcional). Siempre que sea posible, el alumno comentará en el audio del vídeo lo que vaya ocurriendo en la captura. Los vídeos se colocarán en algún servicio de subida de vídeos en Internet (por ejemplo, Vimeo o YouTube). Los vídeos de más de tres minutos tendrán penalización.

Hay muchas herramientas que permiten realizar la captura de pantalla. Por ejemplo, en GNU/Linux puede usarse Gtk-RecordMyDesktop o Istanbul (ambas disponibles en Ubuntu). Es importante que la captura sea realizada de forma que se distinga razonablemente lo que se grabe en el vídeo.

En caso de que convenga editar el vídeo resultante (por ejemplo, para eliminar tiempos de espera) puede usarse un editor de vídeo, pero siempre deberá ser indicado que se ha hecho tal cosa con un comentario en el audio, o un texto en el vídeo. Hay muchas herramientas que permiten realizar esta edición. Por ejemplo, en GNU/Linux puede usarse OpenShot o PiTiVi.

  \item Se han de entregar los siguientes ficheros:

\begin{itemize}
  \item Un fichero README.md que resuma las mejoras, si las hay, y explique cualquier peculiaridad de la entrega (ver siguiente punto).
  \item El repositorio GitHub deberá contener un proyecto Django completo y listo para funcionar en el entorno del laboratorio, incluyendo la base de datos con datos suficientes como para poder probarlo. Estos datos incluirán al menos dos usuarios con sus datos correspondientes, con al menos cinco museos en su página personal, y con al menos cinco comentarios en total.
  \item Cualquier biblioteca Python que pueda hacer falta para que la aplicación funcione, junto con los ficheros auxiliares que utilice, si es que los utiliza.
\end{itemize}

  \item Se incluirán en el fichero README.md los siguientes datos (la mayoría de estos datos se piden también en el formulario que se ha de rellenar para entregar la práctica - se recomienda hacer un corta y pega de estos datos en el formulario):

\begin{itemize}
  \item Nombre y titulación.
  \item Nombre de su cuenta en el laboratorio del alumno.
  \item Nombre de usuario en GitHub.
  \item Resumen de las peculiaridades que se quieran mencionar sobre lo implementado en la parte obligatoria.
  \item Lista de funcionalidades opcionales que se hayan implementado, y breve descripción de cada una.
  \item URL del vídeo demostración de la funcionalidad básica
  \item URL del vídeo demostración de la funcionalidad optativa, si se ha realizado funcionalidad optativa
\end{itemize}

Asegúrate de que las URLs incluidas en este fichero están adecuadamente escritas en Markdown, de forma que la versión HTML que genera GitHub los incluya como enlaces ``pinchables''.

\end{enumerate}


%%----------------------------------------------------------------------------
\subsection{Notas y comentarios}

La práctica deberá funcionar en el entorno GNU/Linux (Ubuntu) del laboratorio de la asignatura con la versión de Django que se ha usado en prácticas.

La práctica deberá funcionar desde el navegador Firefox disponible en el laboratorio de la asignatura.

Los canales (feeds) RSS que produce la aplicación web realizada en la práctica deberán funcionar al menos con el navegador Firefox (considerándolos como canales RSS) disponibles en el laboratorio. Los documentos XML deberán ser correctos desde el punto de vista de la sintaxis XML, y por lo tanto reconocibles por un reconocedor XML, como por ejemplo el reconocedor del módulo xml.sax de Python.


%%----------------------------------------------------------------------------
%%----------------------------------------------------------------------------
\section{Práctica final (2018, junio)}
\label{practica-final-2018-06}

La práctica final para la convocatoria de junio de 2018 será la misma que la descrita para la convocatoria de mayo de 2018, con las siguientes consideraciones:

\begin{itemize}
  \item En vez de /{usuario}/xml: Canal XML para los museos seleccionados por ese usuario, se ofrecerá el canal en formato JSON. El documento JSON tendrá una entrada para cada museo seleccionado por el usuario, y tendrá una estructura similar (pero no necesariamente igual) a la del fichero JSON del portal del Ayuntamiento.
  \item La página principal no mostrará los cinco museos más comentados, sino que mostrará los cinco museos más seleccionados por usuarios para sus páginas personales.
\end{itemize}

Las fechas de entrega, publicación y revisión de esta convocatoria quedan como siguen:

\begin{itemize}
  \item \textbf{Fecha límite de entrega de la práctica:} viernes, 28 de junio de 2018 a las 05:00 (hora española peninsular)\footnote{Entiéndase la hora como jueves por la noche, ya entrado el viernes.}.
       %{\bf Convocatoria de junio:} miércoles, 24 de junio de 2015 a las 23:59 (hora peninsular española).

  \item \textbf{Fecha de publicación de notas:} domingo, 1 de julio de 2018, en la plataforma Moodle.
%{\bf Convocatoria de junio:} viernes, 26 de junio, en la plataforma Moodle.

  \item \textbf{Fecha de revisión:} martes, 3 de julio de 2018 a las 12:00.
%{\bf Convocatoria de junio:} martes, 30 de junio a las 13:30. Se requerirá a algunos alumnos que asistan a la revisión {\bf en persona}; se informará de ello en el mensaje de publicación de notas.
\end{itemize}


\newpage

\newpage

%%----------------------------------------------------------------------------
%%----------------------------------------------------------------------------
\section{Práctica final (2017, mayo)}
\label{practica-final-2017-05}

La práctica final de la asignatura consiste en la creación de una aplicación web que aglutine información sobre aparcamientos en la ciudad de Madrid. A continuación se describe el funcionamiento y arquitectura general de la aplicación, la funcionalidad mínima que debe proporcionar, y otra funcionalidad optativa que podrá tener.

La aplicación se encargará de descargar información sobre los mencionados aparcamientos, disponibles públicamente en formato XML en el portal de datos abiertos de Madrid, y de ofrecerlos a los usuarios para que puedan seleccionar los que les parezca más interesantes, y comentar sobre ellos. De esta manera, un escenario típico es el de un usuario que a partir de los aparcamientos disponibles, elija los que le parezca más adecuados, y comente sobre los que quiera.

%%----------------------------------------------------------------------------
\subsection{Arquitectura y funcionamiento general}

Arquitectura general:

\begin{itemize}

  \item La práctica se construirá como un proyecto Django/Python3, que incluirá una o varias aplicaciones Django que implementen la funcionalidad requerida.

  \item Para el almacenamiento de datos persistente se usará SQLite3, con tablas definidas en modelos de Django.

  \item Se usará la aplicación Django ``Admin Site'' para crear cuenta a los usuarios en el sistema, y para la gestión general de las bases de datos necesarias. Todas las bases de datos que contenga la aplicación tendrá que ser accesible vía este ``Admin Site''.

  \item Se utilizarán plantillas Django (a ser posible, una jerarquía de plantillas, para que la práctica tenga un aspecto similar) para definir las páginas que se servirán a los navegadores de los usuarios. Estas plantillas incluirán en todas las páginas al menos:
  \begin{itemize}
  \item Un \emph{banner} (imagen) del sitio, en la parte superior izquierda.
  \item Una caja para entrar (hacer login en el sitio), o para salir (si ya se ha entrado).
  \begin{itemize}
    \item En caso de que no se haya entrado en una cuenta, esta caja permitirá al visitante introducir su identificador de usuario y su contraseña. 
    \item En caso de que ya se haya entrado, esta caja mostrará el identificador del usuario y permitirá salir de la cuenta (logout). Esta caja aparecerá en la parte superior derecha.
  \end{itemize}
  \item Un menú de opciones, como barra, debajo de los dos elementos anteriores (banner y caja de entrada o salida).
  \item Un pie de página con una nota de atribución, indicando ``Esta aplicación utiliza datos del portal de datos abiertos de la ciudad de Madrid'', y un enlace al XML con los datos, y a la descripción de los mismos (ver enlaces más abajo).
  \end{itemize}

Cada una de estas partes estará marcada con propiedades ``id'' en HTML, para poder ser referenciadas en hojas de estilo CSS.

\item Se utilizarán hojas de estilo CSS para determinar la apariencia de la práctica. Estas hojas definirán al menos el color y el tamaño de la letra, y el color de fondo de cada una de las partes (elementos) marcadas con id que se indican en el apartado anterior.

\item Se utilizará, para componer la información sobre aparcamientos disponibles, la disponible en el portal de datos abiertos de la ciudad de Madrid:

  \item Fichero con los datos abiertos de aparcamientos para residentes proporcionado por el Ayuntamiento de Madrid: \\
    \url{http://datos.munimadrid.es/portal/site/egob/menuitem.ac61933d6ee3c31cae77ae7784f1a5a0/?vgnextoid=00149033f2201410VgnVCM100000171f5a0aRCRD&format=xml&file=0&filename=202584-0-aparcamientos-residentes&mgmtid=e84276ac109d3410VgnVCM2000000c205a0aRCRD&preview=full}

  \item Descripción del fichero: \\
    \url{http://datos.madrid.es/portal/site/egob/menuitem.c05c1f754a33a9fbe4b2e4b284f1a5a0/?vgnextoid=e84276ac109d3410VgnVCM2000000c205a0aRCRD&vgnextchannel=374512b9ace9f310VgnVCM100000171f5a0aRCRD&vgnextfmt=default}
      
  \item Copia del fichero anterior en el repositorio CursosWeb/Code de GitHub: \\
\end{itemize}

Funcionamiento general:

\begin{itemize}
  \item Los usuarios serán dados de alta en la práctica mediante el módulo ``Admin Site'' de Django. Una vez estén dados de alta, serán considerados ``usuarios registrados''.

  \item El listado de aparcamientos se cargará a partir del fichero XML cuando un usuario indique que se carguen. Hasta que algún usuario indique por primera vez que se carguen los datos, no habrá listado de aparcamientos en la base de datos te la aplicación.

  \item Los usuarios registrados podrán crear su selección de aparcamientos. Para ello, dispondrán de una página personal. Llamaremos a esta página la ``página del usuario''.

  \item La selección de aparcamientos en su página personal la realizará cada usuario a partir de información sobre aparcamientos ya disponibles en el sitio.

  \item Cualquier navegador podrá acceder a la interfaz pública del sitio, que ofrecerá la página personal de cada usuario, para todos los usuarios del sitio.

  \item Cualquier usuario podrá indicar que quiere una vista del sitio que incluya sólo los aparcamientos accesibles (los que en XML tienen ``accesibility'' con valor ``1'').
\end{itemize}


%%----------------------------------------------------------------------------
\subsection{Funcionalidad mínima}

Los aparcamientos se obtendrán a partir de la información pública ofrecida por el Ayuntamiento de Madrid en el Portal de Datos Abiertos, en forma de ficheros XML, como se indicaba anteriormente.

La {\bf interfaz pública} contiene los recursos a servir como páginas HTML completas (pensadas para ser vistas en el navegador) para cualquier visitante (sea usuario registrado o no):

\begin{itemize}
  \item /: Página principal de la práctica. Constará de un listado de aparcamientos y otro con enlaces a páginas personales:
  
  \begin{enumerate}
    \item Mostrará un listado de los cinco aparcamientos con más comentarios. Si no hubiera 5 aparcamientos con comentarios, se mostrarán sólo los que tengan comentarios. Para cada aparcamiento, incluirá información sobre:
    \begin{itemize}
      \item su nombre (que será un enlace que apuntará a la url del aparcamiento en el portal esmadrid), 
      \item su dirección,
      \item y un enlace, ``Más información'', que apuntará a la página del aparcamiento en la aplicación (ver más adelante).
    \end{itemize}
   
  \item También se mostrará un listado, en una columna lateral, con enlaces a las páginas personales disponibles. Para cada página personal mostrará el título que le haya dado su usuario (como un enlace a la página personal en cuestión) y el nombre del usuario. Si a una página personal aún no se le hubiera puesto título, este título será ``Página de usuario'', donde ``usuario'' es el identificador de usuario del usuario en cuestión.
    \item Tambień se mostrará un botón, que al pulsarlo se pasará a ver en todos los listados los aparcamientos accesibles, y sólo estos. Si se vuelve a pulsar, se volverán a ver todos los aparcamientos.
   \end{enumerate}

  \item /{usuario}: Página personal de un usuario. Si la URL es ``/usuario'', es que corresponde al usuario ``usuario''. Mostrará los aparcamientos seleccionados por ese usuario (aunque no puede haber más de 5 a la vez; si hay más debería haber un enlace para mostrar las 5 siguientes y así en adelante, siempre de 5 en 5). Para cada aparcamiento se mostrará la misma información que en la página principal. Además, para cada aparcamiento se deberá mostrar la fecha en la que fue seleccionada por el usuario.

  \item /aparcamientos: Página con todos los aparcamientos. Para cada uno de ellos aparecerá sólo el nombre, y un enlace a su página. En la parte superior de la página, existirá un formulario que permita filtrar estos aparcamientos según el distrito. Para poder filtrar por distrito, se buscará en la base de datos cuáles son los distritos con aparcamientos.

  \item /aparcamientos/{id}: Página de un aparcamiento en la aplicación. Mostrará toda la información razonablemente posible de XML del portal de datos abierto del Ayuntamiento de Madrid, incluyendo al menos la que se menciona en otros apartados de este enunciado, la información de latitud y longitud, la descripción, si es accesible o no, el barrio y el distrito, y los datos de contacto. Además, se mostrarán todos los comentarios que se hayan puesto para este aparcamiento.
  
  \item /{usuario}/xml: Canal XML para los aparcamientos seleccionados por ese usuario. El documento XML tendrá una entrada para cada aparcamiento seleccionado por el usuario, y tendrá una estructura similar (pero no necesariamente igual) a la del fichero XML del portal del Ayuntamiento.

  \item /about: Página con información en HTML indicando la autoría de la práctica y explicando su funcionamiento.

\end{itemize}

Todas las páginas de la interfaz pública incluirán un menú desde el que se podrá acceder a todos los aparcamientos (URL /aparcamientos) con el texto ``Todos'' y a la ayuda (URL /about) con el texto ``About''. Todas las página que no sean la principal tendrán otra opción de menú para la URL /, con el texto ``Inicio''.

La {\bf interfaz privada} contiene los recursos a servir como páginas HTML completas para usuarios registrados (una vez se han autenticado):

\begin{itemize}
  \item Todos los recursos de la interfaz pública.
  
  \item /aparcamientos/{id}: Además de la información que se muestra de manera pública:

    \begin{enumerate}
      \item Un formulario para poner comentarios sobre este aparcamiento. Los comentarios serán anónimos, pero sólo se podrán poner por los usuarios registrados, una vez se han autenticado. Por tanto, bastará con que este formulario esté compuesto por una caja de texto, donde se podrá escribir el comentario, y un botón para enviarlo.
  \end{enumerate}

  \item /{usuario}: Además de la información que se muestra de manera pública:
  
  \begin{enumerate}
    \item Un formulario para cambiar el estilo CSS de todo el sitio para ese usuario. Bastará con que se pueda cambiar el tamaño de la letra y el color de fondo. Si se cambian estos valores, quedará cambiado el documento CSS que utilizarán todas las páginas del sitio para este usuario. Este cambio será visible en cuanto se suba la nueva página CSS.

    \item Un formulario para elegir el título de su página personal.
  \end{enumerate}
\end{itemize}

%Si es preciso, se añadirán más recursos (pero sólo si es realmente preciso) para poder satisfacer los requisitos especificados.

Dados los recursos mencionados anteriormente, no se permitirán los nombres de usuario ``aparcamientos'' ni ``about'' (pero no hay que hacer ninguna comprobación para esto: se asume que no se darán de alta esos usuarios en el Admin Site).


%%----------------------------------------------------------------------------
\subsection{Funcionalidad optativa}

De forma optativa, se podrá incluir cualquier funcionalidad relevante en el contexto de la asignatura. Se valorarán especialmente las funcionalidades que impliquen el uso de técnicas nuevas, o de aspectos de Django no utilizados en los ejercicios previos, y que tengan sentido en el contexto de esta práctica y de la asignatura.

En el formulario de entrega se pide que se justifique por qué se considera funcionalidad optativa lo que habeis implementado. Sólo a modo de sugerencia, se incluyen algunas posibles funcionalidades optativas:

\begin{itemize}
  \item Inclusión de un \emph{favicon} del sitio
  
  \item Generación de un canal XML para los contenidos que se muestran en la página principal.

  \item Generación de canales, pero con los contenidos en JSON

  \item Generación de un canal RSS para los comentarios puestos en el sitio.
  
  \item Funcionalidad de registro de usuarios
  
  \item Uso de Javascript o AJAX para algún aspecto de la práctica (por ejemplo, para seleccionar un aparcamiento para una página de usuario).

  \item Puntuación de aparcamientos. Cada visitante (registrado o no) puede dar un ``+1'' a cualquier aparcamiento del sitio. La suma de ``+'' que ha obtenido un aparcamiento se verá cada vez que se vea el aparcamiento en el sitio.
  
  \item Uso de elementos HTML5 (especificar cuáles al entregar)

  \item Atención al idioma indicado por el navegador. El idioma de la interfaz de usuario del planeta tendrá en cuenta lo que especifique el navegador.

\end{itemize}


%%----------------------------------------------------------------------------
\subsection{Entrega de la práctica}

\begin{itemize}
  \item \textbf{Fecha límite de entrega de la práctica:} miércoles, 24 de mayo de 2017 a las 03:00 (hora española peninsular)\footnote{Entiéndase la hora como miércoles por la noche, ya entrado el jueves.}
       %{\bf Convocatoria de junio:} miércoles, 24 de junio de 2015 a las 23:59 (hora peninsular española).

  \item \textbf{Fecha de publicación de notas:} sábado, 27 de mayo de 2017, en la plataforma Moodle.
%{\bf Convocatoria de junio:} viernes, 26 de junio, en la plataforma Moodle.

  \item \textbf{Fecha de revisión:} lunes, 29 de mayo de 2017 a las 13:00.
%{\bf Convocatoria de junio:} martes, 30 de junio a las 13:30. Se requerirá a algunos alumnos que asistan a la revisión {\bf en persona}; se informará de ello en el mensaje de publicación de notas.
\end{itemize}

La entrega de la práctica consiste en rellenar un formulario (enlazado en el Moodle de la asignatura) y en seguir las instrucciones que se describen a continuación.

\begin{enumerate}
  \item El repositorio contendrá todos los ficheros necesarios para que funcione la aplicación (ver detalle más abajo). Es muy importante que el alumno haya realizado un fork del repositorio que se indica a continuación, porque si no, la práctica no podrá ser identificada: 

\url{https://github.com/CursosWeb/X-Serv-Practica-Aparcamientos/}

Los alumnos que no entreguen las práctica de esta forma serán considerados como no presentados en lo que a la entrega de prácticas se refiere. Los que la entreguen podrán ser llamados a realizar también una entrega presencial, que tendrá lugar en la fecha y hora de la revisión. Esta entrega presencial podrá incluir una conversación con el profesor sobre cualquier aspecto de la realización de la práctica.

Recordad que es importante ir haciendo commits de vez en cuando y que sólo al hacer push estos commits son públicos. Antes de entregar la práctica, haced un push. Y cuando la entreguéis y sepáis el nombre del repositorio, podéis cambiar el nombre del repositorio desde el interfaz web de GitHub. 
 
 \item Un vídeo de demostración de la parte obligatoria, y otro vídeo de demostración de la parte opcional, si se han realizado opciones avanzadas. Los vídeos serán de una duración máxima de 3 minutos (cada uno), y consistirán en una captura de pantalla de un navegador web utilizando la aplicación, y mostrando lo mejor posible la funcionalidad correspondiente (básica u opcional). Siempre que sea posible, el alumno comentará en el audio del vídeo lo que vaya ocurriendo en la captura. Los vídeos se colocarán en algún servicio de subida de vídeos en Internet (por ejemplo, Vimeo o YouTube).

Hay muchas herramientas que permiten realizar la captura de pantalla. Por ejemplo, en GNU/Linux puede usarse Gtk-RecordMyDesktop o Istanbul (ambas disponibles en Ubuntu). Es importante que la captura sea realizada de forma que se distinga razonablemente lo que se grabe en el vídeo.

En caso de que convenga editar el vídeo resultante (por ejemplo, para eliminar tiempos de espera) puede usarse un editor de vídeo, pero siempre deberá ser indicado que se ha hecho tal cosa con un comentario en el audio, o un texto en el vídeo. Hay muchas herramientas que permiten realizar esta edición. Por ejemplo, en GNU/Linux puede usarse OpenShot o PiTiVi.

  \item Se han de entregar los siguientes ficheros:

\begin{itemize}
  \item Un fichero README.md que resuma las mejoras, si las hay, y explique cualquier peculiaridad de la entrega (ver siguiente punto).
  \item El repositorio GitHub deberá contener un proyecto Django completo y listo para funcionar en el entorno del laboratorio, incluyendo la base de datos con datos suficientes como para poder probarlo. Estos datos incluirán al menos dos usuarios con sus datos correspondientes, con al menos cinco aparcamientos en su página personal, y con al menos cinco comentarios en total.
  \item Cualquier biblioteca Python que pueda hacer falta para que la aplicación funcione, junto con los ficheros auxiliares que utilice, si es que los utiliza.
\end{itemize}

  \item Se incluirán en el fichero README.md los siguientes datos (la mayoría de estos datos se piden también en el formulario que se ha de rellenar para entregar la práctica - se recomienda hacer un corta y pega de estos datos en el formulario):

\begin{itemize}
  \item Nombre y titulación.
  \item Nombre de su cuenta en el laboratorio del alumno.
  \item Nombre de usuario en GitHub.
  \item Resumen de las peculiaridades que se quieran mencionar sobre lo implementado en la parte obligatoria.
  \item Lista de funcionalidades opcionales que se hayan implementado, y breve descripción de cada una.
  \item URL del vídeo demostración de la funcionalidad básica
  \item URL del vídeo demostración de la funcionalidad optativa, si se ha realizado funcionalidad optativa
\end{itemize}

Asegúrate de que las URLs incluidas en este fichero están adecuadamente escritas en Markdown, de forma que la versión HTML que genera GitHub los incluya como enlaces ``pinchables''.

\end{enumerate}


%%----------------------------------------------------------------------------
\subsection{Notas y comentarios}

La práctica deberá funcionar en el entorno GNU/Linux (Ubuntu) del laboratorio de la asignatura con la versión de Django que se ha usado en prácticas.

La práctica deberá funcionar desde el navegador Firefox disponible en el laboratorio de la asignatura.

Los canales (feeds) RSS que produce la aplicación web realizada en la práctica deberán funcionar al menos con el navegador Firefox (considerándolos como canales RSS) disponibles en el laboratorio. Los documentos XML deberán ser correctos desde el punto de vista de la sintaxis XML, y por lo tanto reconocibles por un reconocedor XML, como por ejemplo el reconocedor del módulo xml.sax de Python.


%%----------------------------------------------------------------------------
%%----------------------------------------------------------------------------
\section{Práctica final (2017, junio)}
\label{practica-final-2017-06}

La práctica final para la convocatoria de junio de 2017 será la misma que la descrita para la convocatoria de mayo de 2017, con las siguientes consideraciones:

\begin{itemize}
  \item La puntuación de aparcamientos será requisito de la práctica básica.
  \item El formulario para poner comentarios deja de ser un requisito de la práctica básica.
\end{itemize}

Las fechas de entrega, publicación y revisión de esta convocatoria quedan como siguen:

\begin{itemize}
  \item \textbf{Fecha límite de entrega de la práctica:} jueves, 29 de junio de 2017 a las 03:00 (hora española peninsular)\footnote{Entiéndase la hora como jueves por la noche, ya entrado el viernes.}.
       %{\bf Convocatoria de junio:} miércoles, 24 de junio de 2015 a las 23:59 (hora peninsular española).

  \item \textbf{Fecha de publicación de notas:} sábado, 1 de julio de 2017, en la plataforma Moodle.
%{\bf Convocatoria de junio:} viernes, 26 de junio, en la plataforma Moodle.

  \item \textbf{Fecha de revisión:} lunes, 4 de julio de 2017 a las 13:00.
%{\bf Convocatoria de junio:} martes, 30 de junio a las 13:30. Se requerirá a algunos alumnos que asistan a la revisión {\bf en persona}; se informará de ello en el mensaje de publicación de notas.
\end{itemize}


\newpage

%%----------------------------------------------------------------------------
%%----------------------------------------------------------------------------
\section{Práctica final (2016, mayo)}
\label{practica-final-2016-05}

La práctica final de la asignatura consiste en la creación de una aplicación web que aglutine información sobre alojamientos en la ciudad de Madrid. A continuación se describe el funcionamiento y arquitectura general de la aplicación, la funcionalidad mínima que debe proporcionar, y otra funcionalidad optativa que podrá tener.

La aplicación se encargará de descargar información sobre los mencionados alojamientos, disponibles públicamente en formato XML en el portal de datos abiertos de Madrid, y de ofrecerlos a los usuarios para que puedan seleccionar los que les parezca más interesantes, y comentar sobre ellos. De esta manera, un escenario típico es el de un usuario que a partir de los alojamientos disponibles, elija los que le parezca más adecuados, y comente sobre los que quiera.

%%----------------------------------------------------------------------------
\subsection{Arquitectura y funcionamiento general}

Arquitectura general:

\begin{itemize}

  \item La práctica se construirá como un proyecto Django, que incluirá una o varias aplicaciones Django que implementen la funcionalidad requerida.

  \item Para el almacenamiento de datos persistente se usará SQLite3, con tablas definidas en modelos de Django.

  \item Se usará la aplicación Django ``Admin Site'' para crear cuenta a los usuarios en el sistema, y para la gestión general de las bases de datos necesarias. Todas las bases de datos que contenga la aplicación tendrá que ser accesible vía este ``Admin Site''.

  \item Se utilizarán plantillas Django (a ser posible, una jerarquía de plantillas, para que la práctica tenga un aspecto similar) para definir las páginas que se servirán a los navegadores de los usuarios. Estas plantillas incluirán en todas las páginas al menos:
  \begin{itemize}
  \item Un \emph{banner} (imagen) del sitio, en la parte superior izquierda.
  \item Una caja para entrar (hacer login en el sitio), o para salir (si ya se ha entrado).
  \begin{itemize}
    \item En caso de que no se haya entrado en una cuenta, esta caja permitirá al visitante introducir su identificador de usuario y su contraseña. 
    \item En caso de que ya se haya entrado, esta caja mostrará el identificador del usuario y permitirá salir de la cuenta (logout). Esta caja aparecerá en la parte superior derecha.
  \end{itemize}
  \item Un menú de opciones, como barra, debajo de los dos elementos anteriores (banner y caja de entrada o salida).
  \item Un pie de página con una nota de atribución, indicando ``Esta aplicación utiliza datos del portal de datos abiertos de la ciudad de Madrid'', y un enlace al XML con los datos, y a la descripción de los mismos (ver enlaces más abajo).
  \end{itemize}

Cada una de estas partes estará marcada con propiedades ``id'' en HTML, para poder ser referenciadas en hojas de estilo CSS.

\item Se utilizarán hojas de estilo CSS para determinar la apariencia de la práctica. Estas hojas definirán al menos el color y el tamaño de la letra, y el color de fondo de cada una de las partes (elementos) marcadas con id que se indican en el apartado anterior.

\item Se utilizará, para componer la información sobre alojamientos disponibles, la disponible en el portal de datos abiertos de la ciudad de Madrid:

  \begin{itemize}
  \item Descripción: \\
    \url{http://bit.ly/1T24Zsq}
  \item Fichero XML con los datos (en español): \\
    \url{http://www.esmadrid.com/opendata/alojamientos_v1_es.xml} \\
    \url{http://cursosweb.github.io/etc/alojamientos_es.xml}
  \item Fichero XML con los datos (en inglés): \\
    \url{http://www.esmadrid.com/opendata/alojamientos_v1_en.xml} \\
    \url{http://cursosweb.github.io/etc/alojamientos_en.xml}
  \item Fichero XML con los datos (en francés): \\
    \url{http://www.esmadrid.com/opendata/alojamientos_v1_fr.xml} \\
    \url{http://cursosweb.github.io/etc/alojamientos_fr.xml}
  \item Hay ficheros XML con los datos en otros idiomas
  \end{itemize}
\end{itemize}

Funcionamiento general:

\begin{itemize}
  \item Los usuarios serán dados de alta en la práctica mediante el módulo ``Admin Site'' de Django. Una vez estén dados de alta, serán considerados ``usuarios registrados''.

  \item El listado de alojamientos se cargará a partir del XML con los datos en español sólo cuando un usuario indique que quiere que se carguen. Hasta que algún usuario indique por primera vez que se carguen los datos, no habrá listado de alojamientos en la base de datos te la aplicación.

  \item Los usuarios registrados podrán crear su selección de alojamientos. Para ello, dispondrán de una página personal. Llamaremos a esta página la ``página del usuario''.

  \item La selección de alojamientos en su página personal la realizará cada usuario a partir de información sobre alojamientos ya disponibles en el sitio.

  \item Cualquier navegador podrá acceder a la interfaz pública del sitio, que ofrecerá la página personal de cada usuario, para todos los usuarios del sitio.

  \item Cualquier usuario, al ver la página de alojamientos de cualquier usuario (incluido él mismo), podrá pedir verla en otro de los idiomas disponibles. En ese caso, la aplicación descargará el documento XML con el listado de alojamientos en el idioma elegido, buscará los alojamientos en cuestión, y usará sus datos para mostrar la misma página, pero con los datos sobre los alojamientos en ese idioma. La aplicación no almacenará estos datos en otro idioma en la base de datos, de forma que si se le vuelve a pedir lo mismo, volverá a descargar el fichero XML. 
\end{itemize}


%%----------------------------------------------------------------------------
\subsection{Funcionalidad mínima}

Los alojamientos se obtendrán a partir de la información pública ofrecida por el Ayuntamiento de Madrid en el Portal de Datos Abiertos, en forma de ficheros XML, como se indicaba anteriormente.

La {\bf interfaz pública} contiene los recursos a servir como páginas HTML completas (pensadas para ser vistas en el navegador) para cualquier visitante (sea usuario registrado o no):

\begin{itemize}
  \item /: Página principal de la práctica. Constará de un listado de alojamientos y otro con enlaces a páginas personales:
  
  \begin{enumerate}
    \item Mostrará un listado de los diez alojamientos con más comentarios. Si no hubiera 10 alojamientos con comentarios, se mostrarán sólo los que tengan comentarios. Para cada alojamiento, incluirá información sobre:
    \begin{itemize}
      \item su nombre (que será un enlace que apuntará a la url del alojamiento en el portal esmadrid), 
      \item su dirección, 
      \item una imagen suya en pequeño formato, 
      \item y un enlace, ``Más información'', que apuntará a la página del alojamiento en la aplicación (ver más adelante).
    \end{itemize}
   
    \item También se mostrará un listado, en una columna lateral derecha, con enlaces a las páginas personales disponibles. Para cada página personal mostrará el título que le haya dado su usuario (como un enlace a la página personal en cuestión) y el nombre del usuario. Si a una página personal aún no se le hubiera puesto título, este título será ``Página de usuario'', donde ``usuario'' es el identificador de usuario del usuario en cuestión.
   \end{enumerate}

  \item /{usuario}: Página personal de un usuario. Si la URL es ``/usuario'', es que corresponde al usuario ``usuario''. Mostrará los alojamientos seleccionados por ese usuario (aunque no puede haber más de 10 a la vez; si hay más debería haber un enlace para mostrar las diez siguientes y así en adelante, siempre de diez en diez). Para cada alojamiento se mostrará la misma información que en la página principal. Además, para cada alojamiento se deberá mostrar la fecha en la que fue seleccionada por el usuario.

  \item /alojamientos: Página con todos los alojamientos. Para cada uno de ellos aparecerá sólo el nombre, y un enlace a su página. En la parte superior de la página, existirá un formulario que permita filtrar estos alojamientos según varios campos, como, por ejemplo, por su categoría (por ejemplo, ``Hoteles'') y su subcategoría  (por ejemplo, ``4 estrellas'') .

  \item /alojamientos/{id}: Página de un alojamiento en la aplicación. Mostrará toda la información de los elementos ``basicData'' y ``geoData'' obtenida del XML del portal de datos abierto del Ayuntamiento de Madrid. Además, se mostrarán cinco fotos entre las que se pueden obtener del mismo documento XML (o menos, si en el documento no hay tantas), y todos los comentarios que se hayan puesto para este alojamiento.
  
  \item /{usuario}/xml: Canal XML para los alojamientos seleccionados por ese usuario. El documento XML tendrá una entrada para cada alojamiento seleccionado por el usuario, y tendrá una estructura similar (pero no necesariamente igual) a la del fichero XML del portal del Ayuntamiento.

  \item /about: Página con información en HTML indicando la autoría de la práctica y explicando su funcionamiento.

\end{itemize}

Todas las páginas de la interfaz pública incluirán un menú desde el que se podrá acceder a todos los alojamientos (URL /alojamientos) con el texto ``Todos'' y a la ayuda (URL /about) con el texto ``About''. Todas las página que no sean la principal tendrán otra opción de menú para la URL /, con el texto ``Inicio''.

La {\bf interfaz privada} contiene los recursos a servir como páginas HTML completas para usuarios registrados (una vez se han autenticado):

\begin{itemize}
  \item Todos los recursos de la interfaz pública.
  
  \item /alojamientos/{id}: Además de la información que se muestra de manera pública:

    \begin{enumerate}
      \item Un formulario para poner comentarios sobre este alojamiento. Los comentarios serán anónimos, pero sólo se podrán poner por los usuarios registrados, una vez se han autenticado. Por tanto, bastará con que este formulario esté compuesto por una caja de texto, donde se podrá escribir el comentario, y un botón para enviarlo.
    \item Un botón para cada uno de los idiomas en que está disponible el documento XML en el portal del Ayuntamiento. En caso de que el usuario pulse uno de esos botones, la aplicación descargará el XML correspondiente al idioma seleccionado, buscará en él la información sobre el alojamiento en cuestión, y si está disponible, la mostrará en pantalla en ese idioma (además de la información que ya estaba disponible). Si el alojamiento no está disponible en ese idioma, se pondrá un mensaje indicándolo. Esta información en otros idiomas no se guardará en la base de datos.
  \end{enumerate}

  \item /{usuario}: Además de la información que se muestra de manera pública:
  
  \begin{enumerate}
    \item Un formulario para cambiar el estilo CSS de todo el sitio para ese usuario. Bastará con que se pueda cambiar el tamaño de la letra y el color de fondo. Si se cambian estos valores, quedará cambiado el documento CSS que utilizarán todas las páginas del sitio para este usuario. Este cambio será visible en cuanto se suba la nueva página CSS.

    \item Un formulario para elegir el título de su página personal.
  \end{enumerate}
\end{itemize}

%Si es preciso, se añadirán más recursos (pero sólo si es realmente preciso) para poder satisfacer los requisitos especificados.

Dados los recursos mencionados anteriormente, no se permitirán los nombres de usuario ``alojamientos'' ni ``about'' (pero no hay que hacer ninguna comprobación para esto: se asume que no se darán de alta esos usuarios en el Admin Site).


%%----------------------------------------------------------------------------
\subsection{Funcionalidad optativa}

De forma optativa, se podrá incluir cualquier funcionalidad relevante en el contexto de la asignatura. Se valorarán especialmente las funcionalidades que impliquen el uso de técnicas nuevas, o de aspectos de Django no utilizados en los ejercicios previos, y que tengan sentido en el contexto de esta práctica y de la asignatura.

En el formulario de entrega se pide que se justifique por qué se considera funcionalidad optativa lo que habeis implementado. Sólo a modo de sugerencia, se incluyen algunas posibles funcionalidades optativas:

\begin{itemize}
  \item Inclusión de un \emph{favicon} del sitio
  
  \item Generación de un canal XML para los contenidos que se muestran en la página principal.

  \item Generación de canales, pero con los contenidos en JSON

  \item Generación de un canal RSS para los comentarios puestos en el sitio.
  
  \item Funcionalidad de registro de usuarios
  
  \item Uso de Javascript o AJAX para algún aspecto de la práctica (por ejemplo, para seleccionar un alojamiento para una página de usuario).

  \item Puntuación de alojamientos. Cada visitante (registrado o no) puede dar un ``+1'' a cualquier alojamiento del sitio. La suma de ``+'' que ha obtenido un alojamiento se verá cada vez que se vea el alojamiento en el sitio.
  
  \item Uso de elementos HTML5 (especificar cuáles al entregar)

  \item Atención al idioma indicado por el navegador. El idioma de la interfaz de usuario del planeta tendrá en cuenta lo que especifique el navegador.

\end{itemize}


%%----------------------------------------------------------------------------
\subsection{Entrega de la práctica}

\begin{itemize}
  \item \textbf{Fecha límite de entrega de la práctica:} lunes, 23 de mayo de 2016 a las 02:00 (hora española peninsular)\footnote{Entiéndase la hora como domingo por la noche, ya entrado el lunes.}
       %{\bf Convocatoria de junio:} miércoles, 24 de junio de 2015 a las 23:59 (hora peninsular española).

  \item \textbf{Fecha de publicación de notas:} martes, 24 de mayo de 2016, en la plataforma Moodle.
%{\bf Convocatoria de junio:} viernes, 26 de junio, en la plataforma Moodle.

  \item \textbf{Fecha de revisión:} miércoles, 25 de mayo de 2016 a las 13:30.
%{\bf Convocatoria de junio:} martes, 30 de junio a las 13:30. Se requerirá a algunos alumnos que asistan a la revisión {\bf en persona}; se informará de ello en el mensaje de publicación de notas.
\end{itemize}

La entrega de la práctica consiste en rellenar un formulario (enlazado en el Moodle de la asignatura) y en seguir las instrucciones que se describen a continuación.

\begin{enumerate}
  \item El repositorio contendrá todos los ficheros necesarios para que funcione la aplicación (ver detalle más abajo). Es muy importante que el alumno haya realizado un fork del repositorio que se indica a continuación, porque si no, la práctica no podrá ser identificada: 

\url{https://github.com/CursosWeb/X-Serv-Practica-Hoteles/}

Los alumnos que no entreguen las práctica de esta forma serán considerados como no presentados en lo que a la entrega de prácticas se refiere. Los que la entreguen podrán ser llamados a realizar también una entrega presencial, que tendrá lugar en la fecha y hora de la revisión. Esta entrega presencial podrá incluir una conversación con el profesor sobre cualquier aspecto de la realización de la práctica.

Recordad que es importante ir haciendo commits de vez en cuando y que sólo al hacer push estos commits son públicos. Antes de entregar la práctica, haced un push. Y cuando la entreguéis y sepáis el nombre del repositorio, podéis cambiar el nombre del repositorio desde el interfaz web de GitHub. 
 
 \item Un vídeo de demostración de la parte obligatoria, y otro vídeo de demostración de la parte opcional, si se han realizado opciones avanzadas. Los vídeos serán de una duración máxima de 3 minutos (cada uno), y consistirán en una captura de pantalla de un navegador web utilizando la aplicación, y mostrando lo mejor posible la funcionalidad correspondiente (básica u opcional). Siempre que sea posible, el alumno comentará en el audio del vídeo lo que vaya ocurriendo en la captura. Los vídeos se colocarán en algún servicio de subida de vídeos en Internet (por ejemplo, Vimeo o YouTube).

Hay muchas herramientas que permiten realizar la captura de pantalla. Por ejemplo, en GNU/Linux puede usarse Gtk-RecordMyDesktop o Istanbul (ambas disponibles en Ubuntu). Es importante que la captura sea realizada de forma que se distinga razonablemente lo que se grabe en el vídeo.

En caso de que convenga editar el vídeo resultante (por ejemplo, para eliminar tiempos de espera) puede usarse un editor de vídeo, pero siempre deberá ser indicado que se ha hecho tal cosa con un comentario en el audio, o un texto en el vídeo. Hay muchas herramientas que permiten realizar esta edición. Por ejemplo, en GNU/Linux puede usarse OpenShot o PiTiVi.

  \item Se han de entregar los siguientes ficheros:

\begin{itemize}
  \item Un fichero README.md que resuma las mejoras, si las hay, y explique cualquier peculiaridad de la entrega (ver siguiente punto).
  \item El repositorio GitHub deberá contener un proyecto Django completo y listo para funcionar en el entorno del laboratorio, incluyendo la base de datos con datos suficientes como para poder probarlo. Estos datos incluirán al menos dos usuarios con sus datos correspondientes, con al menos cinco alojamientos en su página personal, y con al menos cinco comentarios en total.
  \item Cualquier biblioteca Python que pueda hacer falta para que la aplicación funcione, junto con los ficheros auxiliares que utilice, si es que los utiliza.
\end{itemize}

  \item Se incluirán en el fichero README.md los siguientes datos (la mayoría de estos datos se piden también en el formulario que se ha de rellenar para entregar la práctica - se recomienda hacer un corta y pega de estos datos en el formulario):

\begin{itemize}
  \item Nombre y titulación.
  \item Nombre de su cuenta en el laboratorio del alumno.
  \item Nombre de usuario en GitHub.
  \item Resumen de las peculiaridades que se quieran mencionar sobre lo implementado en la parte obligatoria.
  \item Lista de funcionalidades opcionales que se hayan implementado, y breve descripción de cada una.
  \item URL del vídeo demostración de la funcionalidad básica
  \item URL del vídeo demostración de la funcionalidad optativa, si se ha realizado funcionalidad optativa
\end{itemize}

Asegúrate de que las URLs incluidas en este fichero están adecuadamente escritas en Markdown, de forma que la versión HTML que genera GitHub los incluya como enlaces ``pinchables''.

\end{enumerate}


%%----------------------------------------------------------------------------
\subsection{Notas y comentarios}

La práctica deberá funcionar en el entorno GNU/Linux (Ubuntu) del laboratorio de la asignatura con la versión de Django que se ha usado en prácticas.

La práctica deberá funcionar desde el navegador Firefox disponible en el laboratorio de la asignatura.

Los canales (feeds) RSS que produce la aplicación web realizada en la práctica deberán funcionar al menos con el navegador Firefox (considerándolos como canales RSS) disponibles en el laboratorio. Los documentos XML deberán ser correctos desde el punto de vista de la sintaxis XML, y por lo tanto reconocibles por un reconocedor XML, como por ejemplo el reconocedor del módulo xml.sax de Python.


%%----------------------------------------------------------------------------
%%----------------------------------------------------------------------------
\section{Práctica final (2016, junio)}
\label{practica-final-2016-06}

La práctica final para la convocatoria de junio de 2016 será la misma que la descrita para la convocatoria de mayo de 2016, salvo la siguiente cuestión:
 
Los comentarios incluirán información sobre quién los ha introducido, y cada hotel sólo podrá tener un comentario por cada usuario.

Además, las fechas de entrega, publicación y revisión quedan como siguen:

\begin{itemize}
  \item \textbf{Fecha límite de entrega de la práctica:} lunes, 27 de junio de 2016 a las 02:00 (hora española peninsular)\footnote{Entiéndase la hora como domingo por la noche, ya entrado el lunes.}. Se ha de entregar el código en GitHub y rellenar el formulario de entrega (incluyendo los enlaces a los vídeos de presentación).
       %{\bf Convocatoria de junio:} miércoles, 24 de junio de 2015 a las 23:59 (hora peninsular española).

  \item \textbf{Fecha de publicación de notas:} martes, 28 de junio de 2016, en la plataforma Moodle.
%{\bf Convocatoria de junio:} viernes, 26 de junio, en la plataforma Moodle.

  \item \textbf{Fecha de revisión:} jueves, 30 de junio de 2016 a las 13:00.
%{\bf Convocatoria de junio:} martes, 30 de junio a las 13:30. Se requerirá a algunos alumnos que asistan a la revisión {\bf en persona}; se informará de ello en el mensaje de publicación de notas.
\end{itemize}



\newpage

%%----------------------------------------------------------------------------
%%----------------------------------------------------------------------------
\section{Práctica final (2015, mayo y junio)}
\label{practica-final-2015-05}

La práctica final de la asignatura consiste en la creación de una aplicación web que aglutine información sobre actividades culturales y de ocio que tienen lugar en el municipio de Madrid. A continuación se describe el funcionamiento y arquitectura general de la aplicación, la funcionalidad mínima que debe proporcionar, y otra funcionalidad optativa que podrá tener.

La aplicación consiste en descargarse datos de actividades culturales (disponbiles públicamente en formato XML) y ofrecer estos datos a los usuarios de la aplicación para que puedan gestionar la información de la manera con consideren más conveniente. De esta manera, un escenario típico es el de un usuario que a partir de las actividades existentes, incluya en su perfil las que le interesen.

%%----------------------------------------------------------------------------
\subsection{Arquitectura y funcionamiento general}

Arquitectura general:

\begin{itemize}

\item La práctica se construirá como un proyecto Django, que incluirá una o varias aplicaciones Django que implementen la funcionalidad requerida.

\item Para el almacenamiento de datos persistente se usará SQLite3, con tablas definidas según modelos en Django.

\item Se usará la aplicación Django ``Admin Site'' para crear cuenta a los usuarios en el sistema, y para la gestión general de las bases de datos necesarias. Todas las bases de datos que mantenga DeLorean tendrá que ser accesible vía este ``Admin Site''.

\item Se utilizarán plantillas Django (a ser posible, una jerarquía de plantillas, para que la práctica tenga un aspecto similar) para definir las páginas que se servirán a los navegadores de los usuarios. Estas plantillas incluirán en todas las páginas al menos:
  \begin{itemize}
  \item Un banner (imagen) del sitio, en la parte superior.
  \item Un menú de opciones.
  \item Una caja para entrar (hacer login en el sitio), o para salir (si ya se ha entrado). En caso de que no se haya entrado en una cuenta, esta caja permitirá al visitante introducir su identificador de usuario y su contraseña. En caso de que ya se haya entrado, esta caja mostrará el identificador del usuario y permitirá salir de la cuenta (logout).
  \item Un pie de página con una nota de copyright.
  \end{itemize}

Cada una de estas partes estará marcada con propiedades ``id'' en HTML, para poder ser referenciadas en hojas de estilo CSS.

\item Se utilizarán hojas de estilo CSS para determinar la apariencia de la práctica. Estas hojas definirán al menos el color y el tamaño de la letra, y el color de fondo de cada una de las partes (elementos) marcadas con id que se indican en el apartado anterior.
\end{itemize}

Funcionamiento general:

\begin{itemize}
\item Los usuarios serán dados de alta en la práctica mediante el módulo ``Admin Site'' de Django. Una vez estén dados de alta, serán considerados ``usuarios registrados''.

\item Los usuarios registrados podrán crear su selección de actividades de cultura y de ocio. Para ello, dispondrán de una página personal. Llamaremos a esta página la ``página del usuario''.

\item La selección de actividades en su página personal la realizará cada usuario a partir de información sobre actividades de ocio y cultura ya disponibles en el sitio.

\item Las actividades de ocio y cultura se actualizarán sólo cuando un usuario indique que quiere que se actualicen.

\item Cualquier navegador podrá acceder a la interfaz pública del sitio, que ofrecerá la página personal de cada usuario, para todos los usuarios del sitio.

\end{itemize}


%%----------------------------------------------------------------------------
\subsection{Funcionalidad mínima}

Las actividades de ocio y de cultura se toman de interpretar la información pública ofrecida por el Ayuntamiento de Madrid en el Portal de Datos Abiertos, y que es la siguiente:
    \begin{itemize}
      \item Actividades Culturales y de Ocio Municipal en los próximos 100 días: \\
        \url{http://goo.gl/809BPF}
    \end{itemize}

Interfaz pública: recursos a servir como páginas HTML completas (pensadas para ser vistas en el navegador) para cualquier visitante (sea usuario registrado o no):

\begin{itemize}
\item /: Página principal de la práctica. Mostrará un listado de las diez actividades de ocio y cultura más próximas en el tiempo, que incluya información sobre su título, el tipo de evento y la fecha del mismo. También se mostrará un listado, probablemente en un lateral, con las páginas personales disponibles. Para cada página personal mostrará el título (como un enlace a la página personal), el nombre de su usuario y una pequeña descripción. Si a una página personal aún no se le hubiera puesto título, este título será ``Página de usuario'', donde ``usuario'' es el identificador de usuario del usuario en cuestión.

\item /usuario: Página personal de un usuario. Si la URL es ``/usuario'', es que corresponde al usuario ``usuario''. Mostrará las actividades de ocio y de cultura seleccionadas por ese usuario (aunque no puede haber más de 10 a la vez; si hay más debería haber un enlace para mostrar las diez siguientes y así en adelante, siempre de diez en diez). Para cada actividad de ocio y de cultura se mostrará al menos el título y la fecha de los eventos (con un enlace a la página /actividad de cada evento, ver más adelante). Además, para cada actividad se deberá mostrar la fecha en la que fue seleccionada por el usuario.

\item /actividad/{id}: Página de una actividad de cultura o de ocio. Mostrará toda la información obtenida del XML del portal de datos abierto del Ayuntamiento de Madrid. Además, se mostrará su ``información adicional'', conseguida a partir de seguir la URL con información adicional. Esta información adicional es la que se puede encontrar si seguimos el enlace justo debajo de ``Amplíe información''. Se puede hacer uso del módulo \emph{Beautiful Soup} para llevar a cabo esta funcionalidad.

\item /usuario/rss: Canal RSS para las actividades seleccionadas por ese usuario.

\item /ayuda: Página con información HTML explicando el funcionamiento de la práctica.

\item /todas: Página con todas las actividades de ocio y de cultura. En la parte superior de la página, existirá un formulario que permite filtrar estas actividades según varios campos, como, por ejemplo, la fecha, la duración, el precio o el título.
\end{itemize}

Todas las páginas de la interfaz pública incluirán un menú desde el que se podrá acceder a todas las actividades (URL /todas) con el texto ``Todas'' y a la ayuda (URL /ayuda) con el texto ``Ayuda''. Todas las página que no sean la principal tendrán otra opción de menú para la URL /, con el texto ``Inicio''.


Interfaz privada: recursos a servir como páginas HTML completas para usuarios registrados (una vez se han autenticado).

\begin{itemize}
  \item Todos los recursos de la interfaz pública.
  \item /todas: Además de la información que se muestra de manera pública:
  \begin{itemize}
    \item Se mostrará el número de actividades de ocio y de cultura disponibles para el canal, y la fecha en que fue actualizado por última vez.
    \item Existirá un botón para actualizar las actividades a partir del canal de actividades. Si se pulsa este botón, se tratarán de actualizar las actividades accediendo al canal de actividades del Ayuntamiento de Madrid. Al terminar la operación se volverá a mostrar esta misma página /todas, actualizada.
    \item La lista de actividades disponibles en el canal de actividades.
    \item Junto a cada actividad de la lista, se incluirá un botón que permitirá elegir la actividad para la página personal del usuario autenticado. Tras añadir una actividad a la página del usuario, se volverá a ver en el navegador la página /todas.
  \end{itemize}

  \item En la página /usuario que corresponde al usuario autenticado se mostrará, además de lo ya mencionado para la interfaz pública, un formulario en el que se podrá especificar la siguiente información:

  \begin{itemize}
    \item Los parámetros CSS para el usuario autenticado (al menos los indicados anteriormente para ser manejados por un documento CSS). Si el usuario los cambia, a partir de ese momento deberá verse el sitio con los nuevos valores, y para ello deberá servirse un nuevo documento CSS.
    \item El título de su página personal.
  \end{itemize}
\end{itemize}

%Si es preciso, se añadirán más recursos (pero sólo si es realmente preciso) para poder satisfacer los requisitos especificados.

Dados los recursos mencionados anteriormente, no se permitirán los nombres de usuario ``actividad'', ``ayuda'' ni ``todas'' (pero no hay que hacer ninguna comprobación para esto: se asume que no se darán de alta esos usuarios en el Admin Site).



%%----------------------------------------------------------------------------
\subsection{Funcionalidad optativa}

De forma optativa, se podrá incluir cualquier funcionalidad relevante en el contexto de la asignatura. Se valorarán especialmente las funcionalidades que impliquen el uso de técnicas nuevas, o de aspectos de Django no utilizados en los ejercicios previos, y que tengan sentido en el contexto de esta práctica y de la asignatura.

Sólo a modo de sugerencia, se incluyen algunas posibles funcionalidades optativas:

\begin{itemize}
\item Atención al idioma indicado por el navegador. El idioma de la interfaz de usuario del planeta tendrá en cuenta lo que especifique el navegador.

\item Generación de un canal RSS para los contenidos que se muestran en la página principal.

\item Uso de AJAX para algún aspecto de la práctica (por ejemplo, para seleccionar una actividad para una página de usuario).

\item Puntuación de actividades. Cada visitante (registrado o no) puede dar un ``+1'' a cualquier actividad del sitio. La suma de ``+'' que ha obtenido una actividad se verá cada vez que se vea la actividad en el sitio.

\item Comentarios a actividades. Cada usuario registrado puede comentar cualquier actividad del sitio. Estos comentarios se podrán ver luego en la página personal.

\end{itemize}


%%----------------------------------------------------------------------------
\subsection{Entrega de la práctica}

\textbf{Fecha límite de entrega de la práctica:} domingo, 24 de mayo de 2015 a las 23:59 (hora española peninsular). {\bf Convocatoria de junio:} miércoles, 24 de junio de 2015 a las 23:59 (hora peninsular española).

\textbf{Fecha de publicación de notas:} martes, 26 de mayo de 2015, en la plataforma Moodle.
{\bf Convocatoria de junio:} viernes, 26 de junio, en la plataforma Moodle.

\textbf{Fecha de revisión:} viernes, 29 de mayo de 2014 a las 12:00. {\bf Convocatoria de junio:} martes, 30 de junio a las 13:30. Se requerirá a algunos alumnos que asistan a la revisión {\bf en persona}; se informará de ello en el mensaje de publicación de notas.

La práctica se entregará realizando {\bf dos} acciones:

\begin{enumerate}
  \item Rellenando un formulario web, que pedirá la siguiente información:
  \begin{itemize}
    \item Nombre de la asignatura.
    \item Nombre completo del alumno.
    \item Nombre de su cuenta en el laboratorio.
    \item Nombres y contraseñas de los usuarios creados para la práctica. Éstos deberán incluir al menos un usuario con cuenta ``marty'' y contraseña ``marty'' y otro usuario con cuenta ``doc'' y contraseña ``doc''.
    \item Resumen de las peculiaridades que se quieran mencionar sobre lo implementado en la parte obligatoria.
    \item Lista de funcionalidades opcionales que se hayan implementado, y breve descripción de cada una.
    \item URL del vídeo demostración en YouTube que muestre la funcionalidad básica
    \item URL del vídeo demostración en YouTube con la funcionalidad optativa, si se ha realizado funcionalidad optativa
  \end{itemize}

  \item Subiendo la práctica a un repositorio GitHub. El nombre del repositorio se dará al entregar la práctica. Así, para ir realizando la práctica se recomienda crearse un repositorio en GitHub con el nombre que queráis, e ir haciendo commits. Recordad que es importante ir haciendo commits de vez en cuando y que sólo al hacer push estos commits son públicos. Antes de entregar la práctica, haced un push. Y cuando la entreguéis y sepáis el nombre del repositorio, podeis cambiar el nombre del repositorio desde el interfaz web de GitHub. 
  
    El repositorio GitHub deberá contener un proyecto Django completo y listo para funcionar en el entorno del laboratorio, incluyendo la base de datos con datos suficientes como para poder probarlo. Estos datos incluirán al menos dos usuarios con sus datos correspondientes, y con al menos cinco actividades en su página personal.
\end{enumerate}

Los vídeos de demostración serán de una duración máxima de tres minutos (cada uno), y consistirán en una captura de pantalla de un navegador web utilizando la aplicación, y mostrando lo mejor posible la funcionalidad correspondiente (básica u opcional). Se valorará negativamente que los vídeos duren más de 3 minutos (de la experiencia de cursos pasados, tres minutos es un tiempo más que suficiente si uno no entra en detalles que no son importantes). Siempre que sea posible, el alumno comentará en el audio del vídeo lo que vaya ocurriendo en la captura. Los vídeos se colocarán en YouTube y deberán ser accesibles públicamente al menos hasta el 31 de mayo, fecha a partir de la cual los alumnos pueden retirar el vídeo (o indicarlo como privado).

Hay muchas herramientas que permiten realizar la captura de pantalla. Por ejemplo, en GNU/Linux puede usarse Gtk-RecordMyDesktop o Istanbul (ambas disponibles en Ubuntu). Incluso hay alguna aplicación web como Screen-O-Matic. Es importante que la captura sea realizada de forma que se distinga razonablemente lo que se grabe en el vídeo.

En caso de que convenga editar el vídeo resultante (por ejemplo, para eliminar tiempos de espera) puede usarse un editor de vídeo, pero siempre deberá ser indicado que se ha hecho tal cosa con un comentario en el audio, o un texto en el vídeo. Hay muchas herramientas que permiten realizar esta edición. Por ejemplo, en GNU/Linux puede usarse OpenShot o PiTiVi.

Los alumnos que no entreguen las práctica de esta forma serán considerados como no presentados en lo que a la entrega de prácticas se refiere.

%%----------------------------------------------------------------------------
\subsection{Notas y comentarios}

La práctica deberá funcionar en el entorno GNU/Linux (Ubuntu) del laboratorio de la asignatura con la versión de Django que se ha usado en prácticas (Django 1.7.*).

La práctica deberá funcionar desde el navegador Firefox disponible en el laboratorio de la asignatura.

Los canales (feeds) RSS que produce la aplicación web realizada en la práctica deberán funcionar al menos con el navegador Firefox (considerándolos como canales RSS) disponibles en el laboratorio.


\newpage

%%----------------------------------------------------------------------------
%%----------------------------------------------------------------------------
\section{Práctica final (2014, mayo)}
\label{practica-final-2014-04}

La práctica final de la asignatura consiste en la creación de una aplicación web que aglutine información sobre el estado de las carreteras y relacionada. A continuación se describe el funcionamiento y arquitectura general de la aplicación, la funcionalidad mínima que debe proporcionar, y otra funcionalidad optativa que podrá tener. Llamaremos a la aplicación DeLorean, como tributo a los casi 30 años de la primera película ``Regreso al Futuro''.

La aplicación consiste en descargarse datos de tráfico (disponbiles públicamente en formato XML) y ofrecer estos datos a los usuarios de la aplicación para que puedan gestionar la información de la manera con consideren más conveniente. De esta manera, un escenario típico es el de un usuario que indique una provincia (o incluso una carretera) en la que está interesado; en su página personal aparecerán todas las incidencias de tráfico que cumplan esos requisitos, en tiempo real.

%%----------------------------------------------------------------------------
\subsection{Arquitectura y funcionamiento general}

Arquitectura general:

\begin{itemize}

\item DeLorean se construirá como un proyecto Django, que incluirá una o varias aplicaciones Django que implementen la funcionalidad requerida.

\item Para el almacenamiento de datos persistente se usará SQLite3, con tablas definidas según modelos en Django.

\item Se usará la aplicación Django ``Admin Site'' para crear cuenta a los usuarios en el sistema, y para la gestión general de las bases de datos necesarias. Todas las bases de datos que mantenga DeLorean tendrá que ser accesible vía este ``Admin Site''.

\item Se utilizarán plantillas Django (a ser posible, una jerarquía de plantillas, para que DeLorean tenga un aspecto similar) para definir las páginas que se servirán a los navegadores de los usuarios. Estas plantillas incluirán en todas las páginas al menos:
  \begin{itemize}
  \item Un banner (imagen) del sitio, en la parte superior.
  \item Un menú de opciones.
  \item Una caja para entrar (hacer login en el sitio), o para salir (si ya se ha entrado). En caso de que no se haya entrado en una cuenta, esta caja permitirá al visitante introducir su identificador de usuario y su contraseña. En caso de que ya se haya entrado, esta caja mostrará el identificador del usuario y permitirá salir de la cuenta (logout).
  \item Un pie de página con una nota de copyright.
  \end{itemize}

Cada una de estas partes estará marcada con propiedades ``id'' en HTML, para poder ser referenciadas en hojas de estilo CSS.

\item Se utilizarán hojas de estilo CSS para determinar la apariencia de DeLorean. Estas hojas definirán al menos el color y el tamaño de la letra, y el color de fondo de cada una de las partes (elementos) marcadas con id que se indican en el apartado anterior.
\end{itemize}

Funcionamiento general:

\begin{itemize}
\item Los usuarios serán dados de alta en DeLorean mediante el módulo ``Admin Site'' de Django. Una vez estén dados de alta, serán considerados ``usuarios registrados''.

\item Los usuarios registrados podrán crear su selección de estados de carretera de DeLorean. Para ello, dispondrán de una página personal. Llamaremos a esta página la ``página del usuario''.

\item La selección de incidencias en su página personal la realizará cada usuario a partir de información sobre incidencias ya disponibles en el sitio.

\item Las incidencias se actualizarán sólo cuando un usuario indique que quiere que se actualicen.

\item Cualquier navegador podrá acceder a la interfaz pública del sitio, que ofrecerá la página personal de cada usuario, para todos los usuarios del sitio.
\end{itemize}


%%----------------------------------------------------------------------------
\subsection{Funcionalidad mínima}

Interfaz pública: recursos a servir como páginas HTML completas (pensadas para ser vistas en el navegador) para cualquier visitante (sea usuario registrado o no):

\begin{itemize}
\item /: Página principal de DeLorean. Mostrará un listado de las últimas diez incidencias y posteriormente otro listado con las páginas personales disponibles. Para cada página personal mostrará el título (como un enlace a la página personal), el nombre de su usuario y una pequeña descripción. Si a una página personal aún no se le hubiera puesto título, este título será ``Página de usuario'', donde ``usuario'' es el identificador de usuario del usuario en cuestión.

\item /usuario: Página personal de un usuario. Si la URL es ``/usuario'', es que corresponde al usuario ``usuario''. Mostrará las incidencias seleccionadas por ese usuario (aunuque no puede haber más de 10 a la vez, como se indicará más adelante). Para cada incidencia se mostrará la ``información pública de cada incidencia'', ver más adelante.

\item /usuario/rss: Canal RSS para las incidencias seleccionadas por ese usuario.

\item /ayuda: Página con información HTML explicando el funcionamiento de DeLorean.

\item /todas: Página con todas las incidencias. En la parte superior de la página, existirá un formulario que permite filtrar las incidencias según varios campos, como, por ejemplo, provincia, tipo, longitud.
\end{itemize}

Todas las páginas de la interfaz pública incluirán un menú desde el que se podrá acceder a todas las incidiencias (URL /todas) con el texto ``Todas'' y a la ayuda (URL /ayuda) con el texto ``Ayuda''. Todas las página que no sean la principal tendrán otra opción de menú para la URL /, con el texto ``Inicio''.

Interfaz privada: recursos a servir como páginas HTML completas para usuarios registrados (una vez se han autenticado).

\begin{itemize}
\item Todos los recursos de la interfaz pública.
\item /incidencias: Página con la lista de incidencias disponibles en DeLorean:

  \begin{itemize}
  \item Las incidencias se toman de interpretar la información pública ofrecida por la Dirección General de Tráfico (DGT), y que es la siguiente:
    \begin{itemize}
      \item Información de incidencias en carreteras (canal de incidencias): \\
        \url{http://www.dgt.es/incidencias.xml}
    \end{itemize}

  \item Se mostrará el número de incidencias disponibles para el canal, y la fecha en que fue actualizado por última vez.
  \item Existirá un botón para actualizar las incidencias a partir del canal de incidencias. Si se pulsa este botón, se tratarán de actualizar las incidencias accediendo al canal de incidencias de la DGT. Al terminar la operación se volverá a mostrar esta misma página, actualizada.
  \item La lista de incidencias disponibles en el canal de incidencias, incluyendo para cada una la ``información pública', ver más adelante.
  \item Junto a cada incidencia de la lista, se incluirá un botón que permitirá elegir la incidencia para la página personal del usuario autenticado. Tras añadir una incidencia a la página del usuario, se volverá a ver en el navegador la página /incidencias.
  \end{itemize}

\item En la página /usuario que corresponde al usuario autenticado se mostrará, además de lo ya mencionado para la interfaz pública, un formulario en el que se podrá especificar la siguiente información:

  \begin{itemize}
  \item Los parámetros CSS para el usuario autenticado (al menos los indicados anteriormente para ser manejados por un documento CSS). Si el usuario los cambia, a partir de ese momento deberá verse el sitio con los nuevos valores, y para ello deberá servirse un nuevo documento CSS.
  \item El título de su página personal.
  \end{itemize}
\end{itemize}

Si es preciso, se añadirán más recursos (pero sólo si es realmente preciso) para poder satisfacer los requisitos especificados.

Dados los recursos mencionados anteriormente, no se permitirán los nombres de usuario ``incidencias'', ``ayuda'' ni ``todas'' (pero no hay que hacer ninguna comprobación para esto: se asume que no se darán de alta esos usuarios en el Admin Site).


Como información pública de cada incidencia se mostrará:
\begin{itemize}
  \item El tipo de incidencia
  \item La provincia de la incidencia y la carretera
  \item La fecha en que fue publicada la incidencia en el sitio original (junto al texto ``publicada en'').
  \item La fecha en que fue seleccionada para la página personal del usuario (junto al texto ``elegida en'').
  \item La información detallada de la incidencia (toda la demás información de la incidencia que se puede extraer del XML)
\end{itemize}



%%----------------------------------------------------------------------------
\subsection{Funcionalidad optativa}

De forma optativa, se podrá incluir cualquier funcionalidad relevante en el contexto de la asignatura. Se valorarán especialmente las funcionalidades que impliquen el uso de técnicas nuevas, o de aspectos de Django no utilizados en los ejercicios previos, y que tengan sentido en el contexto de esta práctica y de la asignatura.

Sólo a modo de sugerencia, se incluyen algunas posibles funcionalidades optativas:

\begin{itemize}
\item Atención al idioma indicado por el navegador. El idioma de la interfaz de usuario del planeta tendrá en cuenta lo que especifique el navegador.

\item Generación de un canal RSS para los contenidos que se muestran en la página principal.

\item Uso de AJAX para algún aspecto de la práctica (por ejemplo, para seleccionar una incidencia para una página de usuario).

\item Puntuación de incidencias. Cada visitante (registrado o no) puede dar un ``+1'' a cualquier incidencia del sitio. La suma de ``+'' que ha obtenido una incidencia se verá cada vez que se vea la incidencias en el sitio.

\item Comentarios a incidencias. Cada usuario registrado puede comentar cualquier incidencia del sitio. Estos comentarios se podrán ver luego en la página personal.

\end{itemize}


%%----------------------------------------------------------------------------
\subsection{Entrega de la práctica}

\textbf{Fecha límite de entrega de la práctica:} sábado, 24 de mayo de 2014 a las 03:00 (hora española peninsular).

\textbf{Fecha de publicación de notas:} lunes, 26 de mayo de 2014, en la plataforma Moodle.

\textbf{Fecha de revisión:} miércoles, 28 de mayo de 2014 a las 12:00. Se requerirá a algunos alumnos que asistan a la revisión {\bf en persona}; se informará de ello en el mensaje de publicación de notas.

La práctica se entregará subiéndola al recurso habilitado a tal fin en el sitio Moodle de la asignatura. Los alumnos que no entreguen las práctica de esta forma serán considerados como no presentados en lo que a la entrega de prácticas se refiere. Los que la entreguen podrán ser llamados a realizar también una entrega presencial, que tendrá lugar en la fecha y hora exacta se les comunicará oportunamente. Esta entrega presencial podrá incluir una conversación con el profesor sobre cualquier aspecto de la realización de la práctica.

Para entregar la práctica en el Moodle, cada alumno subirá al recurso habilitado a tal fin un fichero tar.gz con todo el código fuente de la práctica. El fichero se habrá de llamar practica-user.tar.gz, siendo ``user'' el nombre de la cuenta del alumno en el laboratorio.

El fichero que se entregue deberá constar de un proyecto Django completo y listo para funcionar en el entorno del laboratorio, incluyendo la base de datos con datos suficientes como para poder probarlo. Estos datos incluirán al menos dos usuarios con sus datos correspondientes, y con al menos cinco incidencias en su página personal. Se incluirá también un fichero README con los siguientes datos:

\begin{itemize}
  \item Nombre de la asignatura.
  \item Nombre completo del alumno.
  \item Nombre de su cuenta en el laboratorio.
  \item Nombres y contraseñas de los usuarios creados para la práctica. Éstos deberán incluir al menos un usuario con cuenta ``marty'' y contraseña ``marty'' y otro usuario con cuenta ``doc'' y contraseña ``doc''.
\item Resumen de las peculiaridades que se quieran mencionar sobre lo implementado en la parte obligatoria.
\item Lista de funcionalidades opcionales que se hayan implementado, y breve descripción de cada una.
\item URL del vídeo demostración en YouTube que muestre la funcionalidad básica
\item URL del vídeo demostración en YouTube con la funcionalidad optativa, si se ha realizado funcionalidad optativa
\end{itemize}

Además, parte de la información del fichero README se incluirá a su vez en un formulario web a la hora de realizar la entrega.

Los vídeos de demostración serán de una duración máxima de 3 minutos (cada uno), y consistirán en una captura de pantalla de un navegador web utilizando la aplicación, y mostrando lo mejor posible la funcionalidad correspondiente (básica u opcional). Se valorará negativamente que los vídeos duren más de 3 minutos (de la experiencia de cursos pasados, tres minutos es un tiempo más que suficiente si uno no entra en detalles que no son importantes). Siempre que sea posible, el alumno comentará en el audio del vídeo lo que vaya ocurriendo en la captura. Los vídeos se colocarán en YouTube y deberán ser accesibles públicamente al menos hasta el 31 de mayo, fecha a partir de la cual los alumnos pueden retirar el vídeo (o indicarlo como privado).

Hay muchas herramientas que permiten realizar la captura de pantalla. Por ejemplo, en GNU/Linux puede usarse Gtk-RecordMyDesktop o Istanbul (ambas disponibles en Ubuntu). Incluso hay alguna aplicación web como Screen-O-Matic. Es importante que la captura sea realizada de forma que se distinga razonablemente lo que se grabe en el vídeo.

En caso de que convenga editar el vídeo resultante (por ejemplo, para eliminar tiempos de espera) puede usarse un editor de vídeo, pero siempre deberá ser indicado que se ha hecho tal cosa con un comentario en el audio, o un texto en el vídeo. Hay muchas herramientas que permiten realizar esta edición. Por ejemplo, en GNU/Linux puede usarse OpenShot o PiTiVi.

%%----------------------------------------------------------------------------
\subsection{Notas y comentarios}

La práctica deberá funcionar en el entorno GNU/Linux (Ubuntu) del laboratorio de la asignatura con la versión de Django que se ha usado en prácticas (Django 1.7.*).

La práctica deberá funcionar desde el navegador Firefox disponible en el laboratorio de la asignatura.

Se recomienda construir una o varias aplicaciones complementarias para probar la descarga y almacenamiento en base de datos de los canales que alimentarán las revistas.

Los canales (feeds) RSS que produce la aplicación web realizada en la práctica deberán funcionar al menos con el navegador Firefox (considerándolos como canales RSS) disponibles en el laboratorio.


\newpage




%%----------------------------------------------------------------------------
%%----------------------------------------------------------------------------
%%----------------------------------------------------------------------------
\section{Ejercicios complementarios de varios temas}

A continuación, algunos ejercicios relacionados con el temario de la asignatura. Algunos de ellos han sido propuestos en exámenes de ediciones previas, o en asignaturas con temarios similares.

%%----------------------------------------------------------------------------
%%----------------------------------------------------------------------------
\subsection{Números primos}

Se pide realizar una aplicación web que, dado un número, calcule si es primo o no. El número se indica como recurso, con URLs de la forma http://primos.org/34 (si el número a probar es ``34''). Para esta aplicación:

\begin{enumerate}
\item Escribir la petición y la respuesta HTTP que se podría observar para el caso de que se pruebe el número 34.
\item Escribir el código de la aplicación (sin usar un entorno de desarrollo de aplicaciones web). Escribir el código en Python, pseudo-Python o pseudocódigo. Puede usarse un método ``IsPrime'', que acepta un número como parámetro, y devuelve True si ese número es primo, y False en caso contrario.
\item Se quiere que la aplicación mantenga una caché de los números ya probados, para evitar volver a probar un número si ya se calculó si era primo. Explicar las modificaciones que se verán en el intercambio HTTP, y en el código de la aplicación.
\item Se quiere que la aplicación, tal y como la describía el enunciado al principio de este ejercicio, siga funcionado en presencia de caídas y posteriores recuperaciones del servidor. ¿Qué cambios habrá que hacerle?
\item Lo mismo, en el caso de la aplicación con caché, tal y como se describe dos aparatados más arriba.
\end{enumerate}

%%----------------------------------------------------------------------------
%%----------------------------------------------------------------------------
\subsection{Autenticación}

Una aplicación web dada permite el acceso a cierto recurso, ``/resource'', sólo a usuarios que se hayan autenticado previamente. Los usuarios se autentican mediante nombre de usuario y contraseña. La autenticación se realiza mediante POST a un recurso ``/login''. Ese mismo recurso, si recibe un GET, sirve un formulario para poder realizar la autenticación. En este caso, se plantean las siguientes preguntas:

\begin{enumerate}
\item Describir (indicando las cabeceras relevantes y el contenido del cuerpo de los mensajes) las interacciones HTTP, desde que un usuario se quiere autenticar, y pincha en la URL para recibir el formulario, hasta que este usuario recibe un mensaje de bienvenida indicando que está ya autenticado.
\item Escribe el código de una vista (view) que de servicio al recurso ``/login''. Escríbelo como se haría en una view Django (pero si prefieres, usando pseudo-Python o pseudocódigo).
\item Describe la interacción HTTP que se producirá desde que un navegador invoca la un GET sobre ``/resource'' hasta que recibe la pertinente respuesta de la aplicación web. Hazlo primero en el caso de que el navegador se haya autenticado previamente como usuario, y luego en caso de que no lo haya hecho.
\end{enumerate}

%%----------------------------------------------------------------------------
%%----------------------------------------------------------------------------
\subsection{Recomendaciones}

Te han pedido que diseñes un servicio en Internet para elegir, comentar y recibir recomendaciones sobre lugares para pasar las vacaciones. Las características principales del sistema serán:

\begin{enumerate}
\item La información, comentarios y recomendaciones siempre estarán referidos a un lugar (un pueblo, una playa, una zona).
\item Cualquier usuario del servicio podrá ``abrir'' un nuevo lugar, simplemente indicando su nombre y subiendo una descripción del mismo. A partir de ese momento, habrá una URL en el servicio que  mostrará esa información. Nos referiremos a esa URL como ``la página del lugar''.
\item Cualquier usuario del servicio (incluyendo el que lo abrió) podrá modificar la descripción de un lugar y/o añadir un comentario. Los comentarios y los cambios en la descripción se reflejarán inmediatamente en la página del lugar correspondiente.
\item Cualquier usuario del servicio podrá ``elegir'' un lugar. Para ello, tendrá un botón que podrá pulsar en la página de ese lugar.
\item Cualquier usuario del servicio podrá pedir que se le recomiende un lugar, según las elecciones pasadas propias y de otros usuarios. El algoritmo que el servicio use para realizar estas recomendaciones no es objeto del diseño.
\item No se quieren mantener cuentas de usuarios, pero sí se quiere poder diferenciar entre usuarios diferentes al menos para las elecciones y las recomendaciones (para que el algoritmo pueda diferenciar entre elecciones propias y elecciones de otros).
\item El sitio ofrecerá, para cada lugar, un canal RSS con los comentarios sobre ese lugar. También  habrá un canal RSS, único para todo el sitio, con los últimos lugares sobre los que se ha comentado.
\end{enumerate}

Salvo cuando se indique otra cosa, se supone que un usuario corresponde con un navegador en un ordenador concreto.

Teniendo en cuenta los requisitos anteriores, se pide:

\begin{enumerate}
\item Detalla un esquema de URLs que permita nombrar, siguiendo en lo posible el diseño REST, todos los elementos del servicio. Procura no usar URLs innecesarias.

\item Describe todas las interacciones HTTP que tendrán lugar en el sistema
  para abrir un lugar. Detalla las URLs implicadas, e indica las cabeceras más
  relevantes.

\item Ídem para realizar un comentario sobre un  lugar. En la página del lugar habrá un  formulario para poner comentarios, el usuario lo rellenará y a continuación lo verá en esa misma página del lugar (no se usa AJAX en este apartado).

\item Ídem para elegir un  lugar. El usuario habrá de estar a la vista del lugar que quiere elegir, y una vez elegido, tendrá que verlo como elegido en esa misma página (no se usa AJAX en este apartado).

\item Cuando un usuario cambie de navegador, querrá seguir siendo reconocido por el sistema. Diseña un mecanismo, lo más simple posible, que le permita hacerlo, manteniendo garantías de que quien no tenga acceso a su navegador no podrá colocarse en su lugar desde otro. Si es posible, diséñalo sin usar el correo electrónico.

\item Describe los cambios que habría que hacer al sistema para que en la página de cada lugar cualquier usuario pueda, además de comentarios, subir fotos.

\item Describe los cambios que habría que hacer en el sistema para que la elección de un lugar se pudiera expresar sin que se produzca una recarga de página, usando AJAX.

\item ¿Se podría construir un gadget, para integrar en un mashup, que mostrase los últimos comentarios que se están poniendo en el servicio? Explica qué partes del servicio especificado en la primera parte del ejercicio usarías, y si es caso, qué modificaciones del servicio harían falta.
\end{enumerate}

%%----------------------------------------------------------------------------
%%----------------------------------------------------------------------------
\subsection{Geolocalización}

Se decide construir un sitio para permitir que sus usuarios realicen
anotaciones geolocalizadas que puedan ser consultadas por
otros usuarios. Las características principales del sistema serán:

\begin{enumerate}
\item Cualquier usuario del sitio podrá subir una anotación
  geolocalizada. Para ello, rellenará un formulario en su navegador en el que
  especificará el texto que constituirá la anotación y sus
  coordenadas (latitud y longitud).
\item Cualquier usuario podrá consultar información
  geolocalizada, de varias formas:
  \begin{itemize}
  \item Especificando unas coordenadas (latitud y longitud) y una
    distancia en un formulario en el navegador. El sistema devolverá
    una página HTML con todas las anotaciones (incluyendo sus
    coordenadas y el texto correspondiente) que estén cerca de las
    coordenadas especificadas (a menos de la distancia indicada).
  \item Especificando unas coordenadas (latitud y longitud) y una
    distancia como parte de una URL del servicio, y obteniendo como
    respuesta un canal GeoRSS con todas las anotaciones (incluyendo sus
    coordenadas y el texto correspondiente) que estén cerca de las
    coordenadas especificadas (a menos de la distancia indicada).
  \item Especificando unas coordenadas (latitud y longitud) y una
    distancia como parte de una URL del servicio, y obteniendo como
    respuesta un mapa con los puntos anotados (en formato PNG).
  \item Especificando una cadena de texto en un formulario en el
    navegador. El sistema devolverá una página HTML con todas las
    anotaciones (incluyendo sus coordenadas y el texto
    correspondiente) que incluyan ese texto.
  \end{itemize}
\item Los usuarios podrán usar el sitio sin tener que abrir cuenta (de
  hecho, el sitio no mantendrá cuentas).
\item Cualquier anotación podrá ser editada (para modificarla o
  eliminarla) las veces que se quiera, si se hace desde el mismo
  navegador desde el que se creó.
\end{enumerate}

En particular, y teniendo en cuenta los requisitos anteriores,
se pide:

\begin{enumerate}
\item Describe todas las interacciones HTTP que tendrán lugar en el sistema
  para crear una anotación. Detalla las URLs implicadas, e indica las cabeceras más relevantes.
\item Ídem para ver como página HTML las anotaciones cercanas a una
  posición dada por sus coordenadas.
\item Ídem para editar una anotación previamente creada desde el mismo
  navegador.
\item Se quiere que si un usuario pierde su ordenador, y pasa a usar
  uno nuevo, pueda seguir editando las anotaciones que creó. Describe
  un mecanismo que lo permita, sin obligar al usuario a crear una
  cuenta en el sistema.
\item Se quiere utilizar el servicio de consulta de anotaciones desde
  un programa de gestión de mapas. El programa ya tiene funcionalidad
  de mostrar mapas, y de mostrar información asociada con un punto
  cualquiera del mapa. Se pretende que se utilice esta funcionalidad
  de mostrar información para mostrar las anotaciones. Explicar cómo
  se podría usar el servicio descrito en la primera parte de este
  ejercicio. Indica las URLs y las transacciones HTTP involucradas
  (indicando sus principales cabeceras) para que la aplicación pueda
  mostrar las anotaciones cercanas a un punto del mapa.
\item Indica cómo se podría usar el servicio descrito en la primera
  parte del ejercicio para que desde la aplicación
  del apartado anterior se puedan también crear anotaciones. ¿Puede
  decirse que la parte del servicio que has usado sigue las
  directrices REST?
\item Pasado un tiempo se plantea la posibilidad de incorporar cuentas
  de usuario para que estos puedan autenticarse en el sitio web.
  Describe brevemente 2 mecanismos (en cuanto a interacción
  navegador-servicio) que podrían usarse con HTTP para realizar la
  autenticación y las principales ventajas e inconvenientes de cada
  uno.
\end{enumerate}

\newpage

%%----------------------------------------------------------------------------
%%----------------------------------------------------------------------------
%%----------------------------------------------------------------------------
\section{Prácticas de entrega voluntaria de cursos pasados}

\subsection{Prácticas de entrega voluntaria (curso 2014-2015)}


%%----------------------------------------------------------------------------
%%----------------------------------------------------------------------------
\subsubsection{Práctica 1 (entrega voluntaria)}
\label{subsec:practica-vol-1-2013}

\textbf{Fecha recomendada de entrega:} Antes del 15 de marzo.

Esta práctica tendrá como objetivo la creación de una aplicación web simple para acortar URLs. La aplicación funcionará únicamente con datos en memoria: se supone que cada vez que la aplicación muera y vuelva a ser lanzada, habrá perdido todo su estado anterior. La aplicación tendrá que realizarse según un esquema de clases similar al explicado en clase.

El funcionamiento de la aplicación será el siguiente:

\begin{itemize}
\item Recurso ``/'', invocado mediante GET. Devolverá una página HTML con un formulario. En ese formulario se podrá escribir una url, que se enviará al servidor mediante POST. Además, esa misma página incluirá un listado de todas las URLs reales y acortadas que maneja la aplicación en este momento.

\item Recurso ``/'', invocado mediante POST. Si el comando POST incluye una qs (query string) que corresponda con una url enviada desde el formulario, se devolverá una página HTML con la url original y la url acortada (ambas como enlaces pinchables), y se apuntará la correspondencia (ver más abajo).

Si el POST no trae una qs que se haya podido generar en el formulario, devolverá una página HTML con un mensaje de error.

Si la URL especificada en el formulario comienza por ``http://'' o ``https://'', se considerará que ésa es la url a acortar. Si no es así, se le añadirá ``http://'' por delante, y se considerará que esa es la url a acortar. Por ejemplo, si en el formulario se escribe ``http://gsyc.es'', la url a acortar será ``http://gsyc.es''. Si se escribe ``gsyc.es'', la URL a acortar será ``http://gsyc.es''.

Para determinar la URL acortada, utilizará un número entero secuencial, comenzando por 0, para cada nueva petición de acortamiento de una URL que se reciba. Si se recibe una petición para una URL ya acortada, se devolverá la URL acortada que se devolvió en su momento.

Así, por ejemplo, si se quiere acortar

\verb|http://docencia.etsit.urjc.es|

y la aplicación está en el puerto 1234 de la máquina ``localhost'', se invocará (mediante POST) la URL

\verb|http://localhost:1234/|

y en el cuerpo de esa petición HTTP irá la qs

\verb|url=http://docencia.etsit.urjc.es|

si el campo donde el usuario puede escribir en el formulario tiene el nombre ``URL''. Normalmente, esta invocación POST se realizará rellenando el formulario que ofrece la aplicación.

Como respuesta, la aplicación devolverá (en el cuerpo de la respuesta HTTP) la URL acortada, por ejemplo

\verb|http://localhost:1234/3|

Si a continuación se trata de acortar la URL

\verb|http://docencia.etsit.urjc.es/moodle/course/view.php?id=25|

mediante un procedimiento similar, se recibirá como respuesta la URL acortada

\verb|http://localhost:1234/4|

Si se vuelve a intentar acortar la URL

\verb|http://docencia.etsit.urjc.es|

como ya ha sido acortada previamente, se devolverá la misma URL corta:

\verb|http://localhost:1234/3|

\item Recursos correspondientes a URLs acortadas. Estos serán números con el prefijo ``/''. Cuando la aplicación reciba un GET sobre uno de estos recursos, si el número corresponde a una URL acortada, devolverá un HTTP REDIRECT a la URL real. Si no la tiene, devolverá HTTP ERROR ``Recurso no disponible''.

Por ejemplo, si se recibe 

\verb|http://localhost:1234/3|

la aplicación devolverá un HTTP REDIRECT a la URL

\verb|http://docencia.etsit.urjc.es|

\end{itemize}

\textbf{Comentario}

Se recomienda utilizar dos diccionarios para almacenar las URLs reales y los números de las URLs acortadas. En uno de ellos, la clave de búsqueda será la URL real, y se utilizará para saber si una URL real ya está acortada, y en su caso saber cuál es el número de la URL corta correspondiente.

En el otro diccionario la clave de búsqueda será el número de la URL acortada, y se utilizará para localizar las URLs reales dadas las cortas. De todas formas, son posibles (e incluso más eficientes) otras estructuras de datos.

Se recomienda realizar la aplicación en varios pasos:

\begin{itemize}
\item Comenzar por reconocer ``GET /'', y devolver el formulario correspondiente.
\item Reconocer ``POST /'', y devolver la página HTML correspondiente (con la URL real y la acortada).
\item Reconocer ``GET /num'' (para cualquier número num), y realizar la redirección correspondiente.
\item Manejar las condiciones de error y realizar el resto de la funcionalidad.
\end{itemize}

%%---------------------------------------------------------------------
%%---------------------------------------------------------------------
\subsubsection{Práctica 2 (entrega voluntaria)}
\label{subsec:practica-vol-2-2013}

\textbf{Fecha recomendada de entrega:} Antes del 19 de abril.

Esta práctica tendrá como objetivo la creación de una aplicación web (de nombre \emph{acorta}) simple para acortar URLs utilizando Django (proyecto \emph{project}). Su enunciado será igual que el de la práctica 1 de entrega voluntaria (ejercicio~\ref{subsec:practica-vol-1-2013}), salvo en los siguientes aspectos:

\begin{itemize}
\item Se implementará utilizando Django.
\item Tendrá que almacenar la información relativa a las URLs que acorta en una base de datos, de forma que aunque la aplicación sea rearrancada, las URLs acortadas sigan funcionando adecuadamente.
\end{itemize}

Repositorio GitLab de entrega: \\
\url{https://gitlab.etsit.urjc.es/CursosWeb/X-Serv-18.2-Practica2}


%%----------------------------------------------------------------------------
%%----------------------------------------------------------------------------
%%----------------------------------------------------------------------------
\subsection{Prácticas de entrega voluntaria (curso 2012-2013)}

%%----------------------------------------------------------------------------
%%----------------------------------------------------------------------------
\subsubsection{Práctica 1 (entrega voluntaria)}
\label{subsec:practica-vol-1-2012}

\textbf{Fecha recomendada de entrega:} Antes del 12 de marzo.

Esta práctica tendrá como objetivo la creación de una aplicación web simple para acortar URLs. La aplicación funcionará únicamente con datos en memoria: se supone que cada vez que la aplicación muera y vuelva a ser lanzada, habrá perdido todo su estado anterior. La aplicación tendrá que realizarse según un esquema de clases similar al explicado en clase.

El funcionamiento de la aplicación será el siguiente:

\begin{itemize}
\item Recursos que comienzan por el prefijo ``/acorta/'' (invocados mediante GET). Estos recursos se utilizarán para devolver URLs acortadas, por el procedimiento de proporcionar un número entero secuencial, comenzando por 0, para cada nueva petición de acortamiento de una URL que se reciba. Si se recibe una petición para una URL ya acortada, se devolverá la URL acortada que se devolvió en su momento. La URL a acortar se especificará como parte del nombre de recurso, justo a partir de ``/acorta/'' (quitando la parte ``http://'' de la URL.

Así, por ejemplo, si se quiere acortar

\verb|http://docencia.etsit.urjc.es|

y la aplicación está en el puerto 1234 de la máquina ``localhost'', se invocará (mediante GET) la URL

\verb|http://localhost:1234/acorta/docencia.etsit.urjc.es|

Como respuesta, la aplicación devolverá (en el cuerpo de la respuesta HTTP) la URL acortada, por ejemplo

\verb|http://localhost:1234/3|

Si a continuación se trata de acortar la URL

\verb|http://docencia.etsit.urjc.es/moodle/course/view.php?id=25|

se invocará para ello la URL

\verb|http://localhost:1234/acorta/docencia.etsit.urjc.es/moodle/course/view.php?id=25|

y se recibirá como respuesta la URL acortada

\verb|http://localhost:1234/4|

Si se vuelve a intentar acortar la URL

\verb|http://docencia.etsit.urjc.es|

como ya ha sido acortada previamente, se devolverá la misma URL corta:

\verb|http://localhost:1234/3|

\item Recursos correspondientes a URLs acortadas. Estos serán números con el prefijo ``/''. Cuando la aplicación reciba un GET sobre uno de estos recursos, si el número corresponde a una URL acortada, devolverá un HTTP REDIRECT a la URL real. Si no la tiene, devolverá HTTP ERROR ``Recurso no disponible''.

Por ejemplo, si se recibe 

\verb|http://localhost:1234/3|

la aplicación devolverá un HTTP redirect a la URL

\verb|http://docencia.etsit.urjc.es|

\item Recurso ``/''. Si se invoca este recurso con GET, se obtendrá un listado de todas las URLs reales y acortadas que maneja la aplicación en este momento.
\end{itemize}

\textbf{Comentario}

Se recomienda utilizar dos diccionarios para almacenar las URLs reales y los números de las URLs acortadas. En uno de ellos, la clave de búsqueda será la URL real, y se utilizará para saber si una URL real ya está acortada, y en su caso saber cuál es el número de la URL corta correspondiente.

En el otro diccionario la clave de búsqueda será el número de la URL acortada, y se utilizará para localizar las URLs reales dadas las cortas. De todas formas, son posibles (e incluso más eficientes) otras estructuras de datos.

%%---------------------------------------------------------------------
%%---------------------------------------------------------------------
\subsubsection{Práctica 2 (entrega voluntaria)}
\label{subsec:practica-vol-2-2012}

\textbf{Fecha recomendada de entrega:} Antes del 9 de abril.

Esta práctica tendrá como objetivo la creación de una aplicación web simple para acortar URLs utilizando Django. Su enunciado será igual que el de la práctica 1 de entrega voluntaria (ejercicio~\ref{subsec:practica-vol-1-2012}), salvo en los siguientes aspectos:

\begin{itemize}
\item Se implementará utilizando Django.
\item Tendrá que almacenar la información relativa a las URLs que acorta en una base de datos, de forma que aunque la aplicación sea rearrancada, las URLs acortadas sigan funcionando adecuadamente.
\end{itemize}


%%----------------------------------------------------------------------------
%%----------------------------------------------------------------------------
%%----------------------------------------------------------------------------
\subsection{Prácticas de entrega voluntaria (curso 2011-2012)}

%%----------------------------------------------------------------------------
%%----------------------------------------------------------------------------
\subsubsection{Práctica 1 (entrega voluntaria)}
\label{subsec:practica-vol-1-2011}

%\textbf{Fecha recomendada de entrega:} Antes del 31 de octubre.

Esta práctica tendrá como objetivo la creación de una aplicación web para acceso a los artículos de Wikipedia con almacenamiento en cache.

La aplicación servirá dos tipos de recursos:

\begin{itemize}
\item ``/decorated/article'': servirá la página correspondiente al artículo ``article'' de la Wikipedia en inglés, decorado con las cajas auxiliares.
\item ``/raw/article'': servirá la página correspondiente al artículo ``article'' de la Wikipedia en inglés, sin decorar con las cajas auxiliares.
\end{itemize}

La página ``decorada'' es accesible mediante URLs de la siguiente forma (para el artículo ``pencil'' de la Wikipedia en inglés):

\begin{verbatim}
http://en.wikipedia.org/w/index.php?title=pencil&action=view
\end{verbatim}

El contenido que sirven estas URLs está previsto para ser directamente mostrado, como página HTML completa, por un navegador.

La página ``no decorada'' es accesible mediante URLs de la siguiente forma (para el artículo ``pencil'' de la Wikipedia en inglés):

\begin{verbatim}
http://en.wikipedia.org/w/index.php?title=pencil&action=render
\end{verbatim}

El contenido que sirven estas URLs está previsto para ser directamente empotrable en una página HTML, dentro del elemento ``body'' (y por lo tanto la aplicación web tendrá que aportar el HTML necesario para acabar teniendo una página HTML correcta).

Cualquiera de los dos tipos de recursos se comportará de la misma forma. Si es invocado mediante GET, usará para responder el artículo que tenga en cache. Si no lo tiene, lo bajará previamente accediendo a la URL adecuada, que se indicó anteriormente, lo almacenará en la cache, y lo usará para responder.

La respuesta, en cada caso, será una página HTML que contenga en la parte superior la siguiente información:

\begin{itemize}
\item Nombre del artículo, junto con la indicación ``(decorated)'' o ``(non decorated)'', según corresponda. Por ejemplo, ``Pencil (decorated)''.
\item Enlaces a las páginas con el artículo en la Wikipedia (versiones decorada y no decorada)
\item Enlace a la historia de modificaciones del artículo en la Wikipedia
\item Enlace al último artículos de la Wikipedia que ha servido la aplicación (al navegador que le hizo la petición, o a cualquier otro).
\item Línea de separación (elemento ``hr'').
\end{itemize}

Y a continuación el texto correspondiente del  artículo de la Wikipedia (decorado o no decorado, según sea el nombre del recurso invocado).

En caso de que se pida un artículo que no exista en la Wikipedia, se devolverá el código de error correspondiente, y se marcará en la cache, de alguna forma, que ese artículo no existe, para no tener que buscarlo en caso de que vuelva a ser pedido. En general, puede usarse algún texto que aparezca en la página que devuelve Wikipedia cuando sirve la página de un artículo que no existe, como por ejemplo:

\begin{verbatim}
<div class="noarticletext">
\end{verbatim}

\textbf{Materiales de apoyo:}

\begin{itemize}
\item Parámetros de index.php en Wikipedia (MediaWiki):
\url{http://www.mediawiki.org/wiki/Manual:Parameters_to_index.php#View_and_render}
\end{itemize}

\textbf{Comentario:}

En algunas circunstancias, el servidor de Wikipedia puede devolver un código de redirección (por ejmeplo, un ``301 Moved permanently''). Téngase en cuenta que la aplicación ha de reconocer esta situación, y repetir el GET en la URL a la que se redirige.

%%----------------------------------------------------------------------------
%%----------------------------------------------------------------------------
\subsubsection{Práctica 2 (entrega voluntaria)}
\label{subsec:practica-vol-2-2011}

%\textbf{Fecha recomendada de entrega:} Antes del 20 de noviembre.

Realiza lo especificado en la práctica 1 (ejercicio~\ref{subsec:practica-vol-1-2011}), pero usando el entorno de desarrollo Django. En particular, utiliza plantillas (templates) para la generación de las páginas HTML, tablas en base de datos para almacenar las páginas de Wikipedia descargada, y añade la siguiente funcionalidad:

\begin{itemize}
\item Utilizando el módulo correspondiente de Django, añade usuarios, que se autenticarán en el recurso ``/login''. Las cuentas de usuario estarán dadas de alta por el administrador (vía módulo Admin de Django). Si una página es bajada por un usuario autenticado se incluirá en la parte superior el mensaje ``Usuario: user (logout)'', siendo ``user'' el identificador de usuario correspondiente, y ``logout'' un enlace al recurso que puede utilizar el usuario para salir de su cuenta. Si la página es bajada sin haberse autenticado previamente, en lugar de ese mensaje se incluirá ``Usuario anónimo (login)'', siendo ``login'' un enlace al recurso ``/login''.
\item La aplicación atenderá el recurso ``/'', en el que ofrecerá (si se invoca con ``GET'') una lista de los artículos de Wikipedia disponibles en la base de datos, junto al enlace correspondiente (bajo ``/decorated'' o bajo ``/raw'') para descargarla, y el mensaje ``decorated'' o ``raw'', según el tipo de artículo descargado.
\end{itemize}


%%----------------------------------------------------------------------------
%%----------------------------------------------------------------------------
%%----------------------------------------------------------------------------
\subsection{Prácticas de entrega voluntaria (curso 2010-2011)}

%%----------------------------------------------------------------------------
%%----------------------------------------------------------------------------
\subsubsection{Práctica 1 (entrega voluntaria)}
\label{subsec:practica-vol-1-2010}

Esta práctica tendrá como objetivo la creación de una aplicación web para acceso a los artículos de Wikipedia, con almacenamiento en cache, y con consulta en varios idiomas.

La aplicación servirá dos tipos de recursos:

\begin{itemize}
\item ``/article'': servirá la página correspondiente al artículo ``article'' de la Wikipedia en inglés.
\item ``/language/article'': servirá la página correspondiente al artículo ``article'' correspondiente al idioma ``language'', expresado mediante el código ISO de dos letras. Bastará con que funcione con los idiomas inglés (en) y español (es).
\end{itemize}

La página que se bajará de la Wikipedia para cada artículo será la ``no decorada'', accesible mediante URLs de la siguiente forma (para el artículo ``pencil'' de la Wikipedia en inglés):

\begin{verbatim}
http://en.wikipedia.org/w/index.php?action=render&title=pencil
\end{verbatim}

El contenido que sirve esta URL está previsto para ser directamente empotrable en una página HTML, dentro del elemento ``body''.

Cualquiera de los dos tipos de recursos se comportará de la misma forma. Si es invocado mediante GET, usará para responder el artículo que tenga en cache. Si no lo tiene, lo bajará previamente accediendo a la URL de página no decorada, que se indicó anteriormente, lo almacenará en la cache, y lo usará para responder.

La respuesta será una página HTML que contenga:

\begin{itemize}
\item Título de la página (nombre del artículo).
\item Enlace a la página con el artículo en la Wikipedia (versión decorada)
\item Enlace a la historia de modificaciones del artículo en la Wikipedia
\item Enlace a los tres últimos artículos de la Wikipedia que ha servido la aplicación (al navegador que le hizo la petición, o a cualquier otro).
\item Texto de la página no decorada del artículo de la Wikipedia.
\end{itemize}

En caso de que se pida un artículo que no exista en la Wikipedia, se devolverá el código de error correspondiente, y se marcará en la cache, de alguna forma, que ese artículo no existe, para no tener que buscarlo en caso de que vuelva a ser pedido. En general, puede usarse algún texto que aparezca en la página que devuelve Wikipedia cuando sirve la página de un artículo que no existe, como por ejemplo:

\begin{verbatim}
<div class="noarticletext">
\end{verbatim}

\textbf{Materiales de apoyo:}

\begin{itemize}
\item Parámetros de index.php en Wikipedia (MediaWiki):
\url{http://www.mediawiki.org/wiki/Manual:Parameters_to_index.php#View_and_render}
\end{itemize}

%%----------------------------------------------------------------------------
%%----------------------------------------------------------------------------
\subsubsection{Práctica 2 (entrega voluntaria)}
\label{subsec:practica-vol-2-2010}

Esta práctica consistirá en la realización de un gestor de contenidos que tenga las siguientes características:

\begin{itemize}
\item Funcionalidad de ``Gestor de contenidos con usuarios, con control estricto de actualización y uso de base de datos'' (ejercicio~\ref{subsec:gestor-contenidos-usuarios-bbdd})

\item Implementación de HEAD para todos los recursos.

\item Terminación de una sesión autenticada. Para ello se usará el recurso ``/logout''.

\item Además, cada página que se obtenga con un GET irá anotada con la siguiente información:
  \begin{itemize}
  \item Sólo si la página no la está viendo un usuario autenticado. Enlace que permita la autenticación del usuario que creó la página (a falta de la contraseña). Aparecerá con la cadena ``Autor: user'', siendo ``user'' el nombre de usuario que creó la página, y estando enlazado a ``/login,user,''.
  \item Enlace que permita ver el mensaje HTTP que envió el navegador para poder ver esa página (se puede suponer que esa fue la última página descargada desde este navegador).
  \item Enlace que permita ver la respuesta HTTP que envió el servidor para poder ver esa página (se puede suponer que esa fue la última página descargada desde este navegador).
  \end{itemize}
\end{itemize}

Además, opcionalmente, podrá tener:

\begin{itemize}
\item Creación de cuentas de usuario. Para ello se usará un recurso ``/signin,user,passwd'', sobre el que un GET creará el usuario ``user'' con la contraseña ``passwd'', si ese usuario no existía ya.
\item Subida de páginas con POST. en lugar de PUT. Se usará un POST para subir una nueva página. No hace falta implementar un formulario HTML que invoque el POST, pero también se podría hacer.
\item Una implementación que no tenga la limitación de que los enlaces al mensaje HTTP del navegador y del servidor sean de la última página descargada, sino de los de la descarga de la página que los tiene, sea la última o no.
\end{itemize}

Realizar la entrega en un fichero tar.gz o .zip, incluyendo además del código fuente los ficheros de SQLite3 necesarios, y un fichero README que resuma la funcionalidad exacta que se ha implementado (en particular, que detalle la funcionalidad opcional implementada).

%%----------------------------------------------------------------------------
%%----------------------------------------------------------------------------
\subsubsection{Práctica 3 (entrega voluntaria)}
\label{subsec:practica-vol-3-2010}

Realiza lo especificado en la práctica 2, pero usando el entorno de desarrollo Django. Donde lo creas oportuno, interpreta las especificaciones en el contexto de las facilidades que proporciona Django. Por ejemplo, la autenticación de usuarios se puede hacer vía un formulario de login (con el POST correspondiente) usando los módulos que proporciona Django para ello.

Igualmente, extiende las especificaciones en lo que te sea simple al usar las facilidades de Django. Por ejemplo, la gestión de usuarios (creación y borrado de usuarios) puede hacerse fácilmente usando módulos Django.

En la medida que sea razonable, usa POST (con los correspondientes formularios) en lugar de PUT. Opcionalmente, mantén ambas funcionalidades (subida de contenidos vía PUT, como se indicaba en la práctica 2, y vía POST, como se está recomendando para ésta).

\textbf{Notas:}

Parte de la especificación requiere almacenar las cabeceras de la respuesta del servidor al navegador. En Django, las cabeceras se van añadiendo al objeto HTTPResponse (o similar), y por tanto será necesario extraerlas de él. La forma más simple, y suficiente para estas prácticas, es simplemente convertir el objeto HTTPResponse en string: ``str(response)''. Si se quiere, se puede manipular el string resultante, para obtener las cabeceras en un formato más parecido al de la práctica 1, pero esto no será necesario para la versión básica.


%%----------------------------------------------------------------------------
%%----------------------------------------------------------------------------
\subsubsection{Práctica 4 (entrega voluntaria)}
\label{subsec:practica-vol-4-2010}

Realización de lo especificado en la práctica 3 de entrega voluntaria, utilizando para la implementación las posibilidades avanzadas de Django, incluyendo especialmente las plantillas, y si es posible el sitio de administración, los usuarios y las sesiones Django. La parte básica seguirá siendo básica, y la opcional, opcional (más la adición, opcional, que se comenta más adelante).

La funcionalidad de esta práctica es, por lo tanto, la misma que la de la práctica 3. Pero a diferencia de la práctica 3, en este caso sí se pide usar los módulos ``de alto nivel'' de Django.

La URL que se usaba en las prácticas 2 y 3 para autenticarse pasa a ser ``/login'', que en el caso de recibir un GET devolverá el formulario para autenticarse, y en caso de recibir un POST gestionará la autenticación.

A la parte opcional de las prácticas 2 y 3, que sigue siendo opcional, se añade la de modificar el contenido de las páginas con formularios (usando métodos POST para la actualización), y de crear nuevas páginas también mediante formularios y POST. Para la actualización se sugiere que se usen nombres de recurso de la forma ``/edit/name'', siendo ``name'' el nombre de la página. Para la creación se sugiere que se use un nombre de recurso de la forma ``/create''.

Con respecto a la opción de crear usuarios, ahora la opción cambia a servir una URL ``/signin'' que devuelva el formulario para crearse una cuenta, y que cuanto reciba un POST gestione la creación de la cuenta.

\newpage

%%----------------------------------------------------------------------------
%%----------------------------------------------------------------------------
%%----------------------------------------------------------------------------

\section{Prácticas finales de cursos pasados}

%%----------------------------------------------------------------------------
%%----------------------------------------------------------------------------
\subsection{Práctica final (2013, mayo)}
\label{practica-final-2013-05}

La práctica final de la asignatura consiste en la creación de un selector de noticias a partir de canales, como aplicación web. A continuación se describe el funcionamiento y arquitectura general de la aplicación, la funcionalidad mínima que debe proporcionar, y otra funcionalidad optativa que podrá tener. Llamaremos a la aplicación MiRevista.

%%----------------------------------------------------------------------------
\subsection{Arquitectura y funcionamiento general}

Arquitectura general:

\begin{itemize}
\item La práctica consistirá en una aplicación web que servirá los datos a los usuarios.

\item MiRevista se construirá como un proyecto Django, que incluirá una o varias aplicaciones Django que implementen la funcionalidad requerida.

\item Para el almacenamiento de datos persistente se usará SQLite3, con tablas definidas según modelos en Django.

\item Se usará la aplicación Django ``Admin Site'' para mantener los usuarios con cuenta en el sistema, y para la gestión general de las bases de datos necesarias. Todas las bases de datos que mantenga MiRevista tendrá que ser accesible vía este ``Admin Site''.

\item Se utilizarán plantillas Django (a ser posible, una jerarquía de plantillas, para que MiRevista tenga un aspecto similar) para definir las páginas que se servirán a los navegadores de los usuarios. Estas plantillas incluirán en todas las páginas al menos:
  \begin{itemize}
  \item Un banner (imagen) del sitio, en la parte superior.
  \item Un menú de opciones.
  \item Una caja para entrar (hacer login en el sitio), o para salir (si ya se ha entrado). En caso de que no se haya entrado en una cuenta, esta caja permitirá al visitante introducir su identificador de usuario y su contraseña. En caso de que ya se haya entrado, esta caja mostrará el identificador del usuario y permitirá salir de la cuenta (logout).
  \item Un pie de página con una nota de copyright.
  \end{itemize}

Cada una de estas partes estará marcada con propiedades ``id'' en HTML, para poder ser referenciadas en hojas de estilo CSS.

\item Se utilizarán hojas de estilo CSS para determinar la apariencia de MiRevista. Estas hojas definirán al menos el color y el tamaño de la letra, y el color de fondo de cada una de las partes (elementos) marcadas con id que se indican en el apartado anterior.
\end{itemize}

Funcionamiento general:

\begin{itemize}
\item Los usuarios serán dados de alta en MiRevista mediante el módulo ``Admin Site'' de Django. Una vez estén dados de alta, serán considerados ``usuarios registrados''.

\item Los usuarios registrados podrán crear su selección de noticias en MiRevista. Para ello, dispondrán de una página personal, en la que trabajarán. Llamaremos a esta página la ``revista del usuario''.

\item La selección de noticias de su revista la realizará cada usuario a partir de canales RSS de sitios web ya disponibles en el sitio.

\item Además, si hay un canal no disponible en el sitio, un usuario podrá indicar sus datos para que pase a estar disponible.

\item Los contenidos de cada canal se actualizarán sólo cuando un usuario indique que quiere que se actualicen (esta indicación se hará por separado para cada canal que se quiera actualizar).

\item Cualquier navegador podrá acceder a la interfaz pública del sitio, que ofrecerá la revista de cada usuario, para todos los usuarios del sitio.
\end{itemize}


%%----------------------------------------------------------------------------
\subsection{Funcionalidad mínima}

Interfaz pública: recursos a servir como páginas HTML completas (pensadas para ser vistas en el navegador) para cualquier visitante (sea usuario registrado o no):

\begin{itemize}
\item /: Página principal de MiRevista. Mostrará la lista de las revistas disponibles, ordenadas por fecha de actualización, en orden inverso (las revistas actualizadas más recientemente, primero). Para cada revista se mostrará su título (como un enlace a la página de la revista), el nombre de su usuario y la fecha de su última actualización (fecha en que se añadió una noticia a esa revista por última vez). Si a una revista aún no se le hubiera puesto título, este título será ``Revista de usuario'', donde ``usuario'' es el identificador de usuario del usuario en cuestión.

\item /usuario: Página de la revista de un usuario. Si la URL es ``/usuario'', es que corresponde al usuario ``usuario''. Mostrará las 10 noticias de la revista de ese usuario (no puede haber más de 10, como se indicará más adelante). Para cada noticia se mostrará la ``información pública de noticia'', ver más adelante.

\item /usuario/rss: Canal RSS para la revista de ese usuario.

\item /ayuda: Página con información HTML explicando el funcionamiento de MiRevista.
\end{itemize}

Además, todas las páginas de la interfaz pública incluirán un menú desde el que se podrá acceder la ayuda (URL /ayuda) con el texto ``Ayuda''.

Además, todas las página que no sean la principal tendrán otra opción de menú para la URL /, con el texto ``Revistas''.

Interfaz privada: recursos a servir como páginas HTML completas para usuarios registrados (una vez se han autenticado).

\begin{itemize}
\item Todos los recursos de la interfaz pública.
\item /canales: Página con la lista de los canales disponibles en MiRevista:

  \begin{itemize}
  \item Para cada canal se mostrará el nombre del canal (apuntando a la página de ese canal en MiRevista, ver más adelante), el logo del canal, el número de mensajes disponibles para el canal, y la fecha en que fue actualizado por última vez.
  \item Además, en esta página se mostrará un formulario en el que se podrá introducir una URL, que se interpretará como la URL de un nuevo canal. Esta será la forma de añadir un nuevo canal para que esté disponible en el sitio. Cuando se añada un nuevo canal se tratarán de actualizar sus contenidos a partir de la URL indicada: si esta operación falla (bien porque la URL no está disponible, bien porque no se puede interpretar su contenido como un documento RSS), no se añadirá el canal como disponible. En cualquier caso, tras tratar de añadir un nuevo canal se volverá a ver la página /canales en el navegador.
  \end{itemize}

\item /canales/num: Página de un canal en MiRevista. ``num'' es el número de orden en que se hizo disponible (si fue el segundo canal que se hizo disponible en el sitio, será /canales/2). Mostrará:

  \begin{itemize}
  \item El nombre del canal (según venga como titulo en el canal RSS correspondiente) como enlace apuntando al sitio web donde se puede ver el contenido del canal (ojo: el contenido original, no el canal RSS)
  \item Junto a él pondrá entre paréntesis ``canal'', como enlace al canal RSS correspondiente en el sitio original
  \item Un botón para actualizar el canal. Si se pulsa este botón, se tratarán de actualizar las noticias de ese canal accediendo al documento RSS correspondiente en su sitio web de origen. Al terminar la operación se volverá a mostrar esta misma página /canales/num.
  \item La lista de noticias de ese canal, incluyendo para cada una la ``información pública de noticia'', ver más adelante.
  \item Junto a cada noticia de la lista, se incluirá un botón que permitirá elegir la noticia para la revista del usuario autenticado. Si al añadirla la lista de noticias de esa revista fuera de más de 10, se eliminarán las que se eligieron hace más tiempo, de forma que no queden más de 10. Tras añadir una noticia a la revista del usuario, se volverá a ver en el navegador la página /canales/num correspondiente al canal en que se seleccionó.
  \end{itemize}

\item En la página /usuario que corresponde al usuario autenticado se mostrará, además de lo ya mencionado para la interfaz pública, un formulario en el que se podrá especificar la siguiente información:

  \begin{itemize}
  \item Los parámetros CSS para el usuario autenticado (al menos los indicados anteriormente para ser manejados por un documento CSS). Si el usuario los cambia, a partir de ese momento deberá verse el sitio con los nuevos valores, y para ello deberá servirse un nuevo documento CSS.
  \item El título de la revista del usuario autenticado.
  \end{itemize}
\end{itemize}

Si es preciso, se añadirán más recursos (pero sólo si es realmente preciso) para poder satisfacer los requisitos especificados.

Dados los recursos mencionados anteriormente, no se permitirán los nombres de usuario ``canales'' ni ``ayuda'' (pero no hay que hacer ninguna comprobación para esto: se asume que no se darán de alta esos usuarios en el Admin Site).


Como información pública de noticia se mostrará:
\begin{itemize}
\item El título de la noticia, como enlace a la noticia en el sitio web original.
\item La fecha en que fue publicada la noticia en el sitio original (junto al texto ``publicada en'').
\item La fecha en que fue seleccionada para esta revista (junto al texto ``elegida en'').
\item El contenido de la noticia.
\item El nombre del canal de donde viene la noticia, como enlace a la página de ese canal en MiRevista.
  \end{itemize}



%%----------------------------------------------------------------------------
\subsection{Funcionalidad optativa}

De forma optativa, se podrá incluir cualquier funcionalidad relevante en el contexto de la asignatura. Se valorarán especialmente las funcionalidades que impliquen el uso de técnicas nuevas, o de aspectos de Django no utilizados en los ejercicios previos, y que tengan sentido en el contexto de esta práctica y de la asignatura.

Sólo a modo de sugerencia, se incluyen algunas posibles funcionalidades optativas:

\begin{itemize}
\item Atención al idioma indicado por el navegador. El idioma de la interfaz de usuario del planeta tendrá en cuenta lo que especifique el navegador.

\item Generación de un canal RSS para los contenidos que se muestran en la página principal.

\item Uso de AJAX para algún aspecto de la práctica (por ejemplo, para seleccionar una noticia para una revista).

\item Puntuación de noticias. Cada visitante (registrado o no) puede dar un ``+1'' a cualquier noticia del sitio. La suma de ``+'' que ha obtenido una noticia se verá cada vez que se vea la noticia en el sitio.

\item Comentarios a revistas. Cada usuario registrado puede comentar cualquier revista del sitio. Estos comentarios se podrán ver luego en la página de la revista.

\item Autodescubrimiento de canales. Dada una URL (de un blog, por ejemplo), busca si en ella hay algún enlace que parece un canal. Si es así, ofrécelo al usuario para que lo pueda elegir. Esto se puede usar, por ejemplo, en la página que muestra el listado de canales, como una opción más para elegir canales (``especifica un blog para buscar sus canales'').
\end{itemize}


%%----------------------------------------------------------------------------
\subsection{Entrega de la práctica}

\textbf{Fecha límite de entrega de la práctica:} 22 de mayo de 2013.

La práctica se entregará subiéndola al recurso habilitado a tal fin en el sitio Moodle de la asignatura. Los alumnos que no entreguen las práctica de esta forma serán considerados como no presentados en lo que a la entrega de prácticas se refiere. Los que la entreguen podrán ser llamados a realizar también una entrega presencial, que tendrá lugar en la fecha y hora exacta se les comunicará oportunamente. Esta entrega presencial podrá incluir una conversación con el profesor sobre cualquier aspecto de la realización de la práctica.

Para entregar la práctica en el Moodle, cada alumno subirá al recurso habilitado a tal fin un fichero tar.gz con todo el código fuente de la práctica. El fichero se habrá de llamar practica-user.tar.gz, siendo ``user'' el nombre de la cuenta del alumno en el laboratorio.

El fichero que se entregue deberá constar de un proyecto Django completo y listo para funcionar en el entorno del laboratorio, incluyendo la base de datos con datos suficientes como para poder probarlo. Estos datos incluirán al menos tres usuarios con sus datos correspondientes, y con al menos cinco noticias en su revista, y al menos tres canales RSS diferentes. Se incluirá también un fichero README con los siguientes datos:

\begin{itemize}
\item Nombre de la asignatura.
\item Nombre completo del alumno.
\item Nombre de su cuenta en el laboratorio.
\item Nombres y contraseñas de los usuarios creados para la práctica. Éstos deberán incluir al menos un usuario con cuenta ``marta'' y contraseña ``marta'' y otro usuario con cuenta ``pepe'' y contraseña ``pepe''.
\item Canales disponibles en el sitio, incluyendo su URL
\item Resumen de las peculiaridades que se quieran mencionar sobre lo implementado en la parte obligatoria.
\item Lista de funcionalidades opcionales que se hayan implementado, y breve descripción de cada una.
\item URL del vídeo demostración de la funcionalidad básica
\item URL del vídeo demostración de la funcionalidad optativa, si se ha realizado funcionalidad optativa
\end{itemize}

El fichero README se incluirá también como comentario en el recurso de subida de la práctica, asegurándose de que las URLs incluidas en él son enlaces ``pinchables''.

Los vídeos de demostración serán de una duración máxima de 2 minutos (cada uno), y consistirán en una captura de pantalla de un navegador web utilizando la aplicación, y mostrando lo mejor posible la funcionalidad correspondiente (básica u opcional). Siempre que sea posible, el alumno comentará en el audio del vídeo lo que vaya ocurriendo en la captura. Los vídeos se colocarán en algún servicio de subida de vídeos en Internet (por ejemplo, Vimeo o YouTube).

Hay muchas herramientas que permiten realizar la captura de pantalla. Por ejemplo, en GNU/Linux puede usarse Gtk-RecordMyDesktop o Istanbul (ambas disponibles en Ubuntu). Es importante que la captura sea realizada de forma que se distinga razonablemente lo que se grabe en el vídeo.

En caso de que convenga editar el vídeo resultante (por ejemplo, para eliminar tiempos de espera) puede usarse un editor de vídeo, pero siempre deberá ser indicado que se ha hecho tal cosa con un comentario en el audio, o un texto en el vídeo. Hay muchas herramientas que permiten realizar esta edición. Por ejemplo, en GNU/Linux puede usarse OpenShot o PiTiVi.

%%----------------------------------------------------------------------------
\subsection{Notas y comentarios}

La práctica deberá funcionar en el entorno GNU/Linux (Ubuntu) del laboratorio de la asignatura con la versión de Django que se ha usado en prácticas (Django 1.7.*).

La práctica deberá funcionar desde el navegador Firefox disponible en el laboratorio de la asignatura.

Se recomienda construir una o varias aplicaciones complementarias para probar la descarga y almacenamiento en base de datos de los canales que alimentarán las revistas.

Los usuarios registrados pueden, en principio, hacer disponible cualquier canal de cualquier blog. Sin embargo, para la funcionalidad mínima es suficiente que MiRevista funcione con blogs de WordPress.com.

Los canales (feeds) RSS que produce la aplicación web realizada en la práctica deberán funcionar al menos con el navegador Firefox (considerándolos como canales RSS) disponibles en el laboratorio.


%%----------------------------------------------------------------------------
%%----------------------------------------------------------------------------
\subsection{Práctica final (2012, diciembre)}
\label{practica-final-2012-12}

La práctica final de la asignatura consiste en la creación de un planeta, o agregador de canales, como aplicación web. A continuación se describe el funcionamiento y arquitectura general de la aplicación, la funcionalidad mínima que debe proporcionar, y otra funcionalidad optativa que podrá tener. Llamaremos a la aplicación MiPlaneta.

%%----------------------------------------------------------------------------
\subsubsection{Arquitectura y funcionamiento general}

Arquitectura general:

\begin{itemize}
\item La práctica consistirá en una aplicación web que servirá los datos a los usuarios.

\item MiPlaneta se construirá como un proyecto Django, que incluirá una o varias aplicaciones Django que implementen la funcionalidad requerida.

\item Para el almacenamiento de datos persistente se usará SQLite3, con tablas definidas según modelos en Django.

\item Se usará la aplicación Django ``Admin Site'' para mantener los usuarios con cuenta en el sistema, y para la gestión general de las bases de datos necesarias (todas las bases de datos que mantenga MiPlaneta tendrá que ser accesible vía este ``Admin Site''.

\item Se utilizarán plantillas Django (a ser posible, una jerarquía de plantillas, para que MiPlaneta tenga un aspecto similar) para definir las páginas que se servirán a los navegadores de los usuarios. Estas plantillas incluirán en todas las páginas al menos:
  \begin{itemize}
  \item Un banner (imagen) del sitio, en la parte superior.
  \item Un menú de opciones.
  \item Un pie de página con una nota de copyright.
  \end{itemize}

Cada una de estas partes estará marcada con propiedades ``id'' en HTML, para poder ser referenciadas en hojas de estilo CSS.

\item Se utilizarán hojas de estilo CSS para determinar la apariencia de MiPlaneta. Estas hojas definirán al menos el color de fondo y del texto, y alguna propiedad para las partes marcadas que se indican en el apartado anterior.
\end{itemize}

Funcionamiento general:

\begin{itemize}
\item Los usuarios serán dados de alta en MiPlaneta mediante el módulo ``Admin Site'' de Django. Una vez estén dados de alta, serán considerados ``usuarios registrados''.

\item Los usuarios registrados podrán especificar en MiPlaneta su número de usuario en el Moodle de la ETSIT. Por ejemplo, si la página de perfil de un usuario en el Moodle de la ETSIT es \url{http://docencia.etsit.urjc.es/moodle/user/profile.php?id=8} (llamaremos a la página a la que apunta esta URL la ``página del usuario en el Moodle de la ETSIT'') su número de usuario es 8. Puede obtenerse el número de usuario en el Moodle de la ETSIT a través de los enlaces a ese usuario en los mensajes que pone en sus foros, por ejemplo.

\item Cada usuario registrado podrá indicar el blog que le representa en MiPlaneta. Para ello, especificará la URL del canal RSS correspondiente a ese blog.

\item Habrá una URL para actualizar los contenidos.

\item Cualquier navegador podrá acceder a la interfaz pública del sitio, que ofrecerá los artículos en la base de datos e información pública para cada usuario.
\end{itemize}


%%----------------------------------------------------------------------------
\subsubsection{Funcionalidad mínima}

Interfaz pública: recursos a servir como páginas HTML completas (pensadas para ser vistas en el navegador) para cualquier visitante (sea usuario registrado o no):

\begin{itemize}
\item /: Página principal de MiPlaneta. Lista de los últimos 20 artículos, por fecha de publicación, en orden inverso (más nuevos primero). Para cada artículo se mostrará la ``información pública de articulo'', ver más abajo.

\item /users: Lista de usuarios registrados de MiPlaneta. Para casa usuario se mostrará la ``información resumida de usuario'', ver más abajo.

\item /users/alias: Información sobre el usuario que tiene el alias ``alias'' en MiPlaneta. El alias es el nombre de usuario que tiene como usuario registrado, fijado con el módulo ``Admin Site''. Se incluirá la ``información completa de usuario'', ver más abajo.

\item /update: Actualización de los artículos de todos los blogs. Cuando sea invocada, se bajarán todos los canales y se almacenarán en la base de datos los artículos correspondientes. Si un artículo ya estaba en la base de datos, no debe almacenarse dos veces. Al terminar, enviará una redirección a la página principal.
\end{itemize}

Además, todas las páginas de la interfaz pública incluirán un formulario para poder autenticarse si se es usuario registrado, y un menú desde el que se podrá acceder a / (con el texto ``página principal''), a /users (con el texto ``listado de usuarios'') y a /update (con el texto ``actualizar'').

Interfaz privada: recursos a servir como páginas HTML completas para usuarios registrados (una vez se han autenticado).

\begin{itemize}
\item Todos los recursos de la interfaz pública.
\item /conf: Configuración de usuario. Tendrá un formulario en el que se podrá especificar:
  \item Un número de usuario del Moodle de la ETSIT
  \item La URL del canal RSS de un blog
  \item El color de fondo de todas las páginas del blog
  \item El color del texto normal de todas las páginas del blog
\end{itemize}

Además, todas las páginas de la interfaz privada incluirán el nombre y la foto del usuario registrado (según aparecen en su perfil el en Moodle de la ETSIT), una opción para cerrar la sesión y un menú que incluirá las mismas opciones que el menú público más otra que permita acceder a /conf con el texto ``configuración''.

Tanto el color de fondo como el del texto normal de las páginas deberán recibirse en el navegador como parte de un documento CSS.

Detalles de las distintas informaciones mencionadas:

\begin{itemize}
\item Información pública de artículo. Se mostrará:
  \begin{itemize}
  \item Del artículo: su título (que será un enlace al artículo en su blog original) y su contenido (tal y como venga especificado en el canal).
  \item Del blog original que lo contiene: el nombre del blog, un enlace al blog, y otro a su canal RSS.
  \item Del usuario del Moodle de la ETSIT correspondiente: el nombre, que será un enlace a ``/users/alias'' (el alias en MiPlaneta) y la foto.
  \end{itemize}

\item Información resumida de usuario. Se mostrará:
  \begin{itemize}
  \item Del usuario del Moodle de la ETSIT correspondiente: el nombre, la foto, el enlace a su sitio web. El nombre será un enlace a ``/users/alias'' (el alias en MiPlaneta).
  \item Del blog original que lo contiene: el nombre del blog, que será un enlace a ese mismo blog.
  \end{itemize}

\item Información completa de usuario. Se mostrará:
  \begin{itemize}
  \item Del usuario del Moodle de la ETSIT correspondiente: el nombre, la foto, el enlace a su sitio web, y un enlace a su perfil en Moodle de la ETSIT.
  \item Del blog original que lo contiene: el nombre del blog, un enlace al blog, y otro a su canal RSS, todos los artículos almacenados para ese blog.
  \item De cada artículo: su título (que será un enlace al artículo en su blog original) y su contenido (tal y como venga especificado en el canal).
  \end{itemize}
\end{itemize}


Además de estos recursos, se atenderá a cualquier otro que sea necesario para proporcionar la funcionalidad indicada.


%%----------------------------------------------------------------------------
\subsubsection{Funcionalidad optativa}

De forma optativa, se podrá incluir cualquier funcionalidad relevante en el contexto de la asignatura. Se valorarán especialmente las funcionalidades que impliquen el uso de técnicas nuevas, o de aspectos de Django no utilizados en los ejercicios previos, y que tengan sentido en el contexto de esta práctica y de la asignatura.

Sólo a modo de sugerencia, se incluyen algunas posibles funcionalidades optativas:

\begin{itemize}
\item Atención al idioma indicado por el navegador. El idioma de la interfaz de usuario del planeta tendrá en cuenta lo que especifique el navegador.

\item Generación de un canal RSS para los contenidos que se muestran en la página principal.

\item Uso de AJAX para algún aspecto de la práctica (por ejemplo, en los formularios de /conf)

\item Puntuación de artículos. Cada usuario registrado puede puntuar cualquier artículo del sitio, por ejemplo entre 1 y 5. Estas puntuaciones se podrán ver luego junto al artículo en cuestión.

\item Comentarios a artículos. Cada usuario registrado puede comentar cualquier artículo del sitio. Estos comentarios se podrán ver luego junto al artículo en cuestión (en la página de ese artículo).

\item Soporte para logos. Cada blog o artículo de un blog se presentará junto con un logo que represente al blog en cuestión.

\item Autodescubrimiento de canales. Dada una URL (de un blog, por ejemplo), busca si en ella hay algún enlace que parece un canal. Si es así, ofrécelo al usuario para que lo pueda elegir. Esto se puede usar, por ejemplo, en la página de configuración de usuario, como una opción más para elegir canales (``especifica un blog para buscar sus canales'').
\end{itemize}


%%----------------------------------------------------------------------------
%%----------------------------------------------------------------------------
\subsection{Práctica final (2011, diciembre)}
\label{practica-final-2011-12}

%[Este enunciado es aún tentativo, incompleto, y está sujeto a cambios]

La práctica final de la asignatura consiste en la creación de un sitio web de creación de revistas con resúmenes de información procedente de sitios terceros, MetaMagazine. A continuación se describe el funcionamiento y arquitectura general de la aplicación, la funcionalidad mínima que debe proporcionar, y otra funcionalidad optativa que podrá tener.

%%----------------------------------------------------------------------------
\subsubsection{Arquitectura y funcionamiento general}

Arquitectura general:

\begin{itemize}
\item La práctica consistirá en una aplicación web que servirá los datos a los usuarios.

\item La aplicación web se construirá como un proyecto Django, que incluirá una o varias aplicaciones Django que implementen la funcionalidad requerida.

\item Para el almacenamiento de datos persistente se usará SQLite3, con tablas definidas según modelos en Django.

\item Se usará la aplicación Django ``Admin site'' para mantener los usuarios con cuenta en el sistema, y para la gestión general de las bases de datos necesarias.

\item Se utilizarán plantillas Django (a ser posible, una jerarquía de plantillas, para que toda la aplicación tenga un aspecto similar) para definir las páginas que se servirán a los navegadores de los usuarios. Estas plantillas incluirán en todas las páginas al menos:
  \begin{itemize}
  \item Un banner (imagen) del sitio, en la parte superior.
  \item Un menú de opciones también en la parte superior.
  \item Un pie de página con una nota de copyright.
  \end{itemize}

\item Se utilizarán hojas de estilo CSS para determinar la apariencia de la aplicación.
\end{itemize}

Funcionamiento general:

\begin{itemize}
\item El sitio MetaMagazine ofrece como servicio la construcción de revistas con resúmenes de información obtenidos a partir de canales RSS de ciertos sitios terceros. Para construir una revista, primero se extraerán noticias de los canales correspondientes. Para cada noticia, se buscarán las URLs incluidas en su texto. Para cada URL, se visitará la página correspondiente, y se extraerá de ella la información (texto, imágenes, etc.) deseada. Con esta información se compondrá una página HTML que será la que se sirva a los navegadores que visiten la revista de ese usuario.

\item Cada usuario autenticado podrá construir una revista indicando en qué información de sitios terceros están interesados (eligiendo los canales RSS correspondientes), e indicando cuántas noticias de cada uno se tomarán como máximo cuando se actualice la revista. Cuando un usuario autenticado indica un nuevo canal en el que está interesado, el sistema genera una revista para ese sitio a partir de su canal (usando el número de noticias que ha seleccionado), y se lo muestra al usuario. Si el usuario lo acepta, se toma nota del sitio y de los contenidos de la revista en la base de datos.

\item Cuando cualquier visitante de MetaMagazine acceda a la revista creada por un usuario, podrá ver la información almacenada para esa revista. Además, la página de la revista incluirá un mecanismo para actualizarla, bajando información de los sitios correspondientes. En la actualización, para cada canal sólo se considerará el número de noticias más actuales que haya seleccionado el creador de la revista (y se ignorarán las más antiguas, salvo que ya estén en la base de datos). No se eliminarán las noticias antiguas de la base de datos al actualizar las revistas.

\item Cuando esté visitando MetaMagazine un visitante sin autenticar, le aparecerá una caja para autenticarse. Si es un usuario autenticado, le aparecerá un mecanismo para salir de la cuenta (``desautenticarse'').
\end{itemize}

%%----------------------------------------------------------------------------
\subsubsection{Funcionalidad mínima}

Esta es la funcionalidad mínima que habrá de proporcionar la aplicación:

\begin{itemize}
\item Para cada revista (correspondiente a un usuario registrado del sitio) se mostrará a cualquier visitante:
  \begin{itemize}
  \item El título de la revista.
  \item Un enlace a los canales y sitios web correspondientes a esos canales, y la fecha de última actualización (para cada uno de ellos).
  \item Para cada canal, un mecanismo para actualizar en la base de datos la información extraída las páginas web que referencie.
  \item El texto de las noticias de los sitios elegidos para esa revista.
  \item Para cada noticia, un mecanismo para desplegar la información extraída las páginas web que referencie.
  \item Un mecanismo para desplegar (de una vez) la información extraída de todas las noticias.
  \end{itemize}

\item Para cada noticia, la información que se mostrará será:
  \begin{itemize}
  \item Enlace a la página de la noticia en MetaMagazine.
  \item Los enlaces a las páginas web cuya URL aparezca en la noticia.
  \item Para cada una de esas páginas, las primeras 50 palabras que incluyan (basta con considerar, por ejemplo, las primeras 50 palabras incluidas en elementos $<p>$).
  \item Para cada una de esas páginas, 5 de las imágenes que incluyan, si las hubiera.
  \item Para cada una de esas páginas, los vídeos de Youtube, si los hubiera.
  \end{itemize}

\item Para cada revista (correspondiente a un usuario registrado del sitio) se mostrará al usuario que la construye:

  \begin{itemize}
  \item Toda la información anterior, que se muestra también para cualquier visitante.
  \item El título de la revista de forma que se pueda cambiar.
  \item Una zona para incluir nuevos canales en la revista, que incluirá:
    \begin{itemize}
    \item Un menú con la opción de sitios de los que se podrán incluir canales.
    \item Un formulario para indicar qué canal del sitio elegido se incluirá.
    \end{itemize}
  \item Para cada canal de la revista, un mecanismo para eliminarlo.
  \end{itemize}

\item Como mínimo, se podrán seleccionar los siguientes tipos de canales:
  \begin{itemize}
  \item Canales RSS correspondientes a usuarios de Twitter\footnote{Para el usuario ``jgbarah'': \\
    \url{https://twitter.com/statuses/user_timeline/jgbarah.rss}}.
  \item Canales RSS correspondientes a usuarios de Identi.ca\footnote{Para el usuario ``jgbarah'': \\
    \url{http://identi.ca/jgbarah/rss}}.
  \item Canales RSS correspondientes a usuarios de Youtube\footnote{Para el usuario ``user'': \\
\url{http://gdata.youtube.com/feeds/api/videos?max-results=5&alt=rss&author=user}}.
  \end{itemize}
\end{itemize}

%%----------------------------------------------------------------------------
\subsubsection{Esquema de recursos servidos (funcionalidad mínima)}

Recursos a servir como páginas HTML completas (pensadas para ser vistas en el navegador):

\begin{itemize}
\item /: Página principal de MetaMagazine, con texto de bienvenida y contenidos de una de las revistas (aleatoriamente, se elegirá una cada vez que se reciba una nueva visita, y se incluirán sus contenidos, que deberán ser iguales a los que se verían en la página de esa revista).
\item /channels: Lista de canales activos, con enlace a los RSS correspondientes
\item /magazines: Lista de revistas disponibles, con enlace a cada una de ellas.
\item /magazines/user: Revista del usuario ``user''
\item /news/id: Página de la noticia ``id'' en MetaMagazine: título de la noticia y elementos a mostrar (enlaces de la noticia, primeras palabras de los sitios web en esos enlaces, imágenes en esos enlaces, etc.)
\end{itemize}

Recursos a servir con texto HTML listo para empotrar en otras páginas (esto es, texto que pueda ir dentro de un elemento $<body>$):

\begin{itemize}
\item /api/news/id: Para la noticia ``id'', elementos a mostrar (enlaces de la noticia, primeras palabras de los sitios web en esos enlaces, imágenes en esos enlaces, etc.)
\end{itemize}

Además de estos recursos, se atenderá a cualquier otro que sea necesario para proporcionar la funcionalidad indicada.


%%----------------------------------------------------------------------------
\subsubsection{Funcionalidad optativa}

De forma optativa, se podrá incluir cualquier funcionalidad relevante en el contexto de la asignatura. Se valorarán especialmente las funcionalidades que impliquen el uso de técnicas nuevas, o de aspectos de Django no utilizados en los ejercicios previos, y que tengan sentido en el contexto de esta práctica y de la asignatura.

Sólo a modo de sugerencia, se incluyen algunas posibles funcionalidades optativas:

\begin{itemize}

\item Recurso /conf: Configuración del usuario, para usuarios registrados. Incluirá campos para editar su nombre público, su contraseña (dos veces, para comprobar).
\item Recurso /conf/skin: Configuración del estilo (skin), para usuarios registrados. Mediante un formulario, el usuario podrá editar el fichero CSS que codificará su estilo, o podrá copiar el de otro usuario. Cada usuario tendrá un estilo (fichero CSS) por defecto, que el sistema le asignará si no lo ha configurado.
\item Recurso /rss/user: Canal RSS para la revista del usuario ``user'', con las 20 últimas entradas (del canal que sea.
\item Uso de AJAX para otros aspectos de la aplicación. Por ejemplo, para indicar qué canales se quieren.
\item Puntuación de revistas. Cada usuario registrado puede puntuar cualquier revista del sitio, por ejemplo entre 1 y 5. Estas puntuaciones se podrán ver luego junto a la revista en cuestión.
\item Puntuación de noticias. Cada usuario registrado puede puntuar cualquier noticia del sitio, por ejemplo entre 1 y 5. Estas puntuaciones se podrán ver luego junto a la noticia en cuestión.
\item Comentarios a noticias. Cada usuario registrado puede comentar cualquier noticia del sitio. Estos comentarios se podrán ver luego junto a la noticia en cuestión.
\item Soporte para avatares. Cada canal se presentará junto con el avatar (el logo que ha elegido el usuario en el sitio original, como por ejemplo Twitter) del canal.
\item Mejoras en la identificación de la información de las páginas web enlazadas. Por ejemplo, seleccionar las imágenes descartando las que probablemente son pequeños iconos (analizando el tamaño de la imagen), o identificando otros elementos relevantes.
\end{itemize}

%%----------------------------------------------------------------------------
\subsubsection{Notas y comentarios}

La práctica deberá funcionar en el entorno GNU/Linux (Ubuntu) del laboratorio de la asignatura, con la versión de Django instalada en /usr/local/django (Django 1.3.1).

La práctica deberá funcionar desde el navegador Firefox disponible en el laboratorio de la asignatura.

Se recomienda construir una o varias aplicaciones complementarias para probar la descarga y almacenamiento en base de datos de los canales que alimentarán el planeta.

%%----------------------------------------------------------------------------
%%----------------------------------------------------------------------------
\subsection{Práctica final (2012, mayo)}
\label{practica-final-2012-05}

%[Este enunciado es aún tentativo, incompleto, y está sujeto a cambios]

La práctica final a entregar en la convocatoria extraordinaria (mayo de 2012) será como la de la entrega ordinaria (práctica~\ref{practica-final-2011-12}), con las diferencias que se indican en los siguientes apartados.

%%----------------------------------------------------------------------------
\subsubsection{Arquitectura y funcionamiento general}

Con respecto a las de la práctica de la convocatoria ordinaria, el enunciado tiene los siguientes cambios:

\begin{itemize}
\item En lugar de canales RSS se utilizarán canales Atom para descargar las noticias de los sitios terceros.
\item Para construir una revista, en lugar de indicar qué información se quiere de cada sitio tercero, se indicarán cadenas de texto. Estas cadenas se utilizarán como hashtags en los sitios terceros que los soporten, o como cadenas de búsqueda en los que no. Por lo tanto, el usuario especificará una cadena, que se usará para definir qué canales Atom de los sitios terceros habrá que considerar (ver funcionalidad mínima más adelante).
\item Al definir su revista, un usuario podrá por tanto especificar cadenas, igual que antes especificaba canales de un sitio tercero. Ahora, cada cadena indicará qué canales de todos los sitios terceros hay que considerar para esa revista.
\end{itemize}

El resto queda igual.

%%----------------------------------------------------------------------------
\subsubsection{Funcionalidad mínima}

Con respecto a la de la práctica de la convocatoria ordinaria, el enunciado tiene los siguientes cambios:

\begin{itemize}
\item Para cada cadena que un usuario especifique en su revista se bajará información de, como mínimo, los siguientes canales (los ejemplos serían para la cadena ``urjc''):
  \begin{itemize}
  \item Canal Atom correspondiente al hashtag de Twitter definido por esa cadena\footnote{Para el hashtag ``\#urjc'': \\
    \url{http://search.twitter.com/search.atom?q=\%23urjc}}.
  \item Canal Atom correspondientes al hashtag de Identi.ca definido por esa cadena\footnote{Para el hashtag ``\#urjc'': \\
    \url{http://identi.ca/api/statusnet/tags/timeline/urjc.atom}}.
  \item Canal Atom correspondientes a la búsqueda en Youtube de esa cadena\footnote{Para la búsqueda ``urjc'': \\
\url{http://gdata.youtube.com/feeds/api/videos?q=urjc&max-results=5&alt=atom}}.
  \end{itemize}

\end{itemize}

El resto queda igual.



%%----------------------------------------------------------------------------
%%----------------------------------------------------------------------------
\subsection{Práctica final (2010, enero)}

La práctica final de la asignatura consiste en la creación de un planeta, o agregador de canales, como aplicación web. A continuación se describe el funcionamiento y arquitectura general de la aplicación, la funcionalidad mínima que debe proporcionar, y otra funcionalidad optativa que podrá tener.

%%----------------------------------------------------------------------------
\subsubsection{Arquitectura y funcionamiento general}

Arquitectura general:

\begin{itemize}
\item La práctica consistirá en una aplicación web que servirá los datos a los usuarios.

\item La aplicación web se construirá como un proyecto Django, que incluirá una o varias aplicaciones Django que implementen la funcionalidad requerida.

\item Para el almacenamiento de datos persistente se usará SQLite3, con tablas definidas según modelos en Django.

\item Se usará la aplicación Django ``Admin site'' para mantener los usuarios con cuenta en el sistema, y para la gestión general de las bases de datos necesarias.

\item Se utilizarán plantillas Django (a ser posible, una jerarquía de plantillas, para que toda la aplicación tenga un aspecto similar) para definir las páginas que se servirán a los navegadores de los usuarios. Estas plantillas incluirán en todas las páginas al menos:
  \begin{itemize}
  \item Un banner (imagen) del sitio, en la parte superior.
  \item Un menú de opciones también en la parte superior.
  \item Un pie de página con una nota de copyright.
  \end{itemize}

\item Se utilizarán hojas de estilo CSS para determinar la apariencia de la aplicación.
\end{itemize}

Funcionamiento general:

\begin{itemize}
\item Los usuarios indicarán en qué canales (blogs) están interesados. Para ello, cada usuario podrá especificar un número en principio ilimitado de URLs, cada una correspondiente a un canal que le interesa.
\item Cuando un usuario indica que le interesa un blog, se baja el canal correspondiente y se almacenan en la base de datos los artículos referenciados en él.
\item Cuando un usuario acceda a la URL de actualización de sus blogs, se bajan los canales correspondientes a todos ellos, y se almacenan en la base de datos los artículos correspondientes. Si un artículo ya estaba en la base de datos, no debe almacenarse dos veces.
\item Cualquier navegador podrá acceder a la interfaz pública del sitio, que ofrecerá los artículos en la base de datos e información pública para cada usuario.
\item Sólo los navegadores con un usuario autenticado podrán personalizar en qué blogs están interesados.
\end{itemize}

%%----------------------------------------------------------------------------
\subsubsection{Funcionalidad mínima}

\begin{itemize}
\item Para cada artículo en la base de datos del planeta, se mostrarán (salvo que se indique lo contrario) su título (que será un enlace al artículo en su blog original), un enlace al blog original que lo incluye, y su contenido (tal y como venga especificado en el canal).
\item El planeta mostrará en una interfaz pública (visible por cualquiera que no tenga cuenta en el sitio) todos los artículos que tenga en la base de datos, organizados en los siguientes recursos:

  \begin{itemize}
  \item /: Lista de los últimos 20 artículos, por fecha de publicación, en orden inverso (más nuevos primero).
  \item /blog: Lista de los últimos 20 artículos del blog ``blog'', por fecha de publicación, en orden inverso (más nuevos primero).
  \item /blog/num: Artículo número ``num'' del blog ``blog'', siendo ``0'' el artículo más antiguo de ese blog que se tiene en la base de datos.
  \end{itemize}

\item Además, el planeta mostrará en una interfaz privada (visible sólo para un usuario concreto cuando se autentica como tal) los artículos que éste haya seleccionado, organizados en los siguientes recursos:

  \begin{itemize}
  \item /custom: Lista de los últimos 20 artículos, por fecha de publicación, en orden inverso (más nuevos primero), de los blogs seleccionados por el usuario.
  \end{itemize}

\item Además, habrá ciertos recursos donde los usuarios registrados podrán (una vez autenticados) configurar ciertos aspectos del sitio:

  \begin{itemize}
  \item /conf: Configuración del usuario. Incluirá campos para editar su nombre público, su contraseña (dos veces, para comprobar), y los blogs en los que está interesado. Estos blogs se podrán elegir bien de un menú desplegable (en el que estarán los que ya se están bajando) o indicando sus datos (la URL de su canal correspondiente).
  \item /conf/skin: Configuración del estilo (skin) con el que el usuario quiere ver el sitio. Mediante un formulario, el usuario podrá editar el fichero CSS que codificará su estilo, o podrá copiar el de otro usuario. Cada usuario tendrá un estilo (fichero CSS) por defecto, que el sistema le asignará si no lo ha configurado.
  \item /update: Actualizará los artículos de los blogs en los que está interesado el usuario.
  \end{itemize}

\item Para cada usuario, se mantendrán ciertos recursos públicos con información relacionada con ellos:

  \begin{itemize}
  \item /user: Nombre de usuario y lista de blogs que interesan al usuario ``user''.
  \item /user/feed: Canal RSS  con los 20 últimos artículos que interesan al usuario ``user''.
  \end{itemize}

\item El idioma de la interfaz de usuario del planeta tendrá en cuenta lo que especifique el navegador, y podrá ser especificado también en la URL /conf para los usuarios registrados (entre opciones para indicar un idioma particular, o ``por defecto'', que respetará lo que indique el navegador).

\end{itemize}

%%----------------------------------------------------------------------------
\subsubsection{Funcionalidad optativa}

De forma optativa, se podrá incluir cualquier funcionalidad relevante en el contexto de la asignatura. Se valorarán especialmente las funcionalidades que impliquen el uso de técnicas nuevas, o de aspectos de Django no utilizados en los ejercicios previos, y que tengan sentido en el contexto de esta práctica y de la asignatura.

Sólo a modo de sugerencia, se incluyen algunas posibles funcionalidades optativas:

\begin{itemize}
\item Uso de AJAX para algún aspecto de la práctica (por ejemplo, para elegir un nuevo blog, o para subir comentarios)
\item Puntuación de artículos. Cada usuario registrado puede puntuar cualquier artículo del sitio, por ejemplo entre 1 y 5. Estas puntuaciones se podrán ver luego junto al artículo en cuestión.
\item Comentarios a artículos. Cada usuario registrado puede comentar cualquier artículo del sitio. Estos comentarios se podrán ver luego junto al artículo en cuestión (en la página de ese artículo).
\item Soporte para logos. Cada blog o artículo de un blog se presentará junto con un logo que represente al blog en cuestión.
\item Autodescubrimiento de canales. Dada una URL (de un blog, por ejemplo), busca si en ella hay algún enlace que parece un canal. Si es así, ofrécelo al usuario para que lo pueda elegir. Esto se puede usar, por ejemplo, en la página de configuración de usuario, como una opción más para elegir canales (``especifica un blog para buscar sus canales'').
\end{itemize}

%%----------------------------------------------------------------------------
\subsubsection{Entrega de la práctica}

La práctica se entregará el día del examen de la asignatura, o un día posterior si así se acordase. La entrega se realizará presencialmente, en el laboratorio donde tienen lugar las clases de la asignatura habitualmente.

Cada alumno entregará su práctica en un fichero tar.gz, que tendrá preparado antes del comienzo del examen, y cuya localización mostrará al profesor durante el transcurso del mismo. El fichero se habrá de llamar practica-user.tar.gz, siendo ``user'' el nombre de la cuenta del alumno en el laboratorio.

El fichero que se entregue deberá constar de un proyecto Django completo y listo para funcionar en el entorno del laboratorio, incluyendo la base de datos con datos suficientes como para poder probarlo. Estos datos incluirán al menos tres usuarios, y cinco blogs con sus noticias correspondientes. Se incluirá también un fichero README con los siguientes datos:

\begin{itemize}
\item Nombre de la asignatura.
\item Nombre completo del alumno.
\item Nombre de su cuenta en el laboratorio.
\item Nombres y contraseñas de los usuarios creados para la práctica.
\item Resumen de las peculiaridades que se quieran mencionar sobre lo implementado en la parte obligatoria.
\item Lista de funcionalidades opcionales que se hayan implementado, y breve descripción de cada una.
\end{itemize}

%%----------------------------------------------------------------------------
\subsubsection{Notas y comentarios}

La práctica deberá funcionar en el entorno GNU/Linux (Ubuntu) del laboratorio de la asignatura, con la versión de Django instalada en /usr/local/django (Django 1.1.1).

La práctica deberá funcionar desde el navegador Firefox disponible en el laboratorio de la asignatura.

Se recomienda construir una o varias aplicaciones complementarias para probar la descarga y almacenamiento en base de datos de los canales que alimentarán el planeta.

Los canales (feeds) RSS que produce la aplicación web realizada en la práctica deberán funcionar al menos con el agregador Liferea.

%%----------------------------------------------------------------------------
%%----------------------------------------------------------------------------
\subsection{Práctica final (2010, junio)}

La práctica final para entrega en la convocatoria extraordinaria de junio será similar a la especificada para la convocatoria ordinaria de enero. En particular, deberá cumplir las siguientes condiciones:

\begin{itemize}
\item La arquitectura general será la misma, salvo:
  \begin{itemize}
  \item En lugar de incluir en las plantillas Django un menú de opciones en la parte superior de las páginas, ese menú estará en una columna en la parte derecha de cada página.
  \end{itemize}
\item El funcionamiento general será el mismo, salvo:
  \begin{itemize}
  \item Cuando un usuario indica que le interesa un blog, no se almacenan en la base de datos los artículos de ese blog.
  \item No habrá URL de actualización de los blogs de un usuario.
  \item Los artículos correspondientes a un blog se actualizarán sólo cuando se visualice una página del planeta que incluya artículos de ese blog. En ese momento, los artículos nuevos (los que no estaban ya en la base de datos) se bajarán a dicha base de datos.
  \end{itemize}
\item La funcionalidad mínima será la misma, salvo:
  \begin{itemize}
  \item No se implementará el recurso ``/update'', dado que el funcionamiento de la actualización será diferente, como se ha indicado anteriormente.
  \item El recurso ``/user'' incluirá la lista de los últimos 20 artículos, por fecha de publicación, en orden inverso (más nuevos primero), de los blogs seleccionados por el usuario, además del nombre de usuario.
  \item Cada usuario registrado podrá puntuar cualquier artículo del sitio entre 1 y 5. Estas puntuaciones se podrán ver junto al artículo en cuestión, en todos los sitios donde aparece un enlace a él en el planeta.
  \end{itemize}
\item La funcionalidad optativa será la misma, salvo la puntuación de artículos, que ya ha sido mencionada como funcionalidad mínima.
\end{itemize}

El resto de condiciones serán iguales que en la convocatoria de enero de 2010.

%%----------------------------------------------------------------------------
%%----------------------------------------------------------------------------
\subsection{Práctica final (2010, diciembre)}

La entrega de esta práctica será necesaria para poder optar a aprobar la asignatura.
 Este enunciado corresponde con la convocatoria de diciembre.

La práctica final de la asignatura consiste en la creación de una aplicación web de resumen y cache de micronotas (microblogs). En este enunciado, llamaremos a esa aplicación ``MiResumen'', y a los resúmenes de micronotas para cada usuario, ``microresumen''.

Los sitios de microblogs permiten a sus usuarios compartir notas breves (habitualmente de 140 caracteres o menos). Entre los más populares pueden mencionarse Twitter\footnote{\url{http://twitter.com}} e Identi.ca\footnote{\url{http://identi.ca}}. La aplicación web a realizar se encargará de mostrar las micronotas que se indiquen, junto con información relacionada. A continuación se describe el funcionamiento y arquitectura general de la aplicación, la funcionalidad mínima que debe proporcionar, y otra funcionalidad optativa que podrá tener.

%%----------------------------------------------------------------------------
\subsubsection{Arquitectura y funcionamiento general}

Arquitectura general:

\begin{itemize}
\item La práctica consistirá en una aplicación web que servirá los datos a los usuarios.

\item La aplicación web se construirá como un proyecto Django, que incluirá una o varias aplicaciones Django que implementen la funcionalidad requerida.

\item Para el almacenamiento de datos persistente se usará SQLite3, con tablas definidas según modelos en Django.

\item Se usará la aplicación Django ``Admin site'' para mantener los usuarios con cuenta en el sistema, y para la gestión general de las bases de datos necesarias.

\item Se utilizarán plantillas Django (a ser posible, una jerarquía de plantillas, para que toda la aplicación tenga un aspecto similar) para definir las páginas que se servirán a los navegadores de los usuarios. Estas plantillas incluirán en todas las páginas al menos:
  \begin{itemize}
  \item Un banner (imagen) del sitio, en la parte superior.
  \item Un menú de opciones también en la parte superior, a la derecha del banner del sitio.
  \item Un pie de página con una nota de copyright.
  \end{itemize}

\item Se utilizarán hojas de estilo CSS para determinar la apariencia de la aplicación.
\end{itemize}

Funcionamiento general:

\begin{itemize}
\item Se considerarán sólo micronotas en Identi.ca. Llamaremos a los usuarios de Identi.ca ``micronoteros''.
\item MiResumen mantendrá usuarios, que habrán de autenticarse para poder configurar la aplicación.
\item Cada usuario de MiResumen indicará qué micronoteros de Identi.ca le interesan, configurando una lista de micronoteros.
\item Cuando un usuario indica que le interesa un micronotero, MiResumen bajará el canal RSS correspondiente, y se almacenarán en la base de datos las micronotas referenciadas en él.
\item Cuando un usuario acceda a la URL de actualización de su microresumen, se bajan los canales correspondientes a todos los micronoteros que tiene especificados, y se almacenan en la base de datos las micronotas correspondientes. Si una micronota ya estaba en la base de datos, no debe almacenarse dos veces.
\item Cualquier navegador podrá acceder a la interfaz pública del sitio, que ofrecerá los microresúmenes de cada usuario.
\end{itemize}

%%----------------------------------------------------------------------------
\subsubsection{Funcionalidad mínima}

\begin{itemize}
\item Para cada micronota en la base de datos del planeta, se mostrarán (salvo que se indique lo contrario) su texto, un enlace a la micronota en Identi.ca, el nombre del micronotero que la puso (con un enlace a su página en Identi.ca), y la fecha en que la puso, 
\item MiResumen mostrará en una interfaz pública (visible por cualquiera que no tenga cuenta en el sitio) todas las micronotas que tenga en la base de datos, organizadas en los siguientes recursos:

  \begin{itemize}
  \item /: Microresumen de las últimas 50 micronoticias, ordenadas por fecha inversa de publicación, en orden inverso (más nuevos primero).
  \item /noteros/micronotero: Microresumen de las últimas 50 micronoticias del micronotero ``micronotero'', ordenadas por fecha inversa de publicación, en orden inverso (más nuevos primero).
  \item /usuarios/usuario: Microresumen de las últimas 50 micronoticias de los micronoteros que sigue el usuario ``usuario'', ordenadas por fecha inversa de publicación.
  \item /usuarios/usuario/feed: Canal RSS  con las 50 últimas micronotas que interesan al usuario ``usuario''.
  \end{itemize}

\item Además MiResumen proporcionará ciertos recursos donde los usuarios registrados podrán (una vez autenticados) configurar ciertos aspectos del sitio:

  \begin{itemize}
  \item /conf: Configuración del usuario. Incluirá campos para editar su nombre público, su contraseña (dos veces, para comprobar), y el idioma que prefiere (al menos deberá poder elegir entre español e inglés).
  \item /conf/skin: Configuración del estilo (skin) con el que el usuario quiere ver el sitio. Mediante un formulario, el usuario podrá editar el fichero CSS que codificará su estilo, o podrá copiar el de otro usuario. Cada usuario tendrá un estilo (fichero CSS) por defecto, que el sistema le asignará si no lo ha configurado.
  \item /micronoteros: Lista de los micronoteros seleccionados por el usuario, junto con enlace a su página en Identi.ca. El usuario podrá eliminar un micronotero de la lista, o añadir uno nuevo mediante POST sobre ese recurso. Los micronoteros se podrán elegir bien de un menú desplegable (en el que estarán los que tiene seleccionados cualquier usuario de MiResumen) o indicando su nombre de micronotero en Identi.ca.
  \item /update: Actualizará las micronotas de los micronoteros en los que está interesado el usuario.
  \end{itemize}

\item El idioma de la interfaz de usuario del planeta será el especificado en la URL /conf para los usuarios registrados. Para los visitantes no registrados, será español.

\end{itemize}

Para la generación de canales RSS y para la internacionalización se podrán usar los mecanismos que proporciona Django, o no, según el alumno considere que le sea más conveniente.

%%----------------------------------------------------------------------------
\subsubsection{Funcionalidad optativa}

De forma optativa, se podrá incluir cualquier funcionalidad relevante en el contexto de la asignatura. Se valorarán especialmente las funcionalidades que impliquen el uso de técnicas nuevas, o de aspectos de Django no utilizados en los ejercicios previos, y que tengan sentido en el contexto de esta práctica y de la asignatura.

Sólo a modo de sugerencia, se incluyen algunas posibles funcionalidades optativas:

\begin{itemize}
\item Uso de Ajax para algún aspecto de la práctica (por ejemplo, para solicitar actualización de la lista de micronotas, o para suscribirse a un micronotero picando sobre una micronota suya).
\item Promoción de micronotas. Cada usuario registrado puede promocionar (indicar que le gusta) cualquier micronota del sitio. Cada micronota se verá en el sitio junto con el número de promociones que ha recibido.
\item Soporte para avatares. Cada micronota se presentará junto con el avatar (imagen) correspondiente al micronotero que la ha puesto.
\item Soporte para Twitter y/o otros sitios de microblogging (micronotas) además de Identi.ca
\item Enlace a URLs. Se identificarán en las micronotas los textos que tengan formato de URL, y se mostrará esa URL como enlace.
\item Enlace a micronoteros referenciados. Se identificarán en las micronotas los textos que tengan formato de identificador de micronotero (@nombre), y se mostrarán como enlace a la página del micronotero en cuestión.
\item Suscripción a los mismos micronoteros a los que esté suscrito otro usuario. Un usuario podrá indicar que quiere suscribirse a la misma lista de micronoteros que otro, indicando sólo su identificador de usuario.
\end{itemize}

%%----------------------------------------------------------------------------
\subsubsection{Entrega de la práctica}

La práctica se entregará electrónicamente en una de las dos fechas indicadas:

\begin{itemize}
\item El día anterior al examen de la asignatura, esto es, el 12 de diciembre, a las 18:00.
\item El 30 de diciembre a las 23:00.
\end{itemize}

Además, los alumnos que hayan presentado las prácticas podrán tener que realizar una entrega presencial en una de las dos fechas indicadas:

\begin{itemize}
\item El día del examen, esto es, el 14 de diciembre, al terminar el examen de teoría. La lista de alumnos que tengan que hacer la entrega presencial se indicará durante el examen de teoría.
\item El día 10 de enero, a las 16:00. La lista de alumnos que tengan que hacer la entrega presencial se indicará con anterioridad en el sito web de la asignatura.
\end{itemize}

La entrega presencial se realizará en el laboratorio donde tienen lugar habitualmente las clases de la asignatura.

Cada alumno entregará su práctica colocándola en un directorio en su cuenta en el laboratorio. El directorio, que deberá colgar directamente de su directorio hogar (\$HOME), se llamará ``pf\_django\_2010''.

El directorio que se entregue deberá constar de un proyecto Django completo y listo para funcionar en el entorno del laboratorio, incluyendo la base de datos con datos suficientes como para poder probarlo. Estos datos incluirán al menos tres usuarios, y cinco micronoteros con sus micronotas correspondientes. Entre los usuarios, habrá en la base de datos al menos los dos siguientes.

\begin{itemize}
\item Usuario ``pepe'', contraseña ``XXX''
\item Usuario ``pepa'', contraseña ``XXX''
\end{itemize}

Cada uno de estos usuarios estará ya siguiendo al menos dos micronoteros.

Se incluirá también en el directorio que se entregue un fichero README con los siguientes datos:

\begin{itemize}
\item Nombre de la asignatura.
\item Nombre completo del alumno.
\item Nombre de su cuenta en el laboratorio.
\item Nombres y contraseñas de los usuarios creados para la práctica.
\item Nombres de al menos cinco micronoteros cuyas noticias estén en la base de datos de la aplicación.
\item Resumen de las peculiaridades que se quieran mencionar sobre lo implementado en la parte obligatoria.
\item Lista de funcionalidades opcionales que se hayan implementado, y breve descripción de cada una.
\end{itemize}

Es importante que estas normas se cumplan estrictamente, y de forma especial lo que se refiere al nombre del directorio, porque la recogida de las prácticas, y parcialmente su prueba, se hará con herramientas automáticas.

[Las normas de entrega podrán incluir más detalles en el futuro, compruébalas antes de realizar la entrega.]


%%----------------------------------------------------------------------------
\subsubsection{Notas y comentarios}

La práctica deberá funcionar en el entorno GNU/Linux (Ubuntu) del laboratorio de la asignatura, con la versión 1.2.3 de Django.

La práctica deberá funcionar desde el navegador Firefox disponible en el laboratorio de la asignatura.

Se recomienda construir una o varias aplicaciones complementarias para probar la descarga y almacenamiento en base de datos de los canales que alimentarán MiResumen.

Los canales (feeds) RSS que produce la aplicación web realizada en la práctica deberán funcionar al menos con el agregador Liferea y el que lleva integrado Firefox.

%%----------------------------------------------------------------------------
\subsubsection{Notas de ayuda}

A continuación, algunas notas que podrían ayudar a la realización de la práctica. Gracias a los alumnos que han contribuido a ellas, bien preguntando sobre algún problema que han encontrado, o incluso aportando directamente una solución correcta.

\begin{itemize}
\item \textbf{Conversión de fechas:}

La conversión de fechas, tal y como vienen en el formato de los canales RSS de Identi.ca, al formato de fechas datetime adecuado para almacenarlas en una tabla de la base de datos se puede hacer así:

\begin{verbatim}
from email.utils import parsedate
from datetime import datetime

dbDate = datetime(*(parsedate(rssDate)[:6]))
\end{verbatim}

El uso de ``*'' permite, en este caso, obtener una referencia a la tupla de siete elementos que contiene los parámetros que espera datetime() (que son siete parámetros).

Más información sobre parsedate() en la documentación del módulo email.utils de Python.

\item \textbf{Envío de hojas CSS:}

Para que el navegador interprete adecuadamente una hoja de estilo, puede ser conveniente fijar el tipo de contenidos de la respuesta HTTP en la que la aplicación la envía al navegador. En otras palabras, asegurar que cuando el navegador reciba la hoja CSS, le venga adecuadamente marcada como de tipo ``text/css'' (y no ``text/html'' o similar, que es como vendrá marcado normalmente lo que responda la aplicación).

En código, bastaría con poner la cabecera ``Content-Type'' adecuada al objeto que tiene la respuesta HTTP que devolverá la función que atiende a la URL para servir la hoja CSS (normalmente en \texttt{views.py}):

\begin{verbatim}
myResponse = HttpResponse(cssText)
myResponse['Content-Type'] = 'text/css'
return myResponse
\end{verbatim}

\end{itemize}


%%----------------------------------------------------------------------------
%%----------------------------------------------------------------------------
\subsection{Práctica final (2011, junio)}

La entrega de esta práctica será necesaria para poder optar a aprobar la asignatura.
 Este enunciado corresponde con la convocatoria de junio.

La práctica final de la asignatura consiste en la creación de una aplicación web de resumen y cache de micronotas (microblogs). En este enunciado, llamaremos a esa aplicación ``MiResumen2'', y a los resúmenes de micronotas para cada usuario, ``microresumen''.

Los sitios de microblogs permiten a sus usuarios compartir notas breves (habitualmente de 140 caracteres o menos). Entre los más populares pueden mencionarse Twitter\footnote{\url{http://twitter.com}} e Identi.ca\footnote{\url{http://identi.ca}}. La aplicación web a realizar se encargará de mostrar las micronotas que se indiquen, junto con información relacionada. A continuación se describe el funcionamiento y arquitectura general de la aplicación, la funcionalidad mínima que debe proporcionar, y otra funcionalidad optativa que podrá tener.

%%----------------------------------------------------------------------------
\subsubsection{Arquitectura y funcionamiento general}

Arquitectura general:

\begin{itemize}
\item La práctica consistirá en una aplicación web que servirá los datos a los usuarios.

\item La aplicación web se construirá como un proyecto Django, que incluirá una o varias aplicaciones Django que implementen la funcionalidad requerida.

\item Para el almacenamiento de datos persistente se usará SQLite3, con tablas definidas según modelos en Django.

\item No se mantendrán usuarios con cuenta, ni usando la aplicación Django ``Admin site'' ni de otra manera. Por lo tanto, para usar el sitio no hará falta registrarse, ni entrar en una cuenta.

\item Se utilizarán plantillas Django (a ser posible, una jerarquía de plantillas, para que toda la aplicación tenga un aspecto similar) para definir las páginas que se servirán a los navegadores de los usuarios. Estas plantillas incluirán en todas las páginas al menos:
  \begin{itemize}
  \item Un banner (imagen) del sitio, en la parte superior.
  \item Un menú de opciones justo debajo del banner, formateado en una línea.
  \item Un pie de página con una nota de copyright.
  \end{itemize}

\item Se utilizarán hojas de estilo CSS para determinar la apariencia de la aplicación. Estas hojas se almacenará en la base de datos.
\end{itemize}

Funcionamiento general:

\begin{itemize}
\item Se considerarán sólo micronotas en Identi.ca. Llamaremos a los usuarios de Identi.ca ``micronoteros''.
\item MiResumen2 recordará a todos sus visitantes. A estos efectos, consideraremos como sesión de un visitante todas las interacciones que se hagan con el sitio desde el mismo navegador (por lo tanto, se podrán usar cookies de sesión para mantener esta relación).
\item MiResumen2 mostrará notas de Identi.ca, que se irán actualizando según se indica en el apartado siguiente.
\item Los visitantes de MiResumen2 podrán seleccionar cualquier micronota que aparezca en él.
\item Cada visitante podrá ver las micronotas que ha seleccionado, por orden inverso de publicación en Identi.ca, en un listado que incluirá también la fecha en que seleccionó cada micronota.
\end{itemize}

%%----------------------------------------------------------------------------
\subsubsection{Funcionalidad mínima}

\begin{itemize}
\item Para cada micronota en la base de datos del planeta, se mostrarán (salvo que se indique lo contrario):
  \begin{itemize}
  \item el texto de la micronota
  \item un enlace a la micronota en Identi.ca
  \item el nombre del micronotero que la puso (con un enlace a su página en Identi.ca)
  \item la fecha en que se publicó en Identi.ca
  \item un botón para que cualquier visitante pueda seleccionar esta nota (o deseleccionarla si ya la había seleccionado)
  \item si el usuario ha seleccionado la micronota, la fecha en que la había seleccionado
  \item un número que representará el número de visitantes que han seleccionado esta micronota
  \end{itemize}

\item MiResumen2 mostrará en una interfaz pública (visible por cualquiera que visite el sitio) todas las micronotas que tenga en la base de datos, organizadas en los siguientes recursos:

  \begin{itemize}
  \item /: Microresumen de las últimas 30 micronoticias almacenadas en MiResumen2, ordenadas por fecha inversa de publicación (más nuevos primero). Además, incluirá un enlace al recurso de actualización (ver más abajo), y al microresumen de las 30 siguientes micronoticias (/30, ver más abajo)
  \item /nnn: Microresumen de las micronoticas entre la nnn y la nnn+29, según el orden de fecha inversa de publicación (más nuevos primero, con números más bajos). Se considerará que la micronota más reciente es la micronota 0. Así, /0 mostrará lo mismo que / , /30 mostrará las 30 micronotas siguientes a las mostradas en / y /40 mostrará las micronotas de la 40 a la 67.
  \item /update: Recurso de actualización: cuando se acceda a él, MiResumen2 accederá al RSS de la página principal de Identi.ca y extraerá de él las últimas 20 micronotas (o menos, si no hay tantas micronotas en el canal que no estén ya en la base de datos), almacenándolas en la base de datos y mostrándolas.
  \item /selected: Listado de todas las micronotas seleccionadas por el visitante actual, ordenadas por fecha de publicación inversa (más nuevas primero).
  \item /feed: Canal RSS  con las 10 micronotas más recientes (por fecha de publicación) que ha seleccionado el visitante actual. 
  \item /conf: Configuración del visitante. Incluirá campos para editar el nombre del visitante, que se mostrará en todas las páginas del sitio que se sirvan a ese visitante.
  \item /skin: Configuración del estilo (skin) con el que el visitante quiere ver el sitio. Mediante un formulario, el visitante podrá editar el fichero CSS que codificará su estilo (y que se almacenará en la base de datos). Si no lo han cambiado, los visitantes tendrán el estilo CSS por defecto del sitio.
  \item /cookies: Página HTML que incluirá un listado de las cookies que se están usando con cada uno de los visitantes conocidos para la aplicación, en formato listo para que cada cookie pueda ser copiada y pegada en un editor de cookies.
  \end{itemize}

\end{itemize}

Para la generación de canales RSS y la gestión de sesiones y/o cookies se podrán usar los mecanismos que proporciona Django, o no, según el alumno considere que le sea más conveniente.

%%----------------------------------------------------------------------------
\subsubsection{Funcionalidad optativa}

De forma optativa, se podrá incluir cualquier funcionalidad relevante en el contexto de la asignatura. Se valorarán especialmente las funcionalidades que impliquen el uso de técnicas nuevas, o de aspectos de Django no utilizados en los ejercicios previos, y que tengan sentido en el contexto de esta práctica y de la asignatura.

Sólo a modo de sugerencia, se incluyen algunas posibles funcionalidades optativas:

\begin{itemize}
\item Uso de Ajax para algún aspecto de la práctica (por ejemplo, para seleccionar y deseleccionar una micronota).
\item Votación de micronotas. Cada visitante podrá dar una puntuación entre 0 y 10 a cada micronota. Cuando se muestre cada micronota en el sitio, además de los demás datos que se han mencionado, se incluirá la media de las votaciones que ha tenido, y el número de votaciones que ha tenido esa micronota. Una vez que un visitante ha votado una micronota, no puede volver a votarla, ni cambiar su votación.
\item Soporte para avatares. Cada micronota se presentará junto con el avatar (imagen) correspondiente al micronotero que la ha puesto.
\item Soporte para Twitter y/o otros sitios de microblogging (micronotas) además de Identi.ca
\item Enlace a URLs, etiquetas y micronoteros referenciados. Se identificarán en las micronotas los textos que tengan formato de URL, y se mostrará esa URL como enlace, los que tengan formato de etiqueta (tag, nombres que comienzan por \#), mostrándolos como enlace a la página Identi.ca para ese tag, y los micronoteros referenciados (nombres que comienzan por @), mostrándolos como enlace a la página del micronotero en cuestión en Identi.ca.
\item Recomendación de micronotas. En una página, se mostrarán las micronotas que probablemente interesen al micronotero, basada en la historia de elecciones pasadas. El algoritmo a usarse puede ser: busca los tres visitantes que más notas hayan elegido en común con las del visitante actual, y muestra todas las micronotas que hayan elegido esos visitantes y el visitante actual aún no ha elegido.
\end{itemize}

%%----------------------------------------------------------------------------
\subsubsection{Entrega de la práctica}

La práctica se entregará electrónicamente como muy tarde el día 17 de junio a las 23:00.

Además, los alumnos que hayan presentado las prácticas podrán tener que realizar una entrega presencial el día que esté fijado el examen de teoría de la asignatura. La entrega presencial se realizará en el laboratorio donde tienen lugar habitualmente las clases de la asignatura.

Cada alumno entregará su práctica colocándola en un directorio en su cuenta en el laboratorio. El directorio, que deberá colgar directamente de su directorio hogar (\$HOME), se llamará ``pf\_django\_2010\_2''.

El directorio que se entregue deberá constar de un proyecto Django completo y listo para funcionar en el entorno del laboratorio, incluyendo la base de datos con datos suficientes como para poder probarlo. Estos datos incluirán al menos cinco visitantes diferentes, cada uno con al menos 3 micronotas elegidas, y un total de al menos 50 micronotas en la base de datos de MiResumen2

Se incluirá también en el directorio que se entregue un fichero README con los siguientes datos:

\begin{itemize}
\item Nombre de la asignatura.
\item Nombre completo del alumno.
\item Nombre de su cuenta en el laboratorio.
\item Resumen de las peculiaridades que se quieran mencionar sobre lo implementado en la parte obligatoria.
\item Lista de funcionalidades opcionales que se hayan implementado, y breve descripción de cada una.
\end{itemize}

Es importante que estas normas se cumplan estrictamente, y de forma especial las que se refieren al nombre del directorio, porque la recogida de las prácticas, y parcialmente su prueba, se hará con herramientas automáticas.

[Las normas de entrega podrán incluir más detalles en el futuro, compruébalas antes de realizar la entrega.]


%%----------------------------------------------------------------------------
\subsubsection{Notas y comentarios}

La práctica deberá funcionar en el entorno GNU/Linux (Ubuntu) del laboratorio de la asignatura, con la versión 1.2.3 de Django.

La práctica deberá funcionar desde el navegador Firefox disponible en el laboratorio de la asignatura.

Se recomienda construir una o varias aplicaciones complementarias para probar la descarga y almacenamiento en base de datos del canal que alimentarán MiResumen.

Los canales (feeds) RSS que produce la aplicación web realizada en la práctica deberán funcionar al menos con el agregador Liferea y el que lleva integrado Firefox.

Se recomienda utilizar alguna extensión para Firefox que permita manipular cookies para poder probar la aplicación simulando varios visitantes desde el mismo navegador.

%%----------------------------------------------------------------------------
\subsubsection{Notas de ayuda}

A continuación, algunas notas que podrían ayudar a la realización de la práctica. Gracias a los alumnos que han contribuido a ellas, bien preguntando sobre algún problema que han encontrado, o incluso aportando directamente una solución correcta.

\begin{itemize}
\item \textbf{Conversión de fechas:}

La conversión de fechas, tal y como vienen en el formato de los canales RSS de Identi.ca, al formato de fechas datetime adecuado para almacenarlas en una tabla de la base de datos se puede hacer así:

\begin{verbatim}
from email.utils import parsedate
from datetime import datetime

dbDate = datetime(*(parsedate(rssDate)[:6]))
\end{verbatim}

El uso de ``*'' permite, en este caso, obtener una referencia a la tupla de siete elementos que contiene los parámetros que espera datetime() (que son siete parámetros).

Más información sobre parsedate() en la documentación del módulo email.utils de Python.

\item \textbf{Envío de hojas CSS:}

Para que el navegador interprete adecuadamente una hoja de estilo, puede ser conveniente fijar el tipo de contenidos de la respuesta HTTP en la que la aplicación la envía al navegador. En otras palabras, asegurar que cuando el navegador reciba la hoja CSS, le venga adecuadamente marcada como de tipo ``text/css'' (y no ``text/html'' o similar, que es como vendrá marcado normalmente lo que responda la aplicación).

En código, bastaría con poner la cabecera ``Content-Type'' adecuada al objeto que tiene la respuesta HTTP que devolverá la función que atiende a la URL para servir la hoja CSS (normalmente en \texttt{views.py}):

\begin{verbatim}
myResponse = HttpResponse(cssText)
myResponse['Content-Type'] = 'text/css'
return myResponse
\end{verbatim}

\end{itemize}

\newpage

%%--------------------------------------------------------------------------
%%--------------------------------------------------------------------------
%%--------------------------------------------------------------------------
\section{Pruebas escritas pasadas}

%%--------------------------------------------------------------------------
%%--------------------------------------------------------------------------
\subsection{Examen de ITT-SAT, 7 mayo de 2018}


Se quiere construir un sitio web, MisMuseos, donde se puedan compartir valoraciones sobre museos. La funcionalidad básica del sitio es la siguiente:

\begin{enumerate}
\item Para poder utilizar el sitio hace falta un código de acceso. Los códigos de acceso son cadenas alfanuméricas de 20 caracteres, que se consiguen en los museos. Una vez se ha introducido un código de acceso correcto desde un navegador se puede acceder desde ese navegador a toda la funcionalidad del sitio. Si no, cualquier recurso del sitio devolverá un documento HTML con un formulario para introducir un código de acceso.

\item Una vez se ha introducido un código válido (que llamaremos, a partir de ese momento, ``código activo'' en ese navegador), el sitio sólo proporcionará dos recursos que devuelvan un documento (salvo que haga falta alguno más para proporcionar la funcionalidad descrita en este enunciado):
  \begin{itemize}
  \item El recurso (página) principal, que devolverá el documento HTML que se describe más adelante.
  \item El recurso de valoraciones realizadas, que devolverá un documento XML con un listado de las valoraciones realizadas usando el código de acceso activo en el navegador, ordenadas por fecha de valoración, e incluyendo para cada una de ellas el nombre del museo y la valoración dada.
  \end{itemize}

\item Cualquier otro recurso que se pida desde el navegador causará que se envié una redirección a la página principal.

\item La página principal del sitio mostrará a los visitantes (una vez se ha introducido un código válido):
  \begin{itemize}
  \item Un formulario para elegir un nombre, o el nombre si ya se usó el formulario para elegirlo
  \item Un enlace que, si se pulsa, hará que el código de acceso quede ``desactivado'' (dejando por tanto de ser un ``código activo'' en ese navegador). Cualquier nueva acción en el sitio devolverá el formulario para introducir el código de acceso.
  \item Un listado con todos los museos que tienen MisMuseos, incluyendo para cada museo una foto, el nombre del museo, la puntuación media que le han dado los visitantes del sitio, y un formulario para valorarlo. Este formulario tendrá un botón (``Valorar'') y una caja para poner la valoración (un número entero entre 0 y 4).
  \end{itemize}

\item Además, todas las páginas HTML (incluido el formulario para introducir el código de acceso) incluirán una imagen transparente, de un píxel, que se utilizará para que MisMuseos pueda trazar el número de páginas vistas desde un mismo navegador, esté activo un código en ese navegador o no.

\item Las fotos de cada museo son servidas por el sitio web de cada museo, no por MisMuseos.

\item Al poner un valor en el formulario de valoración de un museo, y pulsar ``Valorar'', se añadirá una nueva valoración al museo en cuestión, si el código de acceso activo en ese navegador nunca había valorado ese museo, o cambiará la valoración anterior, si ya lo había valorado.

\item Un mismo código de acceso puede ser utilizado desde varios navegadores (estar activo en ellos), en periodos diferentes o simultáneos.

\item En un mismo navegador pueden estar activos varios códigos de acceso, pero  no simultáneamente. Sólo si se ``olvida'' (deja de estar activo) el que se está usando, se podrá activar otro, introduciéndolo en el formulario que se recibirá tras pulsar el enlace de ``olvidar''.

\end{enumerate}


Teniendo en cuenta los requisitos anteriores, se pide:

\begin{enumerate}
\item Diseña un esquema REST para proporcionar el servicio descrito. Se habrán de especificar los nombres de recurso empleados, y cómo reaccionará la aplicación cuando reciba los métodos POST o GET sobre esas urls (no se usarán los métodos PUT o DELETE). Coloca la información en una tabla, con las urls en una columna, los métodos en otra, y la descripción de lo que realizará la aplicación al recibirlos en la tercera. Escribe también un fichero similar al fichero urls.py de Django (aunque no es importante que se respete la sintaxis mientras se entienda y la estructura sea similar a la de Django), que refleje el esquema REST anterior (1 punto).

\item Describe el modelo de datos que necesitará esta aplicación. Define las tablas necesarias y los campos necesarios para la funcionalidad descrita. Asegúrate de que incluyes en el modelo de datos la tabla o tablas necesarias para saber el número de páginas vistas por cada navegador, gracias al uso de la imagen transparente que se describe en en enunciado. Hazlo de forma lo más similar posible a lo que tendrías que escribir en el fichero models.py en Django (aunque no es importante que se respete la sintaxis mientras se entienda el modelo de datos que propones) (1 punto).

\item Describe las interacciones HTTP que ocurrirán entre el navegador y cualquier servidor web en el siguiente escenario. El escenario comienza cuando un visitante que accede por primera vez al sitio pone en el navegador la url de la página principal del sitio. A continuación, después de ver esta página principal, rellena el formulario que recibe con un código de acceso válido. El escenario termina cuando el visitante vuelve a ver la página principal del sitio, pero ahora ya con el código activo, y por lo tanto viendo la lista de museos. (1 punto).

\item Describe las interacciones HTTP que ocurrirán entre el navegador y cualquier servidor web en el siguiente escenario. El escenario comienza con un visitante que ya tiene un código activo está viendo la página principal del sitio, con la lista de museos. El visitante rellena el formulario de valoración de un museo, que nunca había valorado antes con ese código, y pulsa el botón ``Valorar''. El escenario termina cuando el visitante vuelve a ver la página principal, con la lista de museos (1 punto).

\item Escribe cómo podría ser el documento XML para un visitante que está usando un código con el que se han realizado valoraciones para los museos ``El Campo'', ``Reina Margarita'' y ``Tisten'' (una valoración para cada uno) (1 punto).
\end{enumerate}

En todos los escenarios, ten en cuenta que tu respuesta debe considerar toda la funcionalidad que ofrece el servicio, y permitir que ésta pueda proporcionarse. Diseña la aplicación de forma que envíe cookies al navegador sólo cuando sea necesario.

En las respuestas donde describas interacciones HTTP indica para cada una de ellas claramente y en este orden:
  \begin{itemize}
  \item La primera línea de la petición HTTP
  \item Si lo hay, el contenido de la petición
  \item La primera línea de la respuesta HTTP
  \item Si lo hay, el contenido de la respuesta
  \item Una brevísima explicación de para qué se usa la interacción
  \item Tanto en la petición como en la respuesta, las cabeceras con cookies, si es que fueran necesarias para la funcionalidad del escenario que se está describiendo (incluyendo el aspecto que han de tener esas cabeceras). Si la cabecera con cookie va o no dependiendo de algún factor ajeno a tu aplicación, explica cuando irá y cuándo no, y cuál es ese factor.
  \end{itemize}

Además, asegúrate de que describes las interacciones HTTP en el orden en que ocurrirían en el escenario.

\section*{Soluciones}

Hay muchas soluciones posibles. A continuación, una de ellas.

\subsection*{Esquema REST}

Este podría ser el esquema REST una vez se ha introducido un código válido:

\begin{tabular}{|l|l|l|}
  \hline
  Recurso & Método & Comentario \\ \hline \hline
  /       & GET    & Página principal (HTML) \\
          & POST   & \verb|museo=id&val=val| \\
          &        & (valoración de museo) \\
          &        & o \\
          &        & \verb|nombre=nombre| \\
          &        & (poner nombre) \\
          &        & Devolverá el mismo HTML que si se invoca con GET \\ \hline
  /val.xml & GET   & Página XML con valoraciones para el código (XML) \\ \hline
  /pixel  & GET    & Pixel para trazar páginas vistas (GIF) \\ \hline
  /salir  & GET    & Desactivación de código \\
          &        & Devolverá el formulario para introducir código (HTML) \\
          &        & (según el enunciado, se invoca con un enlace, luego \\
          &        & ha de ser GET) \\ \hline
\end{tabular}

Antes de introducirlo, todos los recursos devolverán, ante un GET, el formulario para introducir el código, salvo ``/pixel'' (que funcionaría igual) y / que para POST admitiría un código, \verb|codigo=codigo_museo| (para GET devolvería el formulario también).

El recurso / podrá discriminar, si recibe un POST, si se está valorando un museo o si se está poniendo un nombre por el nombre de los campos en la query string recibida.

\vspace{.3cm}

Nota: Alternativamente, podría haber un recurso para valorar para cada muse, por ejemplo ``/valorar/\{museo\}'', sobre el que se haría el POST de valoración. Pero en ese caso, sería recomendable que este recurso, además de aceptar la valoración, devolviese una redirección sobre el recurso /.

\subsection*{urls.py}

Lo escribimos sólo para el esquema REST una vez se ha introducido el código:

\begin{verbatim}
from django.conf.urls import url
from . import views

urlpatterns = [
    url(r'^$', views.pagina_principal),
    url(r'^val.xml$', views.valoraciones),
    url(r'^pixel$', views.trazado),
    url(r'^salir$', views.salir)
]
\end{verbatim}

\subsection*{models.py}

Versión simplificada, que cumple el enunciado, aunque podría optimizarse:

\begin{verbatim}
from django.db import models

class Codigos(models.Model):
    codigo = models.CharField(max_length=20)
    nombre = models.CharField(max_length=100, null==True)

class Museos(models.Model):
    nombre = models.TextField()
    id = models.IntegerField()
    foto = models.CharField(max_length=100)

class Navegadores(models.Model):
    cookie = models.CharField(max_length=32)
    vistas = models.IntegerField()

class Activos(models.Model):
    codigo = models.ForeignKey('Codigos')
    navegador = models.ForeignKey('Navegadores')

class Valoraciones(models.Model):
    codigo = models.ForeignKey('Codigos')
    museo = models.ForeignKey('Museos')
    valoracion = models.IntegerField()
    fecha = models.DateTimeField()
\end{verbatim}

No se incluyen los campos identificador único para cada tabla.

La tabla Codigos tendrá todos los códigos válidos (que se han repartido a los museos). Esta tabla es fuente de ineficiencias, porque la mayoría de los nombres estarán vacíos (dado que corresponderán a códigos no usados o a los que no se les ha puesto nombre), por lo que en producción sería conveniente tener una tabla separada para los nombres. Pero tal y como está definida aquí, funcionaría.

\subsection*{Primer escenario}

Todas las interacciones son entre el navegador y el sitio MisMuseos, salvo cuando se indica otra cosa.

\begin{itemize}
\item Petición GET /

\begin{verbatim}
  GET / HTTP/1.1
  ...
\end{verbatim}

\item Respuesta

\begin{verbatim}
  HTTP/1.1 200 OK

  [Formulario para codigo, HTML]
\end{verbatim}

\item Petición GET /pixel. El navegador, al cargar el documento HTML recibido, encontrará en él la referencia al pixel para trazado, y lo pedirá mediante otra interacción HTTP:

\begin{verbatim}
  GET / HTTP/1.1
  ...
\end{verbatim}

\item Respuesta

\begin{verbatim}
  HTTP/1.1 200 OK
  ...
  Set-Cookie: ....; navegador=12345ABCDE12345ABCDE12345ABCDE12

  [Imagen para trazado, GIF]
\end{verbatim}

\verb|navegador| es un identificador de navegador, que servirá para trazar las páginas vistas (suponemos que este se envía con cabeceras que lo hagan no-cacheable, de forma que pueda realizar su misión). También se utilizará, cuando se haya enviado un código válido, para saber que este navegador se ha autenticado (anotándolo en la tabla Activos).

\item Petición POST / (para proporcionar el código que se ha introducido en el formulario):

\begin{verbatim}
  POST / HTTP/1.1
  ...
  Cookie: navegador=12345ABCDE12345ABCDE12345ABCDE12

  codigo=ABCDE12345ABCDE12345
\end{verbatim}

\item Respuesta

\begin{verbatim}
  HTTP/1.1 200 OK
  ...

  [Página principal con la lista de museos, HTML]
\end{verbatim}

Al recibir esta petición y comprobar que el código es correcto (utilizando la tabla Codigos), MisMuseos apuntará este navegador con este código en la tabla Activos, donde seguirá apuntado hasta que el usuario decida desactivar este código en su navegador.


\item Petición GET /pixel (igual que la anterior):

\begin{verbatim}
  GET / HTTP/1.1
  ...
  Cookie: navegador=12345ABCDE12345ABCDE12345ABCDE12
\end{verbatim}

\item Respuesta

\begin{verbatim}
  HTTP/1.1 200 OK

  [Imagen para trazado, GIF]
\end{verbatim}

En esta ocasión, ya no se recibe una cookie, sino que se envía (ya se recibió, y MisMuseos, al detectar que ya viene con la petición, no la vuelve a enviar). Al recibir esta petición, MisMuseos apuntará una nueva página vista para este navegador.

\item Petición GET de la foto del museo museo (una por cada museo). Cada una de estas interacciones son {\bf con el sitio web de los museos en cuestión}.

\begin{verbatim}
  GET /url_foto HTTP/1.1
  ...
\end{verbatim}

\item Respuesta

\begin{verbatim}
  HTTP/1.1 200 OK

  [Foto, GIF]
\end{verbatim}

\item ...
\end{itemize}


\subsection*{Segundo escenario}

A continuación, las interacciones son entre el navegador y el sitio MisMuseos. Suponemos, como ya se ha indicado, que la imagen se sirve como no-cacheable..

\begin{itemize}
\item Petición POST / (para realizar una valoración):

\begin{verbatim}
  POST / HTTP/1.1
  ...
  Cookie: navegador=12345ABCDE12345ABCDE12345ABCDE12

  museo=5&val=3
\end{verbatim}

\item Respuesta

\begin{verbatim}
  HTTP/1.1 200 OK
  ...

  [Página principal con la lista de museos, HTML]
\end{verbatim}

Al recibir esta petición, MisMuseos comprobará que el código está activo, buscando el identificador de navegador en la tabla Activos, y consiguiendo a partir de la entrada correspondiente el código. A continuación, utilizará el código para añadir una entrada a la tabla Valoraciones con el identificador del museo, el código, y la valoración.

\item Petición GET /pixel (igual que las anteriores):

\begin{verbatim}
  GET / HTTP/1.1
  ...
  Cookie: navegador=12345ABCDE12345ABCDE12345ABCDE12
\end{verbatim}

\item Respuesta

\begin{verbatim}
  HTTP/1.1 200 OK

  [Imagen para trazado, GIF]
\end{verbatim}

\end{itemize}

En este caso no se han incluido las peticiones de las fotos de los museos, porque se hace la suposición razonable de que serán imágenes cacheables. En cualquier caso, si no se hace esta suposición, se pueden incluir, de la misma forma que se incluyeron anteriormente

\subsection*{Canal XML}


\begin{verbatim}
<?xml version="1.0" encoding="UTF-8" ?>

<valoraciones>
  <codigo>ABCDE12345ABCDE12345</codigo>
  <lista_valoraciones>
    <valoracion>
      <museo>El Campo</museo>
      <val>3</val>
      <fecha>2018-05-03 12:20:21</fecha>
    </valoracion>
    <valoracion>
      <museo>Reina Margarita</museo>
      <val>4</val>
      <fecha>2018-05-02 11:10:31</fecha>
    </valoracion>
    <valoracion>
      <museo>Tisten</museo>
      <val>1</val>
      <fecha>2018-05-01 13:23:11</fecha>
    </valoracion>
  </lista_valoraciones>
</valoraciones>
\end{verbatim}

En general, hay que cuidar que la valoración de cada museo sea claramente identificable que corresponde a ese museo, y las convenciones sintácticas de XML.


%%--------------------------------------------------------------------------
%%--------------------------------------------------------------------------
\subsection{Examen de ITT-SARO, 17 mayo de 2018}


Se quiere construir un sitio web, MisMuseos, donde se puedan compartir comentarios sobre museos. La funcionalidad básica del sitio es la siguiente:

\begin{enumerate}
\item El sitio está públicamente accesible para cualquiera que lo quiere consultar, sin necesidad de abrir cuenta ni ningún otro trámite.

\item Además, cualquiera podrá dar un ``me gusta'' a los museos que quiera, en la página del museo (ver más abajo), pero no más de una vez desde el mismo navegador (ver página de museo, más abajo).

\item Para poder poner un comentario sobre un museo, hace falta un código de acceso, disponible en ese museo. Los códigos de acceso son cadenas alfanuméricas de un solo uso, en el sentido de que quien tiene un código puede poner un comentario sobre el museo correspondiente y editarlo cuantas veces quiera desde cualquier navegador, usando ese código. Pero una vez usado para poner un comentario, ese código sólo permitirá cambiar el comentario.

\item La funcionalidad ``en modio consulta'' del sitio es la siguiente:
  \begin{itemize}
  \item El recurso (página) principal del sitio tendrá un listado de todos los museos que se pueden consultar. Para cada museo se mostrará el nombre del museo (que será un enlace a la página del museo, ver más abajo), el último comentario para ese museo, y un icono (en formato PNG) con el número de ``me gusta'' que ha recibido ese museo.
  \item La página de cada museo tendrá el nombre y dirección del museo, un formulario con un botón (sin icono) para indicar ``me gusta'' si no se ha pulsado ya ese botón desde ese mismo navegador, y un formulario para escribir un código de museo, si no se ha utilizado ya desde ese navegador para ese mismo museo (en ese caso, el museo estará ``en modo comentario'' (ver más abajo). Además, tendrá el listado de todos los comentarios que se hayan puesto, con cualquier código válido y desde cualquier navegador, para ese museo.
  \end{itemize}

\item La funcionalidad ``en modo comentario'' del sitio es igual, salvo que en la página de los museos donde se haya introducido un código válido desde ese navegador, en lugar del formulario para introducir el código aparecerá un formulario para introducir un comentario. Si ese código ya  se ha usado (desde cualquier navegador) para introducir un comentario, ese comentario aparecerá ``precargado'' en el formulario. El resto de la página es exactamente igual que en ``modo consulta''.

\item Los iconos que se ven en las páginas del sitio están servidas por el propio sitio.

\item Todas las páginas del sitio (página principal y páginas de museos) tendrán un banner (imagen en formato PNG), que será servido por un sitio tercero que llamaremos ``Servidor del banner''.
\end{enumerate}


Teniendo en cuenta los requisitos anteriores, se pide:

\begin{enumerate}
\item Diseña un esquema REST para proporcionar el servicio descrito. Se habrán de especificar los nombres de recurso empleados, y cómo reaccionará la aplicación cuando reciba los métodos POST o GET sobre esas urls (no se usarán los métodos PUT o DELETE). Coloca la información en una tabla, con las urls en una columna, los métodos en otra, y la descripción de lo que realizará la aplicación al recibirlos en la tercera. Escribe también un fichero similar al fichero urls.py de Django (aunque no es importante que se respete la sintaxis mientras se entienda y la estructura sea similar a la de Django), que refleje el esquema REST anterior (1 punto).

\item Describe el modelo de datos que necesitará esta aplicación. Define las tablas necesarias y los campos necesarios para la funcionalidad descrita. Asegúrate de que incluyes en el modelo de datos la tabla o tablas necesarias para saber el número de páginas vistas por cada navegador, gracias al uso de la imagen transparente que se describe en en enunciado. Hazlo de forma lo más similar posible a lo que tendrías que escribir en el fichero models.py en Django (aunque no es importante que se respete la sintaxis mientras se entienda el modelo de datos que propones) (1 punto).

\item Describe las interacciones HTTP que ocurrirán entre el navegador y cualquier servidor web en el siguiente escenario. El escenario comienza cuando un visitante que accede por primera vez al sitio pone en el navegador la url de la página principal del sitio. A continuación, después de ver esta página principal, pulsa sobre el enlace de un museo, y ve la página de ese museo. El escenario termina cuando el visitante está viendo la página principal de ese museo en su navegador (1 punto).

\item Describe las interacciones HTTP que ocurrirán entre el navegador y cualquier servidor web en el siguiente escenario. El escenario comienza con un visitante que ya ha rellenado, en la página de un museo, el formulario de código con un código válido, y está viendo la página de ese museo. A continuación, rellena el formulario con un comentario (no lo había rellenado nunca antes), y lo envía a MisMuseos. Y a continuación, pone un ``me gusta'' para ese mismo museo (pulsando en el botón correspondiente). El escenario termina cuando el visitante vuelve a ver la página del museo, ya sin el botón de ``me gusta'' y con el comentario pre-relleno en el formulario de comentarios (1 punto).

\item Describe qué tendrá que hacer el ``Servidor del banner'' para poder saber, de forma independiente de MisMuseos, el número de navegadores únicos que están visitando MisMuseos, sin más colaboración por parte de MisMuseos que colocar un banner que él sirva en todas sus páginas (como ya se comentó en la descripción de funcionalidad de MisMuseos).
\end{enumerate}

En todos los escenarios, ten en cuenta que tu respuesta debe considerar toda la funcionalidad que ofrece el servicio, y permitir que ésta pueda proporcionarse. Diseña la aplicación de forma que envíe cookies al navegador sólo cuando sea necesario.

En las respuestas donde describas interacciones HTTP indica para cada una de ellas claramente y en este orden:
  \begin{itemize}
  \item La primera línea de la petición HTTP
  \item Si lo hay, el contenido de la petición
  \item La primera línea de la respuesta HTTP
  \item Si lo hay, el contenido de la respuesta
  \item Una brevísima explicación de para qué se usa la interacción
  \item Tanto en la petición como en la respuesta, las cabeceras con cookies, si es que fueran necesarias para la funcionalidad del escenario que se está describiendo (incluyendo el aspecto que han de tener esas cabeceras). Si la cabecera con cookie va o no dependiendo de algún factor ajeno a tu aplicación, explica cuando irá y cuándo no, y cuál es ese factor.
  \end{itemize}

Además, asegúrate de que describes las interacciones HTTP en el orden en que ocurrirían en el escenario.

\section*{Soluciones}

Hay muchas soluciones posibles. A continuación, una de ellas.

\subsection*{Esquema REST}

Este podría ser el esquema REST:

\begin{tabular}{|l|l|l|}
  \hline
  Recurso & Método & Comentario \\ \hline \hline
  /       & GET    & Página principal (HTML) \\ \hline
  /\{id\_museo\} & GET   & Página de un museo \\
          & POST   & \verb|codigo=codigo| \\
          &        & (introducción de código de museo) \\
          &        & \verb|megusta=True| \\
          &        & (``me gusta'' al museo) \\
          &        & \verb|comentario=texto| \\
          &        & (poner un comentario sobre el museo) \\ \hline
  /iconos/\{num\} & GET & Iconos para los números de ``me gusta'' \\ \hline
\end{tabular}

Las opciones POST para ``me gusta'' y para poner comentario a un museo devolverán 401 (no autorizado) cuando no se haya introducido un código válido para ese museo, y ``funcionarán'' de acuerdo al enunciado cuando se haya introducido.


\subsection*{urls.py}

Lo escribimos sólo para el esquema REST una vez se ha introducido el código:

\begin{verbatim}
from django.conf.urls import url
from . import views

urlpatterns = [
    url(r'^$', views.pagina_principal),
    url(r'^(\d+)$', views.museo),
    url(r'^iconos/(\d+)$', views.trazado),
]
\end{verbatim}

\subsection*{models.py}

Versión simplificada, que cumple el enunciado, aunque podría optimizarse:

\begin{verbatim}
from django.db import models

class Codigos(models.Model):
    codigo = models.CharField(max_length=20)
    museo = models.ForeignKey('Museos')
    navegador = models.ForeignKey('Navegadores', null==True)

class Museos(models.Model):
    nombre = models.TextField()
    id = models.IntegerField()
    direccion = models.TextField()

class Navegadores(models.Model):
    cookie = models.CharField(max_length=32)

class MeGusta(models.Model):
    navegador = models.ForeignKey('Navegadores')
    museo = models.ForeignKey('Museos')

class Comentarios(models.Model):
    codigo = models.ForeignKey('Codigos')
    comentario = models.TextField()
    fecha = models.DateTimeField()
\end{verbatim}

No se incluyen los campos identificador único para cada tabla.

La tabla Codigos tendrá todos los códigos válidos (que se han repartido a los museos), cada uno con su museo correspondiente. Cuando un navegador use un código, se anotará en esta tabla.

\subsection*{Primer escenario}

Todas las interacciones son entre el navegador y el sitio MisMuseos, salvo cuando se indica otra cosa.

\begin{itemize}
\item Petición GET /

\begin{verbatim}
  GET / HTTP/1.1
  ...
\end{verbatim}

\item Respuesta

\begin{verbatim}
  HTTP/1.1 200 OK

  [Página principal, HTML]
\end{verbatim}


\item Petición GET de los iconos de número de ``me gusta'' para cada uno de los museos mostrados. Habrá tantas de estas interacciones como iconos con el número de ``me gusta'' haya en los museos de la página principal. La primera de estas interacciones supone que el número de ``me gusta'' que muestra el icono es 3:

\begin{verbatim}
  GET /iconos/3 HTTP/1.1
  ...
\end{verbatim}

\item Respuesta

\begin{verbatim}
  HTTP/1.1 200 OK
  ...

  [Imagen con 3 "me gusta", PNG]
\end{verbatim}


\item Petición GET para el banner, realizada no a MisMuseos sino al ``Servidor del banner''

\begin{verbatim}
  GET /banner HTTP/1.1
  ...
\end{verbatim}

\item Respuesta

\begin{verbatim}
  HTTP/1.1 200 OK
  ...

  [Imagen de banner, PNG]
\end{verbatim}
  

\item Petición GET de la página de museo, una vez el visitante ha pulsado sobre el enlace correspondiente (suponemos que el museo en cuestión tiene el identificador ``12''):

\begin{verbatim}
  GET /12 HTTP/1.1
  ...
\end{verbatim}

\item Respuesta

\begin{verbatim}
  HTTP/1.1 200 OK

  [Página de museo, HTML]
\end{verbatim}

\end{itemize}

En este escenario no han hecho falta cookies, porque no es necesario identificar al navegador (no se introducen códigos, no se pulsa sobre ``me gusta''...)


\subsection*{Segundo escenario}

A continuación, las interacciones son entre el navegador y el sitio MisMuseos. Suponemos que el banner no es preciso volver a pedirlo, porque estará en la cache del navegador. Además, como el visitante ha introducido ya un código válido de museo, se le habrá enviado (con una cabecera ``Set--Cookie'') una cookie, para poder identificarle. Suponemos que el museo al que corresponde la página que está viendo el visitante es el ``12''.

\begin{itemize}
\item Petición POST /12 (para subir un comentario):

\begin{verbatim}
  POST / HTTP/1.1
  ...
  Cookie: navegador=12345ABCDE12345ABCDE12345ABCDE12
  
  comentario="Me ha gustado mucho este museo"
\end{verbatim}

\item Respuesta

\begin{verbatim}
  HTTP/1.1 200 OK
  ...

  [Página del museo, con el comentario ya puesto, HTML]
\end{verbatim}

\item Petición POST /12 (para indicar ``me gusta''):

\begin{verbatim}
  POST / HTTP/1.1
  ...
  Cookie: navegador=12345ABCDE12345ABCDE12345ABCDE12
  
  megusta=True
\end{verbatim}

\item Respuesta

\begin{verbatim}
  HTTP/1.1 200 OK
  ...

  [Página del museo, con el comentario y sin el boton "me gusta", HTML]
\end{verbatim}

\end{itemize}


\subsection*{Trazado de navegadores únicos desde servidor de banner}

Basta con que, cada vez que sirve por primera vez un banner de MisMuseos a un navegador, le envíe (con cabecera ``Set-Cookie'') una cookie con un identificador de navegador único. Cuando le llegue una petición que ya tenga cookie, no hará más que servir el banner. El número de navegadores únicos será el número de cookies servidas.



%%--------------------------------------------------------------------------
%%--------------------------------------------------------------------------
\subsection{Examen de ITT-SAT, 10 mayo de 2017}

Se quiere construir un sitio web, Mensajitos, donde se pueden poner mensajes para que los vean otras personas. La funcionalidad básica del sitio es la siguiente:

\begin{enumerate}
\item En el sitio no hay cuentas para usuarios: toda la funcionalidad está disponible para cualquiera que lo visite.
\item De todas formas, cualquier visitante podrá reservar un nombre que no esté ya en uso para cuando suba información al sitio. Este nombre se mantendrá mientras el visitante utilice el mismo navegador. Para ello se usará el formulario que aparece en la página principal (ver a continuación).
\item La página principal del sitio mostrará a los visitantes un botón para crear un canal de mensajes, y un formulario para elegir un nombre (si se ha elegido ya, en lugar del formulario aparecerá el nombre elegido). El botón permitirá crear un nuevo canal (cada visitante puede crear tantos como quiera), según se indica más abajo. Además, en esta página principal cada visitante verá la lista de los canales que ha creado previamente. Tras crear un nuevo canal, o elegir un nombre, el visitante volverá a ver la página principal del sitio.
\item Cada canal tendrá un nombre de recurso único, que se generará aleatoriamente cuando se cree. Cualquiera que conozca el nombre de recurso de un canal, podrá leer y escribir en él, simplemente accediendo a ese recurso (lo haya creado quien lo haya creado).
\item El recurso correspondiente a cada canal mostrará una página HTML (la ``página del canal'') con los 10 últimos mensajes en el canal, un formulario para poner un nuevo mensaje, y un formulario para poner la url de una imagen (que puede estar en cualquier sitio de Internet, mientras la haga visible mediante HTTP). Cada mensaje que se muestre, se mostrará con el formato:

\begin{verbatim}
Nombre: mensaje
\end{verbatim}

Donde ``Nombre'' es el nombre del visitante (o ``Anónimo'', si no lo ha elegido), y ``mensaje'' es el mensaje en cuestión.

Las urls de imágenes se considerarán también como mensajes, pero antes de mostrarlos como tales (y de almacenarlos en la base de datos), se convertirán a un elemento IMG de HTML. Por ejemplo, la imagen de url \verb|http://fotos.com/123345.jpg| se convertirá en el HTML siguiente (que se considerará el ``mensaje'' en el formato descrito anteriormente):

\begin{verbatim}
<img src="http://fotos.com/123345.jpg" style="width:200px;height:150px;">
\end{verbatim}

Tras poner un nuevo mensaje (o una imagen) en un canal, el visitante vuelve a ver de nuevo la página del canal.

\item Cada canal tendrá también un recurso asociado donde se podrán descargar todos sus mensajes (incluyendo aquellos que se especificaron como imágenes) en formato XML. Este recurso aparecerá también como enlace en la página del canal. El documento XML correspondiente incluirá al menos todos los mensajes que se han escrito en el canal, el nombre del visitante que puso cada uno de ellos, la fecha en que se puso cada uno de ellos, el enlace a la página HTML del canal, la fecha en que se creó el canal, y el nombre del visitante que creó el canal (si alguno de los visitantes implicados no ha especificado un nombre, se usará ``Anónimo'').
\item Todas las páginas HTML del sitio incluirán una imagen de cabecera (banner) que se alojará en el propio sitio.
\end{enumerate}

Teniendo en cuenta los requisitos anteriores, se pide:

\begin{enumerate}
\item Diseña un esquema REST para proporcionar el servicio descrito. Se habrán de especificar los nombres de recurso empleados, y cómo reaccionará la aplicación cuando reciba los métodos POST o GET sobre esas urls (no se usarán los métodos PUT o DELETE). Coloca la información en una tabla, con las urls en una columna, los métodos en otra, y la descripción de lo que realizará la aplicación al recibirlos en la tercera. Escribe también un fichero similar al fichero urls.py de Django (aunque no es importante que se respete la sintaxis mientras se entienda y la estructura sea similar a la de Django), que refleje el esquema REST anterior (1 punto).

\item Describe el modelo de datos que necesitará esta aplicación. Define las tablas necesarias y los campos necesarios para la funcionalidad descrita. Hazlo de forma lo más similar posible a lo que tendrías que escribir en el fichero models.py en Django (aunque no es importante que se respete la sintaxis mientras se entienda el modelo de datos que propones) (1 punto).

\item Describe las interacciones HTTP que ocurrirán entre el navegador y cualquier servidor web en el siguiente escenario. El escenario comienza cuando un visitante que accede por primera vez al sitio pone en el navegador la url de la página principal del sitio. A continuación, después de ver esta página principal, rellena el formulario para elegir un nombre. En este momento, el navegador vuelve a mostrar la página principal, ya con el nombre elegido en lugar del formulario para elegirlo. A continuación el visitante crea un nuevo canal. El escenario termina cuando el visitante vuelve a ver la página principal del sitio, con el nuevo canal ya creado. (1 punto).

\item Describe las interacciones HTTP que ocurrirán entre el navegador y cualquier servidor web en el siguiente escenario. El escenario comienza con un visitante que ya ha reservado nombre y está viendo la página de un canal que aún no tiene mensajes. El visitante rellena el formulario de imagen, poniendo la url de una imagen válida. El escenario termina cuando el visitante vuelve a ver la página del canal, ya con el nuevo mensaje generado a partir de la url de la imagen (1 punto).

\item Escribe cómo podría ser el documento XML para un canal que tiene tres mensajes, uno de los cuales corresponde a la url de una imagen, para un usuario que tiene nombre (1 punto).
\end{enumerate}

En todos los escenarios, ten en cuenta que tu respuesta debe considerar toda la funcionalidad que ofrece el servicio, y permitir que ésta pueda proporcionarse. Diseña la aplicación de forma que envíe cookies al navegador sólo cuando sea necesario.

En las respuestas donde describas interacciones HTTP indica para cada una de ellas claramente y en este orden:
  \begin{itemize}
  \item La primera línea de la petición HTTP
  \item Si lo hay, el contenido de la petición
  \item La primera línea de la respuesta HTTP
  \item Si lo hay, el contenido de la respuesta
  \item Una brevísima explicación de para qué se usa la interacción
  \item Tanto en la petición como en la respuesta, las cabeceras con cookies, si es que fueran necesarias para la funcionalidad del escenario que se está describiendo (incluyendo el aspecto que han de tener esas cabeceras). Si la cabecera con cookie va o no dependiendo de algún factor ajeno a tu aplicación, explica cuando irá y cuándo no, y cuál es ese factor.
  \end{itemize}

Además, asegúrate de que describes las interacciones HTTP en el orden en que ocurrirían en el escenario.

\section*{Soluciones}

Hay muchas soluciones posibles. A continuación, una de ellas.

\subsubsection*{Esquema REST}

\begin{tabular}{|l|l|l|}
  \hline
  Recurso & Método & Comentario \\ \hline \hline
  /       & GET    & Página principal (HTML) \\ \hline
  /       & POST   & Creación de canal \\
          &        & \verb|canal=True| \\
          &        & Reserva de nombre \\
          &        & \verb|nombre=nombre_visitante| \\
          &        & Devolverá el mismo HTML que si se invoca con GET \\
          &        & Este HTML incluirá ya un enlace al nuevo canal \\ \hline
  /\{id\_canal\} & GET & Página del canal \verb|id_canal| (HTML) \\ \hline
  /\{id\_canal\} & POST & Subir mensaje al canal \verb|id_canal| \\
          &        & \verb|mensaje=texto| \\
          &        & Subir imagen al canal \verb|id_canal| \\ 
          &        & \verb|imagen=url| \\
          &        & Devolverá el mismo HTML que si se invoca con GET \\ \hline
  /\{id\_canal\}.xml & GET & Página del canal \verb|id_canal| (XML) \\ \hline
  /banner & GET & Imagen que se usará como banner del sitio \\ \hline
\end{tabular}

\vspace{.3cm}

Nota: No se indica en el enunciado, pero convendrá que la página HTML que se reciba como respuesta a un POST para crear un canal incluya, de forma prominente, un enlace a dicho canal, dado que el usuario necesita saber cuál es.

\subsubsection*{urls.py}

\begin{verbatim}
from django.conf.urls import url
from . import views

urlpatterns = [
    url(r'^$', views.pagina_principal),
    url(r'^banner$', views.banner),
    url(r'^(.+\.xml)$', views.canal_xml),
    url(r'^(.+)$', views.canal)
]
\end{verbatim}

\subsubsection*{models.py}

Versión simplificada, que cumple el enunciado:

\begin{verbatim}
from django.db import models

class Visitante(models.Model):
    nombre = models.CharField(max_length=20, null==True)
    cookie = models.CharField(max_length=64)

class Canal(models.Model):
    recurso = models.CharField(max_length=50)
    creador = models.ForeignKey('Visitante')

class Mensaje(models.Model):
    canal = models.ForeignKey('Canal')
    autor = models.ForeignKey('Visitante')
    texto = models.TextField()
    fecha = models.DateTimeField() # Para fecha en XML
\end{verbatim}


\subsubsection*{Primer escenario}

Todas las interacciones son entre el navegador y el sitio Mensajitos.

\begin{itemize}
\item Petición GET /

\begin{verbatim}
  GET / HTTP/1.1
  ...
\end{verbatim}

\item Respuesta

\begin{verbatim}
  HTTP/1.1 200 OK
  Set-Cookie: ....; session=session_id

  [Pagina principal, HTML]
\end{verbatim}

session\_id es un identificador de sesión (o de visitante), que se puede enviar también más adelante. Como identificador de sesión que es, será normalmente una cadena de caracteres larga, generada aleatoriamente, y por tanto difícil de adivinar para quien no la conozca.

\item Petición GET /banner (para cargar la imagen del banner)

\begin{verbatim}
  GET /banner HTTP/1.1
  ...
  Cookie: session=session_id
\end{verbatim}

\item Respuesta

\begin{verbatim}
  HTTP/1.1 200 OK
  ...

  [Banner]
\end{verbatim}

\item Petición POST / (para enviar los datos del formulario de nombre)

\begin{verbatim}
  POST / HTTP/1.1
  ...
  Cookie: session=session_id

  nombre=Nombre_Usado
\end{verbatim}

\item Respuesta

\begin{verbatim}
  HTTP/1.1 200 OK
  ...

  [Pagina principal, ya sin formulario para elegir nombre, HTML]
\end{verbatim}

La cookie que se envió anteriormente, en realidad se podría enviar aquí, pues hasta este momento no hay nada que asociar a la sesión.

\item Petición POST / (para enviar los datos del botón de crear canal)

\begin{verbatim}
  POST / HTTP/1.1
  ...
  Cookie: session=session_id

  canal=True
\end{verbatim}

\item Respuesta

\begin{verbatim}
  HTTP/1.1 200 OK
  ...

  [Pagina principal, ahora con el nuevo canal, HTML]
\end{verbatim}

\end{itemize}

\subsubsection*{Segundo escenario}

A continuación, las interacciones son entre el navegador y el sitio Mensajitos. Suponemos que la imagen del banner ya está en la caché del navegador, y por tanto no se pide. Como el navegador ya ha estado visitando el sitio y tiene nombre, ha de haber recibido la cookie de sesión. Suponemos que ``/2732434232'' es el nombre de recurso correspondiente al canal.

\begin{itemize}
\item Petición POST /2732434232 (para poner el mensaje)

\begin{verbatim}
  POST / HTTP/1.1
  ...
  Cookie: session=session_id

  imagen="url"
\end{verbatim}

\item Respuesta

\begin{verbatim}
  HTTP/1.1 200 OK
  ...

  [Pagina del canal, ahora con un nuevo mensaje con el img correspondiente, HTML
   
   <img src="url" style="width:200px;height:150px;">]
\end{verbatim}

\end{itemize}

Ahora, el navegador tendrá que pedir la imagen que se haya incluido anteriormente (url ``url''). Esta interacción será por lo tanto entre el navegador y el sitio al que apunte la url de la imagen. Suponiendo que la url sear \verb|http://sitio.com/imagen|:

\begin{itemize}
\item Petición GET /imagen (para cargar la imagen)

\begin{verbatim}
  GET /imagen HTTP/1.1
  ...
\end{verbatim}

\item Respuesta

\begin{verbatim}
  HTTP/1.1 200 OK
  ...

  [Imagen]
\end{verbatim}

\end{itemize}



\subsubsection*{Canal XML}


\begin{verbatim}
<?xml version="1.0" encoding="UTF-8" ?>

<canal>
  <recurso>http://mensajitos.com/3443344453</recurso>
  <creado>20 de marzo de 2016 23:05:05</creado>
  <creador>Flor de Loto</creador>
  <mensaje>
    <texto>Este es un mensaje</texto>
  </mensaje>
    <texto>
      <img src="url" style="width:200px;height:150px;">
    </texto>
  <mensaje>
  </mensaje>
  <mensaje>
    <texto>Este es el último mensaje</texto>
  </mensaje>
</canal>
\end{verbatim}

El texto para el mensaje de la imagen tendría que ponerse ``codificado'' para que no se confunda con texto XML, pero esto no se ha tenido en cuenta en esta solución..



%%--------------------------------------------------------------------------
%%--------------------------------------------------------------------------
\subsection{Examen de IST-SARO, 10 mayo de 2017}

Se quiere construir un sitio web, Fotogram, donde se pueden poner fotos para que las vean otras personas. La funcionalidad básica del sitio es la siguiente:

\begin{enumerate}
\item En el sitio no hay cuentas para usuarios: toda la funcionalidad está disponible para cualquiera que lo visite.
\item La página principal del sitio mostrará a los visitantes un formulario para crear un nuevo canal de fotos. Este formulario permitirá elegir un nombre para el canal, y el nombre del recurso en que se servirá, que deberá ser (el nombre del recurso) uno que no esté ya en uso en el sitio. Además, en esta página principal cada visitante verá la lista de los canales que ha creado previamente, y junto a cada uno habrá un botón para borrarlo. Tras crear un nuevo canal, o borrarlo, el visitante volverá a ver la página principal del sitio.
\item Cualquiera que conozca el nombre de recurso de un canal, podrá ver sus fotos, y poner fotos en él, simplemente accediendo a ese recurso (lo haya creado quien lo haya creado).
\item El recurso correspondiente a cada canal mostrará una página HTML (la ``página del canal'') con las fotos puestas en ese canal y el comentario asociado a cada foto (si lo hay), y un formulario para poner una nueva foto. Este formulario permitirá especificar la url de una foto (que puede estar en cualquier sitio de Internet, mientras la haga visible mediante HTTP), y opcionalmente un comentario asociado a esa foto. Para cada foto que se muestre se mostrará la foto, el comentario asociado a ella (si lo hay) y la fecha en que se subió la foto.
Tras poner una nueva foto en un canal, el visitante vuelve a ver de nuevo la página de ese canal.
\item El sitio aceptará en un recurso (uno para todo el sitio, no uno por canal), no enlazado en ninguna página del mismo, un documento XML con un listado de fotos a subir a un canal, que se recibirá en el cuerpo de un POST de HTTP. El documento XML incluirá el nombre de recurso del canal donde se subirán las fotos (que deberá existir), y un listado de las fotos a subir. Para cada foto, se incluirá su url y (opcionalmente) su comentario asociado.
\item Todas las páginas HTML del sitio incluirán una imagen de cabecera (banner) que se alojará en el sitio de url \verb|http://banners.com|.

\item Se supone que la cache del navegador está deshabilitada.
\end{enumerate}

Teniendo en cuenta los requisitos anteriores, se pide:

\begin{enumerate}
\item Diseña un esquema REST para proporcionar el servicio descrito. Se habrán de especificar los nombres de recurso empleados, y cómo reaccionará la aplicación cuando reciba los métodos POST o GET sobre esas urls (no se usarán los métodos PUT o DELETE). Coloca la información en una tabla, con las urls en una columna, los métodos en otra, y la descripción de lo que realizará la aplicación al recibirlos en la tercera. Escribe también un fichero similar al fichero urls.py de Django (aunque no es importante que se respete la sintaxis mientras se entienda y la estructura sea similar a la de Django), que refleje el esquema REST anterior (1 punto).

\item Describe el modelo de datos que necesitará esta aplicación. Define las tablas necesarias y los campos necesarios para la funcionalidad descrita. Hazlo de forma lo más similar posible a lo que tendrías que escribir en el fichero models.py en Django (aunque no es importante que se respete la sintaxis mientras se entienda el modelo de datos que propones) (1 punto).

\item Describe las interacciones HTTP que ocurrirán entre el navegador y cualquier servidor web en el siguiente escenario. El escenario comienza cuando un visitante que accede por primera vez al sitio pone en el navegador la url de la página principal del sitio. A continuación, después de ver esta página principal, rellena el formulario para crear un canal. El sitio le devuelve una página donde aparecerá ya el canal recién creado en la lista de canales. El escenario termina cuando el visitante ve en su navegador la página de ese canal, tras haber pulsado sobre él en la lista (1 punto).

\item Describe las interacciones HTTP que ocurrirán entre el navegador y cualquier servidor web en el siguiente escenario. El escenario comienza cuando un visitante que accede por primera vez al sitio pone en el navegador la url de la página de un canal en el que ya hay una foto. A continuación, después de ver la página del canal en cuestión con esa foto, rellena el formulario para poner una nueva foto, indicando su url (no incluye comentario). El escenario termina cuando el visitante ve en su navegador de nuevo la página del canal, ya con la foto que acaba de poner (1 punto).

\item Escribe cómo podría ser el documento XML para subir un listado de fotos a un canal (1 punto).
\end{enumerate}

En todos los escenarios, ten en cuenta que tu respuesta debe considerar toda la funcionalidad que ofrece el servicio, y permitir que ésta pueda proporcionarse. Diseña la aplicación de forma que envíe cookies al navegador sólo cuando sea necesario.

En las respuestas donde describas interacciones HTTP indica para cada una de ellas claramente y en este orden:
  \begin{itemize}
  \item La primera línea de la petición HTTP
  \item Si lo hay, el contenido de la petición
  \item La primera línea de la respuesta HTTP
  \item Si lo hay, el contenido de la respuesta
  \item Una brevísima explicación de para qué se usa la interacción
  \item Tanto en la petición como en la respuesta, las cabeceras con cookies, si es que fueran necesarias para la funcionalidad del escenario que se está describiendo (incluyendo el aspecto que han de tener esas cabeceras). Si la cabecera con cookie va o no dependiendo de algún factor ajeno a tu aplicación, explica cuando irá y cuándo no, y cuál es ese factor.
  \end{itemize}

Además, asegúrate de que describes las interacciones HTTP en el orden en que ocurrirían en el escenario.


\subsection*{Soluciones}

Hay muchas soluciones posibles. A continuación, una de ellas.

\subsubsection*{Esquema REST}

\begin{tabular}{|l|l|l|}
  \hline
  Recurso & Método & Comentario \\ \hline \hline
  /       & GET    & Página principal (HTML) \\ \hline
  /       & POST   & Creación de canal \\
          &        & \verb|canal=Nombre&recurso=Recurso| \\
          &        & Para borrar canal, misma qs, con nombre de canal vacío \\
          &        & Devolverá el mismo HTML que si se invoca con GET \\
          &        & Este HTML incluirá ya un enlace al nuevo canal \\
          &        & cuando se haya creado \\ \hline
  /\{rec\_canal\} & GET & Página del canal \verb|rec_canal| (HTML) \\ \hline
  /\{rec\_canal\} & POST & Subir foto al canal \verb|rec_canal| \\
          &        & \verb|foto=url&comentario=Texto| \\
          &        & Devolverá el mismo HTML que si se invoca con GET \\ \hline
  /subir  & POST   & Página del canal \verb|id_canal| (XML) \\ \hline
\end{tabular}

\subsubsection*{urls.py}

\begin{verbatim}
from django.conf.urls import url
from . import views

urlpatterns = [
    url(r'^$', views.pagina_principal),
    url(r'^(subir)$', views.subir_xml),
    url(r'^(.+)$', views.canal)
]
\end{verbatim}

\subsubsection*{models.py}

Versión simplificada, que cumple el enunciado:

\begin{verbatim}
from django.db import models

class Visitante(models.Model):
    cookie = models.CharField(max_length=64)

class Canal(models.Model):
    recurso = models.CharField(max_length=50)
    nombre = models.CharField(max_length=50)
    creador = models.ForeignKey('Visitante')

class Foto(models.Model):
    canal = models.ForeignKey('Canal')
    url = models.CharField(max_length=50)
    comentario = models.TextField()
    fecha = models.DateTimeField()
\end{verbatim}


\subsubsection*{Primer escenario}

Las interacciones son entre el navegador y el sitio que se indica.

\begin{itemize}
\item Petición GET / (a Fotogram)

\begin{verbatim}
  GET / HTTP/1.1
  ...
\end{verbatim}

\item Respuesta

\begin{verbatim}
  HTTP/1.1 200 OK
  Set-Cookie: ....; session=session_id

  [Pagina principal, HTML]
\end{verbatim}

session\_id es un identificador de sesión (o de visitante), que se puede enviar también más adelante. Como identificador de sesión que es, será normalmente una cadena de caracteres larga, generada aleatoriamente, y por tanto difícil de adivinar para quien no la conozca.

\item Petición GET /banner (a Banners)

\begin{verbatim}
  GET /banner HTTP/1.1
  ...
\end{verbatim}

\item Respuesta

\begin{verbatim}
  HTTP/1.1 200 OK
  ...

  [Banner]
\end{verbatim}

\item Petición POST / (para enviar los datos del formulario de canal)

\begin{verbatim}
  POST / HTTP/1.1
  ...
  Cookie: session=session_id

  canal=Nombre&recurso=Recurso
\end{verbatim}

\item Respuesta

\begin{verbatim}
  HTTP/1.1 200 OK
  ...

  [Pagina principal, ya con enlace al canal recién creado, HTML]
\end{verbatim}

La cookie que se envió anteriormente, en realidad se podría enviar aquí, pues hasta este momento no hay nada que asociar a la sesión.

\item Petición GET /banner (a Banners)

\begin{verbatim}
  GET /banner HTTP/1.1
  ...
\end{verbatim}

\item Respuesta

\begin{verbatim}
  HTTP/1.1 200 OK
  ...

  [Banner]
\end{verbatim}

\item Petición GET /Recurso (para ver la página del canal, a Fotogram)

\begin{verbatim}
  GET / HTTP/1.1
  ...
  Cookie: session=session_id
\end{verbatim}

\item Respuesta

\begin{verbatim}
  HTTP/1.1 200 OK
  ...

  [Pagina del canal, HTML]
\end{verbatim}

\item Petición GET /banner (a Banners)

\begin{verbatim}
  GET /banner HTTP/1.1
  ...
\end{verbatim}

\item Respuesta

\begin{verbatim}
  HTTP/1.1 200 OK
  ...

  [Banner]
\end{verbatim}

\end{itemize}

\subsubsection*{Segundo escenario}

Las interacciones son entre el navegador y el sitio que se indica.

\begin{itemize}
\item Petición GET /Recurso (a Fotogram)

\begin{verbatim}
  GET / HTTP/1.1
  ...
\end{verbatim}

\item Respuesta

\begin{verbatim}
  HTTP/1.1 200 OK
  ...
  Set-Cookie: ....; session=session2_id

  [Pagina del canal, que lleva una foto, HTML]
\end{verbatim}

session2\_id es un identificador de sesión (o de visitante), que se puede enviar también más adelante (o incluso no enviar en este escenario, porque según enunciado no se traza qué visitante sube las fotos).

\item Petición GET /banner (a Banners)

\begin{verbatim}
  GET /banner HTTP/1.1
  ...
\end{verbatim}

\item Respuesta

\begin{verbatim}
  HTTP/1.1 200 OK
  ...

  [Banner]
\end{verbatim}

\item Petición GET /Foto (al sitio donde está la foto)

\begin{verbatim}
  GET /Foto HTTP/1.1
  ...
\end{verbatim}

\item Respuesta

\begin{verbatim}
  HTTP/1.1 200 OK
  ...

  [Foto, JPEG, PNG, GIF, etc.]
\end{verbatim}


\item Petición POST /Recurso (a Fotogram)

\begin{verbatim}
  POST /Recurso HTTP/1.1
  ...
  Cookie: session=session2_id

  foto=url_foto2&comentario=
\end{verbatim}

\item Respuesta

\begin{verbatim}
  HTTP/1.1 200 OK
  ...

  [Pagina del canal, que lleva una foto, HTML]
\end{verbatim}

\item Petición GET /banner (a Banners)

\begin{verbatim}
  GET /banner HTTP/1.1
  ...
\end{verbatim}

\item Respuesta

\begin{verbatim}
  HTTP/1.1 200 OK
  ...

  [Banner]
\end{verbatim}

\item Petición GET /Foto (al sitio donde está la foto)

\begin{verbatim}
  GET /Foto HTTP/1.1
  ...
\end{verbatim}

\item Respuesta

\begin{verbatim}
  HTTP/1.1 200 OK
  ...

  [Foto, JPEG, PNG, GIF, etc.]
\end{verbatim}

\item Petición GET /Foto2 (al sitio donde está la segunda foto)

\begin{verbatim}
  GET /Foto2 HTTP/1.1
  ...
\end{verbatim}

\item Respuesta

\begin{verbatim}
  HTTP/1.1 200 OK
  ...

  [Foto, JPEG, PNG, GIF, etc.]
\end{verbatim}

\end{itemize}


\subsubsection*{Documento XML}


\begin{verbatim}
<?xml version="1.0" encoding="UTF-8" ?>

<fotos>
  <recurso>/Canal</recurso>
  <foto>
    <url>http://sitiodefotos.com/foto1</url>
  </foto>
  <foto>
    <url>http://sitiodefotos2.com/foto2</url>
    <comentario>Esta es una foto</comentario>
  </foto>
  <foto>
    <url>http://sitiodefotos3.com/foto3</url>
    <comentario>Esta es otra foto</comentario>
  </foto>
</fotos>
\end{verbatim}

\newpage

%%--------------------------------------------------------------------------
%%--------------------------------------------------------------------------
%%--------------------------------------------------------------------------
\section{Materiales de interés}

%%--------------------------------------------------------------------------
%%--------------------------------------------------------------------------
\subsection{Material complementario general}

\begin{itemize}
\item Philip Greenspun, \textsl{Software Engineering for Internet Applications}:\\
  \url{http://philip.greenspun.com/seia/} \\
  utilizado en un curso del MIT \\
  \url{http://philip.greenspun.com/teaching/one-term-web}
\end{itemize}

%%--------------------------------------------------------------------------
%%--------------------------------------------------------------------------
\subsection{Introducción a Python}

\begin{itemize}
\item \url{http://www.python.org/doc}

Documentación en línea de Python (incluyendo un Tutorial, los manuales de referencia, HOWTOS, etc. Usa la versión para Python 2.x

\item \url{http://www.diveintopython.org/}

``Dive into Python'', por Mark Pilgrim. Libro para aprender Python, orientado a quien ya sabe programa con lenguajes orientados a objetos.

\item \url{http://wiki.python.org/moin/BeginnersGuide/Programmers}

Otros textos sobre Python, de interés especialmente para quien ya sabe programar en otros lenguajes.

\item \url{http://en.wikibooks.org/wiki/Python_Programming}

``Python Programming'', Wikibook sobre programación en Python.

\item \url{http://en.wikipedia.org/wiki/Python_(programming_language)}

Python en la Wikipedia

\item \url{http://www.python.org/dev/peps/pep-0008/}

Style Guide for Python Code (PEP 8). Esta es la guía de estilo que se puede comprobar con el programa pep8.
\end{itemize}

%%--------------------------------------------------------------------------
%%--------------------------------------------------------------------------
\subsection{Aplicaciones web mínimas}

\begin{itemize}
\item \url{http://docs.python.org/dev/howto/sockets.html}

``Socket Programming HOWTO''. Programación de sockets en Python, guía rápida.

\item \url{http://docs.python.org/library/socket.html}

Documentación de la biblioteca de sockets de Python.

\item \url{https://addons.mozilla.org/en-US/firefox/} 

Lista de add-ons y plugins para Firefox.

\end{itemize}

%%--------------------------------------------------------------------------
%%--------------------------------------------------------------------------
\subsection{SQL y SQLite}


\begin{itemize}
\item \url{http://www.shokhirev.com/nikolai/abc/sql/sql.html}

``SQLite / SQL Tutorials: Basic SQL'', por Nikolai Shokhirev

\end{itemize}


\end{document}

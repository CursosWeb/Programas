%%----------------------------------------------------------------------------
%%----------------------------------------------------------------------------
\section{Proyecto final: DeCharla (2023)}
\label{practica-final-2023-05}

\textbf{Repositorio plantilla.}

\url{https://gitlab.etsit.urjc.es/cursosweb/2022-2023/final-decharla}

[ \textbf{Nota importante:} Enunciado aún en proceso de elaboración, puede sufrir algunos cambios antes de su versión final. Por favor, avisa si detectas algún error o inconsistencia. ]

%[ \textbf{Nota importante:} Este enunciado puede sufrir cambios si se detectan errores en él. ]

Este es el enunciado del proyecto final del curso 2022-2023, tanto para la convocatoria ordinaria como para la extraordinaria.

La práctica final de la asignatura consiste en la creación de una aplicación web, llamada ``DeCharla'', que permitirá poner mensajes, y ver los mensajes que hayan puesto otros. Algunos de los mensajes podrán ser obtenidos automáticamente de ciertas fuentes soportadas. Los mensajes podrán ser un texto o una imagen. Los mensajes se organizarán en ``salas'', que serán accesibles cada una en un recurso. El nombre del recurso de cada sala será decidido por quien la cree, en el momento de la creación. El ``nombre de la sala'' será el nombre del recurso de la sala, quitándole la ``/'' inicial. La ``página de la sala'' será al página HTML que se recibe cuando se invoca mediante GET el recurso de la sala. No habrá autenticación de usuarios para acceder a las salas: cada visitante podrá poner mensajes en una sala si conoce el nombre de su recurso. Cuando en este enunciado se indique que ``hizo algo en el sitio'' (por ejemplo, se diga ``se verán los mensajes escritos anteriormente'' o ``desde que se visitó la página por última vez'') se refiere a que se hizo algo en el sitio desde el mismo navegador en que se está ahora (por ejemplo ``se verán los mensajes escritos anteriormente desde el mismo navegador'' o ``desde que se visitó la página por última vez desde el mismo navegador'').

A continuación se describe el funcionamiento y la arquitectura general de la aplicación, la funcionalidad mínima que debe proporcionar, y otra funcionalidad optativa que podrá tener.

%%----------------------------------------------------------------------------
\subsection{Requisitos generales}

\begin{itemize}

\item La práctica se construirá como un proyecto Django/Python3, que incluirá una o varias aplicaciones (\emph{apps}) Django que implementen la funcionalidad requerida.

\item Para el almacenamiento de datos persistente se usará SQLite3, con tablas definidas en modelos de Django.

\item Todas las bases de datos que contenga la aplicación tendrán que ser accesibles vía la interfaz que proporciona el ``Admin Site'' (además de lo que pueda hacer falta para que funcione al aplicación). Para acceder a este ``Admin Site'' habrá cuentas específicas, creadas vía el propio ``Admin Site''.

\item Se utilizarán plantillas Django (a ser posible, una jerarquía de plantillas, para que la práctica tenga un aspecto similar) para definir las páginas que se servirán a los navegadores de los usuarios (excepto las páginas de ``Admin Site'' (ver ``estructura de las páginas HTML, a continuación).

\end{itemize}

Se servirán al menos las siguientes páginas (recursos) con los contenidos descritos a continuación:

\begin{description}
\item[Página principal.] En la página principal se ofrecerá un listado de todas las salas en las que se ha puesto algún mensaje. Para cada sala se indicará su nombre, que se presentará como un enlace a la página de la sala, y dos números: el número de mensajes total, y el número de mensajes desde que se visitó la sala por última vez.
\item[Página de cada sala.] En la página de cada sala se mostrará:
  \begin{itemize}
  \item Formularios para poner un mensaje en la sala. Como el mensaje podrá ser un texto o una imagen, el formulario incluirá una caja de texto, en el que se podrá escribir el texto de un mensaje o la URL de una imagen, y un ``checkbox''\footnote{Input type checkbox: \\ \url{https://developer.mozilla.org/en-US/docs/Web/HTML/Element/input/checkbox}} que si está seleccionado indicará que el texto es una URL (y el texto de un mensaje si no es así).
  \item Un listado de todos los mensajes puestos en la sala desde cualquier navegador (primero los más recientes). Las imágenes se mostrarán con un tamaño del 50\% del ancho disponible, usando por ejemplo\footnote{Ver detalles sobre cómo definir el tamaños de la imágenes en: \\ \url{https://imagekit.io/blog/how-to-resize-image-in-html/}}:

\begin{verbatim}
img {
  width: 50%;
  height: auto;
}
\end{verbatim}

  \end{itemize}

\item[Página dinámica de cada sala.] En la página dinámica de cada sala se mostrará el mismo contenido que en la página de una sala, pero usando un script JavaScript que actualizará el contenido cada poco tiempo (ver más adelante, en la lista de preguntas frecuentes (apartado \ref{sec:practica-2023-05:preguntas}).
  
\item[Página de configuración.] Permitirá configurar el sitio para ese navegador. Incluirá:
  \begin{itemize}
  \item Un formulario para elegir el nombre con el que se pondrán los mensajes (nombre de charlador).
  \item Un formulario para elegir los elementos de apariencia personalizables del sitio (el tamaño y el tipo de la letra).
  \end{itemize}
\item[Página de ayuda.] Página con información en HTML indicando la autoría de la práctica, explicando su funcionamiento y una brevísima documentación.
\end{description}

Elementos generales que aparecerán en todas las páginas HTML:

\begin{description}
\item[Cabecera.] En la parte superior estará la cabecera del sitio, que consistirá de un texto (el nombre del sitio, ``DeCharla'') en un tipo de letra que resalte claramente, sobre una imagen de fondo. El texto será un enlace al recurso ``/'' del sitio.
\item[Menú.] En una barra debajo del elemento anterior tendremos un menú desde el que se podrá acceder, mediante enlaces, al menos a las siguientes páginas (salvo si la opción apunta a la página en la que se está, en cuyo caso no aparecerá en esa página):
  \begin{itemize}
  \item la página principal (con el texto ``Principal'')
  \item la página de configuración (con el texto ``Configuración'')
  \item la página de ayuda (con el texto ``Ayuda'')
  \item la página de acceso al ``Admin Site'' (con el texto ``Admin''). 
  \end{itemize}
\item[Extra.] En una columna a la derecha (en escritorio) o bajo los elementos anteriores (en móviles):
  \begin{itemize}
  \item El nombre de charlador que se está usando desde ese navegador (o ``Anónimo'' si no se ha especificado uno).
  \item Un formulario para acceder a una sala indicando su nombre. El formulario permitirá escribir el nombre, y tendrá un botón asociado que, al pulsarlo, pasará a mostrar la página de la sala (mediante una redirección). Es importante darse cuenta de que no hace falta que se hayan puesto mensajes previamente en esa sala para que pueda ``verse'': cualquier nombre puesto en el formulario se considerará el nombre de una sala.
  \end{itemize}
\item[Cuerpo.] En una columna más ancha a la izquierda (en escritorio) o bajo los elementos anteriores (en móviles), el cuerpo de la página (los contenidos específicos de esa página, descritos más abajo).
\item[Pie.] Un pie de página con un resumen de métricas: ``Mensajes: XXX (textuales), YYY (imágenes). Salas activas: ZZZ'', siendo XXX el número de mensajes, YYY el número de imágenes, y ZZZ el número de salas con al menos un mensaje en todo el sitio.
\end{description}
  
En lo que tiene que ver con el aspecto de las páginas se tendrá en cuenta:

\begin{description}
\item[Marcado HTML.] Cada uno de los elementos generales descritos anteriormente estará construido dentro de un elemento \texttt{div}, marcado con un atributo \texttt{id} en HTML, para poder ser referido fácilmente en hojas de estilo CSS. Cuando sea conveniente, se podrán utilizar en lugar de \texttt{div} elementos de HTML5 (\texttt{header}, \texttt{footer}, \texttt{nav}, etc).

\item[CSS.] Se utilizarán hojas de estilo CSS para determinar la apariencia del sitio. Estas hojas definirán al menos el color de fondo y de letra de cada una de las partes (elementos) marcadas con un \emph{id} (ver ``Marcado HTML''). Además, elementos que deban tener el mismo aspecto deberían estar en una misma clase CSS, para poder gestionarlo de forma común. También definirán los elementos personalizables del sitio: el tamaño de la letra y el tipo de la letra.

\item[Bootstrap.] Se utilizará Bootstrap para la maquetación (\emph{layout}) de las páginas, de forma que funcionen adecuadamente tanto en navegadores de escritorio como en móviles.
\end{description}

El sitio también proporcionará:

\begin{description}
\item[Salas en formato JSON.] Para cada sala, se ofrecerá un recurso donde el documento servido estará en formato JSON, e incluirá todos los mensajes puestos en la sala, en el siguiente formato:

\begin{verbatim}
[
  {
    "author": "XXX",
    "text": "....",
    isimg: false,
    date: "2023-04-16:14:04:01"
  },
  {
    ....
  }
  ...
]
\end{verbatim}

\item[Actualización de salas en XML.] Para cada sala, se ofrecerá un recurso donde se admitirá recibir un documento XML en el cuerpo de una petición HTTP PUT. Los contenidos de este cuerpo se usarán para recibir nuevos mensajes para la sala. El formato del documento XML será como sigue:

\begin{verbatim}
<?xml version="1.0" encoding="UTF-8"?>

<messages>
  <message isimg="false">
    <text>
       ...
    </text>
  </message>
  <message isimg="true">
    <text>
       https://yyy....
    </text>
  </message>
  ...
</messages>
\end{verbatim}

\end{description}


Autenticación:

\begin{itemize}
\item Todos los recursos del sitio estarán ``protegidos'' de forma que sólo se podrá acceder a ellos (mediante cualquier comando HTTP) si se ha iniciado una sesión.
  
\item Cuando un navegador acceda a cualquier recurso del sitio sin haber iniciado sesión, recibirá un formulario en el que se le solicitará una contraseña. El sitio mantendrá una lista de una o más contraseñas válidas (que serán cadenas alfanuméricas de cualquier longitud), en una tabla de la base de datos. Si la contraseña indicada en el formulario es una de ellas, se iniciará la sesión usando cookies (y se realizará la función correspondiente al recurso invocado). Si no es así, se enviará el formulario de nuevo, con código de error 401 (Unauthorized).

\item Una vez un navegador ha recibido una cookie de autenticación, no tendrá que mostrar ya el formulario de autenticación a ese navegador.
  
\item Los scripts que accedan al sitio para poner noticias mediante la interfaz XML podrán utilizar una contraseña válida, incluyéndola en su petición en la cabecera Authorization como la siguiente (siendo ``xx34d23'' la contraseña):

\begin{verbatim}
Autorization: Basic xx34d23
\end{verbatim}

\end{itemize}

Tests:

\begin{description}
\item[Extremo a extremo.] Para cada tipo de recurso de la práctica habrá que realizar al menos un test ``extremo a extremo''.
\item[Unitarios.] Será conveniente (pero no obligatorio) realizar también tests unitarios para distintas partes del código. Se se hace, es conveniente mencionarlo en el fichero de entrega como una parte opcional.
\end{description}

%%----------------------------------------------------------------------------
\subsection{Despliegue}
\label{sec:practica-2023-05:despliegue}

La práctica deberá estar desplegada en algún sitio de Internet, de forma que pueda accederse a ella. Deberá mantenerse desplegada y activa al menos desde el día de entrega de la práctica, hasta el día del cierre de actas.

Para el despliegue, se puede utilizar Python Anywhere\footnote{Python Anywhere: \url{https://pythonanywhere.com}}, que proporciona un plan gratuito que incluye suficientes recursos como para poder desplegar la práctica.

Si el alumno así lo desea, puede considerarse desplegar en un ordenador dedicado (por ejemplo, una Raspberry Pi accesible directamente desde Internet, alojada en su hogar), o en servicios como Google Computing Engine\footnote{GCP Engine Free: \url{https://cloud.google.com/free/}}. En general, dado que este tipo de despliegues no podrá contar con una ayuda detallada por los profesores, estará algo más valorado.

En el caso de que la práctica se despliegue en Python Anywhere, hay que tener en cuenta que sus máquinas virtuales tienen cortado el acceso a todos los sitios de Internet salvo los que están en una ``lista blanca''\footnote{Lista blanca de Python Anywhere: \url{https://www.pythonanywhere.com/whitelist/}} (\emph{whitelist}). Es de esperar que esto no cause problemas, porque el sitio GitLab de la ETSIT, donde está vuestro código fuente, está ya en la lista blanca, y por ello Python Anywhere debería poder acceder a él para clonar vuestro repositorio. Sin embargo, si vuestra aplicación se conectase a cualquier sitio para obtener información,si este sitio no está en la lista blanca, vuestra aplicación no se podrá conectar. Wikipedia y Flickr, por ejemplo, están ya en la lista blanca. Pero otros sitios que podríais estar usando, no. 

Para lo tanto, si hacéis el despliegue en Python Anywhere y vuestra aplicación ha de obtener datos de un sitio tercero:

\begin{itemize}
  \item Si está ya en la lista blanca de Python Anywhere, funcionará sin problemas sin hacer nada especial. Si os funcionaba ya en las pruebas locales, así que también deberían funcionar en el despliegue.
  \item Si no están en la lista blanca de Python Anywhere, aseguraos de que la base de datos que subís al despliegue de vuestra práctica incluya ya datos de ese sitio tercero..
\end{itemize}

Tened en cuenta que si usáis otras plataformas para el despliegue, puede que os encontréis problemas similares. Y tened en cuenta también que en cualquier caso, nosotros probaremos la práctica en otros despliegues, así que todos los recursos reconocidos que hayáis implementado deben funcionar correctamente si no hay restricciones de conexión.

Tenéis más detalles sobre cómo se hace un despliegue de una aplicación Django en Python Anywhere en el vídeo ``Django: Despliegue en Python Anywhere''\footnote{\url{https://www.youtube.com/watch?v=hlZPC5L2Itc}}, que explica cómo desplegar allí la aplicación \texttt{django-youtube-4}\footnote{\url{https://github.com/CursosWeb/Code/tree/master/Python-Django/django-youtube-4}} (ver también el fichero \texttt{README.md} de esa aplicación para más detalles).

%%----------------------------------------------------------------------------
\subsection{Funcionalidad optativa}

De forma optativa, se podrá incluir cualquier funcionalidad relevante en el contexto de la asignatura. Se valorarán especialmente las funcionalidades que impliquen el uso de técnicas nuevas, o de aspectos de Django no utilizados en los ejercicios previos, y que tengan sentido en el contexto de esta práctica y de la asignatura.

En el formulario de entrega se pide que se justifique por qué se considera funcionalidad optativa lo que habéis implementado. Sólo a modo de sugerencia, se incluyen algunas posibles funcionalidades optativas:

\begin{itemize}
  \item Inclusión de un \emph{favicon} del sitio (FÁCIL).

  \item Permitir que las sesiones autenticadas se puedan terminar. Esto es, que haya una opción para ``terminar la sesión'', de forma que al usarla, si se vuelve a acceder al sitio vuelva a recibirse el formulario para autenticarse (FÁCIL).

  \item Permitir votar las las salas. Para ello, en cada sala aparecerá un botón ``Me gusta'' para indicar que se da un voto a la sala. Cada sala, en el listado general, saldrá junto a los votos que tiene.

  \item Creación de un script para utilizar información de un sitio y subirla a la DeCharla como uno o varios mensajes, usando el formato XML descrito anteriormente. Por ejemplo, se puede usar información de la AEMET para subir a una cierta sala mensajes sobre el tiempo previsto en una localidad. Pueden realizarse varios scripts de este estilo, con diferentes sitios de información (DIFÍCIL).

  \item Igual que la anterior, pero incorporando directamente a DeCharla la funcionalidad de obtener información de sitios terceros e incluirla como mensajes en una sala. En este caso, en cada sala aparecerá uno o varios formularios, cada uno con los datos necesarios para incorporar esa funcionalidad. Por ejemplo, para incorporar previsiones de tiempo de la AEMET, el formulario ofrecerá una caja de texto para poner el nombre de la localidad (DIFÍCIL).
        
  \item Atención al idioma indicado por el navegador. El idioma de la interfaz de usuario de la aplicación tendrá en cuenta lo que especifique el navegador. Para ello, pueden utilizarse las facilidades que proporciona Django para la localización\footnote{Django, Internationalization and localization: \\ \url{https://docs.djangoproject.com/en/4.2/topics/i18n/} } (REGULAR).
    
  \item Mejora de los tests de la práctica, incluyendo test de condiciones de error, test de escenarios con más de una invocación de recurso, tests de API Python, etc. En general, si se realizan tests más allá de los básicos ``extremo a extremo'' que se piden, serán considerados como opciones (REGULAR).
\end{itemize}

%%----------------------------------------------------------------------------
\subsection{Entrega de la práctica}

\begin{itemize}
  \item \textbf{Fecha límite de entrega (convocatoria ordinaria):} día del examen de la asignatura a las 23:55 (hora española peninsular)
  \item \textbf{Fecha límite de entrega (convocatoria extraordinaria):} día del examen de la asignatura, a las 23:55 (hora española peninsular)
  \item \textbf{Notificación de alumnos que tendrán que realizar entrevista:} antes de la fecha de revisión de la asignatura.
  \item \textbf{Realización de entrevistas:} junto con la revisión de la asignatura.    
  % \item \textbf{Fecha de publicación de notas:} jueves, 10 de junio, en el aula virtual.

  % \item \textbf{Fecha de revisión:} viernes, 11 de junio, a las 9:00, en la aplicación Teams.
\end{itemize}

La entrega de la práctica consiste en:

\begin{enumerate}
  
  \item {\bf Subir tu práctica a un repositorio en el GitLab de la Escuela}. El repositorio contendrá todos los ficheros necesarios para que funcione la aplicación (ver detalle más abajo). Es muy importante que el alumno haya realizado una derivación (fork) del repositorio de plantilla de al asignatura, indicado al principio de este enunciado.

Recordad que es importante ir haciendo commits de vez en cuando y que sólo al hacer push estos commits son públicos. Antes de entregar la práctica, haced un push. Y cuando la entreguéis y sepáis el nombre del repositorio, podéis cambiar el nombre del repositorio desde el interfaz web de GitLab. 

Se recomienda mantener el repositorio como privado, hasta el momento en que se entregue la práctica.

\item Para considerar la práctica como entregada es fundamental que en el directorio raíz del repositorio de entrega haya un fichero \texttt{README.md} cuya priumera línea sea:

\begin{verbatim}
# ENTREGA CONVOCATORIA XXX
\end{verbatim}

Siendo \texttt{XXX} ``MAYO'' si se entrega en la convocatoria ordinaria de mayo, y ``JUNIO'' si se entrega en la convocatoria extraordinaria de junio.

\textbf{Atención:} No se considerará entregada la práctica si no ve esta primera línea en el formato adecuado.

 \item {\bf Entregar un vídeo de demostración de la parte obligatoria, y otro vídeo de demostración de la parte opcional}, si se ha realizado parte optativa. Los vídeos serán de una {\bf duración máxima de 3 minutos} (cada uno), y consistirán en una captura de pantalla de un navegador web utilizando la aplicación, y mostrando lo mejor posible la funcionalidad correspondiente (básica u opcional). Siempre que sea posible, el alumno comentará en el audio del vídeo lo que vaya ocurriendo en la captura. Los vídeos se colocarán en algún servicio de subida de vídeos en Internet (por ejemplo, Vimeo, Twitch, o YouTube). Los vídeos de más de tres minutos tendrán penalización.

Hay muchas herramientas que permiten realizar la captura de pantalla. Por ejemplo, en GNU/Linux puede usarse Gtk-RecordMyDesktop o Istanbul (ambas disponibles en Ubuntu). OBS Studio\footnote{OBS Studio: \url{https://obsproject.com/}} está disponible para varias plataformas (Linux, Windows, MacOS). Es importante que la captura sea realizada de forma que se distinga razonablemente lo que se grabe en el vídeo.

En caso de que convenga editar el vídeo resultante (por ejemplo, para eliminar tiempos de espera) puede usarse un editor de vídeo, pero siempre deberá ser indicado que se ha hecho tal cosa con un comentario en el audio, o un texto en el vídeo. Hay muchas herramientas que permiten realizar esta edición. Por ejemplo, en GNU/Linux puede usarse OpenShot o PiTiVi.

\end{enumerate}

Sobre la entrega del repositorio:
\begin{itemize}
  \item Se han de entregar los siguientes ficheros:

\begin{itemize}
  \item El repositorio en la instancia GitLab de la ETSIT deberá tener, en su directorio raíz, un proyecto Django completo y listo para funcionar en el entorno del laboratorio, incluyendo la base de datos. Deberá poder ejecutarse directamente con \verb|python3 manage.py runserver| desde un entorno virtual en el que esté instalado Django~4.1. También ejecutará los tests con \verb|python3 manage.py test|, desde el mismo entorno virtual.

  \item La base de datos habrá de tener datos suficientes como para poder probarlo. Estos datos incluirán al menos mensajes puestos desde dos navegadores, con al menos diez mensajes en total, en al menos tres salas.

  \item Un fichero \verb|requirements.txt|, con un nombre de paquete Python por línea, para indicar cualquier biblioteca Python que pueda hacer falta para que la aplicación funcione, si es que fuera el caso, incluyendo Django. Si es posible, se recomienda escribir este fichero en el formato que entiende \verb|pip install -r requirements.txt|

  \item Cualquier fichero auxiliar que pueda hacer falta para que funcione la práctica, si es que fuera el caso.
\end{itemize}

  \item Se incluirán en el fichero README.md los siguientes datos (atención a lo que ya se han indicado sobre la primera línea de este fichero):

\begin{itemize}
  \item Nombre y titulación.
  \item Nombre de cuenta en laboratorios Linux de la ETSIT.
  \item Nombre de cuenta de la URJC.
  \item URL del vídeo demostración de la funcionalidad básica
  \item URL del vídeo demostración de la funcionalidad optativa, si se ha realizado funcionalidad optativa
  \item URL de la aplicación desplegada
  \item Al menos una contraseña que permita autenticarse para poder usar la aplicación.
  \item Cuenta del superusuario para entrar en el Admin Site.
  \item Resumen de las peculiaridades que se quieran mencionar sobre lo implementado en la parte obligatoria.
  \item Lista de funcionalidades opcionales que se hayan implementado, y breve descripción de cada una.
\end{itemize}

Estos datos se escribirán siguiendo estrictamente el siguiente formato:

\begin{verbatim}
# Entrega practica

## Datos

* Nombre:
* Titulación:
* Cuenta en laboratorios:
* Cuenta URJC:
* Video básico (url):
* Video parte opcional (url):
* Despliegue (url):
* Contraseñas:
* Cuenta Admin Site: usuario/contraseña

## Resumen parte obligatoria

## Lista partes opcionales

* Nombre parte:
* Nombre parte:
* ...
\end{verbatim}

Asegúrate de que el fichero esté adecuadamente formateado en Markdown.
\end{itemize}


%%----------------------------------------------------------------------------
\subsection{Notas y comentarios}

La práctica deberá funcionar en el entorno GNU/Linux (Ubuntu) del laboratorio de la asignatura con la versión de Django que se ha usado en prácticas.

La práctica deberá funcionar desde el navegador Firefox disponible en el laboratorio de la asignatura.

%%----------------------------------------------------------------------------
\subsection{Preguntas frecuentes}
\label{sec:practica-2023-05:preguntas}

A continuación, algunas preguntas relacionadas con el enunciado de esta práctica, junto con sus respuestas:

\begin{itemize}

\item ¿Cómo puedo construir la ``Página dinámica de cada sala''?

  Se hará apoyándose en el recurso ``Sala en formato JSON'', que ha de existir para cada sala. El recurso que sirva la página dinámica de cada sala utilizará código JavaScript para descargar periódicamente (por ejemplo cada 30 segundos) el documento en formato JSON, y actualizar con él un elemento \texttt{div}, en el que se mostrarán todos los mensajes que haya en ese documento JSON. Puede verse un ejemplo en el directorio \texttt{Ajax/messages} del repositorio de código de la asignatura. Puede probarse el contenido de ese directorio lanzando el servidor Python de una línea en él:

\begin{verbatim}
python3 -m http.server
\end{verbatim}

  y a continuación, apuntando el navegador a la URL que sirve.

  En general, bastará con utilizar el mismo fichero \texttt{messages.json} que puede encontrarse en ese directorio, referenciándolo en el documento HTML que sirva el recurso de la página dinámica de cada sala. Para que funcione, ha de haber un elemento \texttt{div} en la página, que será en el que el script JavaScript incluirá los mensajes del documento JSON. La URL del documento JSON de cada página está en el propio HTML, y podrá generarse dinámicamente con la plantilla de la página en cuestión.
  
\item Cuando despliego mi práctica en Python Anywhere, algunos recursos reconocidos no me funcionan, pero otros (YouTube entre ellos), sí. Todo me funciona bien en mi versión local. ¿Qué está pasando?

  Las máquinas virtuales de Python Anywhere están limitadas en cuanto a los sitios a los que se pueden conectar: sólo se pueden conectar a aquellos que están en una cierta ``lista blanca''. Por eso, si el sitio al que tu programa se tiene que conectar para obtener recursos no está en la lista blanca, no va a poder descargarse el documento XML o JSON de esos recursos. Para evitar problemas, en el caso de despliegue en Python Anywhere pedimos que funcionen bien los recursos reconocidos que están en la lista blanca, y para los demás, que tengan recursos en la base de datos de despliegue. Más detalles en el apartado sobre despliegue de este enunciado (\ref{sec:practica-2023-05:despliegue}).
  
\item En la pagina de información que se menciona en el enunciado, ¿qué hay que incluir en el apartado de documentación?

Casi que lo que queráis, lo importante es tener la página. Puede ser por ejemplo un resumen de un párrafo de lo que hace la aplicación.
  
\item ¿Es necesario utilizar los mecanismos provistos por Django para el control de sesiones y autenticación?

  En principio, esa es la solución recomendada. El principal problema suele ser asegurarse de que cualquier mecanismo alternativo funciona al menos tan bien como el de Django, lo que no es en general trivial. De todas formas, salvo muy buenos motivos, la aplicación es una aplicación Django, y por lo tanto cuantas más facilidades de Django se usen (bien usadas), mejor.

\item ¿Dónde puedo realizar el despliegue de la aplicación?

  El despliegue puede realizarse en cualquier ordenador que esté conectado permanentemente a Internet durante el periodo de corrección, en una dirección accesible desde cualquier navegador conectado a su vez a Internet. Esto puede ser por ejemplo un ordenador personal en un domicilio con acceso permanente a Internet, adecuadamente configurado (puede ser una Raspberry Pi o similar, si se busca una solución simple y de bajo coste). También puede ser un servicio en Internet, por ejemplo uno gratuito como los que ofrecen Google (instrucciones\footnote{GCP Quickstart Using a Linux VM:\\ \url{https://cloud.google.com/compute/docs/quickstart-linux}}, precios\footnote{Google Compute Engine Pricing:\\ \url{https://cloud.google.com/compute/pricing}}), o PythonAnywhere (instrucciones\footnote{Capítulo ``Deploy!'' de Django Girls Tutorial:\\ \url{https://tutorial.djangogirls.org/en/deploy/}}, precios\footnote{PythonAnywhere Plans and Pricing:\\ \url{https://www.pythonanywhere.com/pricing/}}). Los profesores podremos ayudar de forma más detallada con PythonAnywhere.

\item Algunos documentos no se descargan de PythonAnywhere. ¿Qué está pasando?

  Algunos documentos estáticos (por ejemplo, la hora de estilo CSS que estoy usando) no se carga en el navegador cuando despliego la práctica en PythonAnywhere. Sin embargo, cuando pruebo en mi ordenador, o en los ordenadores del laboratorio, todo parece ir bien. ¿Qué está pasando.

  Lo que ocurre es que para servir los ficheros estáticos (los que no genera tu aplicación Django, sino que simplemente los sirve a partir de ficheros ya existentes), hay que indicarle a PythonAnywhere qué ficheros son esos, y para qué recursos deben servirse. Es muy típico que esto ocurra, por ejemplo, con el fichero que tenga la hoja de estilo CSS. Puedes ver esto en la consola web, donde indica:

  \begin{verbatim}
Static files:

Files that aren't dynamically generated by your code, like CSS, JavaScript or uploaded files, can be served much faster straight off the disk if you specify them here. You need to Reload your web app to activate any changes you make to the mappings below.
\end{verbatim}

  Añade en esa tabla tus ficheros, y si no hay más problemas te funcionará.

\end{itemize}
